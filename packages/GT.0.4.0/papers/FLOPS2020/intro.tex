\section{Introduction}

Frederic Brooks in his seminal book on software engineering ``The Mythical Man-Month''~\cite{MMM} has characterised the essence of programming with the following words:

\blockquote{``The programmer, like the poet, works only slightly removed from pure thought-stuff. He builds his castles in the air, from air, creating by exertion of the imagination. Few media of
creation are so flexible, so easy to polish and rework, so readily capable of realising grand conceptual structures. (As we shall see later, this very tractability has its own problems.)''}

Indeed, the virtuality of programs and flexibility of their representation call for structuring; the lack of proper structure easily leads to disastrous consequences
(as it happened to some real-world industrial projects in the past). One of commonly used ways to bring a
structure in software are \emph{data types}. Data types allow to describe the properties of data, what can and cannot be done, and to some extent they prescribe
the semantics to data structures. Being kept in runtime, data types make it possible to implement meta-transformations by analysing types (\emph{introspection})
or even creating new types on the fly (\emph{reflection}).

However, in statically typed languages, as a rule, types are completely erased after the compilation and do not retained in runtime. This has a huge advantage over
dynamic typing as, first, programs do not need to inspect types at runtime anymore and, second, a whole class of bad runtime behaviours~--- type errors~---
is eliminated. The other side of the coin, however, is that now some transformations, which in untyped languages can be implemented ``once and for all'',
can not be typed and have to be re-implemented for each type of interest. One way to overcome this deficiency is to develop a more powerful type system in
which more functions can be typed; as an example we may mention the support for \emph{ad-hoc} polymorphism in \textsc{Haskell} in the forms of type
classes~\cite{TypeClasses} and type families~\cite{TypeFamilies}. However, due to the totality of type checking and fundamental undecidability results there
will always be some ``good'' programs which cannot be typed. Another approach, \emph{datatype-generic programming}~\cite{DGP}, is aimed at developing techniques for
implementation of practically important families of type-indexed functions using existing language features. For example, types can be encoded in a substrate
language~\cite{Hinze,InstantGenerics,GenericOCaml}, or a part of type information can be saved for runtime~\cite{SYB,SYBOCaml}, or generic functions for a given
type can be generated at compile-time automatically~\cite{Yallop,PPXLib}. The two approaches we mentioned are in fact complementary~--- the more powerful
type system is the more means for datatype-generic programming the language can incorporate natively. For example, parametric polymorphism makes it possible
to natively express many generic functions like length of list of arbitrary elements, etc.

We present a generic programming library \textsc{GT}\footnote{\url{https://github.com/kakadu/GT/tree/ppx}} (\emph{Generic Transformers}), which has been in an
active development and use since 2014. One of the important observations, which motivated the development of our framework, was that many generic functions
can be considered as a modifications of some other generic functions. While our approach is generative~--- we generate generic functionality from type definitions~---
it also makes possible for end users to easily derive variants of generated functions by redefining some parts of their functionality. This is achieved using
a method-per-constructor encoding for concrete transformations, which resembles the approach of object algebras~\cite{ObjectAlgebras}.

The main properties of our solution are as follows:

\begin{itemize}
\item each transformation is expressed in terms of a \emph{traversal function} and a \emph{transformation object}, which encapsulate the ``interesting part''
  of the transformation;
\item the traversal function is unique for given type and all transformation objects for the type are instances of a unique class;
\item both the traversal function and the class are generated from type definition; we support regular ADTs, structures, polymorphic variants and predefined types;
\item we provide a number of plugins which generate practically important transformations in the form of concrete transformation classes;
\item the plugin system is extensible: end users can implement their own plugins.
\end{itemize}

The library we present is an inheritor of our earlier work~\cite{SYBOCaml} on implementation of ``Scrap Your Boilerplate'' approach~\cite{SYB,SYB1,SYB2}. However,
our experience has shown, that the expressivity and extensibility of SYB is insufficient; in addition the uniform transformations, based solely on type discrimination, turned out to be
inconvenient to use. Our idea initially was to combine combinator and object-oriented approaches~--- the former would provide means for parameterization, while the
latter~--- for extensibility via late binding utilisation. This idea in the form of a certain design pattern was successfully evaluated~\cite{SCICO} and then reified
in a library and a syntax extension~\cite{TransformationObjects}. Our follow-up experience with the library~\cite{OCanren} has (once again) shown some flaws in the
implementation. The version we present here is almost a complete re-implementation with these flaws fixed.

The rest of the paper is organised as follows. In the next section we give an informal, example-driven exposition of
the approach we use. Then we describe the implementation in more details, highlighting some aspects which we find important or
interesting. In the following section we consider some examples, implemented with the aid of the library. Related works
section observes the relevant approaches and frameworks and compares them to ours. The final section discusses the directions for
future development.

%Acknowledgements ...

