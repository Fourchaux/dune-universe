\section{Implementation}

The main components of our solution are syntax extensions (both in terms of \cd{camlp5}~\cite{Camlp5} and \cd{ppxlib}~\cite{PPXLib}), a runtime library and
a plugin system. The syntax extensions process type definitions, attributed by an end user, and generate the following entities:

\begin{itemize}
\item a transformation function (one per a type);
\item a virtual class which is used as a common ancestor for all concrete transformations (one per a type);
\item a number of concrete classes (one per requested plugin);
\item a \emph{typeinfo} structure, which incorporates a type-specific information like the transformation function and a bundle of
  plugin-generated concrete functions, represented as an immediate object.
\end{itemize}

We support the majority of variants in the right-hand side of type definitions with the following limitations:

\begin{itemize}
\item only regular algebraic data types are supported; GADTs are treated as simple algebraic data types;
\item constraints are not taken into account;
\item ``\lstinline{nonrec}'' definitions, object and module types are not supported;
\item extensible datatypes (``\lstinline{...}''/``\lstinline{+=}'') are not supported.
\end{itemize}

For example, for a type ``\lstinline{t}'' with requested plugin ``\lstinline{show}'' the structure with the following skeleton is generated (``$\dots$'' stands for the parts we omit for now):

\begin{figure}[t]
  \center
  \begin{tabular}{l|l}
    \hline
    \multicolumn{2}{c}{\cd{camlp5} version}\\
    \hline
    \lstinline|@type ... $[$ with  $p_1, p_2, \dots$ $]$| & a syntax construct to generate a support for a type \\
                                                         & with plugins $p_1, p_2, \dots$; mutually recursive definitions \\
                                                         & are supported \\
    \lstinline|@$typ$| & the name for the virtual class for type $typ$ \\
    \lstinline|@$plugin$[$typ$]| & the name for a plugin class for type $typ$ and \\
                                 & plugin $plugin$\\
    \hline
        \multicolumn{2}{c}{\cd{ppxlib} version}\\
    \hline
    \lstinline|type ... = ...|  & a syntax construct to generate a support for a type \\
    \lstinline|and  ... = ...|  & with plugins $p_1, p_2, \dots$ \\
    \lstinline|[@@deriving gt ~options:{ $p_1, p_2, \dots$}]|
  \end{tabular}
  \caption{Extended syntax constructs}
  \label{syntax}
\end{figure}

\begin{lstlisting}
   let $\inbr{gcata_t}$ $\dots$ = $\dots$
   
   class virtual [$\dots$] $\inbr{t}$ =
   object
     $\dots$
   end

   class [$\dots$] $\inbr{show_t}$ $\dots$ =
   object inherit [$\dots$] $\inbr{t}$ $\dots$
     $\dots$
   end

   let t = {
     gcata   = $\inbr{gcata_t}$;
     $\dots$
     plugins = object
                 method show = $\dots$
               end
   }
\end{lstlisting}

Using the typeinfo structure ``\lstinline{t}'' we can mimic the type-indexed nature of concrete transformations:

\begin{lstlisting}
   let transform typeinfo = typeinfo.gcata
   let show      typeinfo = typeinfo.plugins#show
\end{lstlisting}

The function ``\lstinline{transform(t)}'' is a top-level transformation function, which can be instantiated for any supported type ``\lstinline{t}''. On the
Figure~\ref{syntax} we describe the concrete constructs, implemented by the syntax extensions. Note, the concrete way of encoding names
for classes and transformation function (represented above as $\inbr{...}$) is not important as long as \cd{camlp5} is
used since it provides corresponding syntax extensions.

\subsection{Types of Transformations}

The design of the library is based on the idea to describe transformations (e.g. catamorphisms~\cite{Bananas}) in terms of attribute grammars~\cite{AGKnuth,AGSwierstra,ObjectAlgebrasAttribute}.
In short, we consider only the transformations of the following type

\[
\iota \to t \to \sigma
\]

where $t$ is the type of a value to transform, $\iota$ and $\sigma$~--- types for \emph{inherited} and \emph{synthesised} attributes. We do not use attribute
grammars as a mean to describe the algorithmic part of transformations; we only utilise their terminology to describe the types. 

When the type under consideration is parameterised, the transformation becomes parameterised as well. From now on we will use a convention to
denote $\left\{...\right\}$ multiple occurrences of a an entity inside the brackets. Under this convention we may stipulate the generic form of
any transformation, representable with the aid of our library, as

\[
  \left\{\iota_i \to \alpha_i \to \sigma_i\right\}\to\iota \to\left\{\alpha_i\right\}\;t \to \sigma
\]

Here $\iota_i\to\alpha_i\to\sigma_i$ is an argument-transforming function for the type parameter $\alpha_i$. In general the argument-transforming functions operate on
inherited values of different types and return synthesised values of different types. The common ancestor class in turn is massively polymorphic: for an $n$-parametric
type it receives $3(n+1)$ type parameters:

\begin{itemize}
\item a triplet $\iota_i$, $\alpha_i$, $\sigma_i$ for each type parameter $\alpha_i$, where $\iota_i$ and $\sigma_i$ are type variables for inherited and
  synthesised attributes for the transformation of $\alpha_i$;
\item a pair of type variables $\iota$ and $\sigma$ for inherited and synthesised attributes for the type itself;
\item an extra type variable $\epsilon$, which is inferred to ``\lstinline|$\{\alpha_i\}$ t|'' for non-polymorphic variant types and to
  an \emph{open} version ``\lstinline|[> $\{\alpha_i\}$ t]|'' for polymorphic variants (see Section~\ref{pv}).
\end{itemize}

For example, if we have a two-parametric type \lstinline{($\alpha$, $\beta$) t} the head of common ancestor class definition will look like

\begin{lstlisting}
  class virtual [$\iota_\alpha$, $\alpha$, $\sigma_\alpha$, $\iota_\beta$, $\beta$, $\sigma_\beta$, $\iota$, $\epsilon$, $\sigma$] $\inbr{t}$
\end{lstlisting}

The concrete transformations inherit from the common ancestor class and, possibly, instantiate some of its type parameters to a more
specific types. Additionally, concrete classes receive a number of functional arguments:

\begin{itemize}
\item $n$ argument-transforming functions: \lstinline|f$_{\alpha_i}$ : $\iota_i$ -> $\alpha_i$ -> $\sigma_i$|;
\item a function to implement open recursion: \lstinline|fself : $\iota$ -> $\epsilon$ ->  $\sigma$|.
\end{itemize}

For example, for the same type as above and a transformation ``\lstinline{show}'' the header of concrete class looks like

\begin{lstlisting}
  class [$\alpha$, $\beta$, $\epsilon$] $\inbr{show_t}$ 
    (f$_\alpha$ : unit -> $\alpha$ -> string)
    (f$_\beta$ : unit -> $\beta$ -> string)
    (fself : unit -> $\epsilon$ -> string) =
  object inherit [unit, $\alpha$, string, unit, $\beta$, string, unit, $\epsilon$, string] $\inbr{t}$
    $\dots$
  end 
\end{lstlisting}

Note, we maintain these conventions for all types although for some of them some of components are superfluous: for example, ``\lstinline{fself}''
is needed only for recursive types. The explanation for this decision is simple: when we \emph{use} a type we generally do not know its
definition. Thus, in order to support separate compilation the interfaces of all entities we generate must have identical structure.

This scheme of typing and pasteurisation looks quite verbose and cumbersome: there are a lot of type parameters which are quite easy to get a mess with. However, end
users would need to deal with this stuff directly only when they desire to implement a transformation \emph{manually} from scratch by immediately inheriting from the common ancestor class.
In the majority of use cases the transformation is implemented either by customising a certain plugin or using the plugin system. In the first case many
type parameters are already instantiated (for example, for ``\lstinline{show}'' the majority of type parameters are instantiated to ground types), in the
second the plugin system takes care of instantiating the parameters correctly (see Section~\ref{plugins}).

We also need to describe the types for the methods of common ancestor classes. The method for a
constructor ``\lstinline|C of a$_1$ * a$_2$ * ... * a$_k$|'' has the following definition:

\begin{lstlisting}
   method virtual $\inbr{C}$ : $\iota$ -> $\epsilon$ -> a$_1$ -> a$_2$ -> ... -> a$_k$ -> $\sigma$
\end{lstlisting}

Note, the method takes not only inherited attribute and the arguments of corresponding constructor, but the value under transformation itself.

Finally, we describe the type of transformation function. This type is slightly different for polymorphic variant types.

For a non-polymorphic variant type ``\lstinline|$\{\alpha_i\}$ t|'' the transformation function has the following type:

\begin{lstlisting}
   val $\inbr{gcata_t}$ : [$\{\iota_{\alpha_i}$, $\alpha_i$, $\sigma_{\alpha_i}\}$, $\iota$, $\{\alpha_i\}$ t, $\sigma$]#$\inbr{t}$ -> $\iota$ -> $\{\alpha_i\}$ t -> $\sigma$
\end{lstlisting}

Thus, it takes a transformation object (which has a type of properly parameterised \emph{subclass} of the common ancestor class), an inherited
attribute, a value to transform, and returns synthesised attribute. The extra type parameter ``$\epsilon$'' is instantiated to the
type itself. For a polymorphic variant type the extra type parameter is instantiated to the \emph{open} version of the
type (``\lstinline{[> $\{\alpha_i\}$ t]}''). This enables the possibility to apply a transformation function for a type to a transformation object for
another type with more constructors.


\subsection{Fixed Point Combinator and Memoization}
\label{memofix}

In our approach we use open recursion: a class for a concrete transformation takes a function for the same transformation as a parameter. In order
to instantiate this function we have to use a fix point combinator. In this section we consider only a simple fix point combinator for an isolated
type definition; in mutually-recursive case a more elaborated combinator is needed (see Section~\ref{murec}).

We repeat here an example from Section~\ref{expo}:

\begin{lstlisting}
   let $\inbr{pretty_{ex pr}}$ e =
     fix (fun fself p e -> $\inbr{gcata_{expr}}$ (new $\inbr{pretty_{expr}}$ fself) p e) min_int e
\end{lstlisting}

As the lambda argument of ``\lstinline{fix}'' is called each time when ``\lstinline{fself}'' is called (virtually, for each node of
an expression), a new transformation object is created for each node. As all these objects are identical, this can be optimised. 

We memoize the creation of transformation objects using lazy evaluation. For this we abstract the object creation sub expression into a
function which takes ``\lstinline{fself}'' as an argument. Then the implementation of the fix point combinator is as follows:

\begin{lstlisting}
   let fix gcata make_obj $\iota$ x =
     let rec obj = lazy (make_obj fself)
     and fself $\iota$ x = gcata (Lazy.force obj) $\iota$ x in
     fself $\iota$ x
\end{lstlisting}

This combinator can be used for all types and is not generated. Now we can fix a little bit the definition of ``\lstinline{transform}'':

\begin{lstlisting}
   let transform typeinfo = fix typeinfo.gcata
\end{lstlisting}

With this definition an end used does not need to deal with the fix point combinator explicitly anymore:

\begin{lstlisting}
   let $\inbr{show_{expr}}$ e =
     transform(expr) (fun fself -> new $\inbr{show_{expr}}$ fself) () e
\end{lstlisting}

\subsection{The Plugin System}
\label{plugins}

\begin{figure}[t]
  \center
  \small
  \begin{tabular}{ccp{4cm}}
    Name & Type of the transformation & Comment \\[3mm]
    \hline\\
    \lstinline|show| & \lstinline|$\{$ unit -> $\alpha_i$ -> string $\}$ -> unit -> $\{\alpha_i\}$ t -> string| & conversion to a string \\[2mm]
    \lstinline|fmt| & \lstinline|$\{$ formatter -> $\alpha_i$ -> unit $\}$ -> formatter -> $\{\alpha_i\}$ t -> unit| & formatted output using the ``\lstinline|Format|'' module \\[2mm]
    \lstinline|html| & \lstinline|$\{$ unit -> $\alpha_i$ -> HTML.t $\}$ -> unit  -> $\{\alpha_i\}$ t -> HTML.t| & conversion to HTML representation\\[2mm]
    \lstinline|compare| & \lstinline| $\{$ $\alpha_i$ -> $\alpha_i$ -> comparison $\}$ -> $\{\alpha_i\}$ t -> $\{\alpha_i\}$ t -> comparison| & comparison \\[2mm]
    \lstinline|eq| & \lstinline|$\{$ $\alpha_i$ -> $\alpha_i$ -> bool $\}$ -> $\{\alpha_i\}$ t -> $\{\alpha_i\}$ t -> bool| & equality test \\[2mm]
    \lstinline|foldl| & \lstinline |$\{$ $\alpha$ -> $\alpha_i$ -> $\alpha$ $\}$ -> $\alpha$ -> $\{\alpha_i\}$ t -> $\alpha$| & threading an inherited attribute through all the nodes using a top-down traversal \\[2mm]
    \lstinline|foldr| & \lstinline |$\{$ $\alpha$ -> $\alpha_i$ -> $\alpha$ $\}$ -> $\alpha$ -> $\{\alpha_i\}$ t -> $\alpha$| & threading an inherited attribute through all the nodes using a bottom-up traversal \\[2mm]
    \lstinline|gmap| & \lstinline|$\{$ unit -> $\alpha_i$ -> $\beta_i$ $\}$ -> unit -> $\{\alpha_i\}$ t -> $\{\beta_i\}$ t| & a functor%\\[2mm]
%    \lstinline|eval| & \lstinline|$\{$ $\epsilon$ -> $\alpha_i$ -> $\beta_i$ $\}$ -> $\epsilon$ -> $\{\alpha_i\}$ t -> $\{\beta_i\}$ t| & a variant of functor with an environment ``$\epsilon$'' passed through a transformation\\[2mm]
%    \lstinline|stateful| & \lstinline|$\{$ $\epsilon$ -> $\alpha_i$ -> $\epsilon$ * $\beta_i$ $\}$ -> $\epsilon$ -> $\{\alpha_i\}$ t -> $\epsilon$ * $\{\beta_i\}$ t| & similar to ``\lstinline|eval|'' but allows to update the environment on the way        
  \end{tabular}
  \caption{The list of predefined plugins}
  \label{listofplugins}
\end{figure}

The default behaviour of our framework is to generate the transformation function, the common ancestor class and the typeinfo structure only. It does not
generate any concrete built-in transformations. All concrete transformations are generated by \emph{plugins}, and the plugin system allows end users to
implement their own. There is a number of predefined plugins (see Figure~\ref{listofplugins}), but none of them receives a special treatment from the
rest of the framework.

Each plugin is implemented as a dynamically-loaded object, and to create a plugin an end user has to properly instantiate a compilation unit using an interface
provided by the framework. The same approach is used in a number of existing frameworks~\cite{PPXLib,Yallop}; however, we claim, that in our case the
implementation of a plugin is much simpler. The reason is that the concrete and generic parts of transformations are properly separated. Thus,
a plugin only instantiates a class, and only a limited assistance from an end-user side is needed. Generally speaking, the following information has to be
provided:

\begin{itemize}
\item What are the types of inherited and synthesised attributes for a given type parameter?
\item What are the types of inherited and synthesised attributes for the type itself?
\item What is the body of the method which transforms given constructor (the arguments of the method and their types are specified by the framework)?
\item What the toplevel method of the typeinfo structure for the plugin is look like?
\end{itemize}

So, there are only a limited number of places where a plugin actually needs to generate a code, and as a rule the generated code is very simple. The
code generation interface the plugin system provides resembles that of \cd{ppxlib} (more precise, \cd{Ast_builder}), which has to be familiar to anyone,
who has ever implemented syntax extensions. In the Section~\ref{pluginExample} we present a complete example of fresh plugin implementation.

\subsection{Mutual Recursion}
\label{murec}

The full support for mutually recursive type definitions requires extra efforts. While, formally, the generation of all needed entities for
mutually recursive definitions can be done in a similar manner as for the isolated case, it would break the extensibility of
transformations. We demonstrate this phenomenon by the following example. Let us have the definition

\begin{lstlisting}
   type expr = $\dots$ | LocalDef of def * expr
   and  def  = Def of string * expr
\end{lstlisting}

where we omitted a non-relevant part (variables, binary operators, etc.) in expression type declaration. It is rather obvious, that generic
transformation functions for both types can be kept as they are; indeed, they only ``outsource'' the transformations to corresponding
methods and do not depend on recursion in type definitions:

\begin{lstlisting}
   let $\inbr{gcata_{expr}}$ $\omega$ $\iota$ = function
   $\dots$
   | LocalDef (d, e) as x -> $\omega$#$\inbr{LocalDef}$ $\iota$ x d e

   let $\inbr{gcata_{def}}$ $\omega$ $\iota$ = function
   | Def (s, e) as x -> $\omega$#$\inbr{Def}$ $\iota$ x s e
\end{lstlisting}

The same is true for the common ancestor classes. However, when we start implementing concrete transformations, we would need to use a transformation
for ``\lstinline{expr}'' inside the class for ``\lstinline{def}'', and vice versa. This can be done with mutually recursive class definitions (we, again,
omit the non relevant parts):

\begin{lstlisting}
   class $\inbr{show_{expr}}$ fself =
   object inherit [unit, _, string] $\inbr{expr}$ fself
     $\dots$
     method $\inbr{LocalDef}$ $\iota$ x d e =
       $\dots$ (fix $\inbr{gcata_{def}}$ (fun fself -> new $\inbr{show_{def}}$ fself) $\dots$) $\dots$
   end
   and $\inbr{show_{def}}$ fself =
   object inherit [unit, _, string] $\inbr{def}$ fself
     method $\inbr{Def}$ $\iota$ x s e =
       $\dots$ (fix $\inbr{gcata_{expr}}$ (fun fself -> new $\inbr{show_{expr}}$ fself) $\dots$) $\dots$
   end
\end{lstlisting}

Note, in both ``\lstinline{fix}'' sub expressions we instantiated \emph{concrete} classes (``$\inbr{show_{def}}$'' and ``$\inbr{show_{expr}}$''). This should
work as expected at the first glance. Strictly speaking, this \emph{concrete} transformation works. But, what happens when we decide to redefine the behaviour of
this default ``$\inbr{show_{expr}}$''? According to our general approach, we would need to inherit from ``$\inbr{show_{expr}}$'', override certain methods and
construct a function using fix point:

\begin{lstlisting}
   class custom_show fself =
   object inherit $\inbr{show_{expr}}$ fself
     method $\inbr{Const}$ $\iota$ x n = "a constant"
   end

   let custom_show e = fix $\inbr{gcata_{expr}}$ (fun fself -> new custom_show fself) () e
\end{lstlisting}

Alas, this won't work as we desire: we did not override the method ``$\inbr{LocalDef}$'', and it still uses the default version for the type ``\lstinline{def}'', which
still uses the default version for the type ``\lstinline{expr}''. Thus, we only redefined the behaviour of default ``$\inbr{show_{expr}}$'' for one component of
mutually recursive type definition~--- the type ``\lstinline{expr}'' as such. All occurrences of ``\lstinline{expr}'' inside other types will still be handled by
the default transformation. In order to make things work as we want we would need to repeat the \emph{whole} mutually-recursive class definition, which invalidates the
very idea of extensibility. 

Our solution for the problem, again, utilises the idea of open recursion. In short, we parameterise the concrete transformation classes with the same transformations
for \emph{all} components of mutually recursive definition. Since this parameterisation violates the conventions on class interfaces we first generate auxiliary classes.
For our example this auxiliary classes look as follows:

\begin{lstlisting}
   class $\inbr{show\_stub_{expr}}$ $f_{expr}$ $f_{def}$ =
   object inherit [unit, _, string] $\inbr{expr}$ $f_{expr}$
     $\dots$
     method $\inbr{LocalDef}$ $\iota$ x d e = $\dots$ ($f_{def}$ $\dots$) $\dots$
   end
     
   class $\inbr{show\_stub_{def}}$ $f_{expr}$ $f_{def}$ =
   object inherit [unit, _, string] $\inbr{def}$ $f_{def}$
     method $\inbr{Def}$ $\iota$ x s e = $\dots$ ($f_{expr}$ $\dots$) $\dots$
   end
\end{lstlisting}

Note the absence of mutually recursive class definitions. Then, we generate a fix point operator for each mutually recursive definition:

\begin{lstlisting}
   let $\inbr{fix_{expr, def}}$ ($c_{expr}$, $c_{def}$) =
     let rec $t_{expr}$ $\iota$ x = $\inbr{gcata_{expr}}$ ($c_{expr}$ $t_{expr}$ $t_{def}$) $\iota$ x
     and $t_{def}$ $\iota$ x = $\inbr{gcata_{def}}$ ($c_{def}$ $t_{expr}$ $t_{def}$) $\iota$ x in
     ($t_{expr}$, $t_{def}$)
\end{lstlisting}

Here $c_{expr}$ and $c_{def}$ are object generators which take the transformation functions for all components of mutually recursive definition
as parameters. Note, the same fix point generator can be used to construct any concrete transformation for given mutually recursive definition.

With auxiliary classes and the fix point operator we can construct the default implementations for any concrete transformation:

\begin{lstlisting}
   let $\inbr{show_{expr}}$, $\inbr{show_{def}}$ =
     $\inbr{fix_{expr,def}}$ (new $\inbr{show\_stub_{expr}}$, new $\inbr{show\_stub_{def}}$) 
\end{lstlisting}

These default implementations, first, are distributed among the typeinfo structures for relevant types and, second, are used to define conventional
transformation classes:

\begin{lstlisting}
   class $\inbr{show_{expr}}$ fself =
   object inherit $\inbr{show\_stub_{expr}}$ fself $\inbr{show_{def}}$ end

   class $\inbr{show_{def}}$ fself =
   object inherit $\inbr{show\_stub_{def}}$ $\inbr{show_{expr}}$ fself end
\end{lstlisting}

Thus, we again made mutually recursive types indistinguishable from the simple ones (in terms of class interfaces), making it possible to
uniformly generate all transformations with separate compilation support.

On the other hand, in order to extend an existing transformation one needs to inherit from \emph{auxiliary} classes and use the custom fix point operator.
For our previously unsuccessful case the implementation is almost as simple as for the single type definition:

\begin{lstlisting}
   let custom_show, _ =
      $\inbr{fix_{expr,def}}$ ((fun $f_{expr}$ $f_{def}$ ->
                     object inherit $\inbr{show\_stub_{expr}}$ $f_{expr}$ $f_{def}$
                       method $\inbr{Const}$ $\iota$ x n = "a constant"
                     end),
                    new $\inbr{show\_stub_{def}}$) 
\end{lstlisting}

In the actual implementation we generate a memoizing fix point combinator, which follows the same pattern we've described in Section~\ref{memofix}. In addition,
we put the fix point combinator into the typeinfo structure, so, for a type ``\lstinline{t}'' the fix point combinator can be addressed as
``\lstinline{fix(t)}''. End users, however, still need to know the structure of mutually-recursive type definitions in order to use the fix point
combinator properly.

There is one subtlety with our support for mutual recursion: we rely on the property, that adding one function per type is enough to implement open recursion.
However, generally speaking, this is not true: take, for example, the following definition:

\begin{lstlisting}
   type ($\alpha$, $\beta$) a = A of $\alpha$ b * $\beta$ b
   and  $\alpha$ b = X of ($\alpha$, $\alpha$) a
\end{lstlisting}

In the parameters of constructor ``\lstinline{A}'' we have here \emph{different} parameterisations of type ``\lstinline{b}'' and, thus, we would need
\emph{two} functions~--- for ``\lstinline{$\alpha$ b}'' and for ``\lstinline{$\beta$ b}''. However, the type ``\lstinline{a}'' is not regular~--- starting with 
the parameterisation ``\lstinline{($\alpha$, $\beta$) a}'' we can end up with ``\lstinline{($\alpha$, $\alpha$) a}'' and ``\lstinline{($\beta$, $\beta$) a}''.
Thus, we have already ruled such definitions out. In this reasoning we assume that mutually recursive definitions are \emph{essential} in the sense that they
can not be split into separate type declarations (i.e. that every pair of types are mutually ``reachable''). If we replace the second definition in the
example above with, say,

\begin{lstlisting}
   ...
   and $\alpha$ b = int
\end{lstlisting}

then we would end up with a case which is not supported by our framework. However, as types ``\lstinline{a}'' and ``\lstinline{b}'' are actually \emph{not}
mutually recursive, the whole definition can be rewritten, which restores the support.

\subsection{Polymorphic Variants}
\label{pv}

We consider the support for polymorphic variants~\cite{PolyVar,PolyVarReuse} as an important feature of our framework since it complements the ability of defining
composable data structures with the ability of creating composable transformations. The main difference between polymorphic variants and usual algebraic
data types is that it is possible to \emph{extend} previously declared polymorphic variants by adding more constructors or to combine a few types into the one. 

Our goal is to provide a \emph{seamless} integration of generic features: when a few types are being combined we would want to acquire all generic
features for the result type by inheriting the same features from the constituent types.

As we said previously, an extra type parameter ``$\epsilon$'' is inferred to an open version of the polymorphic variant type. Thus, the same generic transformation
function can be used to transform a value using a transformation object for a \emph{wider} type\footnote{We refrain from calling this type a ``subtype'' since there is
no subtyping in \textsc{OCaml}.}. This is achieved by a specific form of generic transformation function, which performs an ``opening'':

\begin{lstlisting}
   let $\inbr{gcata_t}$ $\omega$ $\iota$ subj =
     match subj with
     $\dots$
     | C $\dots$ -> $\omega$#$\inbr{C}$ $\iota$ (match subj with #t as subj -> subj) $\dots$
     $\dots$
\end{lstlisting}

This results in applying the methods of transformation object to an opened version of the type, while the transformation function itself still operates only
of the closed version.

When a few polymorphic variant types are combined, the transformation function simply matches a value against type patterns and dispatches the
transformation to the transformation functions of a corresponding constituent type.
