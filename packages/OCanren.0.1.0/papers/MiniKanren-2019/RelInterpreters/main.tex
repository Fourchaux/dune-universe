\documentclass[acmlarge]{acmart}
\usepackage[
    type={CC},           % your choice
    modifier={by-sa},    % your choice
    version={4.0},       % your choice
]{doclicense}            % your choice, see \doclicenseThis below

\settopmatter{printacmref=false}
\fancyfoot{}

\makeatletter
\def\@formatdoi#1{}
\def\@permissionCodeOne{miniKanren.org/workshop}
\def\@copyrightpermission{\doclicenseThis} 
\def\@copyrightowner{Copyright held by the author(s).}
\makeatother

\copyrightyear{2019}
\setcopyright{rightsretained}

\acmMonth{8}
\acmArticle{3} % your article number, same as in HotCRP



%% Bibliography style
\bibliographystyle{ACM-Reference-Format}
%% Citation style
%% Note: author/year citations are required for papers published as an
%% issue of PACMPL.
\citestyle{acmauthoryear}   %% For author/year citations


%%%%%%%%%%%%%%%%%%%%%%%%%%%%%%%%%%%%%%%%%%%%%%%%%%%%%%%%%%%%%%%%%%%%%%
%% Note: Authors migrating a paper from PACMPL format to traditional
%% SIGPLAN proceedings format must update the '\documentclass' and
%% topmatter commands above; see 'acmart-sigplanproc-template.tex'.
%%%%%%%%%%%%%%%%%%%%%%%%%%%%%%%%%%%%%%%%%%%%%%%%%%%%%%%%%%%%%%%%%%%%%%


%% Some recommended packages.
\usepackage{booktabs}   %% For formal tables:
                        %% http://ctan.org/pkg/booktabs
\usepackage{subcaption} %% For complex figures with subfigures/subcaptions
                        %% http://ctan.org/pkg/subcaption
\usepackage{multirow}




\usepackage{listings}
\lstdefinelanguage{ocanren}{
keywords={run, conde, fresh, let, in, match, with, when, class, type,
object, method, of, rec, repeat, until, while, not, do, done, as, val, inherit,
new, module, sig, deriving, datatype, struct, if, then, else, open, private, virtual, include, success, failure,
true, false},
sensitive=true,
commentstyle=\small\itshape\ttfamily,
keywordstyle=\textbf,%\ttfamily\underline,
identifierstyle=\ttfamily,
basewidth={0.5em,0.5em},
columns=fixed,
mathescape=true,
fontadjust=true,
literate={fun}{{$\lambda$}}1 {->}{{$\to$}}3 {===}{{$\equiv$}}1 {=/=}{{$\not\equiv$}}1 {|>}{{$\triangleright$}}3 {\\/}{{$\vee$}}2 {/\\}{{$\wedge$}}2 {^}{{$\uparrow$}}1,
morecomment=[s]{(*}{*)}
}

\lstset{
%mathescape=true,
%basicstyle=\small,
%identifierstyle=\ttfamily,
%keywordstyle=\bfseries,
%commentstyle=\scriptsize\rmfamily,
%basewidth={0.5em,0.5em},
%fontadjust=true,
language=ocanren
}

\newcommand{\lstquot}[1]{``\lstinline{#1}''}
\newcommand{\sembr}[1]{\llbracket{#1}\rrbracket}
\newcommand\false{$f\!alse$}
\newcommand\myif{i\!f}

\sloppy 

\begin{document}

\title[Relational Interpreters for Search Problems]{Relational Interpreters for Search Problems}    

\titlenote{This work was partially suppored by the grant 18-01-00380 from The Russian Foundation for Basic Research} %% \titlenote is optional;


\author{Petr Lozov}
\email{lozov.peter@gmail.com}        

\author{Ekaterina Verbitskaia}
\email{kajigor@gmail.com}

\author{Dmitry Boulytchev}
\email{dboulytchev@math.spbu.ru}    

\affiliation{
  \institution{Saint Petersburg State University}
  \country{Russia}                   
}

\affiliation{
  \institution{JetBrains Research}   
  \country{Russia}                   
}


%% Abstract
%% Note: \begin{abstract}...\end{abstract} environment must come
%% before \maketitle command
\begin{abstract}
We address the problem of constructing a solver for a certain search problem from
its solution verifier. The main idea behind the approach we advocate is to consider a
verifier as an interpreter which takes a data structure to search in as a program and
a candidate solution as this program's input. As a result the interpreter returns
``$true$'' if the candidate solution satisfies all constraints and ``\false''
otherwise. Being implemented in a relational language, a verifier becomes capable of
finding a solution as well. We apply two techniques to make this scenario realistic:
\emph{relational conversion} and \emph{supercompilation}. Relational conversion makes it possible 
to convert a first-order functional program into relational form, while supercompilation (in the
form of conjunctive partial deduction (CPD))~--- to optimize out redundant computations. We demonstrate
our approach on a number of examples using a prototype tool for \textsc{OCanren}~--- an implementation of
\textsc{miniKanren} for \textsc{OCaml},~--- and discuss the results of evaluation.
\end{abstract}


%% 2012 ACM Computing Classification System (CSS) concepts
%% Generate at 'http://dl.acm.org/ccs/ccs.cfm'.
\begin{CCSXML}
<ccs2012>
<concept>
<concept_id>10011007.10011006.10011008.10011009.10011015</concept_id>
<concept_desc>Software and its engineering~Constraint and logic languages</concept_desc>
<concept_significance>500</concept_significance>
</concept>
<concept>
<concept_id>10011007.10011006.10011041.10011047</concept_id>
<concept_desc>Software and its engineering~Source code generation</concept_desc>
<concept_significance>500</concept_significance>
</concept>
</ccs2012>
\end{CCSXML}

\ccsdesc[500]{Software and its engineering~Constraint and logic languages}
\ccsdesc[500]{Software and its engineering~Source code generation}
%% End of generated code


%% Keywords
%% comma separated list
\keywords{relational programming, relational interpreters, search problems}  %% \keywords are mandatory in final camera-ready submission


%% \maketitle
%% Note: \maketitle command must come after title commands, author
%% commands, abstract environment, Computing Classification System
%% environment and commands, and keywords command.
\maketitle

\thispagestyle{empty}

\section{Introduction}
\label{sec:intro}

Algebraic data types (ADT) are an important tool in functional programming which deliver a way to represent flexible and easy to manipulate data structures.
To inspect the contents of an ADT's values a generic construct~--- \emph{pattern matching}~--- is used. The importance of pattern matching efficient
implementation stimulated the development of various advanced techniques which provide good results in practice. The objective of our work is to use these
results as a baseline for a case study of relational synthesis\footnote{We have to note that this term is overloaded and can be used to refer to completely
different approaches than we utilize.}~--- an approach for program synthesis based on application of relational programming~\cite{TRS,WillThesis}, and,
in particular, relational interpreters~\cite{unified} and relational conversion~\cite{conversion}. Relational programming can be considered as a specific form
of constraint logic programming centered around \textsc{miniKanren}\footnote{\url{http://minikanren.org}}, a combinator-based DSL, implemented for a number of host languages.
Unlike \textsc{Prolog}, which employs a deterministic depth-first search, \textsc{miniKanren} advocates a 
%completely 
more
declarative approach, in which a user is not
allowed to rely on a concrete search discipline, which means, that the specifications, written in \textsc{miniKanren}, are understood much more symmetrically.
The distinctive feature of \textsc{miniKanren} is complete \emph{interleaving search}~\cite{search}. The basic constraint is unification with occurs check, although
advanced implementations support other primitive constructs, such as disequality or finite-domain constraints~\cite{CKanren}. Syntactically, \textsc{miniKanren} is mutually
convertible to \textsc{Prolog}, but, unlike latter, makes use of explicit logical connectives (conjunction and disjunction), existential quantification and unification.
 
A distinctive application of relational programming is implementing \emph{relational interpreters}~\cite{Untagged}. Unlike conventional interpreters, which for a program and
input value produce output, relational interpreters can operate in various directions: for example, they are capable of computing an input value for a given
program and a given output, or even synthesize a program for a given pairs of input-output values. The latter case forms a basis for program synthesis~\cite{eigen,unified}.

Our approach is based on relational representation of the source language pattern matching semantics on the one hand, and
the semantics of the intermediate-level implementation language on the other. We formulate the condition necessary for a correct and complete implementation of pattern matching and use it to
construct a top-level goal which represents a search procedure for all correct and complete implementations. We also present a number of techniques which make it possible to come up with an
\emph{optimal} solution as well as optimizations to improve the performance of the search. Similarly to many other prior works we use the size of the synthesized code, which can be measured
statically, to distinguish better programs. Our implementation\footnote{\url{https://github.com/Kakadu/pat-match/tree/aplas2020}} makes use of \textsc{OCanren}\footnote{\url{https://github.com/JetBrains-Research/OCanren}}~---
 a typed implementation of \textsc{miniKanren} for \textsc{OCaml}~\cite{OCanren}, and \textsc{noCanren}\footnote{\url{https://github.com/Lozov-Petr/noCanren}}~--- 
a converter from the subset of plain \textsc{OCaml} into \textsc{OCanren}~\cite{conversion}. An initial  evaluation, performed for a set of benchmarks taken from other papers, showed our synthesizer performing well.
However, being aware of some pitfalls of our approach, we came up with a set of counterexamples on which it did not provide any results in observable time, so we do not consider the problem
completely solved. We also started to work on mechanized 
formalization\footnote{\url{https://github.com/dboulytchev/Coq-matching-workout}},
written in \textsc{Coq}~\cite{Coq}, to make the justification of our approach more solid and easier to verify, but this formalization is not yet complete. 

 

\begin{comment}
We apply relational programming techniques to the problem of synthesizing efficient implementation for a pattern matching construct.
Although in principle pattern matching can be implemented in a trivial way, the result suffers from inefficiency in terms of both
performance and code size. Thus, in implementing functional languages alternative, more elaborate  approaches are widely used.
However, as there are multiple kinds and flavors of pattern matching constructs, these approaches have to be specifically developed
and justified for each concrete inhabitant of the pattern matching ``zoo''. We formulate the pattern matching synthesis problem in
declarative terms and apply relational programming, a specific form of constraint logic programming, to develop a 
develop optimizations which improve the efficiency of the synthesis and guarantee the
optimality of the result. 
\end{comment}

\section{Searching for Paths in a Graph with a Relational Verifier}
\label{sec:example}

In this section we demonstrate how to solve a concrete problem of searching for paths in a directed graph with a relational verifier. 
A directed graph is a tuple $(N, E, start, end)$, where $N$ is a finite set of \emph{nodes}, $E$ is a finite set of \emph{edges}, functions $start, end : E \rightarrow N$ return a start and an end nodes for a given edge respectively.
A path in a directed graph is a sequence:
\[
\langle n_0, e_0, n_1, e_1, \dots, n_k, e_k, n_{k+1} \rangle
\]

such that 
\[
\forall i \in \{ 0 \dots k \}\; :\; n_i = start\,(e_i) \text{ and } n_{i+1} = end\,(e_i).
\]

The problem of searching for paths in a graph is to find a set $\{ p \mid p \text{ is a path in } g\}$, where $g$ is a graph. 
There~are many concrete algorithms which search for paths in a graph. 
Implementing any of them involves determining in which way to traverse the graph, how to ensure one does not get stuck exploring a cycle in the graph (a cycle is a path in the graph of form $\langle n_0, e_0, \dots, n_k, e_k, n_0 \rangle$), how to ensure one path is not processed multiple times, and so~on. 
A much easier task is to implement a simple verifier, which checks if a sequence is indeed a path in a graph, and generate the path searching routine from it by the relational conversion.

Below is the implementation of the verifier ``\lstinline{isPath}''. 
This function takes as an input a list of nodes ``\lstinline{ns}'' and a graph ``\lstinline{g}''. 
We represent the graph as a list of edges, stipulating there are no parallel edges. 
Each edge $e$ is represented as a pair of nodes $(n, m)$, where $n = start(e)$, $m = end(e)$.
Given $ns = [n_0, \dots, n_{k+1}]$ and a graph $g = [e_0, \dots, e_l]$, the function returns true, if $\exists i_0 \dots i_k \text{ such that } \langle n_0, e_{i_0}, n_1, e_{i_1}, \dots, e_{i_k}, n_{k+1} \rangle$ is a path in $g$.

\begin{lstlisting}[numbers=left,numberstyle=\small,escapeinside={@}{@}]
let rec isPath ns g =
  match ns with
  @\label{lst:isPath_5}@| x$_1$ :: x$_2$ :: xs -> elem (x$_1$, x$_2$) g && isPath (x$_2$ :: xs) g 
  @\label{lst:isPath_4}@| [_]            -> true
\end{lstlisting}

The function ``\lstinline{elem}'' checks if an edge ``\lstinline{e}''  exists in the graph ``\lstinline{g}''. 
We omit the definition of equality check for edges ``\lstinline{eq}'', since it is trivial to implement and is not relevant for the example.

\begin{lstlisting}
let rec elem e g =
  match g with
  | []      -> false
  | x :: xs -> if eq e x then true else elem e xs
\end{lstlisting}

We stipulate that a path must include at least two nodes, since searching for shorter paths is trivial. 
Line~\ref{lst:isPath_5} of the ``\lstinline{isPath}'' definition checks that the first two nodes of the list form an edge of the graph. 
Then it checks that what is left after deleting the first node from the list is still a path in the graph.
Line~\ref{lst:isPath_4} may come off a little counterintuitive, since it states that a path which includes a single arbitrary node is in the input graph.
However we only execute this branch by a recursive call of \lstquot{isPath}, which only happens after we have already ensured with the call to the ``\lstinline{elem}'' function that the said node is in the graph. 

The relational conversion of the verifier function ``\lstinline{isPath}'' generates a relation ``\lstinline{isPath$^o$}'' defined for a path \lstquot{ns}, a graph \lstquot{g} and a boolean value \lstquot{res}, which is true if ``\lstinline{ns}'' is a path in the graph ``\lstinline{g}'' and false otherwise. 
The function ``\lstinline{elem}'' is transformed into a relation ``\lstinline{elem$^o$}'' defined for an edge ``\lstinline{e}'', a graph ``\lstinline{g}'' and a boolean value ``\lstinline{res}'', which is true if ``\lstinline{e}'' is an edge in the graph ``\lstinline{g}'' and false otherwise.
The result of the relational conversion of the functions ``\lstinline{isPath}'' and ``\lstinline{elem}'' is presented below.

\begin{lstlisting}[firstnumber=5, numbers=left,numberstyle=\small,escapeinside={@}{@}]
let rec elem$^o$ e g res = conde [
  (g === nil () /\ res === ^false);
  (fresh (x xs resEq) (
    (g === x % xs) /\ 
    (eq$^o$ e x resEq) /\ 
    (conde [
      (resEq === ^true  /\ res === ^true); 
      (resEq === ^false /\ elem$^o$ e xs res)])))]

let rec isPath$^o$ ns g res = conde [
  (fresh (el) (
    (ns === el % nil ()) /\ 
    (res === ^true));
 @\label{isPatho:fst}@(fresh (x$_1$ x$_2$ xs resElem resIsPath) (
    (ns === x$_1$ % (x$_2$ % xs)) /\ 
    (elem$^o$ (pair x$_1$ x$_2$) g resElem) /\
    (isPath$^o$ (x$_2$ % xs) g resIsPath) /\ 
    (conde [
 @\label{isPatho:die}@     (resElem === ^false /\ res === ^false); 
 @\label{isPatho:lst}@     (resElem === ^true  /\ res === resIsPath)])))]
\end{lstlisting}

Here we use the syntax of \textsc{OCanren}. 
A new relation is defined as a recursive function with the keywords ``\lstinline{let rec}''. 
The body of the relation is a goal created with the following goal constructors. 

\begin{itemize}
    \item Disjunction $g_1 \vee g_2$, where $g_1, g_2$ --- some goals. The two goals are evaluated independently and their results are combined.
    \item Disjunction of goal list \lstinline{conde [$g_1; \ldots; g_n$]}, where $g_1; \ldots; g_n$ --- some goals.
    \item Conjunction $g_1 \wedge g_2$, where $g_1, g_2$ --- some goals. The goal $g_2$ is evaluated only if the evaluation of $g_1$ succeeded; the evaluation of $g_2$ uses the results of $g_1$.
    \item Syntactic unification  $t_1 \equiv t_2$, where $t_1, t_2$ --- some terms. Unification is a basic goal constructor. If $t_1$ and $t_2$ can be unified, the goal is considered successful and failed otherwise. 
    \item Relation call $r^n t_1 \dots t_n$ where $r^n$ is a name of some $n$-ary relation, and $t_i$ are terms. 
    \item To introduce fresh variables into scope, one should use $\textbf{fresh} \; (\overline{x}) \; g$, where $\overline{x}$ is a list of variable names.
\end{itemize}

Besides goal constructors we use some syntactic sugar for values and lists. 
``\lstinline{^}'' is used to transform a value into a logic value. 
The empty list is represented as ``\lstinline{nil ()}'', and to construct a new list from a value ``\lstinline{h}'' and a list ``\lstinline{t}'' we use ``\lstinline{h % t}''.
A tuple of ``\lstinline{x}'' and ``\lstinline{y}'' is created with ``\lstinline{pair x y}''.

Regrettably, this relational interpreter suffers from poor performance. 
Query ``\lstinline{isPath$^o$ q <graph> true}'' for path searching takes more than ten minutes even for graphs of 5 nodes. 
This is somewhat expected, considering that the relational conversion generates a relation which can be used for many different queries, which is excessive when any particular query is in question. 
This is, of course, not a desirable behaviour. Fortunately, further transformation of the relation can improve the performance. 

For example, if we consider a query ``\lstinline{isPath$^o$ q <graph> ^true}'', we can simplify lines~\ref{isPatho:fst} through~\ref{isPatho:lst} of its definition. 
First, we notice that, having ``\lstinline{res}'' be equal to ``\lstinline{^true}'', we can safely remove the disjunct in line~\ref{isPatho:die}, after what the whole ``\lstinline{conde}'' becomes unnecessary and can be removed. 
After moving the unifications for ``\lstinline{resElem}'' and ``\lstinline{resIsPath}'' to the top level, we get the following equivalent definition of the ``\lstinline{isPath$^o$}'' relation. 
Note, that the call to the ``\lstinline{elem$^o$}'' relation is done with the last argument being unified with ``\lstinline{^true}'', so further specialization is still possible. 

\begin{lstlisting}[firstnumber=25, numbers=left,numberstyle=\small,escapeinside={@}{@}]
let rec isPath$^o$ ns g res = conde [
  (fresh (el) (
    (ns === el % nil ()) /\ 
    (res === ^true)));
  (fresh (x$_1$ x$_2$ xs resElem resIsPath) (
    (resElem === ^true) /\
    (resIsPath === ^true) /\
    (ns === x$_1$ % (x$_2$ % xs)) /\ 
    (elem$^o$ (pair x$_1$ x$_2$) g resElem) /\
    (isPath$^o$ (x$_2$ % xs) g resIsPath)))]
\end{lstlisting}

The specialized version of the relation is much more performant than the original one.
Before, searching paths of length 5 took more than 10 minutes while the specialized version finds paths of length 10 in the graph with 100 edges in a few seconds. 

This transformation can be performed automatically with conjunctive partial deduction. 
The result of partially deducing the ``\lstinline{isPath$^o$ q p ^true}'', where ``\lstinline{p}'' and ``\lstinline{q}'' are fresh variables is about 40 lines of code long and it has the same performance as the manually transformed relation. 
We omit the transformed program because of the space concerns, but it can be found in the repository\footnote{https://github.com/Lozov-Petr/miniKanren-2019-Relational-Interpreters-for-Search-Problems}.

\section{Relational conversion}
\label{sec:conversion}

In this section we describe how the relational conversion in the form of \emph{unnesting}~\cite{lozov:miniKanren} is done. 
Unnesting constructs a relational program by a first-order functional program. 

First, a new variable for every subexpression is introduced with the \lstinline{let}-expression. 
Then, all pattern matching and if-expressions are translated into disjunctions, in which unifications are generated for the patterns.
Free variables are introduced into scope with the \lstinline{fresh}.
Every $n$-ary function becomes $(n+1)$-ary relation with the last argument unified with the result.
As a final step, unifications are reordered with relation calls such that to be computed as early as it is possible.

\begin{figure}[h!]
  \centering
  \begin{subfigure}[t]{0.4\textwidth}
    \centering
\begin{lstlisting}
let rec append a b =
  match a with
  | []      -> b
  | x :: xs -> 
    x :: append xs b
\end{lstlisting}
\caption{}
\label{unnesting_example_a}
  \end{subfigure}
  ~
  \begin{subfigure}[t]{0.4\textwidth}
        \centering
\begin{lstlisting}
let rec append a b =
  match a with 
  | []      -> b
  | x :: xs -> 
    let q = append xs b in
    x :: q
\end{lstlisting}
\vspace{-1\baselineskip}
\caption{}
\label{unnesting_example_b}
  \end{subfigure}
  \vskip2mm
  \begin{subfigure}[t]{0.4\textwidth}
        \centering
\begin{lstlisting}
let rec append$^o$ a b c =
  (a === [] /\ b === c) \/
  (fresh (x xs q) (
     (a === x :: xs) /\
     (append$^o$ xs b q) /\
     (c === x :: q))
\end{lstlisting}
\caption{}
\label{unnesting_example_c}
  \end{subfigure}
  ~
  \begin{subfigure}[t]{0.4\textwidth}
        \centering
\begin{lstlisting}
let rec append$^o$ a b c =
  (a === [] /\ b === c) \/
  (fresh (x xs q) (
     (a === x :: xs) /\
     (c === x :: q) /\
     (append$^o$ xs b q))
\end{lstlisting}
\caption{}
\label{unnesting_example_d}
  \end{subfigure}  
\caption{Example of unnesting}
\label{unnesting_example}
\end{figure}

The example of unnesting is shown in Fig.~\ref{unnesting_example}. 
The input functional program is presented in Fig.~\ref{unnesting_example_a}. 
The result of introducing fresh variables for subexpressions is in Fig.~\ref{unnesting_example_b}.
The relational program before the conjuncts are reordered is shown in Fig.~\ref{unnesting_example_c} and the result of the unnesting is presented in Fig.~\ref{unnesting_example_d}.

Note, that the unnesting has limitations: it does not support higher-order functions and partial application. 
A more general method of translation which does not impose the same limitations was developed~\cite{lozov:conversion}. 
Unfortunately, it uses higher-order relations which are not currently supported in conjunctive partial deduction, so we use unnesting. 

The forward execution of the relation mimics the execution of the function from which it was constructed by relational conversion.
This makes forward execution quite efficient, to the detriment of the execution in the backwards direction. 
The unnesting can be modified to improve the performance of  backward execution. 
Let us consider the conversion of a functional conjunction ``\lstinline{f$_1$ x$_1$ && f$_2$ x$_2$}''.

\begin{lstlisting}
fun res ->
  fresh (p) (
    (f$_1$ x$_1$ p) /\
    (conde [
      (p === ^false /\ res === ^false);
      (p === ^true  /\ f$_2$ x$_2$ res)]))
\end{lstlisting}

Mimicking the function evaluation, the forward execution of this code first computes ``\lstinline{f$_1$ x$_1$}''. 
If it fails, then the result ``\lstinline{res}'' is unified with ``\lstinline{false}'', otherwise the second conjunct ``\lstinline{f$_2$ x$_2$}'' is executed and its result is unified with the result. 
This strategy is not efficient in the backward direction, when we know what ``\lstinline{res}'' is. 
The~following relation is much more performant when executed in the backward direction:

\begin{lstlisting}
fun res ->
    conde [
      (res === ^false /\ f$_1$ x$_1$ ^false);
      (f$_1$ x$_1$ ^true    /\ f$_2$ x$_2$ res)]
\end{lstlisting}

In particular, if ``\lstinline{res === ^true}'', both conversions execute ``\lstinline{f$_2$ x$_2$ res}'', but when the first conversion computes ``\lstinline{f$_1$ x$_1$ p}'' with fresh ``\lstinline{p}'', the second executes ``\lstinline{f$_1$ x$_1$ ^true}''. 
Using the second conversion is enough to significantly increase the performance in the backward direction. 
For example, the path search takes several minutes if the first conversion strategy is used, whereas it finishes in less than a second in the second case. 

Choosing the second conversion strategy comes with a price for the forward execution. 
Instead of executing ``\lstinline{f$_1$ x$_1$ p}'', where ``\lstinline{p}'' is fresh, the second strategy executes both ``\lstinline{f$_1$ x$_1$ ^false}'' and ``\lstinline{f$_1$ x$_1$ ^true}''.
In the worst case scenario, when the execution of ``\lstinline{f$_1$}'' does not depend on the last argument, it doubles the number of executions of ``\lstinline{f$_1$}''.

To sum up, by choosing different strategies of the relational conversion we can achieve significant performance improvement. 
There is no single right way of doing the conversion which improves the performance of the execution in every possible direction. 
Choosing a strategy per each relation and each direction manually is not feasible, but it can be achieved with a fully-automatic program transformation, such as conjunctive partial deduction.

\section{Conjunctive Partial Deduction}
\label{sec:cpd}
Specialization~\cite{jones1993partial} is a natural way to tackle the problem of redundant computations when a part of the input is known. 
A fully-automatic specialization technique developed in the domain of logic programming is called \emph{partial deduction}~\cite{komorowski1982partial, lloyd1991partial}. 
It is related to the supercompilation of functional languages~\cite{gluck1994partial, turchin1986concept}. 
The particular flavour of the partial deduction we are interested in is called \emph{conjunctive partial deduction}~\cite{de1999conjunctive}.
As opposed to the partial deduction, conjunctive partial deduction handles  conjunctions of atoms, thus being able to perform such optimizations as tupling~\cite{hu1997tupling} and deforestation~\cite{wadler1988deforestation}.
Below we demonstrate by example the features of conjunctive partial deduction.

\emph{Deforestation} is a program transformation which gets rid of intermediate data structures. 
The following example demonstrates deforestation. 
Consider a goal ``\lstinline{append$^o$ xs ys ts /\ append$^o$ ts zs rs}'' (note the shared ``\lstinline{ts}''), where ``\lstinline{append$^o$ x y xy}'' describes concatenation, ``\lstinline{nil ()}'' constructs the empty list, and ``\lstinline{h % t}'' constructs a new list from the value ``\lstinline{h}'' and another list ``\lstinline{t}'' (similarly to ``\lstinline{cons}'' in \textsc{Scheme} and ``\lstinline{::}'' in \textsc{OCaml}).

\begin{lstlisting}[label={cpd:appendo}]
let rec append$^o$ x y xy = conde [
  (x === nil () /\ xy === y);
  (fresh (h t ty) (
     (x  === h % t)  /\  
     (xy === h % ty) /\
     (append$^o$ t y ty)))]
\end{lstlisting}

This goal concatenates three lists: ``\lstinline{xs}'', ``\lstinline{ys}'', ``\lstinline{zs}'', constructing an intermediate list ``\lstinline{ts}''. During the execution of this goal, elements of the list ``\lstinline{xs}'' are examined twice: first when ``\lstinline{ts}'' is constructed, and then when the result ``\lstinline{rs}'' is constructed. What is worse, ``\lstinline{ts}'' is only constructed to be immediately deconstructed. Deforestation gets rid of ``\lstinline{ts}'' in this example.  

A better program would be such that does not construct ``\lstinline{ts}'' at all. 
Such a program be generated from the original definition by conjunctive partial deduction and is shown below: 

\begin{lstlisting}[label={cpd:doubleappendo}]
let rec doubleAppend$^o$ xs ys zs rs = conde [
  (xs === nil () /\ append$^o$ ys zs rs);
  (fresh (h t ts) (
     (xs === h % t)  /\  
     (rs === h % ts) /\
     (doubleAppend$^o$ t ys zs ts)))]
\end{lstlisting}


Conjunctive partial deduction is also capable of \emph{tupling}. 
This transformation makes sure that the same data structure is traversed once even if computing several results. 
The following example demonstrates such a case. 

The goal ``\lstinline{maxLength$^o$ xs m l}'' computes both the maximum value of the list ``\lstinline{xs}'' and its length. 
The elements of the list are Peano numbers with ``\lstinline{zero ()}'' as the zero and ``\lstinline{succ}'' as the successor function.
The third argument ``\lstinline{b}'' of the relation ``\lstinline{le$^o$ x y b}'' is ``\lstinline{^true}'' if ``\lstinline{x}'' is less or equal than ``\lstinline{y}'', and ``\lstinline{^false}'' otherwise. The relation ``\lstinline{gt$^o$ x y b}'' is similar to ``\lstinline{le$^o$ x y b}'', but it checks for ``\lstinline{x}'' to be greater than ``\lstinline{y}''. 

\begin{lstlisting}[label={cpd:maxandlength}]
let maxLength$^o$ xs m l = max$^o$ xs m /\ length$^o$ xs l

let rec length$^o$ xs l = conde [
  (xs === nil () /\ l === zero ());
  (fresh (h t m) (
    xs === h % t /\ l === succ m /\ length$^o$ t m))]

let max$^o$ xs m = max$_1^o$ xs (zero ()) m

let rec max$_1^o$ xs n m = conde [
  (xs === nil () /\ m === n);
  (fresh (h t) (
    (xs === h % t) /\
    (conde [
      (le$^o$ h n ^true /\ max$_1^o$ t n m); 
      (gt$^o$ h n ^true /\ max$_1^o$ t h m)])))]

let rec le$^o$ x y b = conde [
  (x === zero () /\ b === ^true); 
  (fresh (x$_1$) (
    x === succ x$_1$ /\ y === zero () /\ b === ^false)); 
  (fresh (x$_1$ y$_1$) (
    x === succ x$_1$ /\ y === succ y$_1$ /\ le$^o$ x$_1$ y$_1$ b))]

let rec gt$^o$ x y b = conde [
  (x === zero () /\ b === ^false);
  (fresh (x$_1$) (
    x === succ x$_1$ /\ y === zero () /\ b === ^false));
  (fresh (x$_1$ y$_1$) (
    x === succ x$_1$ /\ y === succ y$_1$ /\ gt$^o$ x$_1$ y$_1$ b))]
\end{lstlisting}


Execution of the goal ``\lstinline{maxLength$^o$ xs m l}'' leads to ``\lstinline{xs}'' being traversed twice. 
There is a way to rewrite the program so that ``\lstinline{xs}'' is traversed once, but this requires fusing together the definitions of ``\lstinline{length$^o$}'' and ``\lstinline{max$^o$}'', which either restricts code reuse, or leads to code duplication. 
A better way is to only fuse the definitions when it is needed, and do it automatically by employing tupling. 

The desirable implementation of the ``\lstinline{maxLength$^o$ xs m l}'' relation is the following (the definitions of ``\lstinline{gt$^o$}'' and ``\lstinline{le$^o$}'' are left out for brevity). It can be achieved with conjunctive partial deduction as well: 

\begin{lstlisting}[label={cpd:maxlen}]
let maxLength$^o$ xs m l = maxLength$_1^o$ xs m (zero ()) l

let rec maxLength$_1^o$ xs m n l = conde [
  (xs === nil () /\ m === n /\ l === zero ());
  (fresh (h t l$_1$)
     (xs === h % t) /\
     (l === succ l$_1$) /\
     (conde [
       (le$^o$ h n /\ maxLength$_1^o$ t m n l);
       (gt$^o$ h n /\ maxLength$_1^o$ t m h l)]))]
\end{lstlisting}

\subsection{CPD for Prolog-like languages}

Initially, conjunctive partial deduction was developed for Prolog-like languages.
Conjunctive partial deduction partially evaluates goals, which are conjunctions of atoms, using two levels of control: local and global~\cite{gluck1996controlling}. The global control determines which atoms are to be partially deduced. The local control~--- what the definitions for the atoms selected at the global control shall be.
Both local and global control construct tree structures which represent the input program. 

Local control constructs finite SLD-trees for conjunctions of atoms. 
The construction is guided with an \emph{unfold} operator: it selects a literal from the leaf of the partially constructed SLD-tree and adds its resolvents as children at each step.
Since, in general, SLD-trees are infinite, a decision to stop unfolding should be made at some point. 
There are several techniques for doing this, the most promising of them combine determinacy and either some well-founded or well-quasi order, such as homeomorphic embedding, or other measures. 

Global control determines the set of the conjunctions for which partial SLD-trees are built.
The important goal of the global control is to ensure termination.
The termination is achieved with the \emph{abstraction}.
If there is a goal which is embedded into the current goal, it points to the possibility of nontermination. 
The embedding tells that there is a certain similarity between the two goals, and if a current goal keeps being processed, then their similar subpart will appear again and again, causing nontermination.
Whenever the embedding goals are detected, the current goal is abstracted to remove the common subgoal from consideration. 

When the partial deduction is done, the only thing left is to construct the \emph{residual program}.
The clauses are generated from a partial SLD-tree, one tree per conjunction at the global level. 
A conjunction is uniquely \textit{renamed} to give a name for the predicate being defined. 
All free variables of the root of the tree become arguments of the predicate. 
For each non-failing path in the SLD-tree a clause is generated: a substitution collected along the path is substituted into the head of the clause, and the body is generated from what is in the leaf. 

\subsection{CPD for \textsc{miniKanren}}

In this section we describe how we adapted conjunctive partial deduction for \textsc{miniKanren}. 
We describe the particular unfolding and generalization strategies as well as discuss how the conjunctive partial deduction had to be modified as a response to the differences between \textsc{Prolog} and \textsc{miniKanren}. 

\subsubsection{Local Control}

Goals in \textsc{miniKanren} are different from those in \textsc{Prolog}-like languages: besides conjunction, disjunction and relation calls, there are explicit unification and  introduction of fresh variables. 
We normalize the input goal so that it was a disjunction of conjunctions of relation calls. 
To do so, we first pop all the fresh variables to the top level (``\lstinline{fresh (x) (p (x) /\ fresh (y) (q(x) \/ r(y, x)))}'' becomes ``\lstinline{fresh (x y) (p(x) \/ q(x) /\ r(y, x))}''). 
Then we transform the goal to be a disjuction of conjunctions of relation calls or unifications. 
All unifications in each conjunction are evaluated to some substitution (or the conjunct is discarded, if some unification fails). 
The normalization allows us to only consider conjunctions of relation calls while doing conjunctive partial deduction.

The local control constructs the following tree structure which represents the goal:

\begin{lstlisting}
type local_tree = 
    Fail
  | Success of subst
  | Leaf    of goal list * subst
  | Disj    of local_tree list
  | Conj    of local_tree * goal list
\end{lstlisting}

Leaf nodes can be either ``\lstinline{Fail}'', ``\lstinline{Success}'' or ``\lstinline{Leaf}''. 
The ``\lstinline{Fail}'' node is created whenever the evaluation of the current goal fails. 
When the current goal evaluates to some substitution, we create the ``\lstinline{Success}'' node with this substitution. 
The last leaf node is called ``\lstinline{Leaf}'', it corresponds to some partially evaluated goal. 
This type of node contains a substitution which has been computed up to this point, and a residual goal.
The goal in this type of node is then examined at the global level. 

``\lstinline{Disj}'' node corresponds to a disjunction in a goal: its children are the local control trees constructed for all disjuncts. 
The last type of nodes is a ``\lstinline{Conj}'' node. 
It is a transient node, which keeps track of a conjunction being unfolded. 

In general, unfolding replaces some of the relation calls with their bodies and partially evaluates them.
The particular unfolding strategy we adhere to is the following. 
At each step only one relation call is replaced with its body: the leftmost selectable relation call.
The selectable relation call is the one which does not embed any of its predecessors~--- goals which were unfolded in order to get the current goal. 
Embedding here is the modification of the homeomorphic embedding defined for the conjunctions of goals in conjunctive partial deduction literature~\cite{de1999conjunctive}. 
Since using pure embedding to control unfolding leads to hideously big programs, we also allow only one non-deterministic unfold.

\subsubsection{Global Control}

The conjunctions in the ``\lstinline{Leaf}'' nodes are processed at the global level. 
This step is responsible for the termination of the transformation. 
Generally speaking, the danger for nontermination arises whenever we encounter a subgoal which we have encountered before: processing the same thing will lead to itself over and over again. 
To break the vicious circle, one needs to stop unfolding the encountered subgoal, this is what \emph{abstraction} serves for.

The simplest case here is when we come upon the goal which is equal up to variable renaming to any other goal at the global level. 
When this happens, we stop exploring the goal. 
This is called \emph{variant check} in the literature, and it is done both at the global and the local control levels. 

The more complicated case is when a subpart of the goal repeats. 
This case we test with the modification of the homeomorphic embedding relation (strict homeomorphic embedding), initially developed for conjunctions. 
A conjunction $\overline{A}$ is considered embedded into a conjunction $\overline{B}$ when there is an ordered subconjunction within $\overline{A}$, each conjunct of which is embedded into the corresponding conjunct of $\overline{B}$:
\[
\overline{A} = A_0 \wedge A_1 \wedge \dots \wedge A_n \trianglelefteq B_0 \wedge B_1 \wedge \dots \wedge B_m = \overline{B}, \, \myif \, \exists \{ i_0 \dots i_m \mid \forall j.  i_j < i_{j+1} \}: \forall j \in \{0 \dots m\}. A_{i_j} \trianglelefteq B_j 
\]

A single conjunct is embedded into another ($A_i \trianglelefteq B_j$) when the following relation holds and $A_i$ is \emph{not} a strict instance of the second one $B_j$: 
\[
X \trianglelefteq Y, \text{where } X \text{ and } Y \text{ are variables}
\]
\[
f(x_0, x_1, \dots, x_n) \trianglelefteq f (y_0, y_1, \dots, y_n) \Leftrightarrow \forall i \in \{ 0 \dots n \}. x_i \trianglelefteq y_i
\]
\[
f \trianglelefteq g( y_0, y_1, \dots y_m) \Leftrightarrow \exists i \in \{ 0 \dots m \}. f \trianglelefteq y_i
\]

This check determines two major causes of the growth within the conjunctions. 
The conjunction can grow in some argument of a relation call or the number of conjuncts itself can grow. 
To mitigate the first source of the growth, the bigger conjunction can be replaced with a \emph{most specific generalization} of the two conjunctions.
Otherwise we need to \emph{split} the embedded subconjunction from the rest and start processing them separately. 
This process called \emph{abstraction} removes the subconjunctions which cause potential nontermination, and what is left should indeed be processed further.

\subsubsection{Residualization}

After the transformation is finished, a \emph{residual} program is constructed from the global control tree. 
A relation definition is generated for each conjunction at the global level (this is done with the renaming step of the original conjunctive partial deduction).
First, a unique name is given for each conjunction. 
Then free variables of the conjunction are collected to become the arguments of the relation: the constructors and constants are omitted (for example ``\lstinline{f x (succ y) /\ g (zero ()) z}'' becomes ``\lstinline{fG x y z}''.
The body of the definition is generated from the local control tree which corresponds to the conjunction under consideration.
The body is formed as a disjunction of conjunctions for the non-failure nodes of the local control tree. 
A computed substitution is transformed into a conjunction of unifications.
Suitable definitions are chosen for a goal in a leaf, and the conjunction of their applications is generated. 
As a final step we perform redundant argument filtering as described in~\cite{leuschel1996redundant}, and introduce fresh variables where necessary.

\section{Evaluation}

\label{sec:evaluation}

In this section, we present an evaluation of 
implemented constructive negation on a series of examples.

\subsection{If-then-else}

Using relational if-then-else operator, 
presented in section~\ref{sec:ifte},
we have implemented several 
higher-order relations over lists, namely 
\lstinline{find} (Listing~\ref{lst:eval-find}), 
\lstinline{remove}\footnote{Note, this implementation 
differs from the one in Section~\ref{sec:intro}, but 
it is easy to see that these two are semantically equivalent.} (Listing~\ref{lst:eval-remove}) 
and \lstinline{filter} (Listing~\ref{lst:eval-filter}).
These relations are almost identical (syntactically) to their
functional implementations.
We have tested that these relations can be run
in various directions and produce the expected results.
For example, the goal \lstinline{filter p q q}
with the predicate \lstinline{p} equal to

\begin{lstlisting}
  fun l -> fresh (x) (l === [x])
\end{lstlisting}

stating that the given list should be a singleton list,
starts to generate all singleton lists.
Vice versa, the goal \lstinline{filter p q []} 
with that same \lstinline{p} generates 
all lists, constrained to be not a singleton list.

Listings~\ref{lst:eval-p}-\ref{lst:eval-filter-queries} give 
more concrete examples of queries to these relations.
In the listing the syntax \lstinline{run n q g}
means running a goal \lstinline{g} with 
the free variable \lstinline{q}
taking the first \lstinline{n} answers (``\lstinline{*}'' denotes all answers).
After the sign $\leadsto$ the result of the query is given.
The result \lstinline{fail} means that the query has failed.
The result \lstinline[mathescape]|succ {{a$_1$}; ... {a$_n$}} |
means that the query has succeeded delivering $n$ answers.
Each answer represents a set of constraint on free variables.
Constraints are of two forms: equality constraints, e.g. \lstinline{q = (1, _.$_0$)}, 
or disequality constraints, e.g. \lstinline{q $\neq$ (1, _.$_0$)}.
The terms of the form \lstinline{_.$_i$} in the answer
denote some universally quantified variables.

\begin{minipage}[thb]{.3\textwidth}
\begin{lstlisting}[
  caption={A definition of \code{find} relation},
  label={lst:eval-find}
]
let find p e xs =
  fresh (x xs' ys') (
    xs === x::xs' /\
    ifte (p x)
      (e === x)
      (find p e xs')
  )
\end{lstlisting}
\end{minipage}\hfill
\begin{minipage}[thb]{.3\textwidth}
\begin{lstlisting}[
  caption={A definition of \code{remove} relation},
  label={lst:eval-remove}
]
let remove p xs ys =
  (xs === [] /\ ys === [])
  \/
  fresh (x xs' ys') (
    xs === x::xs' /\
    ifte (p x)
      (ys === xs')
      (ys === x::ys' /\ 
       remove p xs' ys')
  )
\end{lstlisting}
\end{minipage}\hfill
\begin{minipage}[thb]{.3\textwidth}
\begin{lstlisting}[
  caption={A definition of \code{filter} relation},
  label={lst:eval-filter}
]
let filter p xs ys =
  (xs === [] /\ ys === [])
  \/
  fresh (x xs' ys') (
    xs === x::xs' /\
    (ifte (p x)
      (ys === x :: ys')
      (ys === ys')) /\
    filter p xs' ys'
  )
\end{lstlisting}
\end{minipage}

% \vspace{3cm}

\begin{minipage}[thb]{0.4\textwidth}
\begin{lstlisting}[
  caption={Definition of the predicate \lstinline{p}},
  label={lst:eval-p}
]
let p l = fresh (x) (l === [x])
\end{lstlisting}
\begin{lstlisting}[
  caption={Example of queries to \lstinline{find}},
  label={lst:eval-find-queries}
]
run 3 q (fresh (e) find p e q) 
$\leadsto$ succ {
     { q = [_.$_0$] :: _.$_1$ }
     { q = _.$_0$ :: [_.$_1$] :: _.$_2$; 
         _.$_0$ $\neq$ [_.$_3$] }
     { q = _.$_0$ :: _.$_1$ :: [_.$_2$] :: _.$_3$; 
         _.$_0$ $\neq$ [_.$_4$]; _.$_1$ $\neq$ [_.$_5$] }
   }
\end{lstlisting}
\end{minipage}\hfill
\begin{minipage}[thb]{0.4\textwidth}
\begin{lstlisting}[
  caption={Example of queries to \lstinline{remove}},
  label={lst:eval-remove-queries}
]
run * q (fresh (e) remove p q [[ ]]) 
$\leadsto$ succ {
     { q = [[_.$_0$]; [ ]] }
     { q = [[ ]] }
     { q = [[ ]; [_.$_0$]] }
   }

run 3 q (fresh (e) remove p q q) 
$\leadsto$ succ {
     { q = [] }
     { q = [_.$_0$], _.$_0$ $\neq$ [_.$_1$] }
     { q = [_.$_0$; _.$_1$]; 
         _.$_0$ $\neq$ [_.$_2$]; _.$_1$ $\neq$ [_.$_3$] }
   }
\end{lstlisting}
\end{minipage}

\begin{minipage}[thb]{0.4\textwidth}
\begin{lstlisting}[
  caption={Example of queries to \lstinline{filter}},
  label={lst:eval-filter-queries}
]
run 3 q (filter p q q) 
$\leadsto$ succ {
     { q = [ ] }
     { q = [_.$_0$] }
     { q = [_.$_0$; _.$_1$] }
   }

run 3 q (filter p q [1]) 
$\leadsto$ succ {
     { q = [[1]] }
     { q = [_.$_0$; [1]]; _.$_0$ $\neq$ [_.$_1$] }
     { q = [[1]; _.$_0$]; _.$_0$ $\neq$ [_.$_1$] }
   }

run 3 q (filter p q [ ]) 
$\leadsto$ succ {
     { q = [] }
     { q = [_.$_0$]; _.$_0$ $\neq$ [_.$_1$] }
     { q = [_.$_0$; _.$_1$]; 
            _.$_0$ $\neq$ [_.$_2$]; _.$_1$ $\neq$ [_.$_3$] }
   }
\end{lstlisting}
\end{minipage}

\subsection{Universal quantification}

In the Section~\ref{sec:impl-univ} we presented 
the \lstinline{forall} goal constructor 
which is implemented through the double negation.
We have observed, that although \lstinline{forall g}
does not terminate when the goal \lstinline{g x} 
has an infinite number of answers 
(assuming \lstinline{x} is a fresh variable),
it does terminate in the case when \lstinline{g x} has 
a finite number of answers.
The behavior of \lstinline{forall} in this case is sound
even in the presence of disequality constraints or nested quantifiers. 

The Table~\ref{tab:univ} gives some concrete examples.
The left column contains the tested goals\footnote{
We typeset the goals in terms of first-order logic syntax 
instead of \textsc{OCanren} syntax for brevity and clarity.} 
and the right column gives the obtained results.
For the results we use the same notation 
as in the previous section.

\begin{table}[th]
  \centering
  \def\arraystretch{1.5}
  \begin{tabularx}{\textwidth}{|X|X|}
    \hline

    $\forall x\ldotp x = q$ & 
      \texttt{fail} \\
    \hline

    $\forall x\ldotp \exists y\ldotp x = y$ & 
      \texttt{succ \{[q = \_.$_0$]\}} \\
    \hline

    $\forall x\ldotp \exists y\ldotp x = y \wedge y = q$ &
      \texttt{fail} \\
    \hline

    $\forall x\ldotp q = (1, x)$ & 
      \texttt{fail} \\
    \hline

    $\forall x\ldotp \exists y\ldotp y = (1, x)$ & 
      \texttt{succ \{[q = \_.$_0$]\}} \\
    \hline

    $\forall x\ldotp \exists y\ldotp x = (1, y)$ &
      \texttt{fail} \\
    \hline

    $\forall x\ldotp x \neq q$ & \texttt{fail} \\
    \hline

    $\forall x\ldotp \exists y\ldotp x \neq y$ & 
      \texttt{succ \{[q = \_.$_0$]\}} \\
    \hline

    $\forall x\ldotp \exists y\ldotp x \neq y \wedge y = q$ & 
      \texttt{fail} \\
    \hline

    $\forall x\ldotp q \neq (1, x)$ & 
      \texttt{succ \{[q $\neq$ (1, \_.$_0$)]\}} \\
    \hline

    $(\exists x\ldotp q = (1, x)) \wedge (\forall x\ldotp q \neq (1, x))$ & 
      \texttt{fail} \\
    \hline

    $\forall x\ldotp (x, x) \neq (0, 1)$ & 
      \texttt{succ \{[q = \_.$_0$]\}} \\
    \hline

    $\forall x\ldotp (x, x) \neq (1, 1)$ & 
      \texttt{fail} \\
    \hline

    $\forall x\ldotp (x, x) \neq (q, 1)$ & 
      \texttt{succ \{[q $\neq$ 1]\}} \\
    \hline

    $\exists a~ b\ldotp q = (a, b) \wedge \forall x\ldotp (x, x) \neq (a, b)$ & 
      \texttt{succ \{[q = (\_.$_0$, \_.$_1$); \_.$_0$ $\neq$ \_.$_1$]\}} \\
    \hline

  \end{tabularx}
  \caption{\lstinline{forall} evaluation}
  \label{tab:univ}
\end{table}

\section{Future Work}

There are a few possible directions for future work. First, in this paper we did not address the performance issues. As we represent
the transformations in a very generic form with many levels of indirection, obviously, the transformations, implemented with
our framework, are at disadvantage in comparison with hard coded ones in terms of performance. We assume that the performance of transformations
can be essentially improved by applying some techniques like staging~\cite{Staged} or, perhaps, object-specific optimisations.

Another important direction is supporting more kinds of type declarations, in the first hand, GADTs and non-regular types. Although we have some
implementation ideas for this case, the solution we came up with so far makes the interface of the whole framework too cumbersome to use even for
simple cases.

Finally, the typeinfo structure we generate can be used to mimic the \emph{ad-hoc} polymorphism as it contains the implementation of
type-indexed functions. This, together with some proposed extensions~\cite{ModularImplicits}, can open interesting perspectives.



\begin{comment}
%% Acknowledgments
\begin{acks}                            %% acks environment is optional
                                        %% contents suppressed with 'anonymous'
  %% Commands \grantsponsor{<sponsorID>}{<name>}{<url>} and
  %% \grantnum[<url>]{<sponsorID>}{<number>} should be used to
  %% acknowledge financial support and will be used by metadata
  %% extraction tools.
  This material is based upon work supported by the
  \grantsponsor{GS100000001}{Russian Foundation for Basic Research}{https://www.rfbr.ru/rffi/eng} under Grant
  No.~\grantnum{GS100000001}{18-01-00380} and by the grant from JetBrains Research. 
  %Any opinions, findings, and
  %conclusions or recommendations expressed in this material are those
  %of the author and do not necessarily reflect the views of the
  %National Science Foundation.
\end{acks}
\end{comment}

\bibliography{references}

\end{document}
