\section{Relational conversion}
\label{sec:conversion}

In this section we describe how the relational conversion in the form of \emph{unnesting}~\cite{lozov:miniKanren} is done. 
Unnesting constructs a relational program by a first-order functional program. 

First, a new variable for every subexpression is introduced with the \lstinline{let}-expression. 
Then, all pattern matching and if-expressions are translated into disjunctions, in which unifications are generated for the patterns.
Free variables are introduced into scope with the \lstinline{fresh}.
Every $n$-ary function becomes $(n+1)$-ary relation with the last argument unified with the result.
As a final step, unifications are reordered with relation calls such that to be computed as early as it is possible.

\begin{figure}[h!]
  \centering
  \begin{subfigure}[t]{0.4\textwidth}
    \centering
\begin{lstlisting}
let rec append a b =
  match a with
  | []      -> b
  | x :: xs -> 
    x :: append xs b
\end{lstlisting}
\caption{}
\label{unnesting_example_a}
  \end{subfigure}
  ~
  \begin{subfigure}[t]{0.4\textwidth}
        \centering
\begin{lstlisting}
let rec append a b =
  match a with 
  | []      -> b
  | x :: xs -> 
    let q = append xs b in
    x :: q
\end{lstlisting}
\vspace{-1\baselineskip}
\caption{}
\label{unnesting_example_b}
  \end{subfigure}
  \vskip2mm
  \begin{subfigure}[t]{0.4\textwidth}
        \centering
\begin{lstlisting}
let rec append$^o$ a b c =
  (a === [] /\ b === c) \/
  (fresh (x xs q) (
     (a === x :: xs) /\
     (append$^o$ xs b q) /\
     (c === x :: q))
\end{lstlisting}
\caption{}
\label{unnesting_example_c}
  \end{subfigure}
  ~
  \begin{subfigure}[t]{0.4\textwidth}
        \centering
\begin{lstlisting}
let rec append$^o$ a b c =
  (a === [] /\ b === c) \/
  (fresh (x xs q) (
     (a === x :: xs) /\
     (c === x :: q) /\
     (append$^o$ xs b q))
\end{lstlisting}
\caption{}
\label{unnesting_example_d}
  \end{subfigure}  
\caption{Example of unnesting}
\label{unnesting_example}
\end{figure}

The example of unnesting is shown in Fig.~\ref{unnesting_example}. 
The input functional program is presented in Fig.~\ref{unnesting_example_a}. 
The result of introducing fresh variables for subexpressions is in Fig.~\ref{unnesting_example_b}.
The relational program before the conjuncts are reordered is shown in Fig.~\ref{unnesting_example_c} and the result of the unnesting is presented in Fig.~\ref{unnesting_example_d}.

Note, that the unnesting has limitations: it does not support higher-order functions and partial application. 
A more general method of translation which does not impose the same limitations was developed~\cite{lozov:conversion}. 
Unfortunately, it uses higher-order relations which are not currently supported in conjunctive partial deduction, so we use unnesting. 

The forward execution of the relation mimics the execution of the function from which it was constructed by relational conversion.
This makes forward execution quite efficient, to the detriment of the execution in the backwards direction. 
The unnesting can be modified to improve the performance of  backward execution. 
Let us consider the conversion of a functional conjunction ``\lstinline{f$_1$ x$_1$ && f$_2$ x$_2$}''.

\begin{lstlisting}
fun res ->
  fresh (p) (
    (f$_1$ x$_1$ p) /\
    (conde [
      (p === ^false /\ res === ^false);
      (p === ^true  /\ f$_2$ x$_2$ res)]))
\end{lstlisting}

Mimicking the function evaluation, the forward execution of this code first computes ``\lstinline{f$_1$ x$_1$}''. 
If it fails, then the result ``\lstinline{res}'' is unified with ``\lstinline{false}'', otherwise the second conjunct ``\lstinline{f$_2$ x$_2$}'' is executed and its result is unified with the result. 
This strategy is not efficient in the backward direction, when we know what ``\lstinline{res}'' is. 
The~following relation is much more performant when executed in the backward direction:

\begin{lstlisting}
fun res ->
    conde [
      (res === ^false /\ f$_1$ x$_1$ ^false);
      (f$_1$ x$_1$ ^true    /\ f$_2$ x$_2$ res)]
\end{lstlisting}

In particular, if ``\lstinline{res === ^true}'', both conversions execute ``\lstinline{f$_2$ x$_2$ res}'', but when the first conversion computes ``\lstinline{f$_1$ x$_1$ p}'' with fresh ``\lstinline{p}'', the second executes ``\lstinline{f$_1$ x$_1$ ^true}''. 
Using the second conversion is enough to significantly increase the performance in the backward direction. 
For example, the path search takes several minutes if the first conversion strategy is used, whereas it finishes in less than a second in the second case. 

Choosing the second conversion strategy comes with a price for the forward execution. 
Instead of executing ``\lstinline{f$_1$ x$_1$ p}'', where ``\lstinline{p}'' is fresh, the second strategy executes both ``\lstinline{f$_1$ x$_1$ ^false}'' and ``\lstinline{f$_1$ x$_1$ ^true}''.
In the worst case scenario, when the execution of ``\lstinline{f$_1$}'' does not depend on the last argument, it doubles the number of executions of ``\lstinline{f$_1$}''.

To sum up, by choosing different strategies of the relational conversion we can achieve significant performance improvement. 
There is no single right way of doing the conversion which improves the performance of the execution in every possible direction. 
Choosing a strategy per each relation and each direction manually is not feasible, but it can be achieved with a fully-automatic program transformation, such as conjunctive partial deduction.
