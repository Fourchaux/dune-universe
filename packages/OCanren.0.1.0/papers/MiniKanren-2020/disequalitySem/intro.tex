\section{Introduction}

In its initial form~\cite{TRS,MicroKanren} \textsc{miniKanren} introduces a single
form of constraint~--- unification of terms. While from a theoretical standpoint unification together with other primitive constructs (conjunction, disjunction, and
fresh variable introduction) form a Turing-complete basis, in practice of relational programming a number of extensions are often used to make specifications more
expressive, concise or efficient. One of the most important extensions is \emph{disequality constraint}.

A generic concept of domain-specific constraints in logic programming is studied in details in~\cite{CLP}; more specifically, disequality constraint~\cite{Disunification}
introduces one additional type of base goal~--- a disequality of two terms

\[
t_1 \diseq t_2
\]

The informal semantics of disequality constraint is complementary to that of unification: it puts certain restrictions on free variables in the terms which
prevent them from turning into syntactically equal. Similarly to unification, whose evaluation results in a substitution, which is then threaded through
the rest of computations, the effect of disequality constraint is recorded in a \emph{constraint store} which is later used to check the violation of
disequalities~\cite{CKanren}.

We present an extension of our prior work on certified semantics for core \textsc{miniKanren}~\cite{CertifiedSemantics}. In that work, we
defined denotational and operational semantics and proved the soundness and completeness of the latter w.r.t. the former. 
The main advantage of the operational semantics introduced there over the ones developed before~\cite{MechanisingMiniKanren, RelConversion, DivTest}
was its ability to capture the conventional for \textsc{miniKanren} \emph{interleaving search}~\cite{Search} procedure.
This allowed us to give the first to our knowledge formal proof of completeness of the interleaving search
as the capability to reach all the solutions from denotational semantics
(proof of completeness as the fairness of steams interleaving for a specific implementation was given in~\cite{SmallEmbedding}).
The development was formally certified in \textsc{Coq} proof assistant~\cite{Coq}, and a correct-by-construction interpreter was extracted.

To some extent our work follows the conventional roadmap of adding constraints to a pure logic/relational language~\cite{CLP}; the difference is that, first,
we use a specific constraint and a concrete solver, and second, we prove all the results with regard to conventional for \textsc{miniKanren} interleaving
search (versus a very generic and abstract breadth-first search in~\cite{CLP}).

The contribution of our current work is as follows:

\begin{itemize}
\item we extend our denotational semantics to handle disequality constraints;
\item we introduce a new abstraction layer (a constraint store with a number of abstract operations) in our operational semantics;
\item we formulate a set of \emph{sufficient conditions for completeness}, expressed as algebraic properties of constraint store and
  abstract operators, and prove the soundness and completeness of the extended operational semantics w.r.t. the denotational one;
\item we present two concrete implementations of constraint store and abstract operators and show that they satisfy the
  sufficient conditions; thus, the soundness and completeness of the implementation with disequality constraints follow
  immediately, and correct-by-construction interpreter for \textsc{miniKanren} with disequality constraints
  can be extracted
\item we demonstrate how our framework can be used to prove some properties of implementations of disequality constraints.
\end{itemize}

The paper is organized as follows. In Section~\ref{sec:review} we recall our framework from the previous paper~\cite{CertifiedSemantics}, which is extended in this work. The following sections describe the new results. Section~\ref{sec:extension} contains the description of the extensions in semantics and sufficient conditions on abstract definitions for search completeness. Section~\ref{sec:implementations} contains two examples of implementations of constraint stores that satisfy the sufficient conditions for completeness. Section~\ref{sec:applications} presents some applications of the extended semantics. The final section concludes.
