\documentclass[acmlarge]{acmart}

%% Bibliography style
\bibliographystyle{ACM-Reference-Format}
%% Citation style
%% Note: author/year citations are required for papers published as an
%% issue of PACMPL.
\citestyle{acmauthoryear}   %% For author/year citations


%%%%%%%%%%%%%%%%%%%%%%%%%%%%%%%%%%%%%%%%%%%%%%%%%%%%%%%%%%%%%%%%%%%%%%
%% Note: Authors migrating a paper from PACMPL format to traditional
%% SIGPLAN proceedings format must update the '\documentclass' and
%% topmatter commands above; see 'acmart-sigplanproc-template.tex'.
%%%%%%%%%%%%%%%%%%%%%%%%%%%%%%%%%%%%%%%%%%%%%%%%%%%%%%%%%%%%%%%%%%%%%%


%% Some recommended packages.
\usepackage{booktabs}   %% For formal tables:
                        %% http://ctan.org/pkg/booktabs
\usepackage{subcaption} %% For complex figures with subfigures/subcaptions
                        %% http://ctan.org/pkg/subcaption


\usepackage{amsmath,amssymb}
\usepackage[russian,english]{babel}
\usepackage{amssymb}
\usepackage{mathtools}
\usepackage{listings}
\usepackage{comment}
\usepackage{indentfirst}
\usepackage{hyperref}
\usepackage{amsthm}
\usepackage{stmaryrd}
\usepackage{eufrak}
\usepackage{lstcoq}

\newtheorem{theorem}{Theorem}
\newtheorem{lemma}{Lemma}
\newtheorem{corollary}{Corollary}
\newtheorem{hyp}{Hypethesis}
\newtheorem{definition}{Definition}

\lstdefinelanguage{minikanren}{
keywords={fresh},
sensitive=true,
commentstyle=\small\itshape\ttfamily,
keywordstyle=\textbf,
identifierstyle=\ttfamily,
basewidth={0.5em,0.5em},
columns=fixed,
fontadjust=true,
literate={fun}{{$\lambda\;\;$}}1 {->}{{$\to$}}3 {===}{{$\,\equiv\,$}}1 {=/=}{{$\not\equiv$}}1 {|>}{{$\triangleright$}}3 {/\\}{{$\wedge$}}2 {\\/}{{$\vee$}}2,
morecomment=[s]{(*}{*)}
}

\lstset{
mathescape=true,
language=minikanren
}

\usepackage{letltxmacro}
\newcommand*{\SavedLstInline}{}
\LetLtxMacro\SavedLstInline\lstinline
\DeclareRobustCommand*{\lstinline}{%
  \ifmmode
    \let\SavedBGroup\bgroup
    \def\bgroup{%
      \let\bgroup\SavedBGroup
      \hbox\bgroup
    }%
  \fi
  \SavedLstInline
}

\def\transarrow{\xrightarrow}
\newcommand{\setarrow}[1]{\def\transarrow{#1}}

\def\padding{\phantom{X}}
\newcommand{\setpadding}[1]{\def\padding{#1}}

\def\subarrow{}
\newcommand{\setsubarrow}[1]{\def\subarrow{#1}}

\newcommand{\trule}[2]{\frac{#1}{#2}}
\newcommand{\crule}[3]{\frac{#1}{#2},\;{#3}}
\newcommand{\withenv}[2]{{#1}\vdash{#2}}
\newcommand{\trans}[3]{{#1}\transarrow{\padding{\textstyle #2}\padding}\subarrow{#3}}
\newcommand{\ctrans}[4]{{#1}\transarrow{\padding#2\padding}\subarrow{#3},\;{#4}}
\newcommand{\llang}[1]{\mbox{\lstinline[mathescape]|#1|}}
\newcommand{\pair}[2]{\inbr{{#1}\mid{#2}}}
\newcommand{\inbr}[1]{\left<{#1}\right>}
\newcommand{\highlight}[1]{\color{red}{#1}}
%\newcommand{\ruleno}[1]{\eqno[\scriptsize\textsc{#1}]}
\newcommand{\ruleno}[1]{\mbox{[\textsc{#1}]}}
\newcommand{\rulename}[1]{\textsc{#1}}
\newcommand{\inmath}[1]{\mbox{$#1$}}
\newcommand{\lfp}[1]{fix_{#1}}
\newcommand{\gfp}[1]{Fix_{#1}}
\newcommand{\vsep}{\vspace{-2mm}}
\newcommand{\supp}[1]{\scriptsize{#1}}
\newcommand{\sembr}[1]{\llbracket{#1}\rrbracket}
\newcommand{\cd}[1]{\texttt{#1}}
\newcommand{\free}[1]{\boxed{#1}}
\newcommand{\binds}{\;\mapsto\;}
\newcommand{\dbi}[1]{\mbox{\bf{#1}}}
\newcommand{\sv}[1]{\mbox{\textbf{#1}}}
\newcommand{\bnd}[2]{{#1}\mkern-9mu\binds\mkern-9mu{#2}}
\newcommand{\meta}[1]{{\mathcal{#1}}}
\newcommand{\dom}[1]{\mathtt{dom}\;{#1}}
\newcommand{\primi}[2]{\mathbf{#1}\;{#2}}
\renewcommand{\dom}[1]{\mathcal{D}om\,({#1})}
\newcommand{\ran}[1]{\mathcal{VR}an\,({#1})}
\newcommand{\fv}[1]{\mathcal{FV}\,({#1})}
\newcommand{\tr}[1]{\mathcal{T}r_{#1}}
\newcommand{\diseq}{\not\equiv}
\newcommand{\reprfunset}{\mathcal{R}}
\newcommand{\reprfun}{\mathfrak{f}}
\newcommand{\cstore}{\Omega}
\newcommand{\cstoreinit}{\cstore_\epsilon^{\mathit{init}}}
\newcommand{\csadd}[3]{\mathbf{add}\,(#1, #2 \diseq #3)}  %{#1 + [#2 \diseq #3]}
\newcommand{\csupdate}[2]{\mathbf{update}\,(#1, #2)}  %{#1 \cdot #2}
\newcommand{\cupdate}[2]{\mathbf{update_{constr}}\,(#1, #2)}  %{#1 \cdot #2}
\newcommand{\eqrestr}{=_n}

\newcommand{\searchRule}[6] {
  #1, #2 \vdash (#3, #4) \xRightarrow{#5} #6}
\newcommand{\extSearchRule}[8] {
  #1, #2, #3, #4 \vdash (#5, #6) \xRightarrow{#7}_{e} #8}
\newcommand{\q}{\hspace{0.5em}}
\newcommand{\bigcdot}{\boldsymbol{\cdot}}
\newcommand{\bigslant}[2]{{\raisebox{.2em}{$#1$}\left/\raisebox{-.2em}{$#2$}\right.}}

\let\emptyset\varnothing
\let\eps\varepsilon

\sloppy

\begin{document}

%% Title information
\title{Certified Semantics for Disequality} %% [Short Title] is optional;
                                           %% when present, will be used in
                                           %% header instead of Full Title.
\titlenote{The reported study was funded by RFBR, project number 18-01-00380} %% \titlenote is optional;
                                        %% can be repeated if necessary;
                                        %% contents suppressed with 'anonymous'
%\subtitle{Subtitle}                     %% \subtitle is optional
%\subtitlenote{with subtitle note}       %% \subtitlenote is optional;
                                        %% can be repeated if necessary;
                                        %% contents suppressed with 'anonymous'


%% Author information
%% Contents and number of authors suppressed with 'anonymous'.
%% Each author should be introduced by \author, followed by
%% \authornote (optional), \orcid (optional), \affiliation, and
%% \email.
%% An author may have multiple affiliations and/or emails; repeat the
%% appropriate command.
%% Many elements are not rendered, but should be provided for metadata
%% extraction tools.

\author{Dmitry Rozplokhas}
\affiliation{%
  \institution{Higher School of Economics}}
\affiliation{%
  \institution{JetBrains Research}
  \country{Russia}}
\email{darozplokhas@edu.hse.ru}

\author{Dmitry Boulytchev}
\affiliation{%
  \institution{Saint Petersburg State University}}
\affiliation{%
  \institution{JetBrains Research}
  \country{Russia}}
\email{dboulytchev@math.spbu.ru}



%% Abstract
%% Note: \begin{abstract}...\end{abstract} environment must come
%% before \maketitle command
\begin{abstract}
We present an extension of our prior work on certified semantics for core \textsc{miniKanren}, introducing disequality constraints in the language.
Semantics is parameterized by an exact definition of constraint stores, allowing us to cover different implementations, and we provide a list of sufficient conditions on this definition for search completeness.
We also give two examples of concrete implementations of constraint stores that satisfy those sufficient conditions.
The description and proofs for parameterized semantics and both implementations are certified in Coq and two correct-by-construction interpreters are extracted.
\end{abstract}


%% 2012 ACM Computing Classification System (CSS) concepts
%% Generate at 'http://dl.acm.org/ccs/ccs.cfm'.
\begin{CCSXML}
<ccs2012>
<concept>
<concept_id>10003752.10003790.10003795</concept_id>
<concept_desc>Theory of computation~Constraint and logic programming</concept_desc>
<concept_significance>500</concept_significance>
</concept>
<concept>
<concept_id>10003752.10010124.10010131.10010133</concept_id>
<concept_desc>Theory of computation~Denotational semantics</concept_desc>
<concept_significance>500</concept_significance>
</concept>
<concept>
<concept_id>10003752.10010124.10010131.10010134</concept_id>
<concept_desc>Theory of computation~Operational semantics</concept_desc>
<concept_significance>500</concept_significance>
</concept>
</ccs2012>
\end{CCSXML}
%% \ccsdesc[500]{Theory of computation~Constraint and logic programming}
%% \ccsdesc[500]{Theory of computation~Denotational semantics}
%% \ccsdesc[500]{Theory of computation~Operational semantics}
%% End of generated code


%% Keywords
%% comma separated list
%% \keywords{Relational programming, denotational semantics, operational semantics, certified programming}  %% \keywords are mandatory in final camera-ready submission


%% \maketitle
%% Note: \maketitle command must come after title commands, author
%% commands, abstract environment, Computing Classification System
%% environment and commands, and keywords command.
\maketitle
\thispagestyle{empty}

\section{Introduction}

The introductory book on \textsc{miniKanren}~\cite{TRS} describes the language by means of an evolving set of examples. In the
series of follow-up papers~\cite{MicroKanren,CKanren,CKanren1,AlphaKanren,2016,Guided} various extensions of the language were presented with
their semantics explained in terms of \textsc{Scheme} implementation. We argue that this style of semantic definition is
fragile and not self-evident since it requires the knowledge of semantics of concrete implementation language. In addition the justification of
important properties of relational programs (for example, refutational completeness~\cite{WillThesis}) becomes cumbersome. In the
area of programming languages research a formal definition for the semantics of language of interest is a \emph{de-facto} standard, and
in our opinion in its current state \textsc{miniKanren} deviates from this standard.

There were some previous attempts to define a formal semantics for \textsc{miniKanren}. \citet{RelConversion} present a variant of nondeterministic
operational semantics, and~\citet{DivTest} use another variant of finite-set semantics. None of them was capable of reflecting
the distinctive property of \textsc{miniKanren} search~--- \emph{interleaving}~\cite{Search}, thus deviating from the conventional understanding
of the language.

In this paper we present a formal semantics for core \textsc{miniKanren} and prove some its basic properties. First,
we define denotational semantics similar to the least Herbrand model for definite logic programs~\cite{LHM}; then
we describe operational semantics with interleaving in terms of labeled transition system. Finally, we prove the soundness and
completeness of the operational semantics w.r.t the denotational one. We support our development with a formal specification
using \textsc{Coq}~\cite{Coq} proof assistant\footnote{\url{https://github.com/dboulytchev/miniKanren-coq}}, thus outsourcing
the burden of proof checking to the automatic tool. 

The paper organized as follows. In Section~\ref{language} we give the syntax of the language, describe its semantics
informally and discuss some examples. Section~\ref{denotational} contains the description of denotational semantics for
the language, and Section~\ref{operational}~--- the operational semantics. In Section~\ref{equivalence} we overview the
certified proof for soundness and completeness of operational semantics. The final section concludes.

\section{The syntax and semantics of the core language}
\label{sec:review}

In this section, we recall existing definitions of the syntax and the two semantics for the core language without disequality constraints and the main result~--- the equivalence
of these two semantics~\cite{CertifiedSemantics}.

\subsection{The Syntax of Core Language}
\label{subsec_syntax}

\begin{figure}[t]
\centering
\[
\begin{array}{ccll}
  \mathcal{C} & = & \{C_i^{k_i}\} & \mbox{constructors with arities} \\
  \mathcal{T}_X & = & X \cup \{C_i^{k_i} (t_1, \dots, t_{k_i}) \mid t_j\in\mathcal{T}_X\} & \mbox{terms over the set of variables $X$} \\
  \mathcal{D} & = & \mathcal{T}_\emptyset & \mbox{ground terms}\\
  \mathcal{X} & = & \{ x, y, z, \dots \} & \mbox{syntactic variables} \\
  \mathcal{A} & = & \{ \alpha, \beta, \gamma, \dots \} & \mbox{semantic variables} \\
  \mathcal{R} & = & \{ R_i^{k_i}\} &\mbox{relational symbols with arities} \\[2mm]
  \mathcal{G} & = & \mathcal{T_X}\equiv\mathcal{T_X}   &  \mbox{equality} \\
              &   & \mathcal{G}\wedge\mathcal{G}     & \mbox{conjunction} \\
              &   & \mathcal{G}\vee\mathcal{G}       &\mbox{disjunction} \\
              &   & \mbox{\lstinline|fresh|}\;\mathcal{X}\;.\;\mathcal{G} & \mbox{fresh variable introduction} \\
              &   & R_i^{k_i} (t_1,\dots,t_{k_i}),\;t_j\in\mathcal{T_X} & \mbox{relational symbol invocation} \\[2mm]
  \mathcal{S} & = & \{R_i^{k_i} = \lambda\;x_1^i\dots x_{k_i}^i\,.\, g_i;\}\; g & \mbox{specification}
\end{array}
\]
\caption{The syntax of core language}
\label{syntax}
\end{figure}

The syntax of the language is shown in Fig.~\ref{syntax}. First, we fix a set of constructors $\mathcal{C}$ with known arities and consider
a set of terms $\mathcal{T}_X$ with constructors as functional symbols and variables from $X$. We parameterize this set with an alphabet of
variables since in the semantic description we will need \emph{two} kinds of variables. The first kind, \emph{syntactic} variables, is denoted
by $\mathcal{X}$. We also consider an alphabet of \emph{relational symbols} $\mathcal{R}$ which are used to name relational definitions.
The central syntactic category in the language is a \emph{goal}. In our case, there are five types of goals: \emph{equality} of terms,
conjunction and disjunction of goals, fresh variable introduction, and invocation of some relational definition. Thus, equality is used
as a constraint, and multiple constraints can be combined using conjunction, disjunction, and recursion. For the sake of brevity we
abbreviate immediately nested ``\lstinline|fresh|'' constructs into the one, writing ``\lstinline|fresh $x$ $y$ $\dots$ . $g$|'' instead of
``\lstinline|fresh $x$ . fresh $y$ . $\dots$ $g$|''. The final syntactic category is \emph{specification} $\mathcal{S}$. It consists of a set
of relational definitions and a top-level goal. A top-level goal represents a search procedure which returns a stream of substitutions for
the free variables of the goal. The language we defined is first-order, as goals can not be passed as parameters,
returned or constructed at runtime.

As an example consider the specification for the standard \lstinline|append$^o$| relation and a query which splits a list containing
three constants \lstinline|A|, \lstinline|B| and \lstinline|C| into two parts in every possible way:

\begin{lstlisting}
  append$^o$ = fun x y xy .
    ((x === Nil) /\ (xy === y)) \/
    (fresh h t ty .
       (x  === Cons (h, t))  /\
       (xy === Cons (h, ty)) /\
       (append$^o$ t y ty)
    );
  append$^o$ x y (Cons (A, Cons (B, Cons (C, Nil))))
\end{lstlisting}

\subsection{Denotational sematics}

For denotational semantics, we use a simple set-theoretic approach which can be considered analogous to the least Herbrand model for definite logic programs~\cite{LHM}.

Intuitively, the mathematical model for every goal should be a relation between semantic variables that occur free in this goal. We represent this relation as a set of total
functions 

\[
\mathfrak{f}:\mathcal{A}\mapsto\mathcal{D}
\]

from semantic variables to ground terms. We call these functions \emph{representing functions}.

Then, the semantic function for goals is parameterized over environments which prescribe semantic functions to relational symbols:

\[
  \Gamma : \mathcal{R} \to (\mathcal{T_A}^*\to 2^{\mathcal{A}\to\mathcal{D}})
\]

An environment associates with relational symbol a function that takes a string of terms (the arguments of the relation) and returns a set of
representing functions. The signature for semantic brackets for goals is as follows:

\[
\sembr{\bullet}_{\Gamma} : \mathcal{G}\to 2^{\mathcal{A}\to\mathcal{D}}
\]

It maps a goal into the set of representing functions w.r.t. an environment $\Gamma$.

We formulate the following important \emph{completeness condition} for the semantics of a goal $g$: for any goal $g$ and two representing functions ${\mathfrak f}$ and ${\mathfrak f'}$, such that $\left.{\mathfrak f}\right|_{FV(g)} = \left.{\mathfrak f'}\right|_{FV(g)}$
\[ {\mathfrak f} \in \sembr{g} \Leftrightarrow {\mathfrak f'} \in \sembr{g} \]

In other words, representing functions for a goal $g$ restrict only the values of free variables of $g$ and do not introduce any ``hidden'' correlations.
This condition guarantees that our semantics is complete in the sense that it does not introduce artificial restrictions for the relation it defines.
We proved that the semantics of goals always satisfy this condition.

To define the semantic function we need a few operations for representing functions:

\begin{itemize}
\item A homomorphic extension of a representing function 

\[
  \overline{\mathfrak{f}}:\mathcal{T_A}\to\mathcal{D}
\]

which maps terms to terms:

\[
\begin{array}{rcl}

  \overline{\mathfrak f}\,(\alpha) & = & \mathfrak f\,(\alpha)\\
  \overline{\mathfrak f}\,(C_i^{k_i}\,(t_1,\dots.t_{k_i})) & = & C_i^{k_i}\,(\overline{\mathfrak f}\,(t_1),\dots \overline{\mathfrak f}\,(t_{k_i}))
\end{array}
\]

\item A pointwise modification of a function

\[
f\,[x\gets v]\,(z)=\left\{
\begin{array}{rcl}
  f\,(z) &,& z \ne x \\
  v      &,& z = x
\end{array}
\right.
\]

\item A \emph{generalization} operation:

\[
\mathfrak{f}\uparrow\alpha = \{ \mathfrak{f}\,[\alpha\gets d] \mid d\in\mathcal D\}
\]

Informally, this operation generalizes a representing function into a set of representing functions in such a way that the
values of these functions for a given variable cover the whole $\mathcal{D}$. We extend the generalization operation for sets of
representing functions $\mathfrak{F}\subseteq\mathcal{A}\to\mathcal{D}$:

\[
  \mathfrak{F}\uparrow\alpha = \bigcup_{\mathfrak{f}\in\mathfrak{F}}(\mathfrak{f}\uparrow\alpha)
\]

\end{itemize}

The semantics for goals is shown on Fig.~\ref{denotational_semantics_of_goals}.

\begin{figure}[t]
  \[
  \begin{array}{cclr}
    \sembr{t_1\equiv t_2}_\Gamma&=&\{\mathfrak f : \mathcal{A}\to\mathcal{D}\mid \overline{\mathfrak{f}}\,(t_1)=\overline{\mathfrak{f}}\,(t_2)\}& \ruleno{Unify$_D$}\\
    \sembr{g_1\wedge g_2}_\Gamma&=&\sembr{g_1}_\Gamma\cap\sembr{g_1}_\Gamma&\ruleno{Conj$_D$}\\
    \sembr{g_1\vee g_2}_\Gamma&=&\sembr{g_1}_\Gamma\cup\sembr{g_1}_\Gamma&\ruleno{Disj$_D$}\\
    \sembr{\mbox{\lstinline|fresh|}\,x\,.\,g}_\Gamma&=&(\sembr{g\,[\alpha/x]}_\Gamma)\uparrow\alpha,\;\alpha\not\in FV(g)& \ruleno{Fresh$_D$}\\
    \sembr{R\,(t_1,\dots,t_k)}_\Gamma&=&(\Gamma\,R)\,t_1\dots t_k & \ruleno{Invoke$_D$}
  \end{array}
  \]
  \caption{Denotational semantics of goals}
  \label{denotational_semantics_of_goals}
\end{figure}

The final component is the semantics of specifications. Given a specification

\[
\{R_i=\lambda\,x_1^i\dots x_{k_i}^i\,.\,g_i;\}_{i=1}^n\;g
\]

we construct a correct environment $\Gamma_0$ and then take the semantics of the top-level goal:

\[
\sembr{\{R_i=\lambda\,x_1^i\dots x_{k_i}^i\,.\,g_i;\}_{i=1}^n\;g}=\sembr{g}_{\Gamma_0}
\]

As the set of definitions can be mutually recursive we apply the fixed point approach and define $\Gamma_0$ as the least
fixed point of a specific function $F$ that takes an environment $\Gamma$ and returns new environment in which semantics
of a body of each definition is evaluated with environment $\Gamma$.


\subsection{Operational sematics}

The operational semantics of \textsc{miniKanren}, which we described, corresponds to the known
implementations with interleaving search. The semantics is given in the form of a labeled transition system (LTS)~\cite{LTS}.

The states in the transition system have the following shape:

\[
S = \mathcal{G}\times\Sigma\times\mathbb{N}\mid S\oplus S \mid S \otimes \mathcal{G}
\]

A state has a tree-like structure with intermediate nodes corresponding to partially-evaluated conjunctions (``$\otimes$'') or
disjunctions (``$\oplus$''). A leaf in the form $\inbr{g, \sigma, n}$ determines a goal in a context, where $g$~--- a goal, $\sigma$~--- a substitution accumulated so far,
and $n$~--- a natural number, which corresponds to a number of semantic variables used to this point. For a conjunction node, its right child is always a goal since
it cannot be evaluated unless some result is provided by the left conjunct.

We also need extended states

\[
\overline{S} = \diamond \mid S
\]

where $\diamond$ symbolizes the end of the evaluation.

The set of labels is defined as follows:

\[
L = \circ \mid \Sigma\times \mathbb{N}
\]

The label ``$\circ$'' is used to mark those steps which do not provide an answer; otherwise, a transition is labeled by a pair of a substitution and a number of allocated
variables. The substitution is one of the answers, and the number is threaded through the derivation to keep track of the allocated variables.

\begin{figure*}
  \renewcommand{\arraystretch}{1.6}
  \[
  \begin{array}{cr}
    \inbr{t_1 \equiv t_2, \sigma, n} \xrightarrow{\circ} \Diamond , \, \, \nexists\; mgu\,(t_1 \sigma, t_2 \sigma) &\ruleno{UnifyFail} \\
    \inbr{t_1 \equiv t_2, \sigma, n} \xrightarrow{(mgu\,(t_1 \sigma, t_2 \sigma) \circ \sigma),\, n)} \Diamond & \ruleno{UnifySuccess} \\
    \inbr{g_1 \lor g_2, \sigma, n} \xrightarrow{\circ} \inbr{g_1, \sigma, n} \oplus \inbr{g_2, \sigma, n} & \ruleno{Disj} \\
    \inbr{g_1 \land g_2, \sigma, n} \xrightarrow{\circ} \inbr{ g_1, \sigma, n} \otimes g_2 & \ruleno{Conj} \\
    \inbr{\mbox{\lstinline|fresh|}\, x\, .\, g, \sigma, n} \xrightarrow{\circ} \inbr{g\,[\bigslant{\alpha_{n + 1}}{x}], \sigma, n + 1} & \ruleno{Fresh} \\
    \dfrac{R_i^{k_i}=\lambda\,x_1\dots x_{k_i}\,.\,g}{\inbr{R_i^{k_i}\,(t_1,\dots,t_{k_i}),\sigma,n} \xrightarrow{\circ} \inbr{g\,[\bigslant{t_1}{x_1}\dots\bigslant{t_{k_i}}{x_{k_i}}], \sigma, n}} & \ruleno{Invoke}\\
    \dfrac{s_1 \xrightarrow{\circ} \Diamond}{(s_1 \oplus s_2) \xrightarrow{\circ} s_2} & \ruleno{DisjStop}\\
    \dfrac{s_1 \xrightarrow{r} \Diamond}{(s_1 \oplus s_2) \xrightarrow{r} s_2} & \ruleno{DisjStopAns}\\
    \dfrac{s \xrightarrow{\circ} \Diamond}{(s \otimes g) \xrightarrow{\circ} \Diamond} &\ruleno{ConjStop}\\
    \dfrac{s \xrightarrow{(\sigma, n)} \Diamond}{(s \otimes g) \xrightarrow{\circ} \inbr{g, \sigma, n}}  & \ruleno{ConjStopAns}\\
    \dfrac{s_1 \xrightarrow{\circ} s'_1}{(s_1 \oplus s_2) \xrightarrow{\circ} (s_2 \oplus s'_1)} &\ruleno{DisjStep}\\
    \dfrac{s_1 \xrightarrow{r} s'_1}{(s_1 \oplus s_2) \xrightarrow{r} (s_2 \oplus s'_1)} &\ruleno{DisjStepAns}\\
    \dfrac{s \xrightarrow{\circ} s'}{(s \otimes g) \xrightarrow{\circ} (s' \otimes g)} &\ruleno{ConjStep}\\
    \dfrac{s \xrightarrow{(\sigma, n)} s'}{(s \otimes g) \xrightarrow{\circ} (\inbr{g, \sigma, n} \oplus (s' \otimes g))} & \ruleno{ConjStepAns} 
  \end{array}
  \]
  \caption{Operational semantics of interleaving search}
  \label{lts}
\end{figure*}

The transition rules are shown in Fig.~\ref{lts}. The introduced transition system is completely deterministic.

A derivation sequence for a certain state $s$ determines a \emph{trace} $\tr{s}$~--- a finite or infinite sequence of answers. The trace corresponds to the stream of answers
in the reference \textsc{miniKanren} implementations.

\subsection{Semantics Equivalence}

After we defined two different kinds of semantics for \textsc{miniKanren} we related them and showed that the results given by these two semantics are the same for any specification.
By proving this equivalence we established the \emph{completeness} of the search which means that the search will get all answers satisfying the described specification and only those.

To do it we had to relate the answers produced by these two semantics as they have different forms: a trace of substitutions (along with numbers of allocated variables)
for operational and a set of representing functions for denotational. There is a natural way to extend any substitution to a representing function: composing it with an arbitrary representing function will preserve all variable dependencies in the substitution. So we defined a set of representing functions corresponding to substitution as follows:

\[
\sembr{\sigma} = \{\overline{\mathfrak f} \circ \sigma \mid \mathfrak{f}:\mathcal{A}\mapsto\mathcal{D}\}
\]

And \emph{denotational analog} of an operational semantics (a set of representing functions corresponding to answers in the trace) for given extended state $s$ is
then defined as a union of sets for all substitution in the trace:

\[
\sembr{s}_{op} = \cup_{(\sigma, n) \in \tr{s}} \sembr{\sigma}
\]

This allowed us to state the theorem relating two semantics.

\begin{theorem}[Operational semantics soundness and completeness]
For any specification $\{\dots\}\; g$, for which the indices of all free variables in $g$ are limited by some number $n$

\[
\sembr{\inbr{g, \epsilon, n}}_{op} \eqrestr \sembr{\{\dots\}\; g}.
\]
\end{theorem}

Where `$\eqrestr$' means that we compare representing functions of these sets only on the semantic variables from $\{\alpha_1, \dots, \alpha_n\}$:

\[
S_1 \eqrestr S_2 \xLeftrightarrow{def}  \{\mathfrak{f}|_{\{\alpha_1,\dots,\alpha_n\}} \mid \mathfrak{f} \in S_1 \} = \{\mathfrak{f}|_{\{\alpha_1,\dots,\alpha_n\}} \mid \mathfrak{f} \in S_2 \}.
\]

We can not use the usual equality of sets instead of this one, the sets from the theorem statement are actually not equal.
The reason for this is that denotational semantics encodes only dependencies between the free variables of a goal, which is reflected by the completeness condition, while
operational semantics may also contain dependencies between semantic variables allocated in ``\lstinline|fresh|''.
Therefore we have to restrict representing functions on the semantic variables allocated in the beginning (which includes all free variables of a goal). This does not
compromise our promise to prove the completeness of the search as \textsc{miniKanren} provides the result as substitutions only for queried variables,
which are allocated in the beginning.

The proof of this main theorem was certified in \textsc{Coq}.
\section{Extension with disequality constraints}
\label{sec:extension}

In this section, we present extensions of our two semantics for the language with disequality constraints and revised versions of the soundness and completeness theorems.

Disequality constraint introduces one additional type of base goal~--- a disequality of two terms: $t_1 \diseq t_2$

The extension of denotational semantics is straightforward (as disequality constraint is complementary to equality):

\[ \sembr{t_1 \diseq t_2}  =  \{\reprfun \in \reprfunset \mid \overline{\reprfun}\,(t_1) \neq \overline{\reprfun}\,(t_2)\}, \]

This definition for a new type of goals fits nicely into the general inductive definition of denotational semantics of an arbitrary goal
and preserves its properties, such as completeness condition.

In the operational case we deviate from describing one specific search implementation since there are several distinct ways to embed disequality constraints
in the language and we would like to be able to give semantics (and subsequently prove correctness) for all of them. Therefore we base the extended operational
semantics on a number of abstract definitions concerning constraint stores for which different concrete implementations may be substituted.

We assume that we are given a set of constraint store objects, which we denote by $\cstore_\sigma$ (indexing every constraint store with
some substitution $\sigma$ and assuming the store and the substitution are consistent with each other), and three following operations:

\begin{enumerate}
\item Initial constraint store $\cstoreinit$ (where $\epsilon$ is empty substitution), which does not contain any constraints yet.
\item Adding a disequality constraint to a store $\csadd{\cstore_\sigma}{t_1}{t_2}$, which may result in a new constraint store $\cstore^\prime_\sigma$ or a failure $\bot$,
  if the new constraint store is inconsistent with the substitution $\sigma$.
\item Updating a substitution in a constraint store $\csupdate{\cstore_\sigma}{\delta}$ to intergate a new substitution $\delta$ into the current one,
  which may result in a new constraint store $\cstore^\prime_{\sigma \delta}$ or a failure $\bot$, if the constraint store is inconsistent with the new substitution.
\end{enumerate}

The change in operational semantics for the language with disequality constraints is now straightforward: we add a constraint store to a basic (leaf) state $\inbr{g, \sigma, \cstore_\sigma, n}$,
as well as in the label form $(\sigma, \cstore_\sigma, n)$, and this store is simply threaded through all the rules, except those for equality. We change the rules
for equality using $\mathbf{update}$ operation and add the rules for disequality constraint using $\mathbf{add}$. In both cases, the search in the current branch is
pruned if these primitives return $\bot$.

 \[
  \begin{array}{cr}
    \inbr{t_1 \equiv t_2, \sigma, \cstore_\sigma, n} \xrightarrow{\circ} \Diamond , \, \, \nexists\; mgu\,(t_1, t_2, \sigma) &\ruleno{UnifyFailMGU} \\[2mm]
    \inbr{t_1 \equiv t_2, \sigma, \cstore_\sigma, n} \xrightarrow{\circ} \Diamond , \, \, mgu\,(t_1, t_2, \sigma) = \delta, \, \, \csupdate{\cstore_\sigma}{\delta} = \bot &\ruleno{UnifyFailUpdate} \\[2mm]
    \inbr{t_1 \equiv t_2, \sigma, \cstore_\sigma, n} \xrightarrow{(\sigma \delta, \, \cstore'_{\sigma\delta}, \, n)} \Diamond , \, \, mgu\,(t_1, t_2, \sigma) = \delta, \, \, \csupdate{\cstore_\sigma}{\delta} = \cstore'_{\sigma\delta} & \ruleno{UnifySuccess} \\[2mm]
    \inbr{t_1 \diseq t_2, \sigma, \cstore_\sigma, n} \xrightarrow{\circ} \Diamond , \, \, \csadd{\cstore_\sigma}{t_1}{t_2} = \bot &\ruleno{DiseqFail} \\[2mm]
    \inbr{t_1 \diseq t_2, \sigma, \cstore_\sigma, n} \xrightarrow{(\sigma, \, \cstore'_{\sigma}, \, n)} \Diamond , \, \, \csadd{\cstore_\sigma}{t_1}{t_2} = \cstore'_\sigma &\ruleno{DiseqSucess} \\[2mm]
  \end{array}
\]

The initial state naturally contains an initial constraint store $\inbr{g, \eps, \cstoreinit, n}$.

To state the soundness and completeness result now we need to revise our definition of the denotational analog of an answer $(\sigma, \cstore_\sigma, n)$
since we have to take into account the restrictions which a constraint store $\cstore_\sigma$ encodes.
To do this we need one more abstract definition~--- a denotational interpretation of a constraint store $\sembr{\cstore_\sigma}$ as a set of representing functions.
We prove the soundness and completeness w.r.t. this interpretation and expect it to adequately reflect how the restrictions of constraint stores in the answers are presented.
The denotational analog of operational semantics for an arbitrary extended state is then redefined as follows.

 \[
\sembr{s}_{op} = \cup_{(\sigma, \cstore_\sigma, n) \in \tr{s}} \sembr{\sigma} \cap \sembr{\cstore_\sigma}
\]

The statement of the soundness and completeness theorem stays the same with regard to this updated definitions of semantics and denotational analog.

\begin{theorem}[Operational semantics soundness and completeness for extended language]
For any specification $\{\dots\}\; g$, for which the indices of all free variables in $g$ are limited by some number $n$

\[
\sembr{\inbr{g, \epsilon, \cstoreinit, n}}_{op} \eqrestr \sembr{\{\dots\}\; g}.
\]
\end{theorem}

To be able to prove it we, of course, need certain requirements for the given operations on constraint stores. We came up with the following list of sufficient
conditions for soundness and completeness.

\begin{enumerate}
\item $\sembr{\cstoreinit} = \{\mathfrak{f}:\mathcal{A}\mapsto\mathcal{D}\}$;
\item $\csadd{\cstore_\sigma}{t_1}{t_2} = \cstore^\prime_\sigma \implies \sembr{\cstore_\sigma} \cap \sembr{t_1 \diseq t_2} \cap \sembr{\sigma} = \sembr{\cstore^\prime_\sigma} \cap \sembr{\sigma}$;
\item $\csadd{\cstore_\sigma}{t_1}{t_2} = \bot \implies \sembr{\cstore_\sigma} \cap \sembr{t_1 \diseq t_2} \cap \sembr{\sigma} = \emptyset$;
\item $\csupdate{\cstore_\sigma}{\delta} = \cstore^\prime_{\sigma \delta} \implies \sembr{\cstore_\sigma} \cap \sembr{\sigma \delta} = \sembr{\cstore^\prime_{\sigma \delta}} \cap \sembr{\sigma \delta}$;
\item $\csupdate{\cstore_\sigma}{\delta} = \bot \implies \sembr{\cstore_\sigma} \cap \sembr{\sigma \delta} = \emptyset$.
\end{enumerate}

These conditions state that given denotational interpretation and given operations on constraint stores are adequate to each other.

Condition 1 states that interpretation of the initial constraint store is the whole domain of representing function since it does not impose any restrictions.

Condition 2 states that when we add a constraint to a store $\cstore_\sigma$ the interpretation of the result contains exactly those functions which simultaneously belong to
the interpretation of the store $\cstore_\sigma$ and satisfy the constraint if we consider only extensions of the substitution $\sigma$.

Condition 3 states that addition could fail only if no such functions exist.

Conditions 4 state that the result of updating a store with an additional substitution should have the same interpretation if we consider only extensions of the updated substitution.

Condition 5 states that update could fail only if no such functions exist.

The conditions 2-5 describe exactly what we need to prove the soundness and completeness for base goals (equality and disequality); at the same time,
since these conditions have relatively simple intuitive meaning in terms of these two operations they are expected to hold naturally
in all reasonable implementations of constraint stores.

We can prove that this is enough for soundness and completeness to hold for an arbitrary goal. However,
contrary to our expectations, the existing proof can not be just reused for all non-basic types of goals and has to be modified
significantly in the case of \lstinline|fresh|. Specifically, we need one additional condition on constraint store in state $(\sigma, n, \cstore_\sigma)$:
only the values on the first $n$ fresh variables determine whether a representing function belongs to the denotational semantics $\sembr{\sigma} \cap \sembr{\cstore_\sigma}$
of the state (note the similarity to the completeness condition). Luckily, we can infer this property for all states that can be constructed by our operational
semantics from the sufficient conditions above.

Thus for an arbitrary implementation, we need to give a formal definition of constraint store object and its denotational interpretation, provide three
operations for it and prove five conditions on them, and by this, we ensure that for arbitrary specification the interpretations of all solutions found by the
search in this version of \textsc{MiniKanren} will cover exactly the mathematical model of this specification.

As well as our previous development this extension is certified in \textsc{Coq}\footnote{\url{https://github.com/dboulytchev/miniKanren-coq/tree/disequality}}.
We describe operational semantics and its soundness and completeness as modules parametrized by the definitions of constraint
stores and proofs of the sufficient conditions for them.

\section{Concrete Implementations}
\label{sec:implementations}

In this section, we define two concrete implementations of constraint stores which can be incorporated in operational semantics: the trivial one and the one, which is close to existing real implementation in a certain version of \textsc{miniKanren}~\cite{CKanren}. We prove that they satisfy the sufficient conditions for search completeness from the previous section. Both implementations are certified in \textsc{Coq}, which allowed us to extract two correct-by-construction interpreters for \textsc{miniKanren} with
disequality constraints.

\subsection{Trivial Implementation}

This trivial implementation simply stores all pairs of terms, which the search encounters, in a multiset and never uses them:

\[ \cstore_\sigma \subset_m \mathcal{T} \times \mathcal{T} \]

\[ \cstoreinit = \emptyset \]

\[ \csadd{\cstore_\sigma}{t_1}{t_2} = \cstore_\sigma \cup \{(t_1, t_2)\} \]

\[ \csupdate{\cstore_\sigma}{\delta} = \cstore_\sigma \]

The interpretation of such constraint store is the set of all representing functions that does not equate terms in any pair:

\[ \sembr{\cstore_\sigma} = \{\reprfun \colon \mathcal{A}\mapsto\mathcal{D} \mid \forall (t_1, t_2) \in \cstore_\sigma, \; \overline{\reprfun}\,(t_1) \neq \overline{\reprfun}\,(t_2)\} \]

This is a correct implementation (although for the full implementation we should find a way to present restrictions stored this way
in answers adequately) and it satisfies the sufficient conditions for completeness trivially, but it is not very practical.
In particular, it does not use information acquired from disequalities to halt the search in case of contradiction and it can return contradictory answers with the final disequality constraint violated by the final substitution (such as $([\alpha_0 \mapsto 5], [\alpha_0 \neq 5], 1)$): since such answers have empty interpretations, their presence does not affect search completeness.

\subsection{ReaIistic Implementation}

This implementation is more similar to those in existing \textsc{miniKanren} implementations and takes an approach that is close to one described is~\cite{CKanren}.

In this version, every constraint is represented as a substitution containing variable bindings which should \emph{not} be satisfied.

\[ \cstore_\sigma \subset_m \Sigma \]

So if a constraint store $\cstore_\sigma$ contains a substitution $\omega$ the set of representing functions prohibited by it is $\sembr{\sigma \omega}$,
which provides the following denotational interpretation for a constraint store:

\[ \sembr{\cstore_\sigma} = \bigcap_{\omega \in \cstore_\sigma} \overline{ \sembr{\sigma \omega} } \]

We start with an empty store

\[ \cstoreinit = \emptyset \]

When we encounter a disequality for two terms we try to unify them and update constraint store depending on the result of unification:

\[
\csadd{\cstore_\sigma}{t_1}{t_2} =
    \begin{cases}
       \cstore_\sigma                                & \not\exists mgu(t_1 \sigma, t_2 \sigma) \\
       \bot                                                 & mgu(t_1 \sigma, t_2 \sigma) = \epsilon \\
       \cstore_\sigma \cup \{\omega\}      & mgu(t_1 \sigma, t_2 \sigma) = \omega \neq \epsilon
    \end{cases}
\]

If the terms are not unifiable, there is no need to change the constraint store. If they are unified by current substitution the constraint is already violated and we signal a failure.
Otherwise, the most general unifier is an appropriate representation of this constraint.

When updating a constraint store with an additional substitution $\delta$ we try to update each individual constraint substitution by treating it
as a list of pairs of terms that should not be unified (the first element of each pair is a variable), applying $\delta$ to these terms and trying to
unify all pairs simultaniously:

\[ \cupdate{[x_1 \mapsto t_1, \dots, x_k \mapsto t_k]}{\delta} = mgu([\delta(x_1), \dots, \delta(x_k)],[t_1 \delta, \dots, t_k \delta]) \]

We construct the updated constraint store from the results of all constraint updates:

\[
\csupdate{\cstore_\sigma}{\delta} =
\begin{cases}
  \bot                                                 & \exists \omega \in \cstore_\sigma: \cupdate{\omega}{\delta} = \epsilon \\
  \{ \omega' \mid \cupdate{\omega}{\delta} = \omega' \neq \bot, \; \omega \in \cstore_\sigma \}   & \textit{otherwise}
\end{cases}
\]

If any constraint is violated by the additional substitution we signal a failure, otherwise we take in the store the updated constraints
(and some constraints are thrown away as they can no longer be violated).

We proved the sufficient conditions for completeness for this implementation, too, but it required us to prove first that all substitutions constructed by \textsc{miniKanren} search have a specific form. Namely, a current subsitution $\sigma$ at any point of the search (started from the initial state) is always \emph{narrowing} --- which means that $\ran{\sigma} \cap \dom{\sigma} = \emptyset$ --- and every time a current substitution $\sigma$ is updated by composing with some substitution $\delta$ (in rule $\ruleno{UnifySuccess}$) this substitution is \emph{extending} --- which means that $\dom{\delta} \cap \dom{\sigma} = \emptyset \land \ran{\delta} \cap \dom{\sigma} = \emptyset$.

\section{Applications}
\label{sec:applications}

In addition to verification of correctness of different implementations of disequality constraints we can use our framework to formally
state and prove some of its other important properties. Thanks to our completeness result, we can do it in the denotational context,
where the reasoning is much easier.

For example, we can specify contradictory answers with empty interpretation, which we pointed out for the trivial implementation from the previous section,
and prove that there are no such answers in the realistic implementation if and only if there are infinitely many constructors in the language. So, for the realistic implementation the following holds iff the set of constructors is infinite:

\begin{lemma}
For any goal $g$, if all free variables in it belong to the set $\{\alpha_1,\dots,\alpha_n\}$, then

\[ \forall (\sigma, \cstore_\sigma, n_r) \in Tr_{\inbr{g, \epsilon, \cstoreinit, n}}, \quad \sembr{\sigma} \cap \sembr{\cstore_\sigma} \neq \emptyset. \]
\end{lemma}

The proof is based on the following lemma about combining constraints, which we can prove we can prove when there are infinitely many constructors (and otherwise it is not true).

\begin{lemma}
If for a finite constraint store $\cstore_\sigma$
\[ \forall \omega \in \cstore_\sigma,  \sembr{\sigma} \cap \sembr{\omega} \neq \emptyset, \]
then
\[ \sembr{\sigma} \cap \sembr{\cstore_\sigma} \neq \emptyset. \]
\end{lemma}

Another example of application is the justification of optimizations in constraint store implementation. For example, the following obvious (in denotational context) statement
allows deleting subsumed constraints in the realistic implementation.

\begin{lemma}
For any constraint store $\cstore_\sigma$ and two constraint substitutions $\omega$ and $\omega'$, if

\[ \exists \tau, \omega' = \omega \tau \]

then

\[ \sembr{\cstore_\sigma \cup \{\omega, \omega'\}} = \sembr{\cstore_\sigma \cup \{\omega\}}. \]
\end{lemma}

\section{Future Work}

There are a few possible directions for future work. First, in this paper we did not address the performance issues. As we represent
the transformations in a very generic form with many levels of indirection, obviously, the transformations, implemented with
our framework, are at disadvantage in comparison with hard coded ones in terms of performance. We assume that the performance of transformations
can be essentially improved by applying some techniques like staging~\cite{Staged} or, perhaps, object-specific optimisations.

Another important direction is supporting more kinds of type declarations, in the first hand, GADTs and non-regular types. Although we have some
implementation ideas for this case, the solution we came up with so far makes the interface of the whole framework too cumbersome to use even for
simple cases.

Finally, the typeinfo structure we generate can be used to mimic the \emph{ad-hoc} polymorphism as it contains the implementation of
type-indexed functions. This, together with some proposed extensions~\cite{ModularImplicits}, can open interesting perspectives.



%% Acknowledgments
%\begin{acks}                            %% acks environment is optional
                                        %% contents suppressed with 'anonymous'
  %% Commands \grantsponsor{<sponsorID>}{<name>}{<url>} and
  %% \grantnum[<url>]{<sponsorID>}{<number>} should be used to
  %% acknowledge financial support and will be used by metadata
  %% extraction tools.
 % This material is based upon work supported by the
  %\grantsponsor{GS100000001}{National Science
   % Foundation}{http://dx.doi.org/10.13039/100000001} under Grant
  %No.~\grantnum{GS100000001}{nnnnnnn} and Grant
  %No.~\grantnum{GS100000001}{mmmmmmm}.  Any opinions, findings, and
  %conclusions or recommendations expressed in this material are those
  %of the author and do not necessarily reflect the views of the
  %National Science Foundation.
%\end{acks}


%% Bibliography
\bibliography{main}


%% Appendix
%\appendix
%\section{Appendix}

%Text of appendix \ldots

\end{document}
