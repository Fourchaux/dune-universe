\section{Related works}
\label{sec:related}

There are a few different approaches for compiling pattern matching. GHC is using influential paper~\cite{Jones1987}, OCaml is currently based on~\cite{maranget2001} although a work~\cite{maranget2008} can slightly improve effectiveness of generated code. 

Although semantics of pattern matching can be given as a sequence of scrutinee's sub expression comparisons (Figure~\ref{fig:matchpatts}) effective compilers don't follow this approach. One can either optimise run time cost by minimizing amount of checks performed or static cost by minimizing the size of generated code. \emph{Decision trees} are good for the first criteria, because they check every subexpression not more than once. \emph{Backtracking automata} are rather compact but in some cases can perform repeated checks.


Minimizing the size of decision tree is  NP-hard (\cite{baudinet1985tree}, without proof) and usually heuristics are applied during compilation, for example: count of nodes, length of the longest path, average length of all paths. The paper~\cite{Scott2000WhenDM} performs experimental evaluation of 9  heuristics on the base of for Standard ML of New Jersey.


The matching compilers for strict languages can work in \emph{direct} or \emph{indirect} styles. The first ones return effective code immediately. In the second style to construct final answer some post processing is required. It can vary from easy simplifications to complicated supercompilation techniques~\cite{sestoft1996}. The main drawback of indirect style is that size of intermediate data structures can be exponentially large.

For strict languages checking  sub expressions of scrutinee in any order is allowed. For lazy languages pattern matching should evaluate only these sub expressions which are necessary for performing pattern matching. If not careful pattern matching can change termination behavior of the program.  In general lazy languages setup more constraints on pattern matching and because of that allow lesser set of heuristics.

A few approaches for checking sub expressions in lazy languages has been proposed~\cite{augustsson1985,laville1991}. \cite{laville1991} models values in lazy languages using \emph{partial terms}, although this approach doesn't scale to types with infinite constructor sets (like integers). In  the \cite{suarez1993} the similar approach is extended by special treatment of overlapping patterns. Pattern matching has been compiled to decision trees~\cite{maranget1992} and later \cite{maranget1992} into \emph{decision DAGs} that allow in some cases to make code smaller.

The first works on compilation to backtracking automaton are~\cite{augustsson1985,wadler1987}. 

The inefficiency of backtracking automaton has been improved in~\cite{maranget2001}. The approach utilizes matrix representation for pattern matching. It splits current matrix  according to constructors in the first column and reduces the task to compiling matrices with less rows. The technique is indirect, in the end a few optimizations are performed by introducing special \emph{exit} nodes to the compiled representation.
No preprocessing is required for this scheme: or-pattern receive a special treatment during compilation process.
 The approach from this paper is used in current implementation of OCaml compiler.

Previous approach uses first column to split the matrix. In~\cite{maranget2008} has been introduced \emph{necessity} heuristic that recommends which column should be used to perform split. Good decision trees that are constructed in this work can perform better in corner cases than~\cite{maranget2001} but for practical cases the difference is insignificant.

To summarize, compilers can try to optimize pattern matching either for guaranteed code speed or for guaranteed code size. There are distinct techniques to minimize drawbacks of both approaches.

