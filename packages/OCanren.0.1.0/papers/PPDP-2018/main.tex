\documentclass[sigconf]{acmart}

\usepackage{booktabs} % For formal tables
\usepackage{amssymb}
\usepackage{amsmath}
\usepackage{mathrsfs}
\usepackage{mathtools}
\usepackage{multirow}
\usepackage{listings}
\usepackage{indentfirst}
\usepackage{verbatim}
\usepackage{amsmath, amssymb}
\usepackage{graphicx}
\usepackage{xcolor}
\usepackage{url}
\usepackage{stmaryrd}
\usepackage{xspace}
\usepackage{comment}
\usepackage{wrapfig}
\usepackage[caption=false]{subfig}
\usepackage{placeins}
\usepackage{tabularx}
\usepackage{ragged2e}

\newtheorem{theorem}{Theorem}
\newtheorem{lemma}{Lemma}

\def\transarrow{\xrightarrow}
\newcommand{\setarrow}[1]{\def\transarrow{#1}}

\def\padding{\phantom{X}}
\newcommand{\setpadding}[1]{\def\padding{#1}}

\def\subarrow{}
\newcommand{\setsubarrow}[1]{\def\subarrow{#1}}

\newcommand{\trule}[2]{\frac{#1}{#2}}
\newcommand{\crule}[3]{\frac{#1}{#2},\;{#3}}
\newcommand{\withenv}[2]{{#1}\vdash{#2}}
\newcommand{\trans}[3]{{#1}\transarrow{\padding#2\padding}\subarrow{#3}}
\newcommand{\ctrans}[4]{{#1}\transarrow{\padding#2\padding}\subarrow{#3},\;{#4}}
\newcommand{\llang}[1]{\mbox{\lstinline[mathescape]|#1|}}
\newcommand{\pair}[2]{\inbr{{#1}\mid{#2}}}
\newcommand{\inbr}[1]{\left<{#1}\right>}
\newcommand{\highlight}[1]{\color{red}{#1}}
\newcommand{\ruleno}[1]{\eqno[\scriptsize\textsc{#1}]}
\newcommand{\rulename}[1]{\textsc{#1}}
\newcommand{\inmath}[1]{\mbox{$#1$}}
\newcommand{\lfp}[1]{fix_{#1}}
\newcommand{\gfp}[1]{Fix_{#1}}
\newcommand{\vsep}{\vspace{-2mm}}
\newcommand{\supp}[1]{\scriptsize{#1}}
\renewcommand{\G}{\mathfrak G}
\newcommand{\sembr}[1]{\llbracket{#1}\rrbracket}
\newcommand{\cd}[1]{\texttt{#1}}
\newcommand{\miniKanren}{miniKanren\xspace}
\newcommand{\ocanren}{OCanren\xspace}
\newcommand{\free}[1]{\boxed{#1}}
\newcommand{\binds}{\;\mapsto\;}
\newcommand{\dbi}[1]{\mbox{\bf{#1}}}
\newcommand{\sv}[1]{\mbox{\textbf{#1}}}
\newcommand{\bnd}[2]{{#1}\mkern-9mu\binds\mkern-9mu{#2}}

\newcommand{\meta}[1]{{\mathcal{#1}}}
\renewcommand{\emptyset}{\varnothing}

\lstdefinelanguage{ocanren}{
keywords={fresh, let, in, match, with, when, class, type,
object, method, of, rec, repeat, until, while, not, do, done, as, val, inherit,
new, module, sig, deriving, datatype, struct, if, then, else, open, private, virtual, include, success, failure,
true, false},
sensitive=true,
commentstyle=\small\itshape\ttfamily,
keywordstyle=\ttfamily\underbar,
identifierstyle=\ttfamily,
basewidth={0.5em,0.5em},
columns=fixed,
fontadjust=true,
literate={fun}{{$\lambda$}}1 {->}{{$\to$}}3 {===}{{$\equiv$}}1 {=/=}{{$\not\equiv$}}1 {|>}{{$\triangleright$}}3 {\\/}{{$\vee$}}2 {/\\}{{$\wedge$}}2 {^}{{$\uparrow$}}1,
morecomment=[s]{(*}{*)}
}

\lstset{
mathescape=true,
%basicstyle=\small,
identifierstyle=\ttfamily,
keywordstyle=\bfseries,
commentstyle=\scriptsize\rmfamily,
basewidth={0.5em,0.5em},
fontadjust=true,
language=ocanren
}

\usepackage{letltxmacro}
\newcommand*{\SavedLstInline}{}
\LetLtxMacro\SavedLstInline\lstinline
\DeclareRobustCommand*{\lstinline}{%
  \ifmmode
    \let\SavedBGroup\bgroup
    \def\bgroup{%
      \let\bgroup\SavedBGroup
      \hbox\bgroup
    }%
  \fi
  \SavedLstInline
}

\sloppy

\begin{document}

\title{Improving Refutational Completeness\\
of Relational Search via Divergence Test$^*$}

\thanks{$^*\;$This work is supported by RFBR grant No 18-01-00380.}

%\titlenote{Produces the permission block, and
%  copyright information}

\author{Dmitri Rozplokhas}
\affiliation{%
  \institution{St. Petersburg Academic University}
  \streetaddress{Khlopina st., 8-3-А}
  \city{St. Petersburg}
  \state{Russia}
  \postcode{194021}
}
\email{rozplokhas@gmail.com}

\author{Dmitri Boulytchev}
\affiliation{%
  \institution{St. Petersburg State University}
  \streetaddress{Universitetskaya emb., 7-9}
  \city{St. Petersburg}
  \state{Russia}
  \postcode{199034}
}
\email{dboulytchev@math.spbu.ru}

\begin{abstract}
We describe a search optimization technique for implementation of relational programming language
miniKanren which makes more queries converge. Specifically, we address the problem of conjunction
non-commutativity. Our technique is based on a certain divergence criterion that we use to trigger a
dynamic reordering of conjuncts. We present a formal semantics of a miniKanren-like language and prove
that our optimization does not compromise already converging programs, thus, being a proper improvement.
We also present the prototype implementation of the improved search and demonstrate its application for a
number of realistic specifications.
\end{abstract}

%
% The code below should be generated by the tool at
% http://dl.acm.org/ccs.cfm
% Please copy and paste the code instead of the example below.
%
\begin{CCSXML}
<ccs2012>
<concept>
<concept_id>10003752.10003790.10003795</concept_id>
<concept_desc>Theory of computation~Constraint and logic programming</concept_desc>
<concept_significance>500</concept_significance>
</concept>
<concept>
<concept_id>10003752.10010124.10010131.10010134</concept_id>
<concept_desc>Theory of computation~Operational semantics</concept_desc>
<concept_significance>100</concept_significance>
</concept>
<concept>
<concept_id>10011007.10011006.10011008.10011009.10011015</concept_id>
<concept_desc>Software and its engineering~Constraint and logic languages</concept_desc>
<concept_significance>500</concept_significance>
</concept>
</ccs2012>
\end{CCSXML}

\ccsdesc[500]{Theory of computation~Constraint and logic programming}
\ccsdesc[100]{Theory of computation~Operational semantics}
\ccsdesc[500]{Software and its engineering~Constraint and logic languages}

\keywords{relational programming, refutational completeness, divergence test}

\copyrightyear{2018}
\acmYear{2018}
\setcopyright{acmcopyright}
\acmConference[PPDP '18]{The 20th International Symposium on Principles and Practice of Declarative Programming}{September 3--5, 2018}{Frankfurt am Main, Germany}
\acmBooktitle{The 20th International Symposium on Principles and Practice of Declarative Programming (PPDP '18), September 3--5, 2018, Frankfurt am Main, Germany}
\acmPrice{15.00}
\acmDOI{10.1145/3236950.3236958}
\acmISBN{978-1-4503-6441-6/18/09}

\maketitle

\section{Introduction}
\label{sec:intro}

Algebraic data types (ADT) are an important tool in functional programming which deliver a way to represent flexible and easy to manipulate data structures.
To inspect the contents of an ADT's values a generic construct~--- \emph{pattern matching}~--- is used. The importance of pattern matching efficient
implementation stimulated the development of various advanced techniques which provide good results in practice. The objective of our work is to use these
results as a baseline for a case study of relational synthesis\footnote{We have to note that this term is overloaded and can be used to refer to completely
different approaches than we utilize.}~--- an approach for program synthesis based on application of relational programming~\cite{TRS,WillThesis}, and,
in particular, relational interpreters~\cite{unified} and relational conversion~\cite{conversion}. Relational programming can be considered as a specific form
of constraint logic programming centered around \textsc{miniKanren}\footnote{\url{http://minikanren.org}}, a combinator-based DSL, implemented for a number of host languages.
Unlike \textsc{Prolog}, which employs a deterministic depth-first search, \textsc{miniKanren} advocates a 
%completely 
more
declarative approach, in which a user is not
allowed to rely on a concrete search discipline, which means, that the specifications, written in \textsc{miniKanren}, are understood much more symmetrically.
The distinctive feature of \textsc{miniKanren} is complete \emph{interleaving search}~\cite{search}. The basic constraint is unification with occurs check, although
advanced implementations support other primitive constructs, such as disequality or finite-domain constraints~\cite{CKanren}. Syntactically, \textsc{miniKanren} is mutually
convertible to \textsc{Prolog}, but, unlike latter, makes use of explicit logical connectives (conjunction and disjunction), existential quantification and unification.
 
A distinctive application of relational programming is implementing \emph{relational interpreters}~\cite{Untagged}. Unlike conventional interpreters, which for a program and
input value produce output, relational interpreters can operate in various directions: for example, they are capable of computing an input value for a given
program and a given output, or even synthesize a program for a given pairs of input-output values. The latter case forms a basis for program synthesis~\cite{eigen,unified}.

Our approach is based on relational representation of the source language pattern matching semantics on the one hand, and
the semantics of the intermediate-level implementation language on the other. We formulate the condition necessary for a correct and complete implementation of pattern matching and use it to
construct a top-level goal which represents a search procedure for all correct and complete implementations. We also present a number of techniques which make it possible to come up with an
\emph{optimal} solution as well as optimizations to improve the performance of the search. Similarly to many other prior works we use the size of the synthesized code, which can be measured
statically, to distinguish better programs. Our implementation\footnote{\url{https://github.com/Kakadu/pat-match/tree/aplas2020}} makes use of \textsc{OCanren}\footnote{\url{https://github.com/JetBrains-Research/OCanren}}~---
 a typed implementation of \textsc{miniKanren} for \textsc{OCaml}~\cite{OCanren}, and \textsc{noCanren}\footnote{\url{https://github.com/Lozov-Petr/noCanren}}~--- 
a converter from the subset of plain \textsc{OCaml} into \textsc{OCanren}~\cite{conversion}. An initial  evaluation, performed for a set of benchmarks taken from other papers, showed our synthesizer performing well.
However, being aware of some pitfalls of our approach, we came up with a set of counterexamples on which it did not provide any results in observable time, so we do not consider the problem
completely solved. We also started to work on mechanized 
formalization\footnote{\url{https://github.com/dboulytchev/Coq-matching-workout}},
written in \textsc{Coq}~\cite{Coq}, to make the justification of our approach more solid and easier to verify, but this formalization is not yet complete. 

 

\begin{comment}
We apply relational programming techniques to the problem of synthesizing efficient implementation for a pattern matching construct.
Although in principle pattern matching can be implemented in a trivial way, the result suffers from inefficiency in terms of both
performance and code size. Thus, in implementing functional languages alternative, more elaborate  approaches are widely used.
However, as there are multiple kinds and flavors of pattern matching constructs, these approaches have to be specifically developed
and justified for each concrete inhabitant of the pattern matching ``zoo''. We formulate the pattern matching synthesis problem in
declarative terms and apply relational programming, a specific form of constraint logic programming, to develop a 
develop optimizations which improve the efficiency of the synthesis and guarantee the
optimality of the result. 
\end{comment}

\begin{figure*}[t]
\[
\begin{array}{cccll}
  &\mathcal{C} & = & \{C_i^{k_i}\} & \mbox{constructors with arities} \\
  &\mathcal{T}_X & = & X \cup \{C_i^{k_i} (t_1, \dots, t_{k_i}) \mid t_j\in\mathcal{T}_X\} & \mbox{terms over the set of variables $X$} \\
  &\mathcal{D} & = & \mathcal{T}_\emptyset & \mbox{ground terms}\\
  &\mathcal{X} & = & \{ x, y, z, \dots \} & \mbox{syntactic variables} \\
  &\mathcal{A} & = & \{ \alpha, \beta, \gamma, \dots \} & \mbox{semantic variables} \\
  &\mathcal{R} & = & \{ R_i^{k_i}\} &\mbox{relational symbols with arities} \\
  &\mathcal{G} & = & \mathcal{T_X}\equiv\mathcal{T_X}   &  \mbox{unification} \\
  &            &   & \mathcal{G}\wedge\mathcal{G}     & \mbox{conjunction} \\
  &            &   & \mathcal{G}\vee\mathcal{G}       &\mbox{disjunction} \\
  &            &   & \mbox{\lstinline|fresh|}\;\mathcal{X}\;.\;\mathcal{G} & \mbox{fresh variable introduction} \\
  &            &   & R_i^{k_i} (t_1,\dots,t_{k_i}),\;t_j\in\mathcal{T_X} & \mbox{relational symbol invocation} \\
  &\mathcal{S} & = & \{R_i^{k_i} = \lambda\;x_1^i\dots x_{k_i}^i\,.\, g_i;\}\; g & \mbox{specification}
\end{array}
\]
\caption{The syntax of the source language}
\label{syntax}
\end{figure*}

\begin{comment}
\begin{figure}[t]
%\centering
\[
\begin{array}{rcl}
  \mathcal{FV}\,(x)&=&\{x\}\\
  \mathcal{FV}\,(C_i^{k_i}\,(t_1,\dots,t_{k_i}))&=&\bigcup\mathcal{FV}\,(t_i)\\
  \mathcal{FV}\,(t_1\equiv t_2)&=&\mathcal{FV}\,(t_1)\cup\mathcal{FV}\,(t_2)\\
  \mathcal{FV}\,(g_1\wedge g_2)&=&\mathcal{FV}\,(g_1)\cup\mathcal{FV}\,(g_2)\\
  \mathcal{FV}\,(g_1\vee g_2)&=&\mathcal{FV}\,(g_1)\cup\mathcal{FV}\,(g_2)\\
  \mathcal{FV}\,(\mbox{\lstinline|fresh|}\;x\;.\;g)&=&\mathcal{FV}\,(g)\setminus\{x\}\\
  \mathcal{FV}\,(R_i^{k_i}\,(t_1,\dots,t_{k_i}))&=&\bigcup\mathcal{FV}\,(t_i)
\end{array}
\]
\caption{Free variables in terms and goals}
\label{free}
\end{figure}
\end{comment}

\section{The Language}
\label{language}
 
In this section, we introduce the syntax of the language we use throughout the paper, describe the informal semantics, and give some examples.

The syntax of the language is shown in Fig.~\ref{syntax}. First, we fix a set of constructors $\mathcal{C}$ with known arities and consider
a set of terms $\mathcal{T}_X$ with constructors as functional symbols and variables from $X$. We parameterize this set with an alphabet of
variables since in the semantic description we will need \emph{two} kinds of variables. The first kind, \emph{syntactic} variables, is denoted
by $\mathcal{X}$. The second kind, \emph{semantic} or \emph{logic} variables, is denoted by $\mathcal{A}$.
We also consider an alphabet of \emph{relational symbols} $\mathcal{R}$ which are used to name relational definitions.
The central syntactic category in the language is \emph{goal}. In our case, there are five types of goals: \emph{unification} of terms,
conjunction and disjunction of goals, fresh variable introduction, and invocation of some relational definition. Thus, unification is used
as a constraint, and multiple constraints can be combined using conjunction, disjunction, and recursion.
The final syntactic category is a \emph{specification} $\mathcal{S}$. It consists of a set
of relational definitions and a top-level goal. A top-level goal represents a search procedure which returns a stream of substitutions for
the free variables of the goal. The definition for a set of free variables for both terms and goals is conventional;
%given in Figure~\ref{free};
as ``\lstinline|fresh|''
is the sole binding construct the definition is rather trivial. The language we defined is first-order, as goals can not be passed as parameters,
returned or constructed at run time.

We now informally describe how relational search works. As we said, a goal represents a search procedure. This procedure takes a \emph{state} as input and returns a
stream of states; a state (among other information) contains a substitution that maps semantic variables into the terms over semantic variables. Then five types of
scenarios are possible (depending on the type of the goal):

\begin{itemize}
\item Unification ``\lstinline|$t_1$ === $t_2$|'' unifies terms $t_1$ and $t_2$ in the context of the substitution in the current state. If terms are unifiable,
  then their MGU is integrated into the substitution, and a one-element stream is returned; otherwise the result is an empty stream.
\item Conjunction ``\lstinline|$g_1$ /\ $g_2$|'' applies $g_1$ to the current state and then applies $g_2$ to each element of the result, concatenating
  the streams.
\item Disjunction ``\lstinline|$g_1$ \/ $g_2$|'' applies both its goals to the current state independently and then concatenates the results.
\item Fresh construct ``\lstinline|fresh $x$ . $g$|'' allocates a new semantic variable $\alpha$, substitutes all free occurrences of $x$ in $g$ with $\alpha$, and
  runs the goal.
\item Invocation ``$\lstinline|$R_i^{k_i}$ ($t_1$,...,$t_{k_i}$)|$'' finds a definition for the relational symbol \mbox{$R_i^{k_i}=\lambda x_1\dots x_{k_i}\,.\,g_i$}, substitutes
  all free occurrences of a formal parameter $x_j$ in $g_i$ with term $t_j$ (for all $j$) and runs the goal in the current state.
\end{itemize}

We stipulate that the top-level goal is preceded by an implicit ``\lstinline|fresh|'' construct, which binds all its free variables, and that the final substitutions
for these variables constitute the result of the goal evaluation.

Conjunction and disjunction form a monadic~\cite{Monads} interface with conjunction playing role of ``\lstinline|bind|'' and disjunction the role of ``\lstinline|mplus|''.
In this description, we swept a lot of important details under the carpet~--- for example, in actual implementations the components of disjunction are not evaluated in
isolation, but both disjuncts are evaluated incrementally with the control passing from one disjunct to another (\emph{interleaving})~\cite{Search};
the evaluation of some goals can be additionally deferred (via so-called ``\emph{inverse-$\eta$-delay}'')~\cite{MicroKanren}; instead of streams
the implementation can be based on ``ferns''~\cite{BottomAvoiding} to defer divergent computations, etc. In the following sections, we present
a complete formal description of relational semantics which resolves these uncertainties in a conventional way.

As an example consider the following specification. For the sake of brevity we
abbreviate immediately nested ``\lstinline|fresh|'' constructs into the one, writing ``\lstinline|fresh $x$ $y$ $\dots$ . $g$|'' instead of
``\lstinline|fresh $x$ . fresh $y$ . $\dots$ $g$|''.

\begin{tabular}{p{5.5cm}p{5.5cm}}
\begin{lstlisting}
append$^o$ = fun x y xy .
 ((x === Nil) /\ (xy === y)) \/
 (fresh h t ty .
   (x  === Cons (h, t))  /\
   (xy === Cons (h, ty)) /\
   (append$^o$ t y ty));

revers$^o$ x x
\end{lstlisting} &
\begin{lstlisting}
revers$^o$ = fun x xr .
 ((x === Nil) /\ (xr === Nil)) \/
 (fresh h t tr .
   (x === Cons (h, t)) /\
   (append$^o$ tr (Cons (h, Nil)) xr) /\
   (revers$^o$ t tr));
\end{lstlisting}
\end{tabular}

Here we defined\footnote{We respect here a conventional tradition for \textsc{miniKanren} programming to superscript all relational names with ``$^o$''.}
two relational symbols~--- ``\lstinline|append$^o$|'' and ``\lstinline|revers$^o$|'',~--- and specified a top-level goal ``\lstinline|revers$^o$ x x|''.
The symbol ``\lstinline|append$^o$|'' defines a relation of concatenation of lists~--- it takes three arguments and performs a case analysis on the first one. If the
first argument is an empty list (``\lstinline|Nil|''), then the second and the third arguments are unified. Otherwise, the first argument is deconstructed into a head ``\lstinline|h|''
and a tail ``\lstinline|t|'', and the tail is concatenated with the second argument using a recursive call to ``\lstinline|append$^o$|'' and additional variable ``\lstinline|ty|'', which
represents the concatenation of ``\lstinline|t|'' and ``\lstinline|y|''. Finally, we unify ``\lstinline|Cons (h, ty)|'' with ``\lstinline|xy|'' to form a final constraint. Similarly,
``\lstinline|revers$^o$|'' defines relational list reversing. The top-level goal represents a search procedure for all lists ``\lstinline|x|'', which are stable under reversing, i.e.
palindromes. Running it results in an infinite stream of substitutions:

\begin{lstlisting}
   $\alpha\;\mapsto\;$ Nil
   $\alpha\;\mapsto\;$ Cons ($\beta_0$, Nil)
   $\alpha\;\mapsto\;$ Cons ($\beta_0$, Cons ($\beta_0$, Nil))
   $\alpha\;\mapsto\;$ Cons ($\beta_0$, Cons ($\beta_1$, Cons ($\beta_0$, Nil)))
   $\dots$
\end{lstlisting}

where ``$\alpha$'' is a \emph{semantic} variable, corresponding to ``\lstinline|x|'', ``$\beta_i$'' are free semantic variables. Therefore, each substitution represents a set of all palindromes of a certain length.


\section{Refutational Incompleteness and Conjunction Non-Commutativity}
\label{incompleteness}

The language, defined in the previous section, is expected to allow defining computable relations in a 
very concise and declarative form. In particular, it is expected from a relational 
specification to preserve its behavior regardless the order of conjunction/disjunction 
constituents. Regretfully, in general this is not true, and one of the most important
manifestations of this deficiency is \emph{refutational incompleteness}.  

In the context of relational programming, refutational completeness~\cite{WillThesis} is understood as 
a capability of a program to discover the absence of solutions and stop. At the first glance,
the divergence in the case of solution absence does not seem to be a severe problem. However, as
we see shortly, refutational incompleteness leads to many observable negative effects in numerous
practically important cases. 

We demonstrate the effect of refutational incompleteness with a very simple example. Let us take the
definition of \lstinline{append$^o$} from the previous section and try to evaluate the following query:

\begin{lstlisting}
   fresh ($p\;q$) (append$^o$ $p$ $q$ Nil)
\end{lstlisting}

We would expect this query to converge to the single answer \mbox{$p=\lstinline|Nil|$}, \mbox{$q=\lstinline|Nil|$};
however, in the reality the query diverges. We sketch here the explanation, omitting some non-essential technical
details, such as semantic variables allocation, etc.:

\begin{itemize}
\item First we evaluate the first disjunct of \lstinline|append$^o$|'s body and unify $p$ with \lstinline|Nil| (successfully)
and \lstinline|Nil| with $q$ (successfully), which gives us the first (expected) answer.

\item Then we proceed to the second disjunct, which is a conjunction of three simpler goals:

  \begin{itemize} 
     \item in the first one we unify $p$ with \lstinline|Cons ($h$, $t$)| (successfully);
     \item in the second we encounter a recursive call \lstinline|append$^o$ $t$ $q$ Nil|; since its arguments are merely the renamings of the enclosing one, we repeat from the top and never stop.
  \end{itemize} 
\end{itemize}

The problem is that the semantics of conjunction, in fact, is not commutative: when the first conjunct diverges and the second fails, the whole
conjunction diverges. We stress that this is not a deviation of our semantics, but a well-known phenomenon, manifesting itself in all known
miniKanren implementations. In our example, switching two last conjuncts in the definition of \lstinline|append$^o$| solves the problem~---
now the whole search stops after the unsuccessful attempt to unify \lstinline|Nil| and \lstinline|Cons ($h$, $ty$)| with no recursive call.
This, improved version of \lstinline|append$^o$|, is known to be refutationally complete. In fact, there is a conventional ``rule of thumb''
for miniKanren programming to place the recursive call as far right as possible in a list of conjuncts. 

This convention, however, does not always help; to tell the truth, it often makes the things worse. Consider 
as an example yet another relation on lists:

\begin{lstlisting}
   revers$^o$ $\binds$ $\lambda\;x\;x_r$ . 
     (($x$ === $\;\;$Nil) /\ ($x_r$ === $\;\;$Nil)) \/
     (fresh ($h$ $t$ $t_r$)
        ($x$  === $\;$Cons ($h$, $t$)) /\
        (append$^o$ $t_r$ (Cons ($h$, Nil)) $x_r$) /\
        (revers$^o$ $t$ $t_r$)
     )
\end{lstlisting}

This relation corresponds to a relational list reversing; as we see, the recursive call is placed to
the end. However, the following query

\begin{lstlisting}
   fresh ($q$) (revers$^o$ (Cons (A, Nil)) $q$)
\end{lstlisting}

\noindent diverges, while

\begin{lstlisting}
   fresh ($q$) (revers$^o$ $q$ (Cons (A, Nil)))
\end{lstlisting}

\noindent converges to the expected results. If we switch the two last conjuncts in the definition of
\lstinline|revers$^o$|, the situation reverses: the first query converges, while the second diverges. 
This example demonstrates that the desired position of a recursive call (and, in general, the order of
conjuncts) depends on the direction, in which the relation of interest is evaluated.

There are, however, some cases, when the same relation is evaluated in both directions, regardless
the query. We can take as an example relational permutations, which can be implemented by running
relational list sorting in both directions:

\begin{lstlisting}
   sort$^o$ $\binds\lambda\;x\;x_s\;.\; \dots$
   perm$^o$ $\binds\lambda\;x\;x_p\;.$
     fresh ($x_s$) 
       (sort$^o$ $x$ $x_s$) $\wedge$ (sort$^o$ $x_p$ $x_s$) 
\end{lstlisting}

The idea of this implementation is very simple. Let us want to calculate all permutations of some list $l$.
We first sort $l$, obtaining the sorted version $l^\prime$; then we ask for all lists which, being sorted,
become equal to $l^\prime$. Obviously, all such lists are merely permutations of the original list $l$. The
important observation is that the existence of a single list sorting relation is sufficient to implement this
idea.

The concrete definition of the relational list sorting \lstinline|sort$^o$| is not important, so we
omit it due to the space considerations (an interested reader can refer to~\cite{OCanren}). The important part 
is that it is obviously recursive and not refutationally complete, and it is being evaluated 
in \emph{both} directions within the body of \lstinline|perm$^o$|. So, \lstinline|perm$^o$| is expected 
to perform poorly regardless the order of recursive calls in \lstinline|sort$^o$| implementation; it, 
indeed, does. First, if we request all solutions, both \lstinline|fresh ($q$) (perm$^o$ l $q$)| and \lstinline|fresh ($q$) (perm$^o$ $q$ l)| diverge for arbitrary non-empty list \lstinline|l| regardless the implementation of \lstinline|sort$^o$|; second, even if we request only a first few existing solutions, it does not provide any results in a reasonable time even for very small list lengths (4, 5, etc.).

Interesting, that if we interested in all solutions,
we have to accurately precompute their number in order not to request more, than exists. For some problems,
it may be not so simple, as it looks at a first glance (for example, the number of all permutations is
not a factorial, but a number of permutations with repetitions). Finally, getting the number of solutions can 
itself be an objective for writing a relational specification (we provide some examples in Section~\ref{evaluation}),
and without refutational completeness requesting all solutions to calculate their number is out of
reach.

\section{Search Improvement}
\label{improvement}

As we've seen in the previous section, the non-commutativity of conjunction in the presence of recursion
is one of the reasons for refutational incompleteness. Switching arguments of a certain conjunction
can sometimes improve the results; there is, however, no certain static order, beneficial in all cases.
Thus, we can make the following observations:

\begin{itemize}
\item the conjunction to change has to be properly identified;
\item the order of conjunct evaluation has to be a subject of a \emph{dynamic} choice.
\end{itemize}

Our improvement of the search is based on the idea of switching the order of conjuncts only when
the divergence of the first one is detected. More specifically: 

\begin{itemize}
\item during the search, we keep track of all conjunctions being performed;
\item when we detect the divergence, we roll back to the nearest conjunction, for which 
we did not try all orders of constituents yet, switch its constituents, and rerun 
the search from that conjunction.
\end{itemize}

The important detail is the divergence test. Of course, due to the fundamental results in computability
theory, there is no hope to find a \emph{precise} computable test that constitutes the necessary and 
sufficient condition of divergence. However, in our case a sufficient condition is sufficient. Indeed,  
a sufficient condition identifies a case, when the search, being continued, will lead to an incompleteness 
(since a divergence in our semantics always means incompleteness). Thus, it is no harm to try some other way. 

Another important question is the discipline of conjuncts reordering. Indeed, simply switching any two operands
of, for example, \mbox{$(g_1\wedge g_2)\wedge g_3$}, would not allow us to try \mbox{$(g_1\wedge g_3)\wedge g_2$}.
Thus, we have to flatten each ``cluster'' of nested conjunctions into a list of conjuncts\mbox{$\bigwedge g_i$}, 
where none of the goals $g_i$ is a conjunction. Then, it may seem at the first glance that the number of orderings to try 
is exponential on the number of conjuncts; we are going to show that, fortunately, this is not the case, and
a quadratic number of orders is sufficient.

In the rest of the section we address all these issues in details: first, we formally present the divergence
criterion and prove the necessity property; then, we describe an efficient reordering discipline. Finally, we present a
modified version of the semantics with incorporated divergence test and reordering. This semantics can be
considered as a modified version of the search, and we prove that this modification is a proper improvement in terms
of convergence.

\subsection{The Divergence Test}

Our divergence test is based on the following notion:

\begin{definition}
\normalfont 
We say that a vector of terms $\overline{a^{\phantom{x}}_i}$ is more general, than a vector of terms $\overline{b^{\phantom{x}}_i}$ (notation 
$\overline{a^{\phantom{x}}_i}\succeq\overline{b^{\phantom{x}}_i}$), if there is a substitution $\tau$, such that $\forall i\;b_i = a_i \tau$.
\end{definition}

The idea of the divergence test is rather simple: it identifies a recursive call with more general arguments 
than (some) enclosing one. To state it formally and prove it using the semantics from section~\ref{language}, we need several definitions and lemmas.

\begin{definition}
\normalfont
A semantic variable $v$ is \emph{observable} w.r.t. the interpretation $\iota$ and substitution $\sigma$, if there exists 
a syntactic variable $x$, such that \mbox{$v \in FV(\iota(x) \sigma)$}.
\end{definition}

\begin{definition}
A triplet of interpretation, substitution and a set of allocated semantic variables \mbox{$(\iota,\sigma,\delta)$} is
called \emph{coherent}, if \mbox{$dom(\sigma) \subseteq \delta$}, and any semantic variable, observable w.r.t. $\iota$ and $\sigma$,
belongs to $\delta$.  
\end{definition}

\begin{definition}
\normalfont
A semantic statement 

$$
\otrans{\Gamma,\iota}{(\sigma,\,\delta)}{g}{S}
$$ 

\noindent is \emph{well-formed}, if \mbox{$(\iota,\sigma,\delta)$} is a coherent triplet.
\end{definition}

Note, the root semantic statement \mbox{$\otrans{\Gamma,\bot}{(\epsilon,\,\emptyset)}{g}{S}$} is always well-formed.

\begin{lemma}
\label{one}
\normalfont
 For a well-formed semantic statement, every statement in its derivation tree is also well-formed.
\end{lemma}

The proof is by induction on the derivation tree. Note, we need to generalize the statement of the lemma, adding the condition that
\mbox{$(\iota,\sigma_r,\delta_r)$} is a coherent triplet for any \mbox{$(\sigma_r,\,\delta_r) \in S$}.

The next lemma ensures that any substitution in the RHS of a semantic statement is a correct refinement of that in the LHS:

\begin{lemma}
\label{two}
\normalfont
For a well-formed semantic statement 

$$
\otrans{\Gamma,\iota}{(\sigma,\,\delta)}{g}{S}
$$ 

\noindent and any result \mbox{$(\sigma_r,\,\delta_r) \in S$}, there exists a substitution $\Delta$, such that:
  \begin{enumerate}
    \item \mbox{$\sigma_r = \sigma\circ\Delta$};
    \item any semantic variable \mbox{$v\in dom(\Delta)\cup ran(\Delta)$} either is observable w.r.t. $\iota$ and $\sigma$,
 or does not belong to $\delta$ (where \mbox{$ran(\Delta)=\bigcup_{v\in dom(\Delta)}FV(\Delta(v))$}).
  \end{enumerate}   
\end{lemma}

The proof is by induction on the derivation tree; we as well need to generalize the statement of the lemma, adding the condition that the 
set of all allocated semantic variables $\delta$ can only grow during the evaluation.

The final lemma formalizes the intuitive considerations that the evaluation for a certain state $(\sigma^\prime,\delta^\prime)$ cannot
diverge, if the evaluation for a more general state $(\sigma,\delta)$ doesn't diverge:

\begin{lemma}
\label{three}
\normalfont
Let 

$$
\otrans{\Gamma,\iota}{(\sigma,\,\delta)}{g}{S}
$$ 

\noindent be a well-formed semantic statement, \mbox{$(\iota^\prime,\sigma^\prime,\delta^\prime)$} be a coherent triplet,
and let $\tau$ be a substitution, such that \mbox{$\iota^\prime(x) \sigma^\prime = \iota(x) \sigma \tau$} for any syntactic
variable $x$. Then

$$
\otrans{\Gamma,\iota^\prime}{(\sigma^\prime,\,\delta^\prime)}{g}{S^\prime}
$$

\noindent is well-formed and its derivation height is not greater than that for the first statement.
\end{lemma}

The proof is by induction on the derivation tree for the first statement. We need to generalize the statement of the lemma, adding the requirement that 
for any substitution $s^\prime_r$ in the RHS of the second statement, there has to be a substitution $s_r$ in the RHS of the first statement,
such that there exists a substitution $\tau_r$, such that \mbox{$\iota^\prime(x) \sigma^\prime_r = \iota(x) \sigma_r \tau_r$} for any syntactic variable $x$. 
In the cases of $\textsc{Fresh}$ and $\textsc{Invoke}$ rules, some semantic variables can become non-observable, and we need to define a substitution $\tau_r$ 
separately for these ``forgotten'' variables and those, which remain observable, using Lemma~\ref{two}.

Now we are ready to claim and prove the divergence criterion.

\setcounter{theorem}{0}
\begin{theorem}[Divergence criterion]
\label{criterion}
\normalfont
For any well-formed semantic statement 

$$
\otrans{\Gamma,\iota}{(\sigma,\,\delta)}{r^k\,t_1\dots t_k}{S}
$$ 

if its proper derivation subtree has a semantic statement 

$$
\otrans{\Gamma,\iota^\prime}{(\sigma^\prime,\,\delta^\prime)}{r^k\,t^\prime_1\dots t^\prime_k}{S^\prime}
$$

then \mbox{$\overline{t^\prime_i \iota^\prime \sigma^\prime} \not \succeq \overline{t^{\phantom{\prime}}_i \iota \sigma}$}. 
\end{theorem}
\begin{proof}
Assume that \mbox{$\overline{t^\prime_i \iota^\prime \sigma^\prime}\succeq \overline{t^{\phantom{\prime}}_i \iota \sigma}$}. 

By Lemma~\ref{one}, the semantic statement

$$
\otrans{\Gamma,\iota^\prime}{(\sigma^\prime,\,\delta^\prime)}{r^k\,t^\prime_1\dots t^\prime_k}{S^\prime}
$$

\noindent is well-formed.

By Lemma~\ref{three}, the derivation tree for

$$
\otrans{\Gamma,\iota^\prime}{(\sigma^\prime,\,\delta^\prime)}{r^k\,t^\prime_1\dots t^\prime_k}{S^\prime}
$$

\noindent has greater or equal height than that for

$$
\otrans{\Gamma,\iota}{(\sigma,\,\delta)}{r^k\,t_1\dots t_k}{S}
$$ 

\noindent which contradicts the theorem condition.

\end{proof}

The theorem justifies that, indeed, our test constitutes a sufficient condition for a divergence: if the execution
reaches a relation call with more general arguments, than those of some enclosing one, then it has no derivation
in our semantics, and, thus, it is not terminating.

\setarrow{\xRightarrow}
\setsubarrow{_e}
\begin{figure*}
\begin{minipage}[t]{\textwidth}
\small
\[
\cotrans{\Gamma,\,\iota,\,h}{(\sigma,\,\delta)}{t_1\equiv t_2}{\emptyset}{mgu\,(t_1\iota\sigma,\,t_2\iota\sigma) = \bot}\ruleno{UnifyFail$^+$}
\]
\[
\cotrans{\Gamma,\,\iota,\,h}{(\sigma,\,\delta)}{t_1\equiv t_2}{(\sigma\circ\Delta,\,\delta)}{mgu\,(t_1\iota\sigma,\,t_2\iota\sigma) = \Delta\ne\bot}\ruleno{UnifySuccess$^+$}
\]
\[
\trule{\otrans{\Gamma,\,\iota,\,h}{(\sigma,\,\delta)}{g_1}{S_1};\quad
       \otrans{\Gamma,\,\iota,\,h}{(\sigma,\,\delta)}{g_2}{S_2}
      }
      {\otrans{\Gamma,\,\iota,\,h}{(\sigma,\,\delta)}{g_1\vee g_2}{S_1\cup S_2}}\ruleno{Disj$^+$}
\]
\[
\crule{\otrans{\Gamma,\,\iota[x\gets\alpha],\,h}{(\sigma,\,\delta\cup\{\alpha\})}{g}{S^\dagger}}
      {\otrans{\Gamma,\,\iota,\,h}{(\sigma,\,\delta)}{\lstinline|fresh($x$) $\;g$|}{S^\dagger}}
      {\alpha\in\meta{W}\setminus\delta}\ruleno{Fresh$^+$}
\]
\end{minipage}      
\caption{Improved search: inherited rules}
\label{improved-semantics-normal}
\end{figure*}

\begin{figure*}
\begin{minipage}[t]{\textwidth}
\small
\[
   \cotrans{\Gamma,\,\iota,\,h}{(\sigma,\,\delta)}{r^k t_1 \dots t_k}{\dagger}{v_i = t_i \iota \sigma, \; (v_1, \dots, v_k) \succeq h\,r^k}
   \ruleno{InvokeDiv$^+$}
\]

\[
\crule{\otrans{\Gamma,\,\epsilon[x_i\gets v_i],\,h[r^k\gets(v_1, \dots, v_k)]}{(\epsilon,\,\delta)}{g}{\bigcup_j\{(\sigma_j,\,\delta_j)\}}}
      {\otrans{\Gamma,\,\iota,\,h}{(\sigma,\,\delta)}{r^k t_1 \dots t_k}{\bigcup_j\{(\sigma\circ\sigma_j, \delta_j)\}}}
      {v_i=t_i\iota\sigma,\;\Gamma\,r^k=\lambda x_1 \dots x_k. g,\; (v_1, \dots, v_k) \nsucceq h\,r^k}
      \ruleno{Invoke$^+$}
\]
\end{minipage}      
\caption{Improved search: invocation and divergence detection}
\label{improved-semantics-invoke}
\end{figure*}

\begin{figure*}
\begin{minipage}[t]{\textwidth}
\small
\[
\trule{\otrans{\Gamma,\,\iota,\,h}{(\sigma,\,\delta)}{g_1}{\dagger}}
      {\otrans{\Gamma,\,\iota,\,h}{(\sigma,\,\delta)}{g_1\vee g_2}{\dagger}}\ruleno{DivDisjLeft$^+$}
\]
\[
\trule{\otrans{\Gamma,\,\iota,\,h}{(\sigma,\,\delta)}{g_2}{\dagger}}
      {\otrans{\Gamma,\,\iota,\,h}{(\sigma,\,\delta)}{g_1\vee g_2}{\dagger}}\ruleno{DivDisjRight$^+$}
\]
\[
\crule{\otrans{\Gamma,\,\epsilon[x_i\gets v_i],\,h[r^k\gets(v_1, \dots, v_k)]}{(\epsilon,\,\delta)}{g}{\dagger}}
      {\otrans{\Gamma,\,\iota,\,h}{(\sigma,\,\delta)}{r^k t_1 \dots t_k}{\dagger}}
      {v_i=t_i\iota\sigma,\;\Gamma\,r^k=\lambda x_1 \dots x_k. g,\; (v_1, \dots, v_k) \nsucceq h\,r^k}
      \ruleno{DivInvoke$^+$}
\]      
\end{minipage}      
\caption{Improved search: divergence propagation}
\label{improved-semantics-divergence-prop}
\end{figure*}

\subsection{Conjuncts Reordering}
\label{sec:reordering}

In this section we consider the discipline of conjuncts reordering. Recall, we flatten all nested conjunctions in 
clusters $\wedge g_i$, where none of $g_i$ is a conjunction. To evaluate a cluster, we have to evaluate
its conjuncts one after another, threading the results, starting from the initial substitution. Each time we
evaluate a conjunct, we can have three possible outcomes:

\begin{itemize}
\item The evaluation converges with some result. In this case, we can proceed with the next conjunct.
\item The evaluation diverges undetected. In this case, nothing can be done.
\item A divergence is detected by the test. This is the case when the reordering takes place.
\end{itemize}

In a general case, for each cluster there can be some converging prefix $\omega$ we've managed to evaluate so far (initially empty),
and the rest of the conjuncts $g_i$. Since $\omega$ converges, we have some set of substitutions $S_\omega$ that corresponds to the
result of $\omega$ evaluation.

Suppose none of $g_i$ converges on $S_\omega$ (i.e. for each $g_i$ there is at least one substitution in $S_\omega$, on which
$g_i$ diverges). We claim that reordering conjuncts inside $\omega$ would not help. Indeed, with any other order
of conjuncts, $\omega$ either diverges or converges with the same result (up to the renaming of semantic variables). Thus,
making any permutations inside $\omega$ is superfluous.

Next, suppose we have two different goals $g_1$ and $g_2$, which both converge on $S_\omega$ (i.e. both converge on each
substitution in $S_\omega$). Do we need to try both cases ($g_1$ and $g_2$) to extend the converging prefix?
It is rather easy to see that if, say, $g_2$ converges on $S_\omega$, then it will as well converge on the result of evaluation
of $g_1$ on $S_\omega$. Indeed, for arbitrary \mbox{$(\sigma, \delta)\in S_\omega$} we have

\[
\otrans{\dots}{(\sigma, \delta)}{g_1}{S^\prime_\omega}
\]

where each $\sigma^\prime$ (such that \mbox{$(\sigma^\prime, \delta^\prime)\in S^\prime_\omega$}) is a ``more specific'',
than $\sigma$, by Lemma~\ref{two}. By Lemma~\ref{three}, since $g_2$ converges on \mbox{$(\sigma, \delta)\in S_\omega$},
it converges on each \mbox{$(\sigma^\prime, \delta^\prime)\in S^\prime_\omega$} as well.

In other words, to extend a converging prefix we can choose arbitrary conjunct, which converges immediately
after this prefix, and this choice will never have to be undone.

Now we can specify the reordering discipline. Since we never re-evaluate a converging prefix, we do not
represent it. Thus, each cluster we consider from now on is a suffix of some initial cluster after
evaluation of some converging prefix (and, perhaps, after some reorderings performed so far).

Let us have a cluster \mbox{$\bigwedge_{i=1}^k g_i$}. We evaluate it on some substitution $\sigma$ in the context of some integer
value $p$ (initially $p=1$), which describes, which conjunct we have to try next. We operate as follows:

\begin{enumerate}
\item\label{reorder:top} We try to evaluate $g_p$ on $\sigma$. If the evaluation succeeds with a result $S^\prime$, we 
remove $g_p$ from the cluster and evaluate the rest for each substitution in $S^\prime$ and $p=1$.
  
\item If a divergence is detected, and $p\le k$, then increment $p$, and repeat from step~\ref{reorder:top} (which will try the next goal).
  
\item Otherwise, we give up and rollback to the enclosing cluster (if any).
\end{enumerate}

Thus, we apply a greedy approach: each time we have a converging prefix of conjuncts (possibly empty), and some tail.
We try to put each conjunct from the tail immediately after the prefix. If we find a converging conjunct, we attach
it to the prefix and continue; if no, then the list of conjuncts diverges. Thus, we can find a converging order
(if any) in a quadratic time. Note, for different substitutions in the result of a converging prefix evaluation
the order of remaining conjuncts can be different.

\begin{figure*}
\begin{minipage}[t]{\textwidth}
\small
\[
\trule{\setsubarrow{_r}\otrans{\Gamma,\,\iota,\,h,\,1}{(\sigma,\,\delta)}{\bigwedge\limits_{i=1}^n g_i}{S^\dagger}}
      {\otrans{\Gamma,\,\iota,\,h}{(\sigma,\,\delta)}{\bigwedge\limits_{i=1}^n g_i}{S^\dagger}}
      \ruleno{ClusterStart$^+$}
\]
\vskip3mm
\[
\crule{\otrans{\Gamma,\,\iota,\,h}{(\sigma,\,\delta)}{g_p}{\bigcup_j\{(\sigma_j,\,\delta_j)\}};\quad
       \forall j\;:\;\otrans{\Gamma,\,\iota,\,h}{(\sigma_j,\,\delta_j)}{\bigwedge\limits_{i\ne p}g_i}{S_j}
      }
      {\setsubarrow{_r}\otrans{\Gamma,\,\iota,\,h,\,p}{(\sigma,\,\delta)}{\bigwedge\limits_{i=1}^n g_i}{\bigcup S_j}}
      {1 \le p \le n}
\ruleno{ClusterStep$^+$}
\]
\vskip3mm
\[
\crule{\otrans{\Gamma,\,\iota,\,h}{(\sigma,\,\delta)}{g_p}{\bigcup_j\{(\sigma_j,\,\delta_j)\}};\quad
       \exists j\;:\;\otrans{\Gamma,\,\iota,\,h}{(\sigma_j,\,\delta_j)}{\bigwedge\limits_{i\ne p}g_i}{\dagger}
      }
      {\setsubarrow{_r}\otrans{\Gamma,\,\iota,\,h,\,p}{(\sigma,\,\delta)}{\bigwedge\limits_{i=1}^n g_i}{\dagger}}
      {1 \le p \le n}
\ruleno{ClusterDiv$^+$}
\]
\vskip3mm
\[
\crule{\otrans{\Gamma,\,\iota,\,h}{(\sigma,\,\delta)}{g_p}{\dagger};\quad
       {\setsubarrow{_r}\otrans{\Gamma,\,\iota,\,h,\,p+1}{(\sigma,\,\delta)}{\bigwedge\limits_{i=1}^n g_i}{S^\dagger}}
      }
      {\setsubarrow{_r}\otrans{\Gamma,\,\iota,\,h,\,p}{(\sigma,\,\delta)}{\bigwedge\limits_{i=1}^n g_i}{S^\dagger}}
      {1 \le p \le n}
\ruleno{ClusterNext$^+$}
\]
\vskip3mm
\[
{\setsubarrow{_r}\cotrans{\Gamma,\,\iota,\,h,\,p}{(\sigma,\,\delta)}{\bigwedge\limits_{i=1}^n g_i}{\dagger}{p>n}}
\ruleno{ClusterStop$^+$}
\]
\end{minipage}      
\caption{Improved search: conjuncts reordering}
\label{improved-semantics-reordering}
\end{figure*}

\subsection{Improved Search Semantics}

Here we combine all observations, presented in the preceding subsections~--- the divergence test, conjunct clustering
and reordering,~--- and express the improved search in terms of a big-step operational semantics that is an extension
of the initial one, presented in Section~\ref{language}.

We denote ``$\xRightarrow{}_e$'' the semantic relation for the improved search, and we add another component to the
environment~--- a history $h$,~--- which maps a relational symbol to a list of fully interpreted terms as its arguments.
As we are (sometimes) capable of detecting the divergence, besides a regular set of answers $S$ as a result of evaluation
we can have a divergence signal, which we denote $\dagger$; $S^\dagger$ ranges over both the set of answers $S$ and the divergence
signal $\dagger$.

For the convenience of presentation we split the set of semantic rules into a few groups. The first one is the inherited
rules (see Fig.~\ref{improved-semantics-normal})~--- those, which did not change (except for the extension in the
environment and evaluation result). Note, the rule \rulename{Disj$^+$} does not handle the divergence detection
in either of disjuncts.

The next group describes the invocation and divergence detection (see Fig.~\ref{improved-semantics-invoke}). On
relation invocation, we first consult with the history. If the history indicates that the invocation is performed in the
context of the same relation evaluation with more specific arguments, then we raise the divergence signal; otherwise
we perform normally. Note, the rule \rulename{Invoke$^+$} does not handle the divergence in the \emph{body} of
invoked relation.

The next group describes the divergence signal propagation (see Fig.~\ref{improved-semantics-divergence-prop}). Here
the divergence signal, raised in one of the disjuncts or in the body of relational definition, is propagated to the upper
levels of the derivation tree.

The final group handles the conjunct reordering (see Fig.~\ref{improved-semantics-reordering}). As we need a reordering
parameter $p$ (see Section~\ref{sec:reordering}), we introduce another relation ``$\xRightarrow{}_r$'' with environment,
enriched by $p$.

The rule \rulename{ClusterStart$^+$} describes the case, when we make an attempt to evaluate a cluster. It can happen, when
we either first encounter an original cluster or try to evaluate a suffix of some initial cluster past some converging
prefix. As the reordering starts now, we recurse to the reordering relation with the parameter \mbox{$p=1$} (which means,
that the first conjunct will be tried to evaluate next).

Two next rules describe the case, when the $p$-th conjunct, being tried to evaluate, succeeds with some result. In the rule
\rulename{ClusterStep$^+$} we handle the case, when all other conjuncts can be evaluated in the context of that result: we
combine the outcomes, which completes the evaluation of the whole cluster. In the rule \rulename{ClusterDiv$^+$} we consider
the opposite case: now there is some conjunct $g_j$, which raises a divergence signal, being evaluated in the context of
the results, delivered by the evaluation of $g_p$. As we argued in Section~\ref{sec:reordering}, nothing can be done, and we
propagate the divergence signal.

The rule \rulename{ClusterNext$^+$} describes the case, when the $p$-th conjunct raises the divergence signal, and there are
some other conjuncts to try. We increment $p$ and proceed.

Finally, in the rule \rulename{ClusterStop$^+$} we handle the situation, when all available conjuncts in a cluster were tried to
evaluate first and raised the divergence signal. We propagate the signal in this case.

The following theorem is rather easy to prove:

\begin{theorem} For arbitrary $\Gamma$ and $g$ if

  \[{\setsubarrow{}\otrans{\Gamma,\,\bot}{(\epsilon,\,\emptyset)}{g}{S}}\]

  then
  
  \[{\setsubarrow{_e}\otrans{\Gamma,\,\bot,\,\bot}{(\epsilon,\,\emptyset)}{g}{S}}\]

\end{theorem}

Indeed, due to Theorem~\ref{criterion}, from the condition we can conclude that the divergence signal is
never raised during the evaluation, according to ``$\xRightarrow{}_e$''; but in this case the evaluation
steps coincide with those, according to ``$\xRightarrow{}$''. Thus, the improved search preserves the convergence.

\section{Evaluation}

\label{sec:evaluation}

In this section, we present an evaluation of 
implemented constructive negation on a series of examples.

\subsection{If-then-else}

Using relational if-then-else operator, 
presented in section~\ref{sec:ifte},
we have implemented several 
higher-order relations over lists, namely 
\lstinline{find} (Listing~\ref{lst:eval-find}), 
\lstinline{remove}\footnote{Note, this implementation 
differs from the one in Section~\ref{sec:intro}, but 
it is easy to see that these two are semantically equivalent.} (Listing~\ref{lst:eval-remove}) 
and \lstinline{filter} (Listing~\ref{lst:eval-filter}).
These relations are almost identical (syntactically) to their
functional implementations.
We have tested that these relations can be run
in various directions and produce the expected results.
For example, the goal \lstinline{filter p q q}
with the predicate \lstinline{p} equal to

\begin{lstlisting}
  fun l -> fresh (x) (l === [x])
\end{lstlisting}

stating that the given list should be a singleton list,
starts to generate all singleton lists.
Vice versa, the goal \lstinline{filter p q []} 
with that same \lstinline{p} generates 
all lists, constrained to be not a singleton list.

Listings~\ref{lst:eval-p}-\ref{lst:eval-filter-queries} give 
more concrete examples of queries to these relations.
In the listing the syntax \lstinline{run n q g}
means running a goal \lstinline{g} with 
the free variable \lstinline{q}
taking the first \lstinline{n} answers (``\lstinline{*}'' denotes all answers).
After the sign $\leadsto$ the result of the query is given.
The result \lstinline{fail} means that the query has failed.
The result \lstinline[mathescape]|succ {{a$_1$}; ... {a$_n$}} |
means that the query has succeeded delivering $n$ answers.
Each answer represents a set of constraint on free variables.
Constraints are of two forms: equality constraints, e.g. \lstinline{q = (1, _.$_0$)}, 
or disequality constraints, e.g. \lstinline{q $\neq$ (1, _.$_0$)}.
The terms of the form \lstinline{_.$_i$} in the answer
denote some universally quantified variables.

\begin{minipage}[thb]{.3\textwidth}
\begin{lstlisting}[
  caption={A definition of \code{find} relation},
  label={lst:eval-find}
]
let find p e xs =
  fresh (x xs' ys') (
    xs === x::xs' /\
    ifte (p x)
      (e === x)
      (find p e xs')
  )
\end{lstlisting}
\end{minipage}\hfill
\begin{minipage}[thb]{.3\textwidth}
\begin{lstlisting}[
  caption={A definition of \code{remove} relation},
  label={lst:eval-remove}
]
let remove p xs ys =
  (xs === [] /\ ys === [])
  \/
  fresh (x xs' ys') (
    xs === x::xs' /\
    ifte (p x)
      (ys === xs')
      (ys === x::ys' /\ 
       remove p xs' ys')
  )
\end{lstlisting}
\end{minipage}\hfill
\begin{minipage}[thb]{.3\textwidth}
\begin{lstlisting}[
  caption={A definition of \code{filter} relation},
  label={lst:eval-filter}
]
let filter p xs ys =
  (xs === [] /\ ys === [])
  \/
  fresh (x xs' ys') (
    xs === x::xs' /\
    (ifte (p x)
      (ys === x :: ys')
      (ys === ys')) /\
    filter p xs' ys'
  )
\end{lstlisting}
\end{minipage}

% \vspace{3cm}

\begin{minipage}[thb]{0.4\textwidth}
\begin{lstlisting}[
  caption={Definition of the predicate \lstinline{p}},
  label={lst:eval-p}
]
let p l = fresh (x) (l === [x])
\end{lstlisting}
\begin{lstlisting}[
  caption={Example of queries to \lstinline{find}},
  label={lst:eval-find-queries}
]
run 3 q (fresh (e) find p e q) 
$\leadsto$ succ {
     { q = [_.$_0$] :: _.$_1$ }
     { q = _.$_0$ :: [_.$_1$] :: _.$_2$; 
         _.$_0$ $\neq$ [_.$_3$] }
     { q = _.$_0$ :: _.$_1$ :: [_.$_2$] :: _.$_3$; 
         _.$_0$ $\neq$ [_.$_4$]; _.$_1$ $\neq$ [_.$_5$] }
   }
\end{lstlisting}
\end{minipage}\hfill
\begin{minipage}[thb]{0.4\textwidth}
\begin{lstlisting}[
  caption={Example of queries to \lstinline{remove}},
  label={lst:eval-remove-queries}
]
run * q (fresh (e) remove p q [[ ]]) 
$\leadsto$ succ {
     { q = [[_.$_0$]; [ ]] }
     { q = [[ ]] }
     { q = [[ ]; [_.$_0$]] }
   }

run 3 q (fresh (e) remove p q q) 
$\leadsto$ succ {
     { q = [] }
     { q = [_.$_0$], _.$_0$ $\neq$ [_.$_1$] }
     { q = [_.$_0$; _.$_1$]; 
         _.$_0$ $\neq$ [_.$_2$]; _.$_1$ $\neq$ [_.$_3$] }
   }
\end{lstlisting}
\end{minipage}

\begin{minipage}[thb]{0.4\textwidth}
\begin{lstlisting}[
  caption={Example of queries to \lstinline{filter}},
  label={lst:eval-filter-queries}
]
run 3 q (filter p q q) 
$\leadsto$ succ {
     { q = [ ] }
     { q = [_.$_0$] }
     { q = [_.$_0$; _.$_1$] }
   }

run 3 q (filter p q [1]) 
$\leadsto$ succ {
     { q = [[1]] }
     { q = [_.$_0$; [1]]; _.$_0$ $\neq$ [_.$_1$] }
     { q = [[1]; _.$_0$]; _.$_0$ $\neq$ [_.$_1$] }
   }

run 3 q (filter p q [ ]) 
$\leadsto$ succ {
     { q = [] }
     { q = [_.$_0$]; _.$_0$ $\neq$ [_.$_1$] }
     { q = [_.$_0$; _.$_1$]; 
            _.$_0$ $\neq$ [_.$_2$]; _.$_1$ $\neq$ [_.$_3$] }
   }
\end{lstlisting}
\end{minipage}

\subsection{Universal quantification}

In the Section~\ref{sec:impl-univ} we presented 
the \lstinline{forall} goal constructor 
which is implemented through the double negation.
We have observed, that although \lstinline{forall g}
does not terminate when the goal \lstinline{g x} 
has an infinite number of answers 
(assuming \lstinline{x} is a fresh variable),
it does terminate in the case when \lstinline{g x} has 
a finite number of answers.
The behavior of \lstinline{forall} in this case is sound
even in the presence of disequality constraints or nested quantifiers. 

The Table~\ref{tab:univ} gives some concrete examples.
The left column contains the tested goals\footnote{
We typeset the goals in terms of first-order logic syntax 
instead of \textsc{OCanren} syntax for brevity and clarity.} 
and the right column gives the obtained results.
For the results we use the same notation 
as in the previous section.

\begin{table}[th]
  \centering
  \def\arraystretch{1.5}
  \begin{tabularx}{\textwidth}{|X|X|}
    \hline

    $\forall x\ldotp x = q$ & 
      \texttt{fail} \\
    \hline

    $\forall x\ldotp \exists y\ldotp x = y$ & 
      \texttt{succ \{[q = \_.$_0$]\}} \\
    \hline

    $\forall x\ldotp \exists y\ldotp x = y \wedge y = q$ &
      \texttt{fail} \\
    \hline

    $\forall x\ldotp q = (1, x)$ & 
      \texttt{fail} \\
    \hline

    $\forall x\ldotp \exists y\ldotp y = (1, x)$ & 
      \texttt{succ \{[q = \_.$_0$]\}} \\
    \hline

    $\forall x\ldotp \exists y\ldotp x = (1, y)$ &
      \texttt{fail} \\
    \hline

    $\forall x\ldotp x \neq q$ & \texttt{fail} \\
    \hline

    $\forall x\ldotp \exists y\ldotp x \neq y$ & 
      \texttt{succ \{[q = \_.$_0$]\}} \\
    \hline

    $\forall x\ldotp \exists y\ldotp x \neq y \wedge y = q$ & 
      \texttt{fail} \\
    \hline

    $\forall x\ldotp q \neq (1, x)$ & 
      \texttt{succ \{[q $\neq$ (1, \_.$_0$)]\}} \\
    \hline

    $(\exists x\ldotp q = (1, x)) \wedge (\forall x\ldotp q \neq (1, x))$ & 
      \texttt{fail} \\
    \hline

    $\forall x\ldotp (x, x) \neq (0, 1)$ & 
      \texttt{succ \{[q = \_.$_0$]\}} \\
    \hline

    $\forall x\ldotp (x, x) \neq (1, 1)$ & 
      \texttt{fail} \\
    \hline

    $\forall x\ldotp (x, x) \neq (q, 1)$ & 
      \texttt{succ \{[q $\neq$ 1]\}} \\
    \hline

    $\exists a~ b\ldotp q = (a, b) \wedge \forall x\ldotp (x, x) \neq (a, b)$ & 
      \texttt{succ \{[q = (\_.$_0$, \_.$_1$); \_.$_0$ $\neq$ \_.$_1$]\}} \\
    \hline

  \end{tabularx}
  \caption{\lstinline{forall} evaluation}
  \label{tab:univ}
\end{table}

\section{Related works}
\label{sec:related}

There are a few different approaches for compiling pattern matching. GHC is using influential paper~\cite{Jones1987}, OCaml is currently based on~\cite{maranget2001} although a work~\cite{maranget2008} can slightly improve effectiveness of generated code. 

Although semantics of pattern matching can be given as a sequence of scrutinee's sub expression comparisons (Figure~\ref{fig:matchpatts}) effective compilers don't follow this approach. One can either optimise run time cost by minimizing amount of checks performed or static cost by minimizing the size of generated code. \emph{Decision trees} are good for the first criteria, because they check every subexpression not more than once. \emph{Backtracking automata} are rather compact but in some cases can perform repeated checks.


Minimizing the size of decision tree is  NP-hard (\cite{baudinet1985tree}, without proof) and usually heuristics are applied during compilation, for example: count of nodes, length of the longest path, average length of all paths. The paper~\cite{Scott2000WhenDM} performs experimental evaluation of 9  heuristics on the base of for Standard ML of New Jersey.


The matching compilers for strict languages can work in \emph{direct} or \emph{indirect} styles. The first ones return effective code immediately. In the second style to construct final answer some post processing is required. It can vary from easy simplifications to complicated supercompilation techniques~\cite{sestoft1996}. The main drawback of indirect style is that size of intermediate data structures can be exponentially large.

For strict languages checking  sub expressions of scrutinee in any order is allowed. For lazy languages pattern matching should evaluate only these sub expressions which are necessary for performing pattern matching. If not careful pattern matching can change termination behavior of the program.  In general lazy languages setup more constraints on pattern matching and because of that allow lesser set of heuristics.

A few approaches for checking sub expressions in lazy languages has been proposed~\cite{augustsson1985,laville1991}. \cite{laville1991} models values in lazy languages using \emph{partial terms}, although this approach doesn't scale to types with infinite constructor sets (like integers). In  the \cite{suarez1993} the similar approach is extended by special treatment of overlapping patterns. Pattern matching has been compiled to decision trees~\cite{maranget1992} and later \cite{maranget1992} into \emph{decision DAGs} that allow in some cases to make code smaller.

The first works on compilation to backtracking automaton are~\cite{augustsson1985,wadler1987}. 

The inefficiency of backtracking automaton has been improved in~\cite{maranget2001}. The approach utilizes matrix representation for pattern matching. It splits current matrix  according to constructors in the first column and reduces the task to compiling matrices with less rows. The technique is indirect, in the end a few optimizations are performed by introducing special \emph{exit} nodes to the compiled representation.
No preprocessing is required for this scheme: or-pattern receive a special treatment during compilation process.
 The approach from this paper is used in current implementation of OCaml compiler.

Previous approach uses first column to split the matrix. In~\cite{maranget2008} has been introduced \emph{necessity} heuristic that recommends which column should be used to perform split. Good decision trees that are constructed in this work can perform better in corner cases than~\cite{maranget2001} but for practical cases the difference is insignificant.

To summarize, compilers can try to optimize pattern matching either for guaranteed code speed or for guaranteed code size. There are distinct techniques to minimize drawbacks of both approaches.


\section{Future Work}

There are a few possible directions for future work. First, in this paper we did not address the performance issues. As we represent
the transformations in a very generic form with many levels of indirection, obviously, the transformations, implemented with
our framework, are at disadvantage in comparison with hard coded ones in terms of performance. We assume that the performance of transformations
can be essentially improved by applying some techniques like staging~\cite{Staged} or, perhaps, object-specific optimisations.

Another important direction is supporting more kinds of type declarations, in the first hand, GADTs and non-regular types. Although we have some
implementation ideas for this case, the solution we came up with so far makes the interface of the whole framework too cumbersome to use even for
simple cases.

Finally, the typeinfo structure we generate can be used to mimic the \emph{ad-hoc} polymorphism as it contains the implementation of
type-indexed functions. This, together with some proposed extensions~\cite{ModularImplicits}, can open interesting perspectives.



\bibliographystyle{ACM-Reference-Format}
\bibliography{main}

\end{document}
