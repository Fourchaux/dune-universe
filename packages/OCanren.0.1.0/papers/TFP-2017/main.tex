\documentclass{llncs}

\usepackage{makeidx}
\usepackage{amssymb}
\usepackage{listings}
\usepackage{indentfirst}
\usepackage{verbatim}
\usepackage{amsmath}
\usepackage{graphicx}
\usepackage{xcolor}
\usepackage{url}
\usepackage{stmaryrd}
\usepackage{xspace}
\usepackage{comment}
\usepackage{wrapfig}
\usepackage{placeins}
\usepackage{tabularx}
\usepackage{ragged2e}
\usepackage{subcaption}
\captionsetup{compatibility=false}
%\usepackage{natbib}
%\usepackage [sorting = none] {biblatex}
%\addbibresource {main.bib}

\def\transarrow{\xrightarrow}
\newcommand{\setarrow}[1]{\def\transarrow{#1}}

\newcommand{\trule}[2]{\frac{#1}{#2}}
\newcommand{\crule}[3]{\frac{#1}{#2},\;{#3}}
\newcommand{\withenv}[2]{{#1}\vdash{#2}}
\newcommand{\trans}[3]{{#1}\transarrow{#2}{#3}}
\newcommand{\ctrans}[4]{{#1}\transarrow{#2}{#3},\;{#4}}
\newcommand{\llang}[1]{\mbox{\lstinline[mathescape]|#1|}}
\newcommand{\pair}[2]{\inbr{{#1}\mid{#2}}}
\newcommand{\inbr}[1]{\left<{#1}\right>}
\newcommand{\highlight}[1]{\color{red}{#1}}
\newcommand{\ruleno}[1]{\eqno[\scriptsize\textsc{#1}]}
\newcommand{\inmath}[1]{\mbox{$#1$}}
\newcommand{\lfp}[1]{fix_{#1}}
\newcommand{\gfp}[1]{Fix_{#1}}
\newcommand{\vsep}{\vspace{-2mm}}
\newcommand{\supp}[1]{\scriptsize{#1}}
\newcommand{\G}{\mathfrak G}
\newcommand{\sembr}[1]{\llbracket{#1}\rrbracket}
\newcommand{\cd}[1]{\texttt{#1}}
\newcommand{\miniKanren}{miniKanren\xspace}
\newcommand{\ocanren}{OCanren\xspace}
\newcommand{\free}[1]{\boxed{#1}}
\newcommand{\binds}{\;\mapsto\;}
\newcommand{\dbi}[1]{\mbox{\bf{#1}}}

\let\emptyset\varnothing
\setlength{\abovecaptionskip}{3pt plus 3pt minus 2pt}
\setlength{\belowcaptionskip}{-10pt plus 3pt minus 2pt}

\lstdefinelanguage{ocanren}{
keywords={fresh, let, in, match, with, when, class, type,
object, method, of, rec, repeat, until, while, not, do, done, as, val, inherit,
new, module, sig, deriving, datatype, struct, if, then, else, open, private, virtual, include, success, failure,
true, false},
sensitive=true,
commentstyle=\small\itshape\ttfamily,
keywordstyle=\ttfamily\underbar,
identifierstyle=\ttfamily,
basewidth={0.5em,0.5em},
columns=fixed,
fontadjust=true,
literate={fun}{{$\lambda$}}1 {->}{{$\to$}}3 {===}{{$\equiv$}}1 {=/=}{{$\not\equiv$}}1 {|>}{{$\triangleright$}}3 {|||}{{$\vee$}}2 {/\\}{{$\wedge$}}2 {^}{{$\uparrow$}}1,
morecomment=[s]{(*}{*)}
}

\lstset{
mathescape=true,
%basicstyle=\small,
identifierstyle=\ttfamily,
keywordstyle=\bfseries,
commentstyle=\scriptsize\rmfamily,
basewidth={0.5em,0.5em},
fontadjust=true,
language=ocanren
}

\usepackage{letltxmacro}
\newcommand*{\SavedLstInline}{}
\LetLtxMacro\SavedLstInline\lstinline
\DeclareRobustCommand*{\lstinline}{%
  \ifmmode
    \let\SavedBGroup\bgroup
    \def\bgroup{%
      \let\bgroup\SavedBGroup
      \hbox\bgroup
    }%
  \fi
  \SavedLstInline
}
%\addtolength{\parskip}{-2pt}
%\setlength{\parskip}{0pt}
\setlength{\belowcaptionskip}{-15pt}

\pagestyle{plain}
\begin{document}
\sloppy
\mainmatter

\title{Typed Relational Conversion\thanks{This work is supported by RFBR grant No 18-01-00380.}}

\author{
  Petr Lozov\inst{1} \and Andrei Vyatkin\inst{2} \and Dmitry Boulytchev\inst{3}
}

\institute{
St.Petersburg State University,\\
Universitetski pr., 28, 198504, St.Petersburg, Russia,\\
\email{lozov.peter@gmail.com}
\and
St.Petersburg State University,\\
Universitetski pr., 28, 198504, St.Petersburg, Russia,\\
\email{dewshick@gmail.com}
\and
St.Petersburg State University,\\
Universitetski pr., 28, 198504, St.Petersburg, Russia,\\
JetBrains Research\\
Universitetskaya emb., 7-9-11, bldg. 5A, 199034, St.Petersburg, Russia,\\
\email{dboulytchev@math.spbu.ru}}

\maketitle

\begin{abstract}
We address the problem of transforming typed functional programs into relational form. 
In this form, a program can be run in various ``directions'' with some arguments left free, 
making it possible to acquire different behaviors from a single specification. We specify the 
syntax, typing rules and semantics for the source language as well as its relational extension, 
describe the conversion and prove its correctness both in terms of typing and dynamic semantics. 
We also discuss the limitations of our approach, present the implementation of the conversion for 
the subset of OCaml and evaluate it on a number of realistic examples.
\end{abstract}

\section{Introduction}
\label{sec:intro}

Algebraic data types (ADT) are an important tool in functional programming which deliver a way to represent flexible and easy to manipulate data structures.
To inspect the contents of an ADT's values a generic construct~--- \emph{pattern matching}~--- is used. The importance of pattern matching efficient
implementation stimulated the development of various advanced techniques which provide good results in practice. The objective of our work is to use these
results as a baseline for a case study of relational synthesis\footnote{We have to note that this term is overloaded and can be used to refer to completely
different approaches than we utilize.}~--- an approach for program synthesis based on application of relational programming~\cite{TRS,WillThesis}, and,
in particular, relational interpreters~\cite{unified} and relational conversion~\cite{conversion}. Relational programming can be considered as a specific form
of constraint logic programming centered around \textsc{miniKanren}\footnote{\url{http://minikanren.org}}, a combinator-based DSL, implemented for a number of host languages.
Unlike \textsc{Prolog}, which employs a deterministic depth-first search, \textsc{miniKanren} advocates a 
%completely 
more
declarative approach, in which a user is not
allowed to rely on a concrete search discipline, which means, that the specifications, written in \textsc{miniKanren}, are understood much more symmetrically.
The distinctive feature of \textsc{miniKanren} is complete \emph{interleaving search}~\cite{search}. The basic constraint is unification with occurs check, although
advanced implementations support other primitive constructs, such as disequality or finite-domain constraints~\cite{CKanren}. Syntactically, \textsc{miniKanren} is mutually
convertible to \textsc{Prolog}, but, unlike latter, makes use of explicit logical connectives (conjunction and disjunction), existential quantification and unification.
 
A distinctive application of relational programming is implementing \emph{relational interpreters}~\cite{Untagged}. Unlike conventional interpreters, which for a program and
input value produce output, relational interpreters can operate in various directions: for example, they are capable of computing an input value for a given
program and a given output, or even synthesize a program for a given pairs of input-output values. The latter case forms a basis for program synthesis~\cite{eigen,unified}.

Our approach is based on relational representation of the source language pattern matching semantics on the one hand, and
the semantics of the intermediate-level implementation language on the other. We formulate the condition necessary for a correct and complete implementation of pattern matching and use it to
construct a top-level goal which represents a search procedure for all correct and complete implementations. We also present a number of techniques which make it possible to come up with an
\emph{optimal} solution as well as optimizations to improve the performance of the search. Similarly to many other prior works we use the size of the synthesized code, which can be measured
statically, to distinguish better programs. Our implementation\footnote{\url{https://github.com/Kakadu/pat-match/tree/aplas2020}} makes use of \textsc{OCanren}\footnote{\url{https://github.com/JetBrains-Research/OCanren}}~---
 a typed implementation of \textsc{miniKanren} for \textsc{OCaml}~\cite{OCanren}, and \textsc{noCanren}\footnote{\url{https://github.com/Lozov-Petr/noCanren}}~--- 
a converter from the subset of plain \textsc{OCaml} into \textsc{OCanren}~\cite{conversion}. An initial  evaluation, performed for a set of benchmarks taken from other papers, showed our synthesizer performing well.
However, being aware of some pitfalls of our approach, we came up with a set of counterexamples on which it did not provide any results in observable time, so we do not consider the problem
completely solved. We also started to work on mechanized 
formalization\footnote{\url{https://github.com/dboulytchev/Coq-matching-workout}},
written in \textsc{Coq}~\cite{Coq}, to make the justification of our approach more solid and easier to verify, but this formalization is not yet complete. 

 

\begin{comment}
We apply relational programming techniques to the problem of synthesizing efficient implementation for a pattern matching construct.
Although in principle pattern matching can be implemented in a trivial way, the result suffers from inefficiency in terms of both
performance and code size. Thus, in implementing functional languages alternative, more elaborate  approaches are widely used.
However, as there are multiple kinds and flavors of pattern matching constructs, these approaches have to be specifically developed
and justified for each concrete inhabitant of the pattern matching ``zoo''. We formulate the pattern matching synthesis problem in
declarative terms and apply relational programming, a specific form of constraint logic programming, to develop a 
develop optimizations which improve the efficiency of the synthesis and guarantee the
optimality of the result. 
\end{comment}

\documentclass[10pt, oneside, nocopyrightspace]{sigplanconf}

\usepackage{amssymb}
\usepackage{listings}
\usepackage{indentfirst}
\usepackage{verbatim}
\usepackage{amsmath, amsthm, amssymb}
\usepackage{graphicx}
\usepackage[hyphens]{url}
\usepackage[hidelinks]{hyperref}

\topmargin 2.0cm
\setlength{\textheight}{25.3cm}

\lstdefinelanguage{ocaml}{
keywords={fresh, let, begin, end, in, match, type, and, fun, function, try, with, when, class, 
object, method, of, rec, repeat, until, while, not, do, done, as, val, inherit, 
new, module, sig, deriving, datatype, struct, if, then, else, open, private, virtual},
sensitive=true,
basicstyle=\small,
commentstyle=\small\itshape\ttfamily,
keywordstyle=\ttfamily\underbar,
identifierstyle=\ttfamily,
basewidth={0.5em,0.5em},
columns=fixed,
fontadjust=true,
literate={->}{{$\;\;\to\;\;$}}1,
morecomment=[s]{(*}{*)}
}

\lstset{
basicstyle=\small,
identifierstyle=\ttfamily,
keywordstyle=\bfseries,
commentstyle=\scriptsize\rmfamily,
basewidth={0.5em,0.5em},
fontadjust=true,
%escapechar=~,
language=ocaml
}

\sloppy

\newcommand{\miniKanren}{\texttt{miniKanren}}

\begin{document}

\title{Typed Embedding of a Relational Language in OCaml}

\authorinfo{Dmitry Kosarev \and Dmitri Boulytchev}
{St.Petersburg State University \\ 
  Saint-Petersburg, Russia }
{$\mathtt{Dmitrii.Kosarev@protonmail.ch}$ \and $\mathtt{dboulytchev@math.spbu.ru}$}

\maketitle

\small
\begin{abstract}
\small
We present an implementation of the relational programming language miniKanren as a set 
of combinators and syntax extension for OCaml. The key feature of our approach is 
\emph{polymorphic unification}, which can be used to unify data structures of almost 
arbitrary types. In addition we provide a useful generic programming pattern to 
systematically develop relational specifications in a typed manner, and address 
the problem of relational and functional code integration.
\end{abstract}

\section{Introduction}
\label{intro}

Relational programming~\cite{TRS} is an attractive technique, based on the idea 
of constructing programs as relations.  As a result, relational programs can be
``queried'' in various ``directions'', making it possible, for example, to simulate
reversed execution. Apart from being interesting from purely theoretical standpoint, 
this approach may have a practical value: some problems look much simpler, 
if they are considered as queries to relational specification. There is a 
number of appealing examples, confirming this observation: a type checker 
for simply typed lambda calculus (and, at the same time, type inferencer and solver 
for the inhabitation problem), an interpreter (capable of generating ``quines''~--- 
programs, producing themselves as result)~\cite{Untagged}, list sorting (capable of 
producing all permutations), etc. 

Many logic programming languages, such as Prolog, Mercury\footnote{\url{https://mercurylang.org}}, 
or Curry\footnote{\url{http://www-ps.informatik.uni-kiel.de/currywiki}} to some extent
can be considered as relational. We have chosen miniKanren\footnote{\url{http://minikanren.org}} 
as model language, because it was specifically designed as relational DSL, embedded in Scheme/Racket. 
Being rather a minimalistic language, which can be implemented with just a few data structures and
combinators, miniKanren found its way in dozens of host languages, including Haskell, 
Standard ML, and OCaml.

There is, however, a predictable glitch in implementing miniKanren for a strongly typed language. 
Designed in a metaprogramming-friendly and dynamically typed realm of Scheme/Racket, original 
miniKanren implementation pays very little attention to what has a significant importance in (specifically) 
ML or Haskell. In particular, one of capstone constructs of miniKanren~--- unification~--- has to work for 
different data structures, which may have types, different beyond parametricity.

There are a few ways to overcome this problem. The first one is simply to follow the untyped paradigm and
provide unification for some concrete type, rich enough to represent any reasonable data structures.
Some Haskell miniKanren libraries\footnote{\url{https://github.com/JaimieMurdock/HK}, \url{https://github.com/rntz/ukanren}}
as well as existing OCaml implementation\footnote{\url{https://github.com/lightyang/minikanren-ocaml}} take this approach. 
As a result, the original implementation can be retold with all its elegance; relational specifications, however,
become weakly typed. Another approach is to utilize \emph{ad hoc} polymorphism and provide type-specific
unification for each ``interesting'' type; Molog\footnote{\url{https://github.com/acfoltzer/Molog}} and 
MiniKanrenT\footnote{\url{https://github.com/jvranish/MiniKanrenT}}, both for Haskell, can be mentioned as examples.
While preserving strong typing, this approach requires a lot of ``boilerplate'' code to be written, so some
automation, for example, using Template Haskell\footnote{\url{https://wiki.haskell.org/Template_Haskell}},
is desirable. There is, actually, another potential approach, but we do not know, if anybody tried
it: to implement unification for generic representation of types as sum-of-products and fixpoints of 
functors~\cite{InstantGenerics, ALaCarte}. Thus, unification would work for any types, for which representation
is provided. We assume that implementing representation would require less boilerplate code.

As follows from this exposition, typed embedding of miniKanren in OCaml can be done with
a combination of datatype-generic programming~\cite{DGP} and \emph{ad hoc} polymorphism. There are a 
number of generic frameworks for OCaml (for example,~\cite{Deriving}). On the other hand, the support
for \emph{ad hoc} polymorphism in OCaml is weak; there is nothing comparable in power with Haskell 
type classes, and despite sometimes object-oriented layer of the language can be used to mimic
desirable behavior, the result as a rule is far from satisfactory. Existing proposals (for example, 
module implicits~\cite{Implicits}) require patching the compiler, which we tend to avoid.

We present an implementation of a set of relational combinators in OCaml, which, 
technically speaking, corresponds to $\mu$Kanren~\cite{MicroKanren} with disequality 
constraints~\cite{CKanren}; syntax extension for ``\lstinline{fresh}'' construct is
added as well. The contribution of our work is as follows:

\begin{enumerate}
\item Our implementation is based on \emph{polymorphic unification}, which, like polymorphic comparison,
can be used for almost arbitrary types. The implementation of polymorphic unification uses unsafe features and
relies on intrinsic knowledge of runtime representation of values; we show, however, that this does not
compromise type safety. Practically, we applied purely \emph{ad hoc} approach since the features, 
which would provide less \emph{ad hoc} solution, are not yet integrated into the mainstream language.

\item We describe a uniform and scalable pattern for using types for relational programming, which
helps in converting typed data to- and from relational domain. With this pattern, only one
generic feature (``\lstinline{map/morphism/Functor}'') is needed, and thus virtually any generic 
framework for OCaml can be used. Despite being rather a pragmatic observation, this pattern, as we
believe, would lead to  more regular and easy to maintain relational specifications.

\item We provide a simplified way to integrate relational and functional code. Our approach utilizes
well-known pattern~\cite{Unparsing, DoWeNeed} for variadic function implementation and makes it
possible to hide refinement of answers phase from an end-user.
\end{enumerate}

The rest of the paper is organized as follows: in the next section we discuss polymorphic
unification, and show, that standard unification with triangular substitution respects
typing. Then we present our approach to handle user-defined types by injecting them 
into logic domain. Next section describes top-level primitives and addresses the problem of
relational and functional code integration. Then, we present a complete example of relational
specification, written with the aid of our library. The final section concludes.

We expect from reader some familiarity with basic concepts behind original miniKanren 
implementation as well as principles of relational programming.

\section{Polymorphic Unification}
\label{polyuni}

We consider it rather natural to employ polymorphic unification in the
language, already equipped with polymorphic comparison~--- a convenient, but
somewhat disputable\footnote{See, for example, \url{https://blogs.janestreet.com/the-perils-of-polymorphic-compare}} 
feature. Like polymorphic comparison, polymorphic unification performs traversal
of values, exploiting intrinsic knowledge of their runtime representation. 
The undeniable benefit of this solution is that in order to perform unification 
for user types no ``boilerplate'' code is needed. On the other hand, all pitfalls of
polymorphic comparison are inherited as well; in particular, unification can loop 
for cyclic data structures and does not work for functional values. Since we generally 
do not expect any reasonable outcome in these cases, the only remaining problem is that
the compiler is incapable of providing any assistance in identifying 
and avoiding them. Another drawback is that the implementation of polymorphic unification
relies on runtime representation of values and have to be fixed every time the representation changes. 
Finally, as it is written in unsafe manner using \lstinline{Obj} interface, it has to be
carefully developed and tested.

An important difference between polymorphic comparison and unification is that the former 
only inspects its operands, while the results of unification are recorded in a substitution
(mapping from logical variables to terms), which later is used to refine answers and reify 
constraints. So, generally speaking, we have to show, that no ill-typed terms are constructed 
as a result.

Polymorphic unification is introduced via the following function:

\begin{lstlisting}[mathescape=true]
   val unify : $\alpha$ logic -> $\;\;\alpha$ logic -> $\;\;$subst option -> 
     $\;\;$subst option
\end{lstlisting}

\noindent where ``\lstinline[mathescape=true]{$\alpha$ logic}'' stands for the type $\alpha$, 
injected into the logic domain, ``\lstinline{subst}''~--- for the type of substitution. Unification can 
fail (hence ``\lstinline{option}'' in the result type), is performed in the context of
existing substitution (hence ``\lstinline{subst}'' in the third argument) and
can be chained (hence ``\lstinline{option}'' in the third argument). Note, the 
type of substitution is not polymorphic, which means, that compiler completely loses the 
track of types for values, stored in a substitution. These types are recovered later during
refinement of answers.

To justify the correctness of unification, we consider a set of typed terms, each of which
has one of two forms

$$
x^\tau \mid C^\tau(t_1^{\tau_1},\dots,t_k^{\tau_k})
$$

\noindent where $x^\tau$ denotes a logical variable of type $\tau$, 
$C^\tau$~--- some constructor of type $\tau$, $t_i^{\tau_i}$~--- some terms of types $\tau_i$.
We reflect by $t_1^\tau[t_2^\rho]$ the fact of $t_2^\rho$ being a subterm of $t_1^\tau$, and
assume, that $\rho$ is unambiguously determined by $t_1$, $\tau$, and a position of $t_2$ 
``inside'' $t_1$.

Outside unification the compiler maintains typing, which means, that all 
terms, subterms, and variables agree in their types in all contexts. However, as 
our implementation resorts to unsafe features, we have to manually repeat this work for 
unification code.

We argue, that the following three invariants are maintained for any substitution $s$, involved 
in unification:

\begin{enumerate}
\item if \mbox{$t_1^{\_}[x^{\tau}]$} and \mbox{$t_2^{\_}[x^{\rho}]$}~--- two arbitrary terms (in particular, 
$t_1^{\_}$ and $t_2^{\_}$ may be the same), bound in $s$ and containing occurrences of variable $x$, 
then $\rho=\tau$ (different occurrences of the same variable in $s$ are attributed with the same type);

\item if \mbox{$(s\;\;x^\tau)$} is defined, then \mbox{$(s\;\;x^\tau) = t^\tau$} (a substitution always
binds a variable to a term of the same type);

\item each variable in $s$ preserves its type, assigned by the compiler (from the first two invariants 
it follows, that this type is unique; note also, that all variables are created and have their 
types assigned outside unification, in a type-safe world).
\end{enumerate}

The initial (empty) substitution trivially fulfills these invariants; hence, it is sufficient
to show, that they are preserved by unification.

The following snippet presents the implementation of unification with triangular 
substitution in only a little bit more abstract form, than actual code (for example, 
``occurs check'' is omitted):

\begin{lstlisting}[mathescape=true,numbers=left,numberstyle=\tiny,stepnumber=1,numbersep=-5pt]
   let rec walk $s$ = function
   | $x^\tau$ when $x\in dom(s)$ -> $\;\;$walk $s$ $(s\;\;x)^\tau$
   | $t^\tau$ -> $\;\;t^\tau$

   let rec unify $t_1^\tau$ $t_2^\tau$ = function
   | None -> None
   | Some $s$ as $sub$  ->
       match walk $s$ $t_1$, walk $s$ $t_2$ with
       | $x_1^\tau$, $x_2^\tau$ when $x_1$ = $x_2$ -> $\;\;sub$
       | $x_1^\tau$, $(t_2^\prime)^\tau$ -> $\;\;$Some ($s[x_1 \gets t_2^\prime]$)
       | $(t_1^\prime)^\tau$, $x_2^\tau$ -> $\;\;$Some ($s[x_2 \gets t_1^\prime]$)
       | $C^\tau(t_1^{\tau_1},\dots,t_k^{\tau_k})$, $C^\tau(p_1^{\tau_1},\dots,p_k^{\tau_k})$ -> 
           unify $t_k^{\tau_k}$ $p_k^{\tau_k}$(.. (unify $t_1^{\tau_1}$ $p_1^{\tau_1}$ $sub$)$..$)
       | $\_$, $\_$ -> $\;\;$None
\end{lstlisting}

Type annotations, included in the snippet above, can be justified by the following 
reasonings\footnote{We omit verbal description of unification algorithm; 
the details can be found in~\cite{MicroKanren}.}:

\begin{enumerate}
\item Line 2: the type of \mbox{$(s\;\;x^\tau)$} is $\tau$ due to invariant 2; hence, 
the type of \lstinline{walk} result coincides with the type of its second argument (technically,
an induction on the number of recursive invocations of \lstinline{walk} is needed).

\item Line 9: the substitution is left unchanged, hence all invariants are preserved.

\item Line 10 (and, symmetrically, line 11): first, note, that \mbox{$(s\;\;x_1)$} is undefined
(otherwise \lstinline{walk} would not return $x_1$). Then, $x_1$ and $t_2^\prime$ have the
same type, which justifies the preservation of invariant 2. Finally, either \mbox{$x_1=t_1$}
(and, then, $\tau$ is the type of $x_1$, assigned by the compiler), or $x_1$ is retrieved
from $s$ with type $\tau$~--- both cases justify invariants 1 and 3. The same applies to 
the pair $t_2^\prime$ and $t_2$.

\item The previous paragraph justifies the base case for inductive proof on the number of
recursive invocations of \lstinline{unify}.
\end{enumerate}

Function \lstinline{unify} is not directly accessible at the user level; it used
to implement both unification (``\lstinline{===}'') and disequality (``\lstinline{=/=}'') 
goals. The implementation generally follows~\cite{CKanren}.

\section{Logic Variables and Injection}
\label{logics}

Unification, considered in Section~\ref{polyuni}, works for values of type \lstinline[mathescape=true]{$\alpha$ logic}. 
Any value of this type can be seen as either value of type $\alpha$, or logical variable of type $\alpha$. The type 
itself is made abstract, but its values can be uncovered after refinement (see Section~\ref{refinement}).

Free variables solely can be created using ``\lstinline{fresh}'' construct of miniKanren. Note,  
since the unification is implemented in untyped manner, we can not use simple pattern matching to
distinguish logical variables from other logical values. Special attention was paid to implement
variable recognition in constant time.

Apart from variables, other logical values can be obtained by injection; conversely, sometimes
logical value can be projected to a regular one. We supply two functions\footnote{``\lstinline{inj}'' and ``\lstinline{prj}'' in concrete syntax.}
for these purposes

\begin{lstlisting}[mathescape=true]
   val ($\uparrow$) : $\alpha$ -> $\;\;\alpha$ logic
   val ($\downarrow$) : $\alpha$ logic -> $\;\;\alpha$
\end{lstlisting}

As expected, injection is total, while projection is partial. Using these functions and type-specific
``\lstinline{map}'', which can be derived automatically using a number of existing frameworks for
generic programming, one can easily provide injection and projection for user-defined datatypes. We
consider user-defined list type as an example:

\begin{lstlisting}[mathescape=true]
   type ($\alpha$, $\beta$) list = Nil | Cons of $\alpha$ * $\beta$
   
   type $\alpha$ glist = ($\alpha$, $\alpha$ glist) list
   type $\alpha$ llist = ($\alpha$ logic, $\alpha$ llist) list logic

   let rec inj_list l = $\uparrow$(map$_{\mbox{\texttt{list}}}$ ($\uparrow$) inj_list l) 
   let rec prj_list l = map$_{\mbox{\texttt{list}}}$ ($\downarrow$) prj_list ($\downarrow$ l)
\end{lstlisting}

Here ``\lstinline{list}'' is a custom type for lists; note, that it is made more
polymorphic, than usual~--- we abstracted it from itself and made it non-recursive 
(pragmatically speaking, it is desirable to make a type fully abstract, thus logic variables 
can be placed in arbitrary positions).

Then we provided two specialized versions~--- ``\lstinline{glist}'' (``ground'' list), which 
corresponds to regular, non-logic lists, and ``\lstinline{llist}'' (``logical'' list), which
corresponds to logical lists with logical elements. Using a single type-specific function
\lstinline[mathescape=true]{map$_{\mbox{\texttt{list}}}$}, we easily provided injection 
(of type {\lstinline[mathescape=true]{$\alpha$ glist -> $\;\;\alpha$ llist}}) and
projection (of type {\lstinline[mathescape=true]{$\alpha$ llist -> $\;\;\alpha$ glist}}).

In context of these definitions, now we can implement relational list concatenation, 
which is one of first-step examples of miniKanren programming:

\begin{lstlisting}[mathescape=true]
   let rec append$^o$ x y xy =
     conde [
       (x === $\uparrow$ Nil) &&& (xy === y);
       fresh (h t ty)
         (x  === $\uparrow$(Cons (h, t))
         (xy === $\uparrow$(Cons (h, ty))
         (append$^o$ t y ty)
     ]
\end{lstlisting}

Note, in the definition of \lstinline[mathescape=true]{append$^o$} we
used only default injection (``$\uparrow$''). Customized version most likely would 
appear in some top-level goal, for example:

\begin{lstlisting}[mathescape=true]
   (fun q -> append$^o$ (inj_list [1; 2; 3]) 
                    (inj_list [4; 5; 6]) 
                    q
   )
\end{lstlisting}

\section{Refinement and Top-Level Primitives}
\label{refinement}

The result of a relational program is a stream of substitutions, each of which represents
a certain answer. As a rule, a substitution binds many intermediate logical variables, 
created by ``\lstinline{fresh}'' in the course of execution. A meaningful answer has to be
\emph{refined}.

In our implementation refinement is represented by the following function:

\begin{lstlisting}[mathescape=true]
   val refine : subst -> $\alpha$ logic -> $\;\;\alpha$ logic
\end{lstlisting}

This function takes a substitution and a logical value and recursively substitutes
all logical variables in that value w.r.t. the substitution until no occurrences of 
bound variables are left. Since in our implementation the type of substitution is
not polymorphic, \lstinline{refine} is also implemented in an unsafe manner. However,
it is easy to see, that \lstinline{refine} does not produce ill-typed terms. Indeed,
all original types of variables are preserved in a substitution due to invariant
3 from Section~\ref{polyuni}. Unification does not change unified terms, so all terms, 
bound in a substitution, are well-typed. Hence, \lstinline{refine} always substitutes
some subterm in a well-typed term with another term of the same type, which preserves
well-typedness.

In addition to performing substitutions, \lstinline{refine} also \emph{reifies} 
disequality constrains. Reification attaches to each free variable in a refined
term a list of \emph{refined} terms, describing disequality constraint for that
free variable. Note, disequality can be established only for equally typed
terms, which justifies type-safety of reification. Note also, additional care has 
to be taken to avoid infinite looping, since refinement and reification are
mutually recursive, and refinement of a variable can be potentially invoked from 
itself due to a chain of disequality constraints.

After refinement, the content of a logical value can be inspected via the following 
function:

\begin{lstlisting}[mathescape=true]
   val destruct : $\alpha$ logic -> 
     [`Var of int * $\alpha$ logic list | `Value of $\alpha$]
\end{lstlisting}

Constructor \lstinline{`Var} corresponds to a free variable with unique
integer identifier and a list of terms, representing all disequality constraints
for this variable. These terms are refined as well.

We did not make \lstinline{refine} accessible for an end-user; instead we provided
a set of top-level combinators, which should be used to surround relational code
and perform refinement in a transparent manner. Note, from pragmatic
standpoint only variables, supplied as arguments for the top-level goal, have
to be refined (the original miniKanren implementation follows the same convention).

The toplevel primitive in our implementation is \lstinline{run}, which takes three
arguments. The exact type of \lstinline{run} is rather complex and non-instructive, 
so we better describe the typical form of its application:

\begin{lstlisting}[mathescape=true]
   run $\overline{n}$ (fun $l_1\dots l_n$ -> $\;\;G$) (fun $a_1\dots a_n$ -> $\;\;H$)
\end{lstlisting}

Here $\overline{n}$ stands for \emph{numeral}, which describes the number of
parameters for two other arguments of \lstinline{run}, \mbox{$l_1\dots l_n$}~---
free logical variables, $G$~--- a goal (which can make use of \mbox{$l_1\dots l_n$}), 
\mbox{$a_1\dots a_n$}~--- refined answers for \mbox{$l_1\dots l_n$}, respectively, and, 
finally, $H$~--- a \emph{handler} (which can make use of \mbox{$a_1\dots a_n$}). The types of 
\mbox{$l_1\dots l_n$} are inferred from $G$, and the types of \mbox{$a_1\dots a_n$} are
inferred from types of \mbox{$l_1\dots l_n$}: if $l_i$ has type \lstinline[mathescape=true]{$t$ logic}, then
$a_i$ has type \lstinline[mathescape=true]{$t$ logic stream}. In other words, user-defined handler
takes streams of refined answers for all variables, supplied to the top-level goal. All streams $a_i$ contains
coherent elements, so they all have the same length and $n$-th elements of all streams correspond 
to the $n$-th answer, produced by the goal $G$.

There are a few predefined numerals for one, two, etc. arguments (called, by tradition, 
\lstinline{q}, \lstinline{qr}, \lstinline{qrs} etc.), and a successor function, which 
can be applied to existing numeral to increment the number of expected arguments. The
technique, used to implement them, generally follows~\cite{Unparsing, DoWeNeed}.

\section{An Example}
\label{example}

Here we present an example of relational specification, written with the aid of our library. 
For this example we take list sorting; specifically, we present sorting for lists of
natural numbers in Peano form since our library already contains built-in
support for them. However our example can be easily extended for arbitrary (but
linearly ordered) types.

List sorting can be implemented in miniKanren in a variety of ways~--- 
virtually any existing algorithm can be rewritten relationally. We, however, 
try to be as much declarative as possible to demonstrate the
advantages of relational approach. From this standpoint, we can
claim, that sorted version of empty list is empty list, and sorted version
of non-empty list is its smallest element, concatenated with sorted
version of list, containing all its remaining elements.

The following snippet literally implements this definition:

\begin{lstlisting}[mathescape=true]
   let rec sort$^o$ x y = conde [
       (x === $\uparrow$Nil) &&& (y === $\uparrow$Nil);
       fresh (s xs xs')
         (y === $\uparrow$(Cons (s, xs')))
         (sort$^o$ xs xs')       
         (smallest$^o$ x s xs)
   ]
\end{lstlisting}

The meaning of the expression

\begin{lstlisting}[mathescape=true]
   smallest$^o$ x s xs
\end{lstlisting}

is

\begin{quotation}
\noindent ``\lstinline{s}'' is the smallest element of a (non-empty) list ``\lstinline{x}'', and 
``\lstinline{xs}'' is the list of all its remaining elements.
\end{quotation}

Now, \lstinline[mathescape=true]{smallest$^o$} can be implemented
using case analysis (note, that ``\lstinline{l}'' here is a non-empty 
list):

\begin{lstlisting}[mathescape=true]
   let rec smallest$^o$ l s l' = conde [       
       (l === $\uparrow$(Cons (s, $\uparrow$Nil))) &&& (l' === $\uparrow$Nil);
       fresh (h t s' t' max)
         (l' === $\uparrow$(Cons(max,t')))
         (l === $\uparrow$(Cons(h,t)))
         (minmax$^o$ h s' s max)
         (smallest$^o$ t s' t')
   ] 
\end{lstlisting}

Finally, we implement relational minimum-maximum calculation
primitive:

\begin{lstlisting}[mathescape=true]
   let minmax$^o$ a b min max = conde [
      (min === a) &&& (max === b) &&& (le$^o$ a b);
      (max === a) &&& (min === b) &&& (gt$^o$ a b)]
\end{lstlisting}

Here ``\lstinline[mathescape=true]{le$^o$}'' and ``\lstinline[mathescape=true]{gt$^o$}'' are
built-in comparison goals for natural numbers in Peano form.

Having relational \lstinline[mathescape=true]{sort$^o$}, we can implement 
sorting for regular integer lists:

\begin{lstlisting}[mathescape=true]
   let sort l =
     run q (sorto @@ inj_nat_list l)
           (fun qs -> prj_nat_list @@ Stream.hd qs)
\end{lstlisting}

Here \lstinline{Stream.hd} is a function, which takes a head from a 
lazy stream of answers. 

It is interesting, that since \lstinline[mathescape=true]{sort$^o$} is
relational, it can be used to calculate the list of all \emph{permutations}
for a given list. Indeed, each permutation, being sorted, results in the same list. 
So, the problem of finding all permutations can be relationally reformulated into 
the problem of finding all lists, which are converted by sorting into the given one:

\begin{lstlisting}[mathescape=true]
let perm l = map prj_nat_list @@
  run q (fun q -> fresh (r)
                    (sort$^o$ (inj_nat_list l) r) 
                    (sort$^o$ q r)
        )
        (Stream.take ~n:(fact @@ length l))
\end{lstlisting}

Note, for sorting original list we used exactly the same primitive. Note also, 
we requested exactly \lstinline{fact @@ length l} answers; requesting more
would result in infinite search for non-existing answers. This concludes our example.

\section{Conclusion}

We presented strongly typed implementation of miniKanren for OCaml. Our implementation
passes all tests, written for miniKanren (including those for disequality constraints);
in addition we implemented many interesting relational programs, known from
the literature. We claim, that our implementation can be used both as a convenient
relational DSL for OCaml and an experimental framework for future research in the area of
relational programming. 

The source code of our implementation is accessible from \url{https://github.com/dboulytchev/OCanren}.

We also want to express our gratitude to William Byrd, who infected us with relational programming, 
and for the extra time he sacrificed as both our tutor and friend.

\begin{thebibliography}{99}
\bibitem{TRS}
Daniel P. Friedman, William E.Byrd, Oleg Kiselyov. The Reasoned Schemer. The MIT
Press, 2005.

\bibitem{MicroKanren}
Jason Hemann, Daniel P. Friedman. $\mu$Kanren: A Minimal Core for Relational Programming //
Proceedings of the 2013 Workshop on Scheme and Functional Programming (Scheme '13).

\bibitem{CKanren}
Claire E. Alvis, Jeremiah J. Willcock, Kyle M. Carter, William E. Byrd, Daniel P. Friedman.
cKanren: miniKanren with Constraints // 
Proceedings of the 2011 Workshop on Scheme and Functional Programming (Scheme '11).

\bibitem{Untagged}
William E. Byrd, Eric Holk, Daniel P. Friedman.
miniKanren, Live and Untagged: Quine Generation via Relational Interpreters (Programming Pearl) //
Proceedings of the 2012 Workshop on Scheme and Functional Programming (Scheme '12).

\bibitem{Implicits}
Leo White, Fr\'ed\'eric Bour, Jeremy Yallop. 
Modular Implicits // Workshop on ML, 2014, arXiv:1512.01438.

\bibitem{Unparsing}
Olivier Danvy.
Functional Unparsing // Journal of Functional Programming, Vol.~8, Issue~6, November 1998.

\bibitem{DoWeNeed}
Daniel Fridlender, Mia Indrika.
Do we need dependent types? // Journal of Functional Programming, Vol.~10, Issue~4, July 2000.

\bibitem{DGP}
Jeremy Gibbons. Datatype-generic Programming //
Proceedings of the 2006 International Conference on Datatype-generic Programming.

\bibitem{Deriving}
Jeremy Yallop. 
Practical Generic Programming in OCaml // Proceedings of 2007 Workshop on ML.

\bibitem{InstantGenerics}
Manuel M. T. Chakravarty, Gabriel C. Ditu, Roman Leshchinskiy. 
Instant Generics: Fast and Easy. \url{http://www.cse.unsw.edu.au/~chak/papers/CDL09.html}, 2009.

\bibitem{ALaCarte}
Wouter Swierstra. Data Types \'a la Carte  // Journal of Functional Programming, Vol.~18, Issue~4, 2008.
\end{thebibliography}

\end{document}


\begin{figure*}[t]
\[
\begin{array}{cccll}
  &\mathcal{C} & = & \{C_i^{k_i}\} & \mbox{constructors with arities} \\
  &\mathcal{T}_X & = & X \cup \{C_i^{k_i} (t_1, \dots, t_{k_i}) \mid t_j\in\mathcal{T}_X\} & \mbox{terms over the set of variables $X$} \\
  &\mathcal{D} & = & \mathcal{T}_\emptyset & \mbox{ground terms}\\
  &\mathcal{X} & = & \{ x, y, z, \dots \} & \mbox{syntactic variables} \\
  &\mathcal{A} & = & \{ \alpha, \beta, \gamma, \dots \} & \mbox{semantic variables} \\
  &\mathcal{R} & = & \{ R_i^{k_i}\} &\mbox{relational symbols with arities} \\
  &\mathcal{G} & = & \mathcal{T_X}\equiv\mathcal{T_X}   &  \mbox{unification} \\
  &            &   & \mathcal{G}\wedge\mathcal{G}     & \mbox{conjunction} \\
  &            &   & \mathcal{G}\vee\mathcal{G}       &\mbox{disjunction} \\
  &            &   & \mbox{\lstinline|fresh|}\;\mathcal{X}\;.\;\mathcal{G} & \mbox{fresh variable introduction} \\
  &            &   & R_i^{k_i} (t_1,\dots,t_{k_i}),\;t_j\in\mathcal{T_X} & \mbox{relational symbol invocation} \\
  &\mathcal{S} & = & \{R_i^{k_i} = \lambda\;x_1^i\dots x_{k_i}^i\,.\, g_i;\}\; g & \mbox{specification}
\end{array}
\]
\caption{The syntax of the source language}
\label{syntax}
\end{figure*}

\begin{comment}
\begin{figure}[t]
%\centering
\[
\begin{array}{rcl}
  \mathcal{FV}\,(x)&=&\{x\}\\
  \mathcal{FV}\,(C_i^{k_i}\,(t_1,\dots,t_{k_i}))&=&\bigcup\mathcal{FV}\,(t_i)\\
  \mathcal{FV}\,(t_1\equiv t_2)&=&\mathcal{FV}\,(t_1)\cup\mathcal{FV}\,(t_2)\\
  \mathcal{FV}\,(g_1\wedge g_2)&=&\mathcal{FV}\,(g_1)\cup\mathcal{FV}\,(g_2)\\
  \mathcal{FV}\,(g_1\vee g_2)&=&\mathcal{FV}\,(g_1)\cup\mathcal{FV}\,(g_2)\\
  \mathcal{FV}\,(\mbox{\lstinline|fresh|}\;x\;.\;g)&=&\mathcal{FV}\,(g)\setminus\{x\}\\
  \mathcal{FV}\,(R_i^{k_i}\,(t_1,\dots,t_{k_i}))&=&\bigcup\mathcal{FV}\,(t_i)
\end{array}
\]
\caption{Free variables in terms and goals}
\label{free}
\end{figure}
\end{comment}

\section{The Language}
\label{language}
 
In this section, we introduce the syntax of the language we use throughout the paper, describe the informal semantics, and give some examples.

The syntax of the language is shown in Fig.~\ref{syntax}. First, we fix a set of constructors $\mathcal{C}$ with known arities and consider
a set of terms $\mathcal{T}_X$ with constructors as functional symbols and variables from $X$. We parameterize this set with an alphabet of
variables since in the semantic description we will need \emph{two} kinds of variables. The first kind, \emph{syntactic} variables, is denoted
by $\mathcal{X}$. The second kind, \emph{semantic} or \emph{logic} variables, is denoted by $\mathcal{A}$.
We also consider an alphabet of \emph{relational symbols} $\mathcal{R}$ which are used to name relational definitions.
The central syntactic category in the language is \emph{goal}. In our case, there are five types of goals: \emph{unification} of terms,
conjunction and disjunction of goals, fresh variable introduction, and invocation of some relational definition. Thus, unification is used
as a constraint, and multiple constraints can be combined using conjunction, disjunction, and recursion.
The final syntactic category is a \emph{specification} $\mathcal{S}$. It consists of a set
of relational definitions and a top-level goal. A top-level goal represents a search procedure which returns a stream of substitutions for
the free variables of the goal. The definition for a set of free variables for both terms and goals is conventional;
%given in Figure~\ref{free};
as ``\lstinline|fresh|''
is the sole binding construct the definition is rather trivial. The language we defined is first-order, as goals can not be passed as parameters,
returned or constructed at run time.

We now informally describe how relational search works. As we said, a goal represents a search procedure. This procedure takes a \emph{state} as input and returns a
stream of states; a state (among other information) contains a substitution that maps semantic variables into the terms over semantic variables. Then five types of
scenarios are possible (depending on the type of the goal):

\begin{itemize}
\item Unification ``\lstinline|$t_1$ === $t_2$|'' unifies terms $t_1$ and $t_2$ in the context of the substitution in the current state. If terms are unifiable,
  then their MGU is integrated into the substitution, and a one-element stream is returned; otherwise the result is an empty stream.
\item Conjunction ``\lstinline|$g_1$ /\ $g_2$|'' applies $g_1$ to the current state and then applies $g_2$ to each element of the result, concatenating
  the streams.
\item Disjunction ``\lstinline|$g_1$ \/ $g_2$|'' applies both its goals to the current state independently and then concatenates the results.
\item Fresh construct ``\lstinline|fresh $x$ . $g$|'' allocates a new semantic variable $\alpha$, substitutes all free occurrences of $x$ in $g$ with $\alpha$, and
  runs the goal.
\item Invocation ``$\lstinline|$R_i^{k_i}$ ($t_1$,...,$t_{k_i}$)|$'' finds a definition for the relational symbol \mbox{$R_i^{k_i}=\lambda x_1\dots x_{k_i}\,.\,g_i$}, substitutes
  all free occurrences of a formal parameter $x_j$ in $g_i$ with term $t_j$ (for all $j$) and runs the goal in the current state.
\end{itemize}

We stipulate that the top-level goal is preceded by an implicit ``\lstinline|fresh|'' construct, which binds all its free variables, and that the final substitutions
for these variables constitute the result of the goal evaluation.

Conjunction and disjunction form a monadic~\cite{Monads} interface with conjunction playing role of ``\lstinline|bind|'' and disjunction the role of ``\lstinline|mplus|''.
In this description, we swept a lot of important details under the carpet~--- for example, in actual implementations the components of disjunction are not evaluated in
isolation, but both disjuncts are evaluated incrementally with the control passing from one disjunct to another (\emph{interleaving})~\cite{Search};
the evaluation of some goals can be additionally deferred (via so-called ``\emph{inverse-$\eta$-delay}'')~\cite{MicroKanren}; instead of streams
the implementation can be based on ``ferns''~\cite{BottomAvoiding} to defer divergent computations, etc. In the following sections, we present
a complete formal description of relational semantics which resolves these uncertainties in a conventional way.

As an example consider the following specification. For the sake of brevity we
abbreviate immediately nested ``\lstinline|fresh|'' constructs into the one, writing ``\lstinline|fresh $x$ $y$ $\dots$ . $g$|'' instead of
``\lstinline|fresh $x$ . fresh $y$ . $\dots$ $g$|''.

\begin{tabular}{p{5.5cm}p{5.5cm}}
\begin{lstlisting}
append$^o$ = fun x y xy .
 ((x === Nil) /\ (xy === y)) \/
 (fresh h t ty .
   (x  === Cons (h, t))  /\
   (xy === Cons (h, ty)) /\
   (append$^o$ t y ty));

revers$^o$ x x
\end{lstlisting} &
\begin{lstlisting}
revers$^o$ = fun x xr .
 ((x === Nil) /\ (xr === Nil)) \/
 (fresh h t tr .
   (x === Cons (h, t)) /\
   (append$^o$ tr (Cons (h, Nil)) xr) /\
   (revers$^o$ t tr));
\end{lstlisting}
\end{tabular}

Here we defined\footnote{We respect here a conventional tradition for \textsc{miniKanren} programming to superscript all relational names with ``$^o$''.}
two relational symbols~--- ``\lstinline|append$^o$|'' and ``\lstinline|revers$^o$|'',~--- and specified a top-level goal ``\lstinline|revers$^o$ x x|''.
The symbol ``\lstinline|append$^o$|'' defines a relation of concatenation of lists~--- it takes three arguments and performs a case analysis on the first one. If the
first argument is an empty list (``\lstinline|Nil|''), then the second and the third arguments are unified. Otherwise, the first argument is deconstructed into a head ``\lstinline|h|''
and a tail ``\lstinline|t|'', and the tail is concatenated with the second argument using a recursive call to ``\lstinline|append$^o$|'' and additional variable ``\lstinline|ty|'', which
represents the concatenation of ``\lstinline|t|'' and ``\lstinline|y|''. Finally, we unify ``\lstinline|Cons (h, ty)|'' with ``\lstinline|xy|'' to form a final constraint. Similarly,
``\lstinline|revers$^o$|'' defines relational list reversing. The top-level goal represents a search procedure for all lists ``\lstinline|x|'', which are stable under reversing, i.e.
palindromes. Running it results in an infinite stream of substitutions:

\begin{lstlisting}
   $\alpha\;\mapsto\;$ Nil
   $\alpha\;\mapsto\;$ Cons ($\beta_0$, Nil)
   $\alpha\;\mapsto\;$ Cons ($\beta_0$, Cons ($\beta_0$, Nil))
   $\alpha\;\mapsto\;$ Cons ($\beta_0$, Cons ($\beta_1$, Cons ($\beta_0$, Nil)))
   $\dots$
\end{lstlisting}

where ``$\alpha$'' is a \emph{semantic} variable, corresponding to ``\lstinline|x|'', ``$\beta_i$'' are free semantic variables. Therefore, each substitution represents a set of all palindromes of a certain length.


\section{Relational Conversion}
\label{conversion}
\def\arraystretch{1}

Before we describe the relational conversion itself, we formulate some limitations for the source
programs. Functional programs tend to operate with higher-order values, while miniKanren is
limited by a first-order unification. Therefore, it would be unreasonable to expect, that arbitrary
functional program can be converted into a relational form (at least using reasonably simple 
transformations). 

We introduce the set of ground types $\mathcal G$:

$$
\mathcal G=\alpha \mid T^k(g_1,\dots,g_k)
$$

Informally, a value of a ground type cannot contain closures. Then we formulate the following limitations for
the programs to be converted into a relational form:

\begin{itemize}
  \item all constructor parameter types must be type variables;
  \item constructors and polymorphic equality can only be applied to the values of ground types;
  \item all \lstinline|match|-expressions must be of ground types.
\end{itemize}

The first condition means, that all algebraic datatypes (which we consider as defined implicitly, see Section~\ref{source_language}) 
have to be fully-polymorphic. The first two limitations then allow us to specify the polymorphism restriction for 
relational programs, which we mentioned informally in Section~\ref{ocanren}: all type variables are bounded to
range only over ground types (this condition, of course, is sufficient, but not necessary).

The third limitation is not essential and introduced only to simplify the presentation. If a \lstinline|match|-expression does not
have a ground type, it can always be transformed to have one by applying $\eta$-expansion:

\begin{lstlisting}
   match $e$ with {$p_i$ -> $e_i$} $\leadsto$ fun $\bar{x}$.match $e$ with {$p_i$ -> $e_i\,\bar{x}$}
\end{lstlisting}

\noindent where $\bar{x}$ is a vector of new variables, different from those in $e$, $e_i$, and $p_i$. In fact, our implementation,
described in Section~\ref{evaluation}, performs this expansion as long as a non-ground type \lstinline|match|-expression is encountered. 
This is the single case when we actually inspect types and perform $\eta$-expansion.

The general idea behind the conversion can be illustrated on a type level: an expression of type $t$ in the source
language is transformed into the expression of type $\sembr{t}^t$ in relational extension, where
the transformation $\sembr{\bullet}^t$ is defined as follows:

$$
\begin{array}{rcl}
\sembr{g}^t                     & = & g \to \G \\
\sembr{t_1 \to t_2}^t           & = & \sembr{t_1}^t \to \sembr{t_2}^t \\
%\sembr{\forall \alpha. \: t} & = & \forall \alpha. \: \sembr{t}
\end{array}
$$

In other words, an expression of a ground type is converted into a goal-returning function. The informal semantics
of this function is to make its argument respect a certain contract. As the argument can have some free variable occurrences, 
the goal tries to substitute these variables with some values in order to respect the contract this goal represents. 
For example, a constant \lstinline|Nil| is converted into a function \lstinline|fun $q$ . $q\,$=== ^Nil|.

The conversion itself is described in terms of transformation $\sembr{\bullet}^c$, see Fig.~7. %\ref{relational_conversion}. 
The first five rules
simply propagate the conversion through the expression; the last three actually do the work. These rules themselves may look complicated,
but the idea is rather simple.

\begin{figure}[t]
  \centering
  \begin{tabular}{rcp{6cm}}
     $\sembr{x}^c$                &=&$x$\\
     $\sembr{\lambda x.e}^c$      &=&$\lambda x.\sembr{e}^c$\\
     $\sembr{f\;e}^c$             &=&$\sembr{f}^c\;\sembr{e}^c$\\
     $\sembr{\lstinline|let $\;x\;$ = $\;e_1\;$ in $\;e_2$|}^c$&=&\lstinline|let $x$ = $\sembr{e_1}^c$ in $\sembr{e_2}^c$|\\
     $\sembr{\lstinline|let rec $\;f\;$ = $\lambda x.e_1\;$ in $\;e_2$|}^c$&=&\lstinline|let rec $f$ = $\sembr{\lambda x.e_1}^c$ in $\sembr{e_2}^c$|\\[2mm]
     $\sembr{C^k (e_1,\dots,e_k)}^c$&=&\lstinline|fun $q$.fresh ($q_1 \dots q_k$)|
\begin{lstlisting}
  ($\sembr{e_1}^c\; q_1$) /\
  ...
  ($\sembr{e_k}^c\; q_k$) /\
  ($q$ === $\;\uparrow(C^n (q_1, \dots, q_k)$))
\end{lstlisting}\\[-2mm]
     $\sembr{\lstinline|match $\;e\;$ with \{$C^{n_i}_i(x^i_1,\dots,x^i_{n_i})\;$ -> $\;e_i$\}|}^c$&=&\lstinline|fun $q$.fresh ($q_e$)|
\begin{lstlisting}
    ($\sembr{e}^c\;q_e$) /\
    $\bigvee_i$ ((fresh ($q^i_1\dots q^i_{n_i}$)
           ($q_e$ === $\;\uparrow C^{n_i}_i(q^i_1,\dots,q^i_{n_i})$) /\
           (fun $x^i_1\dots x^i_{n_i}$.$\sembr{e_i}^c$) ($\equiv q^i_1$) ... ($\equiv q^i_{n_i}$) $q$
     ) 
    )
\end{lstlisting}\\[-2mm]
     $\sembr{\lstinline|$e_1\,$=$\,e_2$|}^c$&=&\lstinline|fun $q$.fresh ($q_1\,q_2$)|
\begin{lstlisting}
  $\sembr{e_1}^c\,q_1$ /\
  $\sembr{e_2}^c\,q_2$ /\
  (($q_1$ === $\;q_2$ /\ $q$ === $\;$^true) |||
   ($q_1$ =/= $\;q_2$ /\ $q$ === $\;$^false)
  )
\end{lstlisting}
  \end{tabular}
\label{relational_conversion}
\caption{Relational conversion}
\end{figure}

In the case of constructor we know, that all expressions $e_i$ have ground types. Thus, their relational images are goal-returning
functions. We create a set of fresh variables (one for each expression) and pass them as arguments to these functions to associate
them with the values of the expressions. The result of conversion for the constructor application itself has to be a 
goal-returning function as well. We surround expression constructed so far with abstraction and unify its argument $q$ with the
constructor, applied to corresponding logical variables. We also apply logical constructor $\uparrow$ to respect the typing rule
for unification.

The rule for pattern-matching conversion operates similarly. First, the scrutinee must have a ground type (since it is matched against
constructors). We create a fresh variable $q_e$ and associate it with the value of the scrutinee exactly as in the previous
case. Then, for each branch we create a number of fresh variables (one for each variable in the pattern for the branch) and
express pattern-matching in terms of unification, using these variables and corresponding constructor. Finally, the body $e_i$ of the branch
is an expression with free variables, corresponding to those in the pattern. We, therefore, convert $e_i$ and surround the result with
lambdas, closing all these variables. To pass the bindings $q^i_j$ for pattern variables to the body, we apply this function to
 goal-returning functions $(\equiv q^i_j)$. This, again, gives us a goal-returning function, which we apply to the topmost result variable $q$.

The last rule follows the same pattern: both arguments of polymorphic equality are transformed into goal-returning functions, and we know, that
the arguments of these functions are of some ground type. We apply these functions to fresh variables and perform case analysis. Note, this is
the only case when we actually use disequality constraints.

An interesting property of relational conversion is that it does not change terms, which do not use constructors, equality, and pattern-matching. Thus,
a lot of useful higher-order functions~--- application, composition, fixed point, etc.~--- are already relational and can be used in
relational specifications.

Another observation is that our transformation is compositional (a relational image of application is an application of relational
images). This means, that relational conversion is compatible with separate compilation~--- multiple source files can be
converted independently without losing the possibility to work properly when combined.

Then, it is interesting, that the result of relational conversion runs in a forward direction
deterministically. Thus, relational conversion imposes only a constant-time slowdown in a forward
direction.

Finally, we formulate the following properties for relational conversion:

\begin{itemize}
\item Static correctness: if an expression $e$ has a type $t$ in the source language, then $\sembr{e}^c$ has a 
type $\sembr{t}^t$ in relational extension. In other words, relational conversion transforms properly typed
programs into properly typed. Proof is by structural induction (and trivial).
\item Partial semantic correctness: if an expression $e$ has a ground type $t$ and \mbox{$e \leadsto^f v$} for some
  value $v$, then \mbox{$\lstinline|fresh($x$)($\sembr{e}^c\;x$)| \leadsto^r (\theta,\emptyset)$}, and 
\mbox{$\theta(\mathfrak{s})=v$}, where $\mathfrak{s}$ is a semantic variable, associated with $x$ on the
first step of the relational evaluation.
%The essential part of the proof is given in the Appendix~\ref{appendix}.
%Proof
%is by induction on the length of derivation sequence (a number of lemmas have to be justified on the way).
\end{itemize}

In order to prove the complete correctness, we need some means to interpret the results of relational 
derivation with free variables in functional case. This is a subject of future research.

\section{Evaluation}

\label{sec:evaluation}

In this section, we present an evaluation of 
implemented constructive negation on a series of examples.

\subsection{If-then-else}

Using relational if-then-else operator, 
presented in section~\ref{sec:ifte},
we have implemented several 
higher-order relations over lists, namely 
\lstinline{find} (Listing~\ref{lst:eval-find}), 
\lstinline{remove}\footnote{Note, this implementation 
differs from the one in Section~\ref{sec:intro}, but 
it is easy to see that these two are semantically equivalent.} (Listing~\ref{lst:eval-remove}) 
and \lstinline{filter} (Listing~\ref{lst:eval-filter}).
These relations are almost identical (syntactically) to their
functional implementations.
We have tested that these relations can be run
in various directions and produce the expected results.
For example, the goal \lstinline{filter p q q}
with the predicate \lstinline{p} equal to

\begin{lstlisting}
  fun l -> fresh (x) (l === [x])
\end{lstlisting}

stating that the given list should be a singleton list,
starts to generate all singleton lists.
Vice versa, the goal \lstinline{filter p q []} 
with that same \lstinline{p} generates 
all lists, constrained to be not a singleton list.

Listings~\ref{lst:eval-p}-\ref{lst:eval-filter-queries} give 
more concrete examples of queries to these relations.
In the listing the syntax \lstinline{run n q g}
means running a goal \lstinline{g} with 
the free variable \lstinline{q}
taking the first \lstinline{n} answers (``\lstinline{*}'' denotes all answers).
After the sign $\leadsto$ the result of the query is given.
The result \lstinline{fail} means that the query has failed.
The result \lstinline[mathescape]|succ {{a$_1$}; ... {a$_n$}} |
means that the query has succeeded delivering $n$ answers.
Each answer represents a set of constraint on free variables.
Constraints are of two forms: equality constraints, e.g. \lstinline{q = (1, _.$_0$)}, 
or disequality constraints, e.g. \lstinline{q $\neq$ (1, _.$_0$)}.
The terms of the form \lstinline{_.$_i$} in the answer
denote some universally quantified variables.

\begin{minipage}[thb]{.3\textwidth}
\begin{lstlisting}[
  caption={A definition of \code{find} relation},
  label={lst:eval-find}
]
let find p e xs =
  fresh (x xs' ys') (
    xs === x::xs' /\
    ifte (p x)
      (e === x)
      (find p e xs')
  )
\end{lstlisting}
\end{minipage}\hfill
\begin{minipage}[thb]{.3\textwidth}
\begin{lstlisting}[
  caption={A definition of \code{remove} relation},
  label={lst:eval-remove}
]
let remove p xs ys =
  (xs === [] /\ ys === [])
  \/
  fresh (x xs' ys') (
    xs === x::xs' /\
    ifte (p x)
      (ys === xs')
      (ys === x::ys' /\ 
       remove p xs' ys')
  )
\end{lstlisting}
\end{minipage}\hfill
\begin{minipage}[thb]{.3\textwidth}
\begin{lstlisting}[
  caption={A definition of \code{filter} relation},
  label={lst:eval-filter}
]
let filter p xs ys =
  (xs === [] /\ ys === [])
  \/
  fresh (x xs' ys') (
    xs === x::xs' /\
    (ifte (p x)
      (ys === x :: ys')
      (ys === ys')) /\
    filter p xs' ys'
  )
\end{lstlisting}
\end{minipage}

% \vspace{3cm}

\begin{minipage}[thb]{0.4\textwidth}
\begin{lstlisting}[
  caption={Definition of the predicate \lstinline{p}},
  label={lst:eval-p}
]
let p l = fresh (x) (l === [x])
\end{lstlisting}
\begin{lstlisting}[
  caption={Example of queries to \lstinline{find}},
  label={lst:eval-find-queries}
]
run 3 q (fresh (e) find p e q) 
$\leadsto$ succ {
     { q = [_.$_0$] :: _.$_1$ }
     { q = _.$_0$ :: [_.$_1$] :: _.$_2$; 
         _.$_0$ $\neq$ [_.$_3$] }
     { q = _.$_0$ :: _.$_1$ :: [_.$_2$] :: _.$_3$; 
         _.$_0$ $\neq$ [_.$_4$]; _.$_1$ $\neq$ [_.$_5$] }
   }
\end{lstlisting}
\end{minipage}\hfill
\begin{minipage}[thb]{0.4\textwidth}
\begin{lstlisting}[
  caption={Example of queries to \lstinline{remove}},
  label={lst:eval-remove-queries}
]
run * q (fresh (e) remove p q [[ ]]) 
$\leadsto$ succ {
     { q = [[_.$_0$]; [ ]] }
     { q = [[ ]] }
     { q = [[ ]; [_.$_0$]] }
   }

run 3 q (fresh (e) remove p q q) 
$\leadsto$ succ {
     { q = [] }
     { q = [_.$_0$], _.$_0$ $\neq$ [_.$_1$] }
     { q = [_.$_0$; _.$_1$]; 
         _.$_0$ $\neq$ [_.$_2$]; _.$_1$ $\neq$ [_.$_3$] }
   }
\end{lstlisting}
\end{minipage}

\begin{minipage}[thb]{0.4\textwidth}
\begin{lstlisting}[
  caption={Example of queries to \lstinline{filter}},
  label={lst:eval-filter-queries}
]
run 3 q (filter p q q) 
$\leadsto$ succ {
     { q = [ ] }
     { q = [_.$_0$] }
     { q = [_.$_0$; _.$_1$] }
   }

run 3 q (filter p q [1]) 
$\leadsto$ succ {
     { q = [[1]] }
     { q = [_.$_0$; [1]]; _.$_0$ $\neq$ [_.$_1$] }
     { q = [[1]; _.$_0$]; _.$_0$ $\neq$ [_.$_1$] }
   }

run 3 q (filter p q [ ]) 
$\leadsto$ succ {
     { q = [] }
     { q = [_.$_0$]; _.$_0$ $\neq$ [_.$_1$] }
     { q = [_.$_0$; _.$_1$]; 
            _.$_0$ $\neq$ [_.$_2$]; _.$_1$ $\neq$ [_.$_3$] }
   }
\end{lstlisting}
\end{minipage}

\subsection{Universal quantification}

In the Section~\ref{sec:impl-univ} we presented 
the \lstinline{forall} goal constructor 
which is implemented through the double negation.
We have observed, that although \lstinline{forall g}
does not terminate when the goal \lstinline{g x} 
has an infinite number of answers 
(assuming \lstinline{x} is a fresh variable),
it does terminate in the case when \lstinline{g x} has 
a finite number of answers.
The behavior of \lstinline{forall} in this case is sound
even in the presence of disequality constraints or nested quantifiers. 

The Table~\ref{tab:univ} gives some concrete examples.
The left column contains the tested goals\footnote{
We typeset the goals in terms of first-order logic syntax 
instead of \textsc{OCanren} syntax for brevity and clarity.} 
and the right column gives the obtained results.
For the results we use the same notation 
as in the previous section.

\begin{table}[th]
  \centering
  \def\arraystretch{1.5}
  \begin{tabularx}{\textwidth}{|X|X|}
    \hline

    $\forall x\ldotp x = q$ & 
      \texttt{fail} \\
    \hline

    $\forall x\ldotp \exists y\ldotp x = y$ & 
      \texttt{succ \{[q = \_.$_0$]\}} \\
    \hline

    $\forall x\ldotp \exists y\ldotp x = y \wedge y = q$ &
      \texttt{fail} \\
    \hline

    $\forall x\ldotp q = (1, x)$ & 
      \texttt{fail} \\
    \hline

    $\forall x\ldotp \exists y\ldotp y = (1, x)$ & 
      \texttt{succ \{[q = \_.$_0$]\}} \\
    \hline

    $\forall x\ldotp \exists y\ldotp x = (1, y)$ &
      \texttt{fail} \\
    \hline

    $\forall x\ldotp x \neq q$ & \texttt{fail} \\
    \hline

    $\forall x\ldotp \exists y\ldotp x \neq y$ & 
      \texttt{succ \{[q = \_.$_0$]\}} \\
    \hline

    $\forall x\ldotp \exists y\ldotp x \neq y \wedge y = q$ & 
      \texttt{fail} \\
    \hline

    $\forall x\ldotp q \neq (1, x)$ & 
      \texttt{succ \{[q $\neq$ (1, \_.$_0$)]\}} \\
    \hline

    $(\exists x\ldotp q = (1, x)) \wedge (\forall x\ldotp q \neq (1, x))$ & 
      \texttt{fail} \\
    \hline

    $\forall x\ldotp (x, x) \neq (0, 1)$ & 
      \texttt{succ \{[q = \_.$_0$]\}} \\
    \hline

    $\forall x\ldotp (x, x) \neq (1, 1)$ & 
      \texttt{fail} \\
    \hline

    $\forall x\ldotp (x, x) \neq (q, 1)$ & 
      \texttt{succ \{[q $\neq$ 1]\}} \\
    \hline

    $\exists a~ b\ldotp q = (a, b) \wedge \forall x\ldotp (x, x) \neq (a, b)$ & 
      \texttt{succ \{[q = (\_.$_0$, \_.$_1$); \_.$_0$ $\neq$ \_.$_1$]\}} \\
    \hline

  \end{tabularx}
  \caption{\lstinline{forall} evaluation}
  \label{tab:univ}
\end{table}

\section{Future Work}

There are a few possible directions for future work. First, in this paper we did not address the performance issues. As we represent
the transformations in a very generic form with many levels of indirection, obviously, the transformations, implemented with
our framework, are at disadvantage in comparison with hard coded ones in terms of performance. We assume that the performance of transformations
can be essentially improved by applying some techniques like staging~\cite{Staged} or, perhaps, object-specific optimisations.

Another important direction is supporting more kinds of type declarations, in the first hand, GADTs and non-regular types. Although we have some
implementation ideas for this case, the solution we came up with so far makes the interface of the whole framework too cumbersome to use even for
simple cases.

Finally, the typeinfo structure we generate can be used to mimic the \emph{ad-hoc} polymorphism as it contains the implementation of
type-indexed functions. This, together with some proposed extensions~\cite{ModularImplicits}, can open interesting perspectives.



%\begin{comment}
\begin{thebibliography}{10}

\bibitem{CKanren}
C.~E. Alvis, J.~J. Willcock, K.~M. Carter, W.~E. Byrd, and D.~P. Friedman.
\newblock {cKanren}: {miniKanren} with Constraints.
\newblock In {\em Proceedings of the 2011 Annual Workshop on Scheme and
  Functional Programming}, Oct. 2011.

\bibitem{Lambda}
H.~P. Barendregt.
\newblock Lambda Calculi with Types.
\newblock In {\em Handbook of Logic in Computer Science (vol. 2)}, 
pages 117--309. Oxford University Press, Inc., New York, NY, USA, 1992.

\bibitem{WillOnHM}
W.~E. Byrd.
\newblock Private communications.

\bibitem{WillThesis}
W.~E. Byrd.
\newblock Relational Programming in miniKanren: Techniques, Applications,
  and Implementations.
\newblock PhD thesis, Indiana University, September 2009.

\bibitem{unified}
W.~E. Byrd, M.~Ballantyne, G.~Rosenblatt, and M.~Might.
\newblock A Unified Approach to Solving Seven Programming Problems (functional
  pearl).
\newblock {\em Proc. ACM Program. Lang.}, 1(ICFP):8:1--8:26, Aug. 2017.

\bibitem{alphaKanren}
W.~E. Byrd and D.~P. Friedman.
\newblock {$\alpha$Kanren}: A Fresh Name in Nominal Logic Programming.
\newblock In {\em Proceedings of the 2007 Annual Workshop on Scheme and
  Functional Programming}, pages 79--90, 2007.

\bibitem{Untagged}
W.~E. Byrd, E.~Holk, and D.~P. Friedman.
\newblock miniKanren, Live and Untagged: Quine Generation via Relational
  Interpreters (programming pearl).
\newblock In {\em Proceedings of the 2012 Annual Workshop on Scheme and
  Functional Programming}, Scheme '12, pages 8--29, New York, NY, USA, 2012.
  ACM.

\bibitem{cardelli}
L.~Cardelli and P.~Wegner.
\newblock On Understanding Types, Data Abstraction, and Polymorphism.
\newblock {\em ACM Comput. Surv.}, 17(4):471--523, Dec. 1985.

\bibitem{TRS}
D.~P. Friedman, W.~E. Byrd, and O.~Kiselyov.
\newblock The Reasoned Schemer.
\newblock The MIT Press, 2005.

\bibitem{MicroKanren}
J.~Hemann and D.~P. Friedman.
\newblock $\mu$Kanren: A Minimal Functional Core for Relational Programming.
\newblock In {\em Proceedings of the 2013 Annual Workshop on Scheme and
  Functional Programming}, 2013.

\bibitem{SmallEmbedding}
J.~Hemann, D.~P. Friedman, W.~E. Byrd, and M.~Might.
\newblock A Small Embedding of Logic Programming with a Simple Complete Search.
\newblock {\em SIGPLAN Not.}, 52(2):96--107, Nov. 2016.

\bibitem{ocanren}
D.~Kosarev and D.~Boulytchev.
\newblock Typed Embedding of a Relational Language in OCaml.
\newblock {\em ACM SIGPLAN Workshop on ML}, 2016.

\bibitem{UnificationRevisited}
J.-L. Lassez, M.~J. Maher, and K.~Marriott.
\newblock Unification Revisited.
\newblock In {\em Foundations of Deductive Databases and Logic Programming},
pages 587--625. Morgan Kaufmann Publishers Inc., San Francisco, CA, USA, 1988.

\bibitem{Types}
B.~C. Pierce.
\newblock Types and Programming Languages.
\newblock The MIT Press, 1st edition, 2002.

\bibitem{Unification}
F.~Baader and W.~Snyder. 
\newblock{Unification Theory.}
\newblock In {\em Handbook of Automated Reasoning},
Elsevier Science Publishers B. V., Amsterdam, The Netherlands, The Netherlands, 2001.

\bibitem{Felleisen}
A.~Wright and M.~Felleisen.
\newblock A Syntactic Approach to Type Soundness.
\newblock {\em Inf. Comput.}, 115(1):38--94, Nov. 1994.

\end{thebibliography}
%\end{comment}

%\renewcommand{\clearpage}{} 
%\bibliographystyle{abbrv}
%\bibliography{main}

%\clearpage
%\appendix
%\section{Appendix}
\label{appendix}

In this appendix we present a proof of partial semantic correctness of relational conversion, or, to be precise, 
a number of observations, definitions, and claims, which, we believe, are sufficient to reconstruct
the complete proof. 

We remind, that our goal is to prove the following statement:

\begin{theorem} 
\normalfont For arbitrary functional program $p$ of a ground type $t$, arbitrary value $v$, and
arbitrary variable $x$

$$
\begin{array}{c}
p\leadsto^f v \Rightarrow \lstinline|fresh ($x$) ($\sembr{p}^c x$)| \leadsto^r (\theta, \emptyset)\\
\mbox{and}\\
\theta(\mathfrak{s})=v
\end{array}
$$

\noindent where $\mathfrak{s}$ is a semantic variable, associated with
$x$ on the first step of the relational evaluation.
\end{theorem}
  
We first comment on the empty set as the set of negative substitutions. A disequality constraint can
come only from a polymorphic equality, which is applied when both its operands are reduced to
values. In the relational counterpart, being run in a forward direction, this corresponds to the evaluation of disequality constraints for
closed terms only, which, in turn, means, that they will immediately succeed or fail. Both cases
add nothing to the set of negative substitutions, which is initially empty. 

Next, we cannot prove the theorem, using an induction by a derivation length, since in the case of
application, for example, the type of the term in the head position is not ground. This 
obstacle could be lifted, if we could prove the following generalization:

$$
p\leadsto^f f \Rightarrow \sembr{p}^c\leadsto^r\sembr{f}^c
$$ 

\noindent for arbitrary $p$ of any type. This claim, however, turned out to be false~--- a term
\lstinline|C ((fun x.x) A)| can be taken as an example.  

The origin of the problem is that we \emph{functionalize} the constructors, \lstinline|match|, and
equality expressions, and, hence, change the order of reductions in the relational counterpart in 
comparison with the original functional program. Thus, we need to take this change into account.

First, we develop a modified functional semantics, which corresponds better to the reduction
order in the relational case. We call this semantics \emph{deferred}, as it defers the evaluation
of constructors, \lstinline|match|, and equality expressions. This semantics can be acquired in
two steps: first, we consider a reduced version of the original functional semantics, in which
we treat arbitrary constructor, \lstinline|match|, and equality expressions as values. Then, the
deferred semantics is just an iterative application of the reduced version to the arguments 
of these new values (arguments of constructors or equality operator, or scrutinees of \lstinline|match| 
expressions).

Next, we claim, that if a term of some ground type is reduced to some value by the original semantics,
then it as well is reduced to the same value by the deferred one. This claim is based on the following
observations:

\begin{itemize}
\item progress and type preservation properties for both semantics (which can be proven in a standard
way);
\item Church-Rosser property for lambda-calculus;
\item the fact, that the reduced semantics applies a proper subset of rules of the original one.
\end{itemize}

Now, we are going to prove the theorem by a simulation between the deferred semantics for the original program
and the relational one for the relationally converted. Before that, we formulate the number of lemmas and 
definitions.

\begin{lemma}
\label{stack_split}
\normalfont Let us separate all the contexts into two disjoint kinds: 

\begin{itemize}
\item functional

$$
C_f = \Box\;e\mid v\;\Box\mid\lstinline|let $x$ = $\Box$ in $e$|
$$

\item ground

$$
C_g = \lstinline|match $\;\Box\;$ with $\{p_i$->$e_i\}$|\mid C^n(\bar{v},\Box,\bar{e})\mid\Box\lstinline|=e|\mid\lstinline|v=|\Box
$$

Let $\left<{\mathcal S},\,e\right>$ be an arbitrary state in a derivation sequence w.r.t. the deferred
semantics. Then $\mathcal S=C_f^*C_g^*$.
\end{itemize}

In other words, during the evaluation w.r.t. the deferred semantics, the stack of contexts is separated into the two
(possibly empty) segments: all ground contexts reside below all functional. The proof is by the induction on the
length of derivation sequence.
\end{lemma}

\begin{definition}
\normalfont
We as well separate all terms of the source language into the two disjoint kinds:

\begin{itemize}
\item functional

$$
e_1\,e_2\mid \lambda x.e \mid \mu f.\lambda x.e \mid \lstinline|let $x$ = $e_1$ in $e_2$| \mid \lstinline|let rec $f$ = $\lambda x.e_1$ in $e_2$|
$$

\item ground

$$
e_1 = e_2 \mid \lstinline|match $e$ with {$p_i$ -> $e_i$ }| \mid \lstinline|C$^k$ ($e_1\dots e_k$)|
$$

\end{itemize}

\end{definition}

\begin{definition}
\normalfont Augmented conversion of a term w.r.t. to a substitution $\sembr{\bullet}_\theta$ is defined as follows: 

$$
\begin{array}{rcl}
\sembr{p}_\theta&=&\sembr{p}^c\\
\sembr{v}_\theta&=&(\lambda x.x\equiv\mathfrak{s}),\,\mbox{if}\;\;\theta(\mathfrak s)=v
\end{array}
$$

Here $\theta$ is a substitution, $p$~--- arbitrary functional term, $v$~--- arbitrary value of a
ground type in the sense of the original semantics (i.e. the composition of constructors). Note, the
cases in this definition are not disjoint, and in the second case there can be more, than one
variable with the requested property, so augmented conversion defines a set of relational terms.
\end{definition}

\begin{lemma}
\label{substitution}
\normalfont Let $f$, $e$ be two arbitrary terms of the source language, $\theta$~--- arbitrary
substitution. Then

$$
\sembr{f[x\gets e]}_\theta=\sembr{f}_\theta[x\gets\sembr{e}_\theta]
$$

The equality here is understood in a set-theoretic sense. The proof is by structural 
induction.
\end{lemma}

\begin{definition}
\normalfont For arbitrary substitution $\theta$ define a conversion of a functional context  
$\sembr{\bullet}_\theta$ as follows:

$$
\begin{array}{rcl}
\sembr{\Box\,e}_\theta&=&\Box\,\sembr{e}_\theta\\
\sembr{v\,\Box}_\theta&=&\sembr{v}_\theta\,\Box\\
\sembr{\lstinline|let $\;x\; = \;\Box\;$ in $\;e$|}_\theta&=&\lstinline|let $\;x\; = \;\Box\;$ in $\;\sembr{e}_\theta$|
\end{array}
$$

Here $e$ is an arbitrary functional term, $v$~--- abstraction. This conversion is an extension of augmented
conversion for functional contexts, hence the same denotation.
\end{definition}

\begin{definition}
\normalfont For arbitrary semantic variables ${\mathfrak s}_1$, ${\mathfrak s}_2$ and arbitrary substitution $\theta$ 
define a conversion of ground context $\sembr{\bullet}^{{\mathfrak s}_1{\mathfrak s}_2}_\theta$ as follows:

$$ 
\begin{array}{rcl}
\sembr{C^k(v_1, \ldots, v_{i-1}, \Box, e_{i+1}, \ldots, e_k)}^{{\mathfrak s}_1{\mathfrak s}_2}_\theta&=&\Box \; \wedge \\
       & & (\sembr{e_{i+1}}_\theta \; {\mathfrak s}^\prime_{i+1}) \; \wedge \\
       & & \ldots  \\
       & & (\sembr{e_k}_\theta \; {\mathfrak s}^\prime_k) \; \wedge \\
       & & ({\mathfrak s}_2 \equiv\; \uparrow C^k({\mathfrak s}^\prime_1, \ldots, {\mathfrak s}^\prime_{i-1}, {\mathfrak s}_1, {\mathfrak s}^\prime_{i+1}, \ldots, {\mathfrak s}_k)),\,\mbox{if}\;\theta({\mathfrak s}^\prime_j)=v_j,\,j<i
\end{array}
$$

$$
\begin{array}{rcl}
\sembr{\Box = e}^{{\mathfrak s}_1{\mathfrak s}_2}_\theta&=&\Box\, \wedge \\
 & & (\sembr{e}_\theta\; {\mathfrak s}^\prime) \wedge \\
 & & ((({\mathfrak s}_1 \equiv {\mathfrak s}^\prime) \wedge ({\mathfrak s}_2 \equiv \lstinline|^true|))\, \vee \\ 
 & & (({\mathfrak s}_1 \not \equiv {\mathfrak s}^\prime) \wedge ({\mathfrak s}_2 \equiv \lstinline|^false|))) 
\end{array}
$$

$$
\begin{array}{rcl}
\sembr{v = \Box}^{{\mathfrak s}_1{\mathfrak s}_2}_\theta&=&\Box\,\wedge \\
 & & ((({\mathfrak s}^\prime \equiv {\mathfrak s}_1) \wedge ({\mathfrak s}_2 \equiv \lstinline|^true|))\, \vee \\ 
 & & (({\mathfrak s}^\prime \not \equiv {\mathfrak s}_1) \wedge ({\mathfrak s}_2 \equiv \lstinline|^false|))),\,\mbox{if}\;\theta({\mathfrak s})=v 
\end{array}
$$

$$
\begin{array}{rcl}
\sembr{\lstinline|match $\;\Box\;$ with \{$C^{n_i}_i$($y^i_1$, ..., $y^i_{n_i}$) -> $\;e_i$\}|}^{{\mathfrak s}_1{\mathfrak s}_2}_\theta&=&\Box \; \wedge \bigvee_i\\
& &(\lstinline|fresh ($s^i_1 \ldots s^i_{n_i}$)| \\
& &\qquad({\mathfrak s}_1 \equiv \;\uparrow C_i^{n_i}(s^i_1, \ldots, s^i_{n_i})) \\
& &\qquad(\lambda y^i_1. \ldots \lambda  y^i_{n_i}. \sembr{e_i}_\theta) \; (\equiv s^i_1) \ldots (\equiv s^i_{n_i})\;{\mathfrak s}_2)
\end{array}
$$

Here we assume ${\mathfrak s}^\prime$ and ${\mathfrak s}^\prime_i$ to be arbitrary semantic variables, $v_i$~--- arbitrary values w.r.t. the original 
functional semantics, $e_i$~--- arbitrary terms of the source language. We also claim, that $\theta$ is
undefined for all mentioned semantic variables, unless the opposite is specified explicitly.

\end{definition}

\begin{definition}
\normalfont For arbitrary substitution $\theta$, arbitrary semantic variable ${\mathfrak s}_m$ and a functional 
term $e$ define a conversion of a stack $\sembr{\bullet}^{e,{\mathfrak s}_m}_\theta$ as follows:

$$
\def\arraystretch{1.5}
\sembr{f_n\dots f_1g_m\dots g_1}^{e,{\mathfrak s}_m}_\theta=\left\{
\begin{array}{lcl}
\sembr{g_m}^{{\mathfrak s}_m{\mathfrak s}_{m-1}}_\theta\dots\sembr{g_1}^{{\mathfrak s}_1{\mathfrak s}_0}_\theta&,&n=0\;\;\mbox{and $e$~--- ground}\\
\sembr{f_n}_\theta\dots\sembr{f_1}_\theta(\Box\,{\mathfrak s}_m)\sembr{g_m}^{{\mathfrak s}_m{\mathfrak s}_{m-1}}_\theta\dots\sembr{g_1}^{{\mathfrak s}_1{\mathfrak s}_0}_\theta&,&\mbox{otherwise}
\end{array}
\right.
$$

Here ${\mathfrak s}_0\dots {\mathfrak s}_{m-1}$ designate arbitrary distinct semantic variables.
\end{definition}

\begin{definition}
\normalfont For arbitrary substitution $\theta$ and arbitrary semantic variable ${\mathfrak s}_m$ define a simulation
conversion $\sembr{\bullet}^{{\mathfrak s}_m}_\theta$ of the source language term as follows:

$$
\begin{array}{rcl}
\sembr{e_1 = e_2}^{{\mathfrak s}_m}_\theta&=& (\sembr{e_1}_\theta\; {\mathfrak s}^\prime_1) \wedge \\
                           & & (\sembr{e_2}_\theta\; {\mathfrak s}^\prime_2) \wedge \\
                           & & ((({\mathfrak s}^\prime_1 \equiv {\mathfrak s}^\prime_2) \wedge ({\mathfrak s}_m \equiv \lstinline|^true|))\, \vee \\ 
                           & & (({\mathfrak s}^\prime_1 \not \equiv {\mathfrak s}^\prime_2) \wedge ({\mathfrak s}_m \equiv \lstinline|^false|)))
\end{array}
$$

$$
\begin{array}{rcl}
\sembr{v = e}^{{\mathfrak s}_m}_\theta&=& (\sembr{e}_\theta\; {\mathfrak s}^\prime_2) \wedge \\
                        & & ((({\mathfrak s}^\prime_1 \equiv {\mathfrak s}^\prime_2) \wedge ({\mathfrak s}_m \equiv \lstinline|^true|))\, \vee \\ 
                        & & (({\mathfrak s}^\prime_1 \not \equiv {\mathfrak s}^\prime_2) \wedge ({\mathfrak s}_m \equiv \lstinline|^false|))),\,\mbox{if}\;\theta({\mathfrak s}^\prime_1)=v
\end{array}
$$

$$
\begin{array}{rcl}
\sembr{v_1 = v_2}^{{\mathfrak s}_m}_\theta&=& ((({\mathfrak s}^\prime_1 \equiv {\mathfrak s}^\prime_2) \wedge ({\mathfrak s}_m \equiv \lstinline|^true|))\, \vee \\ 
                           & & (({\mathfrak s}^\prime_1 \not \equiv {\mathfrak s}^\prime_2) \wedge ({\mathfrak s}_m \equiv \lstinline|^false|))),\,\mbox{if}\;\theta({\mathfrak s}^\prime_j)=v_j
\end{array}
$$

$$ 
\begin{array}{rcl}
\sembr{C^k(v_1, \ldots, v_{i-1}, e_i, \ldots, e_k)}^{{\mathfrak s}_m}_\theta&=&(\sembr{e_i}_\theta \; {\mathfrak s}^\prime_i) \; \wedge \\
       & & \ldots  \\
       & & (\sembr{e_k}_\theta \; {\mathfrak s}^\prime_k) \; \wedge \\
       & & ({\mathfrak s}_m \equiv\; \uparrow C^k({\mathfrak s}^\prime_1, \ldots, {\mathfrak s}^\prime_k)),\,\mbox{if}\;\theta({\mathfrak s}^\prime_j)=v_j,\,j<i
\end{array}
$$

$$ 
\sembr{C^k(v_1, \ldots, v_k)}^{{\mathfrak s}_m}_\theta = ({\mathfrak s}_m \equiv\; \uparrow C^k({\mathfrak s}^\prime_1, \ldots, {\mathfrak s}^\prime_k)),\,\mbox{if}\;\theta({\mathfrak s}^\prime_j)=v_j
$$

$$ 
\sembr{C^k(v_1, \ldots, v_k)}^{{\mathfrak s}_m}_\theta = ({\mathfrak s}_m \equiv\; {\mathfrak s}^\prime),\;\mbox{if}\;\theta({\mathfrak s}^\prime)=C^k(v_1, \ldots, v_k)
$$

$$
\begin{array}{rcl}
\sembr{\lstinline|match $\;e\;$ with \{$C^{n_i}_i$($y^i_1$, ..., $y^i_{n_i}$) -> $\;e_i$\}|}^{{\mathfrak s}_m}_\theta&=&\sembr{e}_\theta\;{\mathfrak s}^\prime\;\wedge\;\bigvee_i\\
& &(\lstinline|fresh ($s^i_1 \ldots s^i_{n_i}$)| \\
& &\qquad({\mathfrak s}^\prime \equiv \;\uparrow C_i^{n_i}(s^i_1, \ldots, s^i_{n_i})) \\
& &\qquad(\lambda y^i_1. \ldots \lambda  y^i_{n_i}. \sembr{e_i}_\theta) \; (\equiv s^i_1) \ldots (\equiv s^i_{n_i})\;{\mathfrak s}_m)
\end{array}
$$

$$
\begin{array}{rcl}
\sembr{\lstinline|match $\;v\;$ with \{$C^{n_i}_i$($y^i_1$, ..., $y^i_{n_i}$) -> $\;e_i$\}|}^{{\mathfrak s}_m}_\theta&=&\bigvee_i\\
& &(\lstinline|fresh ($s^i_1 \ldots s^i_{n_i}$)| \\
& &\qquad({\mathfrak s}^\prime \equiv \;\uparrow C_i^{n_i}(s^i_1, \ldots, s^i_{n_i})) \\
& &\qquad(\lambda y^i_1. \ldots \lambda  y^i_{n_i}. \sembr{e_i}_\theta) \; (\equiv s^i_1) \ldots (\equiv s^i_{n_i})\;{\mathfrak s}_m),\,\mbox{if}\;\theta({\mathfrak s}^\prime)=v
\end{array}
$$

Here all ${\mathfrak s}^\prime$ and ${\mathfrak s}^\prime_i$ designate arbitrary semantic variables, $e$~--- arbitrary term, $v$~--- arbitrary value w.r.t. the
original semantics. We also claim, that $\theta$ is undefined for all mentioned semantic variables, unless the opposite is specified explicitly.
\end{definition}

\begin{definition}
\normalfont Let 
\begin{itemize}
\item \mbox{$\left<\mathcal S,\,e\right>$}~--- a state w.r.t. the deferred semantics;
\item \mbox{$\left<\Sigma, \hat{\mathcal S}, \hat{e}, (\theta, \emptyset)\right>$}~--- a state w.r.t. the
relational semantics.
\end{itemize} 

We say, that these states are connected, if there exists a semantic variable $q_m$, such, that:\vspace{1mm}

\begin{enumerate}
\item \mbox{$\hat{\mathcal S}\in\sembr{\mathcal S}^{e,{\mathfrak s}_m}_\theta$}\vspace{1mm}
\item \mbox{$\hat{e}\in\left\{
                          \begin{array}{lcl}
                            \sembr{e}^{{\mathfrak s}_m}_\theta&,&e\mbox{~--- ground and }\mathcal S\mbox{ does not contain functional contexts}\\[1mm]
                            \sembr{e}_\theta&,&\mbox{otherwise}
                          \end{array}
                       \right.
            $} 
\item $\Sigma$ contains all semantic variables from $\hat{e}$, $\hat{\mathcal S}$, and $\theta$.
\end{enumerate}

\end{definition}

\begin{lemma}
\label{constructor}
\normalfont Let $v=\lstinline|C$^k$($v_1$,...,$v_k$)|$ be a value. Then
for arbitrary $\Sigma$, $\mathcal S$, $\theta$, $\hat{v}\in \sembr{v}_\theta$, and 
semantic variable ${\mathfrak s}$, such, that ${\mathfrak s}\not\in dom(\theta)$ either

$$
\left<\Sigma,\,\mathcal S, (\hat{v}\,{\mathfrak s}),\, (\theta,\,\emptyset)\right>\leadsto^*\left<\Sigma^\prime,\,\mathcal S,\,{\mathfrak s}\equiv\lstinline|C$^k$(${\mathfrak s}^\prime_1$,...,${\mathfrak s}^\prime_k$)|,\,(\theta^\prime,\,\emptyset)\right>\;\mbox{and}\;\theta^\prime({\mathfrak s}^\prime_i)=v_i
$$

or

$$
\left<\Sigma,\,\mathcal S, (\hat{v}\,{\mathfrak s}),\, (\theta,\,\emptyset)\right>\leadsto\left<\Sigma,\,\mathcal S,\,{\mathfrak s}\equiv {\mathfrak s}^\prime,\,(\theta,\,\emptyset)\right>\;\mbox{and}\;\theta({\mathfrak s}^\prime)=v
$$
 
The proof is by induction on the height of $v$.
\end{lemma}

\begin{lemma}
\label{evaluation_lemma}
\normalfont Let $s=\left<\mathcal S=g_m\dots g_1,\,e\right>$ be a state w.r.t. the deferred semantics, 
$g_i$~--- ground contexts, $e$~--- expression of a ground type, $\theta$~--- some substitution,
${\mathfrak s}_m$~--- some semantic variable, \mbox{$\hat{\mathcal{S}}\in\sembr{\mathcal S}^{e,\,{\mathfrak s}_m}_\theta$}, 
\mbox{$\hat{e} \in \sembr{e}_\theta$}. Then there is a sequence of steps w.r.t. the relational
semantics, such, that

$$
\left<\Sigma, \hat{\mathcal S}, (\hat{e} \, {\mathfrak s}_m), (\theta,\,\emptyset) \right>\leadsto^*\hat{s}
$$

\noindent and $s$ and $\hat{s}$ are connected. Here we assume $\Sigma$ to contain all semantic variables from
$\hat{\mathcal S}$ and $\theta$. The proof is by case analysis on $e$, using Lemma~\ref{constructor}.
\end{lemma}

\begin{lemma} 
\label{connection}
\normalfont Let \mbox{$s_1 \to s_2$}~--- a single evaluation step w.r.t. the deferred semantics,
$\hat{s_1}$~--- a state of the relational semantics, such, that $s_1$ and $\hat{s_1}$ are connected. Then
there exists a sequence of steps in the relational semantics \mbox{$\hat{s_1}\leadsto^*\hat{s_2}$}, such, 
that $s_2$ and $\hat{s_2}$ are connected. The proof is by case analysis and definition of connection
relation, using Lemmas~\ref{substitution},~\ref{constructor},~\ref{evaluation_lemma}. 
\end{lemma}

\begin{lemma}
\label{prefix}
\normalfont Let $s_0=\left<\emptyset,\,\epsilon,\,\lstinline|fresh ($x$) $(\sembr{e}^c\;x)$|,\,\iota\right>$ be an
initial state of evaluation w.r.t. the relational semantics. Then there is a sequence of steps
\mbox{$s_0\leadsto^*\hat{s}$}, such, that \mbox{$\left<\epsilon,\,e\right>$} (an initial state of
evaluation of $e$ w.r.t. the deferred semantics) and $\hat{s}$ are connected. Immediately follows from
Lemma~\ref{evaluation_lemma}.
\end{lemma}

Now we can prove the partial correctness theorem. Let us have a term $e$ of a ground type in the source language, which
reduces to a value $v=\lstinline|C$^k$($v_1$,...,$v_k$)|$ w.r.t. the original call-by-value semantics. Then it reduces to the same value w.r.t. the
deferred semantics: 

$$
\left<\epsilon,\,e\right>\to^*\left<\epsilon,\,v\right>
$$

By Lemma~\ref{prefix} 

$$
\left<\emptyset,\,\epsilon,\lstinline|fresh ($x$) $(\sembr{e}^c\;x)$|,\iota\right>\leadsto^*\hat{s}
$$

\noindent where \mbox{$\left<\epsilon,\,e\right>$} and $\hat{s}$ are connected. By Lemma~\ref{connection}, there is
a state $\hat{s^\prime}$ w.r.t. the relational semantics, such, that

$$
\hat{s}\leadsto^*\hat{s^\prime}
$$

\noindent and \mbox{$\left<\epsilon,\,v\right>$} and $\hat{s^\prime}$ are connected. By the definition of
the connection relation, $\hat{s^\prime}$ has one of the following forms:

$$
\left<\Sigma,\,\epsilon,\,{\mathfrak s}_0\equiv\lstinline|C$^k$(${\mathfrak s}^\prime_1$,...,${\mathfrak s}^\prime_k$)|,\,(\theta,\,\emptyset)\right>,\,\theta({\mathfrak s}^\prime_i)=v_i
$$

\noindent or

$$
\left<\Sigma,\,\epsilon,\,{\mathfrak s}_0\equiv {\mathfrak s}^\prime,\,(\theta,\,\emptyset)\right>,\,\theta({\mathfrak s}^\prime)=v
$$

\noindent where ${\mathfrak s}_0$ is the first semantic variable, added to $\Sigma$, and \mbox{${\mathfrak s}_0\not\in dom(\theta)$}. In
both cases, we can make the one last step in the relational semantics, which completes the proof. 


\end{document}

