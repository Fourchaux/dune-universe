\section{Relational Programming in \miniKanren}
\label{ocanren}

In the context of this paper, we will use a certain concrete implementation of \miniKanren~--- a shallow DSL for 
OCaml\footnote{https://github.com/dboulytchev/ocanren}, called \ocanren~\cite{ocanren}. \ocanren corresponds to \miniKanren with
disequality constraints~\cite{CKanren}, and (modulo typing) follows the original implementation~\cite{MicroKanren,SmallEmbedding}. Here we describe the external view 
on \ocanren, giving the only intuitive meaning of its constructs; the formal description will be presented in Section~\ref{relational_extension}.
We also use a simplified syntax, which is a little bit different from the concrete syntax in actual implementation, but assumed to
be easier to read.

The central notion of \miniKanren is \emph{goal}; in \ocanren a goal can be an arbitrary expression of reserved goal type, which we denote $\G$.
There are only five syntactic forms of goals (denoted below as $g, g_1, g_2$, etc.):

\begin{itemize}
  \item conjunction $g_1\wedge g_2$;
  \item disjunction $g_1\vee g_2$;
  \item fresh variable introduction $\lstinline|fresh ($x$) $\;g$|$;
  \item unification $t_1\;\equiv\;t_2$;
  \item disequality constraint $t_1\;\not\equiv\;t_2$.
\end{itemize}

Two last forms of goals constitute a basis for goal construction; here $t_1$ and $t_2$ are \emph{terms}. In \ocanren a term is an arbitrary expression of polymorphic logic type $\alpha^o$. The postfix notation $\Box^o$ is a traditional way to denote relational entities, and we will use it for types as well\footnote{In the real implementation the terms have a more complex two-parametric type, which encodes a tagging, needed to be performed when the results of the relational program are returned into the functional world; these details, however, are irrelevant to the objectives of the paper, and we stick with the simplified version.}.

The simplest expression of logic type is a variable, bound in \lstinline|fresh|. Another example is a primitive value, \emph{injected} into the logic domain with a built-in primitive ``$\uparrow$'', such as $\uparrow\!3$ (of type \lstinline|int$^o$|) or \lstinline|$\uparrow$true| (of type \lstinline|bool$^o$|).
Other types (pairs, lists, user-defined algebraic datatypes, etc.) can be used in relational specifications as well, being injected by the same primitive. For example, expression \lstinline{^(1, "abc")} has the type \lstinline{(int * string)$^o$}, \lstinline{^[1; 2; 3]}~--- the type \lstinline{(int list)$^o$}, etc. The subtle part is that (since the unification only works for logical types) the placement of ``$^o$'' determines the granularity of unification. Indeed, a logical variable can only be placed where logical type is expected. Thus, in unification one can use a value of type \lstinline{(int * int)$^o$} as \emph{a whole}, but in order to control the \emph{contents} of the pair relationally, the type \lstinline{(int$^o$, int$^o$)$^o$} is required. This makes it impossible to reuse some built-in or standard types in relational code~--- for example, predefined list type is not flexible enough, since it does not allow the tail of the list to be logical. Instead, logical list type has to be introduced:

\begin{lstlisting}
   type $\alpha$ llist = Nil | Cons of $\alpha^o$ * ($\alpha$ llist)$^o$
\end{lstlisting}

With logical list type, we can implement some relations for lists:

\begin{lstlisting}
   val append : ($\alpha$ llist)$^o$ -> ($\alpha$ llist)$^o$ -> ($\alpha$ llist)$^o$ -> $\G$
   let rec append$^o$ x y xy =
     (x === ^Nil /\ xy === y) |||
     (fresh (h t ty)
        x  === ^(Cons (h, t)) /\
        xy === ^(Cons (h, ty)) /\
        append$^o$ t y ty
     ) 
\end{lstlisting}

Here we defined relational list concatenation \lstinline{append$^o$}, a canonical example in the field. This ternary relation is constructed,
using case analysis and recursion:

\begin{enumerate}
\item If the first list is empty, then the second and the third lists must be equal.
\item Otherwise, the first list can be split into a head and a tail, and two fresh variables \lstinline{h} and \lstinline{t} are needed to denote them.
We also need a fresh variable \lstinline{ty} to denote the list, such that appending \lstinline{y} to \lstinline{t} equals \lstinline{ty}. To ensure this
property, we use a recursive call to \lstinline{append$^o$}. Finally, we acquire the final result by consing \lstinline{h} and \lstinline{ty}. 
\end{enumerate}

The definition of \lstinline{append$^o$} takes three logical lists \lstinline{x}, \lstinline{y} and \lstinline{xy} as arguments, and constructs a goal,
which can be executed or combined with other goals. In the former case, a stream of \emph{answers} is returned. An element of the stream contains the description
of certain constraints for logical variables, which have to be respected in order for the relation to hold. We denote the running primitive ``$\leadsto$'', so

\begin{lstlisting}
  fresh (q) append$^o$ ^(Cons (^1, ^Nil)) q ^Nil $\leadsto$ []
\end{lstlisting}

\noindent returns an empty stream, since there is no list \lstinline{q}, such that appending \lstinline{Cons (1, Nil)} and \lstinline{q} gives empty list, while

\begin{lstlisting}
  fresh (q) append$^o$ q ^Nil ^(Cons (^1, ^Nil)) $\leadsto$ [q$\binds$Cons (1, Nil)]
\end{lstlisting}

\noindent discovers the expected constraint for the variable \lstinline{q}.

As it can be seen from the type, relational concatenation is polymorphic, like its functional counterpart. However, the query

\begin{lstlisting}
   append$^o$ ^(Cons (^$\lambda$x.x, ^Nil)) q ^(Cons (^$\lambda$y.y, ^Nil))  
\end{lstlisting}

\noindent ends with a run-time error due to inability to unify closures. This is a fundamental limitation in original \miniKanren as well,
as it deals only with first-order syntactic unification~\cite{Unification}. This example demonstrates, that, unlike pure OCaml, the typing
in OCanren is somewhat weak. In order to restore the strong typing, some of the type variables have to be bounded to range over only non-functional
types. The lack of direct support for bounded polymorphism~\cite{cardelli} in OCaml makes this step problematic. Our experience, however, shows, that
in practice this deficiency rarely gets in the way. In the following development, we assume, that in polymorphic types some type variables may be
implicitly bounded by the set of non-function types, and these boundings are respected in all instantiations of those type variables.

\begin{figure}[t]
  \centering
  \begin{subfigure}[t]{0.4\textwidth}
    \centering
\begin{lstlisting}
let rec append x y =
  match x with
  | Nil -> y
  | Cons (h, t) ->
     Cons (h, append t y)     
\end{lstlisting}
\caption{}
  \end{subfigure}
  ~
  \begin{subfigure}[t]{0.4\textwidth}
        \centering
\begin{lstlisting}
let rec append x y =
  match x with 
  | Nil -> y
  | Cons (h, t) -> 
     let ty = append t y in
     Cons (h, ty)
\end{lstlisting}
\vspace{-1\baselineskip}
\caption{}
  \end{subfigure}
  \vskip2mm
  \begin{subfigure}[t]{0.4\textwidth}
        \centering
\begin{lstlisting}
let rec append$^o$ x y xy =
  (t === ^Nil /\ xy === y) |||
  (fresh (h t ty)
    (x  === ^Cons (h, t)) /\
    (xy === ^Cons (h, ty)) /\
    (append$^o$ t y ty)
  )
\end{lstlisting}
\vspace{-1\baselineskip}
\caption{}
  \end{subfigure}
  \vskip4mm
\caption{Unnesting example}
\label{unnesting_example}
\end{figure}

Finally, we describe the unnesting technique~\cite{TRS}, which was introduced as a method for manual transformation
of functional programs into relational form. Unnesting introduces a new name for each nested subexpression; now, when the value of
each subexpression is bound to a certain variable, the conversion is straightforward: each pattern-matching construct is
transformed into a disjunction, new names, introduced in pattern bindings and unnestings, are transformed into \lstinline|fresh| variables, and
each converted function is supplied with the additional argument, unified with the result. As a result we consider, again, the list
concatenation function (see Fig.~\ref{unnesting_example}a). The result of unnesting is shown on Fig.~\ref{unnesting_example}b, while the
final relational form~--- on Fig.~\ref{unnesting_example}c.

\begin{figure}[t]
\centering
\begin{subfigure}[t]{0.4\textwidth}
 \centering
\begin{lstlisting}
let bar y =
  let f x = x in
  let g a = f in
  g A y
\end{lstlisting}
\vspace{-1\baselineskip}
    \caption{}
  \end{subfigure}
  ~
  \begin{subfigure}[t]{0.4\textwidth}
    \centering
\begin{lstlisting}
let bar$^o$ y r =
  let f x r = x === r in
  let g a r = f === r in
  g ^A y r
\end{lstlisting}
\vspace{-1\baselineskip}
  \caption{}
  \end{subfigure}
  \vskip4mm
\caption{Unnesting: invalid case}
\label{unnesting_invalid}
\end{figure}

However, not every definition can be converted to a relational form by unnesting. Consider, for example, the definition on Fig.~\ref{unnesting_invalid}a.
Unnesting would transform this program into the form, shown on Fig.~\ref{unnesting_invalid}b, which is obviously invalid, since it unifies a
function $f$ with a logical variable $r$. In order to apply unnesting, one needs to $\eta$-expand the definition of $g$, making the functional nature of
its return type syntactically visible.
We stress, that relational conversion, described in Section~\ref{conversion}, is essentially different from unnesting. In particular, 
we use $\eta$-expansion in a very limited manner (only in one case).

\begin{comment}

and for the aforementioned example the result of relational
conversion looks as follows:

\begin{lstlisting}
   let bar$^o$ y =
     let f x = x in
     let g a = f in
     g (fun q. q === ^A) y
\end{lstlisting}

Note, the majority of definitions are left intact; the only difference with the functional version comes from the use of the constructor 
\lstinline|A|, which was transformed into a goal-returning function.

\end{comment}

