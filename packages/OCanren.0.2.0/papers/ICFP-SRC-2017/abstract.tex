\documentclass[preprint,numbers,10pt]{sigplanconf}

\usepackage{makeidx}
\usepackage{amssymb}
\usepackage{listings}
\usepackage{indentfirst}
\usepackage{verbatim}
\usepackage{amsmath}

\usepackage{mathtools}
\usepackage{multirow}

\usepackage{graphicx}
\usepackage{color}
\usepackage{url}
\usepackage{xspace}
\usepackage[belowskip=-15pt,aboveskip=0pt]{caption}
%\usepackage{times}

%\setlength{\abovecaptionskip}{1ex}
%\setlength{\belowcaptionskip}{1ex}
%\setlength{\floatsep}{1ex}
%\setlength{\textfloatsep}{1ex}

\definecolor{light-gray}{gray}{0.90}
\newcommand{\graybox}[1]{\colorbox{light-gray}{\texttt{#1}}}
\newcommand{\miniKanren}{\texttt{miniKanren}\xspace}
\newcommand{\ocanren}{\texttt{OCanren}\xspace}
\newcommand{\ocaml}{\texttt{OCaml}\xspace}

\lstdefinelanguage{ocanren}{
keywords={fresh, let, in, match, with, when, class, type,
object, method, of, rec, repeat, until, while, not, do, done, as, val, inherit,
new, module, sig, deriving, datatype, struct, if, then, else, open, private, virtual, include, success, failure},
sensitive=true,
commentstyle=\small\itshape\ttfamily,
keywordstyle=\ttfamily\underbar,
identifierstyle=\ttfamily,
basewidth={0.5em,0.5em},
columns=fixed,
fontadjust=true,
literate={fun}{{$\lambda$}}1 {->}{{$\to$}}3 {===}{{$\equiv$}}1 {=/=}{{$\not\equiv$}}1 {|>}{{$\triangleright$}}3 {/\\}{{$\wedge$}}2 {\\/}{{$\vee$}}2 {^}{{$\uparrow$}}1,
morecomment=[s]{(*}{*)}
}

\lstset{
mathescape=true,
basicstyle=\small,
identifierstyle=\ttfamily,
keywordstyle=\bfseries,
commentstyle=\scriptsize\rmfamily,
basewidth={0.5em,0.5em},
fontadjust=true,
language=ocanren
}

\sloppy
\newcommand{\cc}{\mbox{ :: }}
\newcommand{\opt}[1]{\mbox{[ }#1\mbox{ ] }}
\newcommand{\etc}{\mbox{ }\ldots}
\newcommand{\term}[1]{\mbox{\texttt{#1}}}
\newcommand{\nt}[1]{$#1$}
\newcommand{\meta}[1]{\textsc{#1}}
\newcommand{\spc}{\mbox{ }}
\newcommand{\alt}{\mbox{ | }}
\newcommand{\trans}[3]{{#1}\vdash{#2}\to{#3}}
\newcommand{\cd}[1]{\texttt{#1}}
\newcommand{\kw}[1]{\texttt{\textbf{#1}}}
\newcommand{\comm}[1]{\textit{\small{#1}}}
\newcommand{\tv}[1]{\mbox{$'{#1}$}}
\newcommand{\inmath}[1]{\mbox{\lstinline{#1}}}
\newcommand{\goal}{\mathfrak G}

\begin{document}

%\mainmatter

\title{Improving Refutational Completeness of Relational Specifications\\
via Non-Termination Test\\
{\small (Extended Abstract)}}

\authorinfo{Dmitri Rozplokhas}
{St. Petersburg Academic University, St. Petersburg, Russia}
{rozplokhas@gmail.com\\
Postal address: Khlopina st., 8 bldg. 3, lit. A, 194021, St. Petersburg, Russia\\
Scientific advisor: Dmitri Boulytchev\\
ACM student member number: 9150868\\
Category: undergraduate
}

%\author{}
%\institute{St. Petersburg State University, St. Petersburg, Russia\\
%\email{lozov.peter@gmail.com}\\
%Postal address: 70/3, Botanicheskaya street, 198504, St. Petersburg, Russia\\
%Scientific advisor: Dmitri Boulytchev\\
%ACM student member number: 4970108\\
%Category: graduate}

\maketitle

%\begin{multicols}{2}
%\section{Introduction}

Relational programming is a technique, based on the idea of describing programs not as functions, but 
as relations, without distinguishing between the arguments and the result value. This technique makes it 
possible to ``query'' programs in various ways, for example, to execute them ``backwards'', finding
all sets of arguments for a given result. This behavior can be simulated using a number of
logic programming languages, such as Prolog, Mercury\footnote{\url{https://mercurylang.org}}, 
or Curry\footnote{\url{http://www-ps.informatik.uni-kiel.de/currywiki}}. 
There is also a family of embedded DSLs, specifically designed for writing declarative relational
programs, which originates from \miniKanren~\cite{TRS}. \miniKanren is a minimalistic 
declarative language initially developed for Scheme/Racket. The smallest implementation of \miniKanren 
is reported to comprise of only 40 LOC~\cite{MicroKanren, 2016}; there are also more elaborate versions, including
\miniKanren with constraints~\cite{CKanren, CKanren1}, user-assisted search~\cite{Guided}, nominal unification~\cite{AlphaKanren},
etc. Due to its simplicity \miniKanren was implemented for more than 50 other languages, such as
Haskell, Go, Smalltalk, and OCaml.

\miniKanren has proven to be a useful tool to provide elegant solutions for various problems, otherwise considered as non-trivial~\cite{WillThesis};
one of the main applications of \miniKanren for now lies is the area of \emph{relational interpreters}~--- the interpreters, which (for a given language)
are capable of providing programs, delivering specified results~\cite{Untagged}.

Despite being quite simple and easy to use by the design, in implementation \miniKanren introduces some subtleties. Under the hood, \miniKanren 
uses complete interleaving search~\cite{Search}. This search is guaranteed to find all existing solutions; however, it can diverge, when no 
solution exists. In reality this amounts to a divergence in a number of important cases~--- for example, when the program is asked to 
return \emph{all} existing solutions, or when the number of requested solutions exceeds the number of existing ones. Often, for a 
concrete query, it is possible to refactor the specification to avoid the divergence, but this means, that the specification has to be adjusted for every 
execution ``direction'' of interest, which to some extent compromises the idea of fully declarative relational programming. 

The specifications, which do not diverge even when no solutions exist, are called \emph{refutationally complete}~\cite{WillThesis}. Writing 
refutationally complete relational specifications nowadays requires the knowledge of \miniKanren implementation intrinsics, and is not always
possible due to the undecidability of the fundamental computability problems. However, by developing a more advanced search it is possible
to make more specifications refutationally complete.

In this work we present an optimization technique, which is based on a certain non-termination test. Our optimization is \emph{online} (performed during the
search), \emph{non-intrusive} (does not introduce new constructs and does not require any changes to be made to the existing specifications), and \emph{conservative} 
(applied only when the divergence is detected). We prove, that for the queries, which return a finite number of answers, our optimization is non-degrading. 
We also demonstrate the application of the optimization for a number of interesting and important problems.

%\section{Problem}

\begin{figure*}
\centering
\begin{tabular}{p{8cm}p{8cm}}
\begin{lstlisting}
   let rec append$^o$ x y xy =
     (x === [] /\ y === xy) \/
     (fresh (h t)
       (x === h :: t) /\
       (fresh (ty)
         (h :: ty === xy) /\ (append$^o$ t y ty)
       )
     )
\end{lstlisting} &
\begin{lstlisting}
   let rec revers$^o$ a b =     
     (a === [] /\ b === []) \/
     (fresh (h t)
       (a === h :: t) /\
       (fresh (a')
         (append$^o$ a' [h] b) /\ (revers$^o$ t a')
       )
     )
\end{lstlisting}
\end{tabular}
\caption{Relational List Concatenation and Reverse}
\label{example}
\end{figure*}

\begin{comment}
\begin{figure*}
\centering
\begin{tabular}{|l|c|c|c|}
    \hline
    Query & List lenght & Unoptimized search & Optimized search \\
    \hline
    \multirow{5}{*}{Sorting}
        & 3  & 3                   & 11     \\ \cline{2-4}
        & 4  & 183               & 48     \\ \cline{2-4}
        & 5  & $> 300000$  & 107   \\ \cline{2-4}
        & 10 & $> 300000$ & 1399 \\ \cline{2-4}
        & 15 & $> 300000$ & 9226 \\ \hline
    \multirow{4}{*}{Permutations generation}
        & 2   & 3                  & 3       \\ \cline{2-4}
        & 3   & 18                & 11     \\ \cline{2-4}
        & 4   & $> 300000$ & 81     \\ \cline{2-4}
        & 6   & $> 300000$ & 1193 \\ \hline
\end{tabular}
\end{figure*}
\end{comment}

In this work we use an implementation of \miniKanren for OCaml, called \ocanren~\cite{Ocanren}. A relational specification in \ocanren is written with the aid of
a few simple combinators and syntax extensions, which are designated to build \emph{goals}. A goal is a function, which can be run to produce a (lazy) stream of
\emph{answers}; each answer contains the bindings for some free variables of the goal. The set of combinators contains conjunction (``$\wedge$'') and disjunction
(``$\vee$'') of goals, fresh variable introduction, and unification (``$\equiv$'') of two terms. Other important constructs, such as function definitions and calls, are 
reused from the substrate language. An example of two conventional introductory relational definitions~--- list concatenation and reverse~--- is shown on 
Fig.~\ref{example}\footnote{Here we use a simplified syntax; the concrete syntax of \ocanren is a little bit different and requires additional comments and 
descriptions to be supplied}.

Given these definitions, we may perform some queries:

\begin{lstlisting}
run 1 (fun q -> append$^o$ [1; 2] [3] q) $\leadsto$ {q=[1; 2; 3]}
run 1 (fun q -> append$^o$ q [3] [1; 2; 3]) $\leadsto$ {q=[1; 2]}
run 1 (fun q -> revers$^o$ [1; 2; 3] q) $\leadsto$ {q=[3; 2; 1]}
run 1 (fun q -> revers$^o$ q [1; 2; 3]) $\leadsto$ {q=[3; 2; 1]}
\end{lstlisting}

Here in each query we requested only one answer; as we can see, all queries returned meaningful results. However, for the following query

\begin{lstlisting}
run * (fun q -> revers$^o$ [1; 2; 3] q)
\end{lstlisting}

\noindent the program diverges; here we requested \emph{all} answers, and, although one answer exists, \ocanren failed to ``convince'' itself, 
that this answer is unique. However, the following query

\begin{lstlisting}
run * (fun q -> revers$^o$ q [1; 2; 3]) $\leadsto$ {q=[3; 2; 1]}
\end{lstlisting}

\noindent converges with the single answer. The reason of this effect is that the calculation of conjunction is in fact non-commutative: the calculation 
of the first goal is performed in isolation, without any information about the results of the second goal, and then received information is transferred to the 
calculation of the second goal. Thus, when the first goal diverges and the second fails, the conjunction will diverge.

%\footnote{It does not, however, 
%\emph{always} diverge due to the interleaving; the actual behavior depends on the program.}. 

Our optimization is based on the idea of testing the current state of the search process for divergence, rolling back to some previously taken conjunction
(if the divergence was detected), and trying to swap the subgoals of that conjunction. For non-termination test we use the following consideration: if, during
the search, we arrived at a recursive function call with \emph{more general} arguments, than the enclosing one, then this call will diverge. In order to substantiate
this idea formally, we developed a simple big-step operational semantics of \miniKanren, which describes only the behavior of queries with finite number of answers.
This semantics introduces a relation $\Delta\vdash\sigma\xRightarrow{g} S$, where $\Delta$ is a set of function definitions, $\sigma$~--- current substitution, 
$g$~--- a goal, and $S$~--- a finite stream of answers. Using this semantics, we have proven, that if during the derivation of some branch we arrived at the 
following configuration

$$
\begin{array}{c}
  \Delta\vdash\sigma_2\xRightarrow{f(x_1,\dots,x_n)}\\
  \ldots\\
  \Delta\vdash\sigma_1\xRightarrow{f(y_1,\dots,y_n)} 
\end{array}
$$

\noindent and there is some substitution $\tau$, such that $y_i\sigma_1=(x_i\sigma_2)\tau$, then there is no $S$, such that $\Delta\vdash\sigma_2\xRightarrow{f(x_1,\dots,x_n)} S$.

Each time we encounter a divergence, we roll back to the nearest conjunction, which was not yet reordered, reorder it and repeat the search. The semantics of optimized search 
can be described by a relation $\Delta\vdash\sigma\xRightarrow{g}_{opt} S$ (with the similar meaning of the constituents), and we have proven the following claim: for arbitrary
$\Delta$, $\sigma$, $g$ and $S$ 

$$\Delta\vdash\sigma\xRightarrow{g}S\Rightarrow\Delta\vdash\sigma\xRightarrow{g}_{opt} S$$

\noindent which means, that our optimization preserves the refutational completeness.

We implemented our optimization as a prototype, based on \ocanren, but using a deep embedding, and evaluated it on a number of benchmarks:
basic relations for lists, including concatenation and reversing; list sorting and, at the same time, permutations generator;
multiplication of Peano numbers; simple binary arithmetic, including addition, multiplication, division and comparisons. 
For all these problems we used simple relations, which turned out to be refutationally complete with the optimized search, regardless the order of the goals. 
At the same time, these implementations (except for the list concatenation) are known to be not refutationally complete with 
non-optimized search.

\begin{thebibliography}{99}
\bibitem{TRS}
Daniel P. Friedman, William E.Byrd, Oleg Kiselyov. The Reasoned Schemer. The MIT
Press, 2005.

\bibitem{MicroKanren}
Jason Hemann, Daniel P. Friedman. $\mu$Kanren: A Minimal Core for Relational Programming //
Proceedings of the 2013 Workshop on Scheme and Functional Programming (Scheme '13).

\bibitem{Guided}
Cameron Swords, Daniel P. Friedman. rKanren: Guided Search in miniKanren //
Proceedings of the 2013 Workshop on Scheme and Functional Programming (Scheme '13). 

\bibitem{AlphaKanren}
William E. Byrd, Daniel P. Friedman. alphaKanren: A Fresh Name in Nominal Logic Programming //
Proceedings of the 2007 Workshop on Scheme and Functional Programming (Scheme '07).

\bibitem{2016}
Jason Hemann, Daniel P. Friedman, William E. Byrd, Matthew Might.
A Small Embedding of Logic Programming with a Simple Complete Search //
Proceedings of the 12th Symposium on Dynamic Languages (DLS 2016).

\bibitem{Untagged}
William E. Byrd, Eric Holk, Daniel P. Friedman.
miniKanren, Live and Untagged: Quine Generation via Relational Interpreters (Programming Pearl) //
Proceedings of the 2012 Workshop on Scheme and Functional Programming (Scheme '12).

\bibitem{CKanren}
Claire E. Alvis, Jeremiah J. Willcock, Kyle M. Carter, William E. Byrd, Daniel P. Friedman.
cKanren: miniKanren with Constraints // 
Proceedings of the 2011 Workshop on Scheme and Functional Programming (Scheme '11).

\bibitem{CKanren1}
Jason Hemann, Daniel P. Friedman. A Framework for Extending microKanren with Constraints //
Proceedings of the 2015 Workshop on Scheme and Functional Programming (Scheme '15).

\bibitem{WillThesis}
William E. Byrd. Relational Programming in miniKanren: Techniques, Applications, and Implementations.
Indiana University, Bloomington, IN, September 30, 2009.

%\bibitem{Barendregt}
%Henk Barendregt. Lambda Calculi with Types // Handbook of Logic in Computer Science, Volume II, Oxford University Press, 
%1993.

\bibitem{Search}
Oleg Kiselyov, Chung-chieh Shan, Daniel P. Friedman, Amr Sabry. Backtracking, Interleaving, and Terminating Monad Transformers (functional pearl) //
Proceedings of the 10th ACM SIGPLAN International Conference on Functional Programming (ICFP '05).

\bibitem{Ocanren}
Dmitry Kosarev, Dmitry Boulytchev. Typed Embedding of a Relational Language in OCaml // ACM SIGPLAN Workshop on ML, 2016.

%\bibitem{PE}
%Neil D. Jones, Carsten K. Gomard, Peter Sestoft. Partial Evaluation and Automatic Program Generation. Prentice-Hall, 1993.

%\bibitem{Turchin}
%Valentin F. Turchin. The Concept of a Supercompiler // ACM Transactions on Programming Languages and Systems, Vol.~8, 1986.

%\bibitem{Supercompilation}
%Morten Heine S{\o}rensen, Robert Gl\"{u}ck. Introduction to Supercompilation // Partial Evaluation~--- Practice and Theory, 
%DIKU 1998 International Summer School, 1999.
\end{thebibliography}
%\end{multicols}
\end{document}
