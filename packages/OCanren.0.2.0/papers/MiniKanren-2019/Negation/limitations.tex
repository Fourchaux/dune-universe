\section{Limitations and Future Work}

\label{sec:limitations}

In this section we discuss the limitations 
of constructive negation in general 
and our implementation in particular. 
Also we consider possible directions for future work.

\subsection{Type Constraints}

Although the program written in \textsc{OCanren} typechecks statically 
(thus, for example, preventing the user from unifying two terms of distinct types),
at runtime the type information is erased.
In the presence of even regular disequality constraints
it can lead to the incorrect results.
As an example, consider the following program:

\begin{minipage}[h]{\textwidth}
\begin{lstlisting}[
  % caption={An example of unsoundness in the presence of types},
  label={lst:types-unsound}
]
type bool = true | false

let g = 
  fresh (x y z : bool) (
    (x =/= y)
    (y =/= z)
    (z =/= x)
  )
\end{lstlisting}
\end{minipage}

The goal \lstinline{g} states that there
exists at least three different non-equal 
terms of type \lstinline{bool},
which, as we know, is not true.
Yet the query \lstinline{run g} will succeed.

In order to prevent unsoundness in cases like this,
type information in the form of \emph{type constraints}
should be somehow attached to the variables at runtime.
The satisfiability of type constraints then should 
be rechecked each time when the new disequality is added to some variable.
An extension of \textsc{OCanren} with type constraints
is a direction for future work.

\subsection{Non-stratified Programs}

As we have already discussed in the section~\ref{sec:strat}
our current implementation handles only stratified logic programs.
One of the possible extensions is to support 
non-stratified programs, such as one given in Listing~\ref{lst:game},
with respect to well-founded and/or stable model semantics 
(see section~\ref{sec:related-works} for the details).

\subsection{Negation of Goals With an Infinite Number of Answers}

Consider the following program:

\begin{minipage}[h]{\textwidth}
\begin{lstlisting}[
  % caption={Relation defining a list of zeros},
  label={lst:zeros}
]
let zeros l =
  l === [0] 
  \/
  fresh (l') (
    (l === 0 :: l')
    (zeros l')
  )
\end{lstlisting}
\end{minipage}

The unary relation \lstinline{zeros} defines lists consisting of zeros.
Now, intuitively, the query \lstinline{run ~(zeros q)} should enumerate
all lists that are not built out of zeros only.
Yet this query will fail to deliver even a single answer.
Why? Consider its operational behavior.
First the positive version of the goal, that is \lstinline{zeros q}, should be executed.
Then all answers to this goal should be collected and complemented.
However, there is an infinite number of answers to \lstinline{zeros q}
and thus this process will never terminate. 

It is a significant drawback of constructive negation
that the negation of the goal cannot be computed
if the goal has an infinite number of answers.
This limitation cannot be avoided in general,
however in some cases it is possible to narrow 
the number of answers to some subgoal 
by the reordering of surrounding subgoals.
For example, the query \lstinline{run ~(zeros q) /\ (q === [1])}
can be executed in finite time by the reordering of conjuncts.
It seems that the best strategy is to delay 
negative subgoals as long as possible,
but we do not have a formal proof of that.
