\documentclass[acmlarge]{acmart}

\usepackage[
    type={CC},           % your choice
    modifier={by},       % your choice
    version={4.0},       % your choice
]{doclicense}            % your choice, see \doclicenseThis below

\settopmatter{printacmref=false}
\fancyfoot{}

\makeatletter
\def\@formatdoi#1{}
\def\@permissionCodeOne{miniKanren.org/workshop}
\def\@copyrightpermission{\doclicenseThis} % your choice of text
\def\@copyrightowner{Copyright held by the author(s).} % your choice
\makeatother

\copyrightyear{2019}
\setcopyright{rightsretained}

% Metadata Information
\acmJournal{PACMPL}
% \acmVolume{1}
% \acmNumber{ICFP} % CONF = POPL or ICFP or OOPSLA
\acmYear{2019}
\acmMonth{8}
\acmArticle{4}
% \acmDOI{} % \acmDOI{10.1145/nnnnnnn.nnnnnnn}

% \startPage{1}

\usepackage{booktabs} 
\usepackage{listings}
\usepackage[justification=centering]{caption}
\usepackage{subcaption}
% \usepackage{cite}
\usepackage{amssymb}
\usepackage{amsmath}
\usepackage{amsthm}
\usepackage{mathtools}
\usepackage{xspace}
\usepackage{bussproofs}
\usepackage{tikz}
\usepackage{float}
\usepackage{graphicx}
\usepackage{array}
\usepackage{tabularx}
\usepackage{collcell}
\usepackage[ruled]{algorithm2e} 

% \SetAlFnt{\small}
% \SetAlCapFnt{\small}
% \SetAlCapNameFnt{\small}
% \SetAlCapHSkip{0pt}
% \IncMargin{-\parindent}

% Paper history
% \received{February 2007}
% \received{March 2009}
% \received[accepted]{June 2009}

\bibliographystyle{ACM-Reference-Format}

\citestyle{acmauthoryear}

\newcommand{\rulehskip}{\hskip 1.5em}
\newcommand{\rulevspace}{\vspace{1em}}

\newcommand{\pvfill}{\pause\vfill}

% \mathchardef\mhyphen="2D

\theoremstyle{definition}

\newtheorem{example}{Example}[section]
\newtheorem{definition}{Definition}
\newtheorem{lemma}{Lemma}
\newtheorem{remark}{Remark}
\newtheorem{theorem}{Theorem}

\newenvironment{subproof}[1][\proofname]{%
  \renewcommand{\qedsymbol}{$\blacksquare$}%
  \begin{proof}[#1]%
}{%
  \end{proof}%
}

% \newtheorem{prop}{Proposition}

%% \counterwithin{lemma}{section}

\newcommand{\textdef}[1]{\textit{#1}}

\newcommand{\imm}{{\textrm IMM}~}

% inline code 
\newcommand{\code}[1]{\texttt{#1}}

% tuple with angle brackets
\newcommand{\tup}[1]{\langle #1 \rangle}

% semantics brackets
\newcommand{\sem}[1]{\llbracket #1 \rrbracket}

% equality by definition
\newcommand{\defeq}{\triangleq}

% function arrow
\newcommand{\fun}{\rightarrow}

% partial function arrow
\newcommand{\pfun}{\rightharpoonup}

% some math sets
\newcommand{\N}{{\mathbb{N}}}
\newcommand{\Q}{{\mathbb{Q}}}

% domain/codomain notation
\newcommand{\dom}[1]{\textit{dom}{({#1})}}
\newcommand{\codom}[1]{\textit{codom}{({#1})}}

\newcommand{\isground}[1]{\textit{is\_ground}({#1})}

\newcommand{\mgu}{\textit{mgu}}

\newcommand{\vars}[1]{\textit{Vars}({#1})}

\newcommand{\sapp}[2]{{#2}{#1}}
\newcommand{\subs}{\sqsubseteq}

% some logical notation
%\newcommand{\implies}{{\Rightarrow}}
%\newcommand{\iff}{{\Leftrightarrow}}

% check-mark and cross-mark
\newcommand{\cmark}{\text{\color{green!60!black}\ding{51}}}
\newcommand{\xmark}{\text{\color{red!60!black}\ding{55}}}

%% axiom labels

\newcounter{mylabelcounter}

\makeatletter
\newcommand{\labelAxiom}[2]{%
\hfill{\normalfont\textsc{(#1)}}\refstepcounter{mylabelcounter}
\immediate\write\@auxout{%
  \string\newlabel{#2}{{\unexpanded{\normalfont\textsc{#1}}}{\thepage}{{\unexpanded{\normalfont\textsc{#1}}}}{mylabelcounter.\number\value{mylabelcounter}}{}}
}%
}
\makeatother

%% warning

\colorlet{colorWARNING}{yellow!90!black}

% \newcommand{\warning}[1]{{\color{colorWARNING}\texttt{WARNING}}: #1}
% \newcommand{\app}[1]{{\color{blue}\textbf{ANTON: #1}}}
% \newcommand{\note}[1]{{\color{cyan}\textbf{EVG: #1}}}

\newcommand\ExecScaleFactor{1}

\newcommand{\todo}[1]{{\color{red}\textbf{TODO: #1}}}

%% OCanren's listings

\lstdefinelanguage{ocanren}{
    keywords={fresh, let, in, match, with, when, class, type,
    object, method, of, rec, repeat, until, while, not, do, done, as, val, inherit,
    new, module, sig, deriving, datatype, struct, if, then, else, open, private, virtual, include, success, failure,
    true, false},
    sensitive=true,
    commentstyle=\small\itshape\ttfamily,
    identifierstyle=\ttfamily,
    keywordstyle=\bfseries,
    basewidth={0.5em,0.5em},
    columns=fixed,
    fontadjust=true,
    abovecaptionskip=\bigskipamount,
    literate={->}{{$\to$}}3 {===}{{$\equiv$}}1 {=/=}{{$\not\equiv$}}1 {|>}{{$\triangleright$}}3  {/\\}{{$\wedge$}}2 {\\/}{{$\vee$}}2 {^}{{$\uparrow$}}1 {'}{{$^\prime$}}1 {~}{{$\neg$}}1 {=>}{{$\Rightarrow$}}2, 
    morecomment=[s]{(*}{*)}
}

\lstset{
    mathescape=true,
    %basicstyle=\small,
    commentstyle=\scriptsize\rmfamily,
    language=ocanren,
    captionpos=b,
    % escapeinside={(*}{*)},
}

\newcolumntype{H}{>{\collectcell\lstinline}l<{\endcollectcell}}

% Document starts
\begin{document}

% Title portion
\title{Constructive Negation for MiniKanren} 

\author{Evgenii Moiseenko}
%\authornote{with author2 note}          %% \authornote is optional;
                                        %% can be repeated if necessary
\orcid{nnnn-nnnn-nnnn-nnnn}             %% \orcid is optional
\affiliation{
  %\position{Position2a}
  %\department{}             %% \department is recommended
  \institution{Saint Petersburg State University}           %% \institution is required
  %\streetaddress{Street2a Address2a}
  \city{Saint Petersburg}
  %\state{Russia}
  %$postcode{Post-Code2a}
  %\country{Russia}                   %% \country is recommended
}
\email{e.moiseenko@2012.spbu.ru}         %% \email is recommended
\affiliation{
  %\position{Position2b}
  %\department{Department2b}             %% \department is recommended
  \institution{JetBrains Research}           %% \institution is required
  %\streetaddress{Street3b Address2b}
  %\city{City2b}
  %\state{State2b}
  %\postcode{Post-Code2b}
  %\country{Country2b}                   %% \country is recommended
  \country{Russia}
}
%\email{first2.last2@inst2b.org}         %% \email is recommended

\begin{abstract}

We present an extension of \textsc{MiniKanren} with the negation operator
based on the method of \emph{constructive negation}.
The idea of this method is to constructively 
build a stream of answers for the negated goal
by collecting and negating individual answers
to the positive version of the goal.
As we demonstrate on the series of examples 
constructive negation suits to pure logical nature of \textsc{MiniKanren}:  
the relations involving the negation operator 
still can be ``run'' in various directions.

\end{abstract}


%
% The code below should be generated by the tool at
% http://dl.acm.org/ccs.cfm
% Please copy and paste the code instead of the example below. 
%
\begin{CCSXML}
  <ccs2012>
    <concept>
      <concept_id>10011007.10011006.10011008.10011009.10011012</concept_id>
      <concept_desc>Software and its engineering~Functional languages</concept_desc>
      <concept_significance>500</concept_significance>
    </concept>
    <concept>
      <concept_id>10011007.10011006.10011008.10011009.10011015</concept_id>
      <concept_desc>Software and its engineering~Constraint and logic languages</concept_desc>
      <concept_significance>500</concept_significance>
    </concept>
  </ccs2012>
\end{CCSXML}

\ccsdesc[500]{Software and its engineering~Functional languages}
\ccsdesc[500]{Software and its engineering~Constraint and logic languages}

\keywords{relational programming, constructive negation, OCanren}

\maketitle
\thispagestyle{empty}

\section{Introduction}

The introductory book on \textsc{miniKanren}~\cite{TRS} describes the language by means of an evolving set of examples. In the
series of follow-up papers~\cite{MicroKanren,CKanren,CKanren1,AlphaKanren,2016,Guided} various extensions of the language were presented with
their semantics explained in terms of \textsc{Scheme} implementation. We argue that this style of semantic definition is
fragile and not self-evident since it requires the knowledge of semantics of concrete implementation language. In addition the justification of
important properties of relational programs (for example, refutational completeness~\cite{WillThesis}) becomes cumbersome. In the
area of programming languages research a formal definition for the semantics of language of interest is a \emph{de-facto} standard, and
in our opinion in its current state \textsc{miniKanren} deviates from this standard.

There were some previous attempts to define a formal semantics for \textsc{miniKanren}. \citet{RelConversion} present a variant of nondeterministic
operational semantics, and~\citet{DivTest} use another variant of finite-set semantics. None of them was capable of reflecting
the distinctive property of \textsc{miniKanren} search~--- \emph{interleaving}~\cite{Search}, thus deviating from the conventional understanding
of the language.

In this paper we present a formal semantics for core \textsc{miniKanren} and prove some its basic properties. First,
we define denotational semantics similar to the least Herbrand model for definite logic programs~\cite{LHM}; then
we describe operational semantics with interleaving in terms of labeled transition system. Finally, we prove the soundness and
completeness of the operational semantics w.r.t the denotational one. We support our development with a formal specification
using \textsc{Coq}~\cite{Coq} proof assistant\footnote{\url{https://github.com/dboulytchev/miniKanren-coq}}, thus outsourcing
the burden of proof checking to the automatic tool. 

The paper organized as follows. In Section~\ref{language} we give the syntax of the language, describe its semantics
informally and discuss some examples. Section~\ref{denotational} contains the description of denotational semantics for
the language, and Section~\ref{operational}~--- the operational semantics. In Section~\ref{equivalence} we overview the
certified proof for soundness and completeness of operational semantics. The final section concludes.

\section{Examples}

In this section we present some examples, written with the aid of our library. In this examples we will use \cd{camlp5} syntax extension,
although \cd{ppxlib} plugin can be used equally. As we said, the library is a direct inheritor of our prior work~\cite{TransformationObjects}, and
all examples from that paper can be implemented using the new version. Here we show some more.

\subsection{Typed Logic Values}

The first example arose in the context of our work on strongly typed logical DSL for \textsc{OCaml}~\cite{OCanren}. One of the
most important construct there was a unification of terms with free logical variables, and dealing with such data structures
involves a lot of tedious and error-prone work. The typical scenario of interaction between a logical and non-logical worlds
is constructing a \emph{goal} containing a data structure with free logical variables and solving it. The solution
provides bindings for these variables, which, in optimistic scenario, do not contain free variables anymore. To construct
a goal one would need a systematic way to introduce logic variables in some typed data structure, and to recover answers~---
a systematic way to return to a plan, non-logical representation.

The (simplified) type for logic values can be defined as follows:

\begin{lstlisting}
   @type 'a logic =
   | V     of int
   | Value of 'a
   with show, gmap
\end{lstlisting}

A logic value can either be a free logic variable (``\lstinline{V}'') or a some other value (``\lstinline{Value}'') which is not
a free variable (but which can possibly contain free variables inside). To convert to- and from- the logic domain we can use the following
functions:

\begin{lstlisting}
   let lift x = Value x
  
   let reify  = function
   | V     _ -> invalid_arg "Free variable"
   | Value x -> x
\end{lstlisting}

The function ``\lstinline{reify}'' raises and exception on a free variable; indeed, if an occurrence of a free variable
is encountered the logic value can no longer be considered as a regular (non-logical) data structure and has to be interpreted
in some other way.

When we dealing with logic data structures we need to have an opportunity to put a free variable in an arbitrary
position. This means that we have to switch to another type, ``lifted'' into the logic domain. For example,
for arithmetic expressions, which we use as an example through the paper, we would need to construct a value like

\begin{lstlisting}
   Value (
     Binop (
       V 1, 
       Value (Const (V 2)),
       V 3
    )
   )
\end{lstlisting}

which has a type ``\lstinline{lexpr}'', defined as

\begin{lstlisting}
   type expr' = Var of string logic | Const of int logic | Binop of lexpr * lexpr
   and  lexpr = expr' logic
\end{lstlisting}

We also need to implement two conversion functions. All these definitions present a typical example of boilerplate code.

With our framework the solution is almost purely declarative\footnote{But we need to switch the compiler into \cd{-rectypes} mode}.
First, we abstract the type of interest, replacing all positions, in which we may desire to place a type variable, with
fresh type parameters:

\begin{lstlisting}
   @type ('string, 'int, 'expr) a_expr =
   | Var   of 'string
   | Const of 'int
   | Binop of 'string * 'expr * 'expr with show, gmap
\end{lstlisting}

Here we abstract the type of everything, but we could equally abstract it only of itself. Note, we make use of two
generic features~--- ``\lstinline{show}'' and ``\lstinline{gmap}''. The first one is needed for debugging purposes, while
the second is essential for our solution.

Now we can define the logical and non-logical counterparts as customised versions of the abstracted type:

\begin{lstlisting}
   @type expr  = (string, int, expr) a_expr with show, gmap
   @type lexpr = (string logic, int logic, lexpr) a_expr logic with show, gmap
\end{lstlisting}

Note, the ``new'' type ``\lstinline{expr}'' is equivalent to the ``old'' one, thus, this transformation makes no
harm to the existing code.

Finally, the definitions of conversion functions make use of the generic ``\lstinline{gmap}'' feature the
framework provides:

\begin{lstlisting}
   let rec to_logic   expr = gmap(a_expr) lift  lift  to_logic  expr
   let rec from_logic expr = gmap(a_expr) reify reify from_logic @@ reify expr
\end{lstlisting}

As we can see, the support for type constructor application is vital for the success of this scenario. In our prior
implementation~\cite{TransformationObjects} type constructor application was not supported and could not be easily added.

\subsection{Conversion to a Nameless Representation}

Polymorphic variant types make it possible to define composable statically typed and separately compiled data structures~\cite{PolyVarReuse}.
Dealing with them to implement composable statically typed and separately compiled transformations looks like a natural idea. The problem of
constructing transformations from separately compiled, strongly typed components is known as ``The Expression Problem''~\cite{ExpressionProblem}, which
is often used as a ``litmus test'' for generic programming frameworks~\cite{ObjectAlgebras,ALaCarte}. In this section we show the solution for
the expression problem with the aid of our framework. For a concrete problem we take the transformation from named to a nameless representations
for lambda terms.

First, we define the non-binding part of the terms:

\begin{lstlisting}
   @type ('name, 'lam) lam = [
   | `App of 'lam * 'lam
   | `Var of 'name
   ] with show
\end{lstlisting}

Separating this type looks a natural idea since potentially there can be many binding constructs (lambdas, lets, etc.) and by combining them
with the non-binding part (and with themselves) one can acquire a variety of languages with a coherent behaviour.

The type ``\lstinline{lam}'' is polymorphic: the first parameter is used to represent names or de Bruijn indices, the second one is needed
for open recursion (we here follow the known technique for describing extensible data structures with polymorphic variants~\cite{PolyVarReuse}).

What would the transformation to the nameless representation look like for this type? In our terms, what the transformation class is? It is shown
below:

\begin{lstlisting}
   class ['lam, 'nameless] lam_to_nameless
     (flam : string list -> 'lam -> 'nameless) =
   object
     inherit [string list, string, int,
              string list, 'lam, 'nameless,
              string list, 'lam, 'nameless] $\inbr{lam}$
     method $\inbr{App}$ env _ l r = `App (flam env l, flam env r)
     method $\inbr{Var}$ env _ x   = `Var (index env x)
   end
\end{lstlisting}

First, we use a list of strings as an environment, and we pass it as an inherited attribute. Then, we use a function ``\lstinline{index}'' to find a
position of a string in the environment (thus, it translates names to the de Bruijn indices). The interesting part is the typing of the common ancestor
class ``$\inbr{lam}$''. The first triple of its parameters describes the transformation for the first type parameter of the type. As we can see, we
transform strings into integers, using an environment. Next, the type variable ``\lstinline{'lam}'', as we know, will be set to the open version of the ``\lstinline{lam}''.
Finally, the result of the transformation is typed as ``\lstinline{'nameless}''. This is because the result will be, indeed, a different type, as we
will see soon. As the type parameter ``\lstinline{'lam}'' designates the type itself, the last three parameters repeat the next to last three.

Now we define a binding construct~--- abstraction:

\begin{lstlisting}
   @type ('name, 'lam) abs = [ `Abs of 'name * 'lam ] with show
\end{lstlisting}

The same reasoning applies here: we use an open recursion and a parameterization over name representation. The transformation class can be
implemented in a similar manner:

\begin{lstlisting}
  class ['lam, 'nameless] abs_to_nameless
    (flam : string list -> 'lam -> 'nameless) =
  object
    inherit [string list, string, int,
             string list, 'lam, 'nameless,
             string list, 'lam, 'nameless] $\inbr{abs}$
    method $\inbr{Abs}$ env name term = `Abs (flam (name :: env) term)
  end
\end{lstlisting}

Note, the method ``$\inbr{Abs}$'' constructs a value which has a \emph{different} type, than any parameterization of ``\lstinline{abs}''. Indeed, in a
nameless representation abstraction does not keep any name.

We can now combine two type definitions to build a type for terms with binders:

\begin{lstlisting}
   @type ('name, 'lam) term = [ ('name, 'lam) lam | ('name, 'lam) abs) ] with show
\end{lstlisting}

We can also provide two new types for named and nameless representation\footnote{We need to enable \cd{-rectypes} mode for these definitions to compile.}:

\begin{lstlisting}
   @type named    = (string, named) term with show
   @type nameless = [ (int, nameless) lam | `Abs of nameless] with show
\end{lstlisting}

Finally, we build a transformation for converting a named to a nameless representation:

\begin{lstlisting}
   class to_nameless
     (fself : string list -> named -> nameless) =
   object
     inherit [string list, named, nameless] $\inbr{named}$
     inherit [named, nameless] lam_to_nameless fself
     inherit [named, nameless] abs_to_nameless fself
   end
\end{lstlisting}

This transformation is constructed by inheriting all relevant counterparts: a common ancestor class for all transformations for the type ``\lstinline{named}'' and
two concrete transformations for its counterparts. The transformation function can be build in a standard way:

\begin{lstlisting}
   let to_nameless term =
     transform(named) (fun fself -> new to_nameless fself) [] term
\end{lstlisting}

Thus, we constructed a solution for a type from the solutions for its counterparts. This partial solutions can be separately compiled, and the whole
system remains strongly statically typed.

\subsection{A Custom Plugin}
\label{pluginExample}

Finally we demonstrate the utilisation of the plugin system using the example of a fresh custom plugin implementation. For this purpose we
take a well-known \emph{hash-consing} transformation~\cite{HC}. This transformation converts a data structure to its maximally shared
representation, when structurally equal substructures are represented by the same physical object. For example, an expression tree

\begin{lstlisting}
   let t =
     Binop ("+",
       Binop ("-",
         Var "b",
         Binop ("*", Var "b", Var "a")),
       Binop ("*", Var "b", Var "a"))
\end{lstlisting}

can be rewritten into

\begin{lstlisting}
   let t =
     let b  = Var "b" in
     let ba = Binop ("*", b, Var "a") in
     Binop ("+", Binop ("-", b, ba), ba)  
\end{lstlisting}

where equal sub expressions are represented by shared sub trees.

Our plugin for a type ``\lstinline|$\left\{\alpha_i\right\}$ t|'' will provide a hash-consing function ``\lstinline{hc(t)}'' of the type

\begin{lstlisting}
    $\{$ H.t -> $\alpha_i$ -> H.t * $\alpha_i$ $\}$ -> H.t -> $\left\{\alpha_i\right\}$ t -> H.t * $\left\{\alpha_i\right\}$ t
\end{lstlisting}

where ``\lstinline{H.t}''~--- a heterogeneous hash table for values of arbitrary types. The interface for the hash table is
as follows:

\begin{lstlisting}
   module H :
   sig
     type t
     val hc : t -> 'a -> t * 'a
   end
\end{lstlisting}

The function ``\lstinline{H.hc}'' takes a hash table and some value and returns a possibly updated table and a structurally equivalent value
of the same type. For now we postpone the description of this module implementation and consider an example of constructor transformation
method:

\begin{lstlisting}   
   method $\inbr{Binop}$ h _ op l r =
     let h, op = hc(string) h op in
     let h, l  = fself h l in
     let h, r  = fself h r in
     H.hc h (Binop (op, l, r))
\end{lstlisting}

The method takes an inherited attribute~---this time a hash table ``\lstinline{h}'',~--- the whole expressions node (which we do not
need in this case, hence underscore), and three arguments of the constructor: ``\lstinline{op}'' of type \lstinline{string}, and
``\lstinline{l}'' and ``\lstinline{r}'' of type \lstinline{expr}. We first hash-cons all three arguments (which gives us a possibly updated
hash table and three hash-consed values of the same types), then we apply the constructor and hash-cons the value again. To hash-cons
the arguments of the constructor we can use the functions provided by the framework~--- for the type \lstinline{string} it is
``\lstinline{hc(string)}''\footnote{Generally speaking, we would need to implement a hash-consing function for each primitive type; in
  our case, however, we could equally use ``\lstinline{H.hc}''.}, and for both sub expressions it is ``\lstinline{fself}''.

As a final component we need to decide on the type parameters for a plugin class for a type ``\lstinline|$\{\alpha_i\}$ t|''. Clearly,
all inherited attribute types has to be ``\lstinline{H.t}'', and synthesised attribute types has to be ``\lstinline{H.t * $a$}'' for the
type of interest ``$a$''. This gives us the following plugin class definition:

\begin{lstlisting}
   class [$\{\alpha_i\}$, $\epsilon$] $\inbr{hc_t}$ $\dots$ =
   object
     inherit [$\{$ H.t, $\alpha_i$, H.t * $\alpha_i$ $\}$, H.t, $\epsilon$, H.t * $\epsilon$] $\inbr{t}$
     $\dots$
   end
\end{lstlisting}

For simplicity we omitted the specification of functional parameters for the class since their types can be trivially
recovered.

Now we need to generate this logic using a plugin.

The infrastructure code for the plugin implementation is shown below:

\begin{lstlisting}
   let trait_name = "hc"
  
   module Make (AstHelpers : GTHELPERS_sig.S) =
     struct
     
       open AstHelpers

       module P = Plugin.Make (AstHelpers)

       class g tdecls =
       object (self : 'self)
         inherit P.with_inherited_attr tdecls as super
         $\ldots$
       end

     end

   let _ =
     Expander.register_plugin trait_name (module Make : Plugin_intf.Plugin)
\end{lstlisting}

To implement a plugin, one needs to implement a functor parameterised by a helper module, which resembles ``\cd{Ast_builder}'' from
\cd{ppxlib} to create \textsc{OCaml} syntax trees. We need to use a functor since we have to provide two implementations for
a plugin~--- for \cd{camlp5} syntax extension as well as for \cd{ppxlib} itself. The main entity in the body of the functor is
a class ``\lstinline{g}'' declaration (``generator''), which for simplicity can be inherited from one of generic classes 
from the framework. In this case we, first, instantiate the generic plugin ``\lstinline{P}'' for ``\lstinline{AstHelpers}'' and
then inherit from the class ``\lstinline{P.with_inherited_attr}'', which means that we are going to implement a plugin
making use of inherited attribute. The class takes a type declaration as a parameter. Finally, we register the functor as a
first-class module in the framework to make it accessible.

Now we show what the methods of the generator class look like. First, we need to specify what are the types of inherited and
synthesised attributes for the plugin:

\begin{lstlisting}
   method main_inh ~loc _tdecl = ht_typ ~loc

   method main_syn ~loc ?in_class tdecl =
     Typ.tuple ~loc
       [ ht_typ ~loc
       ; Typ.use_tdecl tdecl
       ]

   method inh_of_param tdecl _name =
       ht_typ ~loc:(loc_from_caml tdecl.ptype_loc)

   method syn_of_param ~loc s =
     Typ.tuple ~loc
       [ ht_typ ~loc
       ; Typ.var ~loc s
       ]
\end{lstlisting}

where we assume ``\lstinline{ht_typ}'' is defined as

\begin{lstlisting}
   let ht_typ ~loc =
     Typ.of_longident ~loc (Ldot (Lident "H", "t"))
\end{lstlisting}

In other words, we say here that the type of inherited attribute is always ``\lstinline{H.t}'' and the type of a synthesised attribute for
a type of interest ``\lstinline{t}'' is ``\lstinline{H.t * t}''.

The next group of methods specifies the behaviour of plugin class type parameters:

\begin{lstlisting}
   method plugin_class_params tdecl =
     let ps =
       List.map tdecl.ptype_params ~f:(fun (t, _) -> typ_arg_of_core_type t)
     in
     ps @
     [ named_type_arg ~loc:(loc_from_caml tdecl.ptype_loc) @@
       Naming.make_extra_param tdecl.ptype_name.txt
     ]

   method prepare_inherit_typ_params_for_alias ~loc tdecl rhs_args =
     List.map rhs_args ~f:Typ.from_caml
\end{lstlisting}

The first method specifies the type parameters for the plugin class itself: this time they are exactly the type parameters of the type declaration plus
the extra parameter ``$\epsilon$''. The second one describes the method of recalculation of type parameters for application of type constructor: when
the type declaration looks like

\begin{lstlisting}
   type $\{\alpha_i\}$ t = $\{a_i\}$ tc
\end{lstlisting}

we need to acquire the implementation of the plugin for ``\lstinline{t}'' from the implementation of the same plugin for ``\lstinline{tc}'', inheriting
from properly instantiated corresponding class. As for our plugin the class is parameterised by the same types as the type, we just keep the parameters.

The last group of methods generate the bodies of constructor transformation. As we support regular constructors with both tuple and record
argument specifications as well as top-level tuples and records, there are four methods, which as a rule share many details of implementation. We show the
skeleton for one of them:

\begin{lstlisting}
method on_tuple_constr ~loc ~is_self_rec ~mutual_decls ~inhe tdecl constr_info ts =
  $\dots$ 
  match ts with
  | [] -> Exp.tuple ~loc [ inhe; c [] ]
  | ts ->
     let res_var_name = sprintf "%s_rez" in
     let argcount = List.length ts in
     let hfhc =
       Exp.of_longident ~loc (Ldot (Lident "H", "hc"))
     in
     List.fold_right
       (List.mapi ~f:(fun n x -> (n, x)) ts)
       ~init:$\dots$
       ~f:(fun (i, (name, typ)) acc ->
            Exp.let_one ~loc
              (Pat.tuple ~loc [ Pat.sprintf ~loc "ht%d" (i+1)
                              ; Pat.sprintf ~loc "%s" @@ res_var_name name])
              (self#app_transformation_expr ~loc
                 (self#do_typ_gen ~loc ~is_self_rec ~mutual_decls tdecl typ)
                 (if i = 0 then inhe else Exp.sprintf ~loc "ht%d" i)
                 (Exp.ident ~loc name)
              )
              acc
          )
  $\dots$
\end{lstlisting}

This implementation makes use of the generic method ``\lstinline{self#app_transformation_expr}'' from the framework, which generates an application of
the transformation in question for a given type.

The final component for the implementation is module ``\lstinline{H}'' itself. The standard functor ``\lstinline{Hashtbl.Make}'' instantiates a
hash table making use of some hash function and equality predicate, supplied by an end user. In a whole, we follow a conventional pattern:
for the hash function we use polymorphic ``\lstinline{Hashtbl.hash}'' and for the equality we use physical equality ``\lstinline{==}''. There are, however, two
subtleties:

\begin{itemize}
\item Since our hash table is heterogeneous, we have to utilise unsafe coercion ``\lstinline{Obj.magic}''.
\item Our implementation for equality has to be a little more complex than simple ``\lstinline{==}'': we need to compare the top-level constructors and
  the number of their arguments \emph{structurally}, and only then compare the corresponding arguments by physical equality. Technically this
  may result in hash-consing structurally equal values of \emph{different} types.
\end{itemize}

We rely here on the follow observation: as hash-consing is only consistent with referentially-transparent data structures, we can assume
that structurally equal data structures can be interchangeable regardless their types. The complete implementation for this plugin can be seen in the main project
repository; it occupies 164 LOC, including comments and blank lines.

\section{Implementation}

\label{sec:implementation}

In this section we present our implementation of constructive negation.
We start with the general ideas behind the method (Section~\ref{sec:negation}).
We describe how the constructive negation behaves
in concrete examples, starting from trivial ones
and moving to more sophisticated.
During this presentation, we will observe, 
that in order to implement constructive negation, 
we need a solver for universally quantified disequality constraints.
We will show that such solver can be implemented 
on top of existing \textsc{MiniKanren} disequality solver
with just a few modifications (Section~\ref{sec:ctr-solver}).
In the Section~\ref{sec:search}, 
we describe how the \textsc{OCanren} search can be extended
to support constructive negation.
We will also discuss how negation interacts with recursion
and present the notion of \emph{stratification} (Section~\ref{sec:strat}). 

\subsection{General Ideas}

\label{sec:negation}

Constructive negation is based on the following idea: 
given a goal \lstinline{~g}, one can construct an answer for this goal 
by collecting all answers to its positive version \lstinline{g} 
and then taking their complementation.
In order to do that, a notion of ``negation'' of an answer is needed.
Since each answer can be matched to some logical formula~
\cite{przymusinski1989constructive, stuckey1991constructive}, 
a ``negation'' of an answer corresponds to the logical negation of
this formula. 

\begin{example}
  \label{ex:disequality}
  Consider the goal \lstinline{~(q === 1 /\ r === 2)}.
  Its positive version \lstinline{(q === 1) /\ (r === 2)} 
  has single answer, a substitution 
  $\mathtt{\{q \mapsto 1, r \mapsto 2 \}}$
  which corresponds to the formula $q = 1 \wedge r = 2$.
  By negating this formula we obtain $q \neq 1 \vee r \neq 2$.
  This formula still can be represent by a single substitution
  $\mathtt{\{q \mapsto 1, r \mapsto 2 \}}$.
  However, we now treat this substitution differently, as a \emph{disequality constraint}. 
\end{example}

Some \textsc{MiniKanren} implementations (including \textsc{OCanren})
already have the support for disequality constraints.
A programmer can use them with the help of 
\lstinline{=/=} primitive, as we have seen 
in the \lstinline{remove} example (Listing~\ref{lst:remove-correct}).
Usually, the support for disequalities is implemented as follows.

\begin{itemize}
  \item A current state is maintained during the search.
        The state consists of a substitution, 
        which represents positive information,
        and a disequality constraint store, 
        which represents negative information.
        A constraint store can be implemented simply 
        as a list of substitutions,
        or as a more efficient data structure.        

  \item Each time a subgoal of the form \lstinline{t =/= u} 
        is encountered during the search,
        its satisfiability in current substitution is checked.
        If it is satisfiable, then disequality 
        is added to the constraint store.

  \item Each time a subgoal of the form \lstinline{t === u} 
        is encountered during the search, 
        the current substitution is refined by the result 
        of unification of terms \lstinline{t} and \lstinline{u}. 
        Then the satisfiability of disequality constraints is rechecked 
        in the refined substitution.
\end{itemize}

Various optimizations can be applied to the scheme above.
For example, there is no need to recheck every 
disequality in the store after each unification.
We will not discuss these optimizations here, 
as they are irrelevant to our goals. 

Unfortunately, as the next example illustrates, 
disequalities presented above are not sufficient
to implement constructive negation.

\begin{example}
  \label{ex:univ-disequality}
  Consider a goal \lstinline{~(fresh (x) (q === (x, x))},
  which states that \lstinline{q} should not be equal to
  some pair of identical terms.
  The subgoal \lstinline{fresh (x) (q === (x, x))} succeeds,
  delivering the substitution $\mathtt{\{q \mapsto (x, x)\}}$.
  Because the variable \lstinline{x} occurs under \lstinline{fresh},
  the corresponding formula is existentially quantified:
  $\exists x\ldotp q = (x, x)$.
  By the negation of this formula we obtain 
  $\forall x\ldotp q \neq (x, x)$.
  This formula differs from disequality formula 
  from example~\ref{ex:disequality} as it 
  contains universally quantified variable \lstinline{x}.
\end{example}

Thereby, in order to support the negation of goals,
containing \lstinline{fresh},
we need to extend disequality constraint solver,
so it can check the satisfiability of 
\emph{universally quantified disequality constraints}
in the form $\forall \overline{x}\ldotp t \neq u$~\footnote{
$\overline{x}$ notation denotes a vector of variables,
$t$ and $u$ are terms that may or may not contain variables from $\overline{x}$
}~\cite{chan1988constructive, stuckey1991constructive}. 
Later, in Section~\ref{sec:ctr-solver} we will 
show how it can be done, for now let us assume we have such a solver. 

We took care about \lstinline{fresh} under negation.
It led us to a more complicated representation of the state. 
During the search we maintain a pair of a current substitution and 
a universally quantified disequality constraint store.
But now an interesting question arises: 
is this representation closed under negation? 
If we perform negation one more time, 
will we obtain a finite number of states in a similar form? 

Lucky for us, it is the case. In order to verify it, let us consider
a logical formula which corresponds to the representation of the state:

\begin{equation}
  \label{eq:state-positive}
  \exists\overline{x}\ldotp \left(
  \bigwedge_{i}(v_i = t_i) \wedge
  \bigwedge_{j}\forall\overline{y_j}\ldotp
  \bigvee_{k}(w_{jk} \neq u_{jk})
  \right)
\end{equation}

Here $v_i$ and $w_{jk}$ denote some variables,
$t_i$ and $u_{jk}$ denote some terms.
Existentially quantified variables $\overline{x}$
correspond to the variables occurred under \lstinline{fresh}.
The left conjunct corresponds to the substitution,
the right conjunct corresponds to the constraint store.
The constraint store itself is represented as a conjunction
of individual universally quantified disequalities.
As we have seen in example~\ref{ex:disequality},
each disequality corresponds to a disjunction
of individual inequalities over variables\footnote{
We will give further explanation in section~\ref{sec:ctr-solver}}.
Besides existential variables $\overline{x}$ 
and universal variables $\overline{y_j}$,
there are free variables $\overline{q}$ that may occur in the formula
(such as variables \lstinline{q} and \lstinline{r} in 
the examples~\ref{ex:disequality},~\ref{ex:univ-disequality}).

By logical negation of the above formula, we get the following:

\begin{equation}
  \label{eq:state-negation}  
  \forall\overline{x}\ldotp \left(
  \bigvee_{i}(v_i \neq t_i) \vee
  \bigvee_{j}\exists\overline{y_j}\ldotp
  \bigwedge_{k}(w_{jk} = u_{jk})
  \right)
\end{equation}

The left disjunct, which corresponded
to the substitution in the original formula, 
now became a disequality constraint. 
Looking at the right disjunct, 
we can see that each disequality 
has transformed into a substitution.
However, there is one subtlety here.
Variables $w_{jk}$ may be universally quantified,
that is $w_{jk} \in \overline{x}$ for some $j, k$.
Moreover, the terms $u_{jk}$ may contain universally quantified
variables as well, 
$\vars{u_{jk}} \subseteq \overline{x}$ for some $j, k$.
In the section~\ref{sec:ctr-solver} we will show,
that each disjunct  
$\exists\overline{y_j}\ldotp\bigwedge_{k}(w_{jk} = u_{jk})$ 
is either unsatisfiable or could be rewritten as 
$\exists\overline{y_j}\ldotp\bigwedge_{k^*}(w^*_{jk^*} = u^*_{jk^*})$,
such that universally quantified variables do not occur 
among variables $w^*_{jk^*}$ or in terms $u^*_{jk^*}$~\cite{liu1999constructive}. 

Taking this into account, we can rewrite the formula~\ref{eq:state-negation} as follow:

\begin{equation}
  \label{eq:state-negation-simpl}  
  (\bigvee_{j}\exists\overline{y_j}\ldotp
  \bigwedge_{k^*}(w^*_{jk^*} = u^*_{jk^*})) \vee
  \forall\overline{x}\ldotp
  \bigvee_{i}(v_i \neq t_i)
\end{equation}

In the obtained formula each disjunct corresponds to one state
in the form, similar to the given in formula~\ref{eq:state-positive}.
In the left disjunct, each sub-disjunct
corresponds to a substitution with an empty disequality constraint, 
the right disjunct corresponds to the 
single universally quantified disequality constraint 
with an empty substitution.
Thereby, the proposed representation of states
is closed under negation.

One can perform further manipulations on the formula~\ref{eq:state-negation-simpl}.
Given that equivalence $a \vee \neg b = a \wedge b \vee \neg b$ 
holds in classical logic, 
% and also 
% $\not\forall\overline{x}\ldotp\bigvee_{i}(u_i \neq t_i) = \exists\overline{x}\ldotp\bigwedge_{i}(u_i = t_i)$,
we can rewrite formula~\ref{eq:state-negation-simpl} in the following way:

\begin{equation}
  \label{eq:state-negation-rewritten}  
  (\exists\overline{x}\ldotp \bigwedge_{i}(v_i = t_i) \wedge 
   \bigvee_{j}\exists\overline{y_j}\ldotp
  \bigwedge_{k^*}(w^*_{jk^*} = u^*_{jk^*})) \vee
  \forall\overline{x}\ldotp
  \bigvee_{i}(v_i \neq t_i)
\end{equation}

The latter transformation,
while vacuous from the logical point of view,
could improve the performance of the search in practice.
It follows from the fact, that the subpart 
$\exists\overline{x}\ldotp \bigwedge_{i}(v_i = t_i)$ 
of the formula extends each substitution 
with additional mappings, 
thus delivering more positive information.
If the negation constitutes a subpart of some larger goal,
this positive information could lead to 
the earlier failure during the search.

Finally let us consider the negation in general case.
A goal can be matched to a logical formula in the following way.
Each answer to the goal corresponds to the state
which itself corresponds to a logical formula
in the form similar to one in formula~\ref{eq:state-positive}.
Goal can have multiple answers (even an infinite number).
In the corresponding logical formula these answers 
will be connected by the disjunction:

\begin{equation}
  \label{eq:positive-general}
  \bigvee_{n}\left(
  \exists\overline{x_{n}}\ldotp \left(
  \bigwedge_{i}(v_{ni} = t_{ni}) \wedge
  \bigwedge_{j}\forall\overline{y_{nj}}\ldotp
  \bigvee_{k}(w_{njk} \neq u_{njk})
  \right) \right)
\end{equation}

The negation of this formula after an application of the 
transformations described above will become:

\begin{equation}
  \label{eq:negation-general}  
  \bigwedge_{n}\left(
  (\exists\overline{x_{n}}\ldotp \bigwedge_{i}(v_{ni} = t_{ni}) \wedge 
   \bigvee_{j}\exists\overline{y_{nj}}\ldotp
  \bigwedge_{k^*}(w^*_{njk^*} = u^*_{njk^*})) \vee
  \forall\overline{x_n}\ldotp
  \bigvee_{i}(v_{ni} \neq t_{ni})
  \right)
\end{equation}

In this way, we have obtained the result of the negation of the goal
as a conjunction of the negation of individual answers to the 
positive version of the goal.
However, one of the pitfalls of this construction is 
that the process will not terminate if
the positive version of the goal has an infinite number of answers.
Thus, in the general case, it makes the \textsc{OCanren} search incomplete for the goals involving negation.

\subsection{Constraint Solver, Formally}

\label{sec:ctr-solver}

In this section we will formally define 
the satisfiability of universally quantified 
disequality constraints and quantified equalities,
mentioned in section~\ref{sec:negation}.
We will also present a simple decision procedure 
for satisfiability checking.
To do that we need a rather standard notions of  
terms, substitutions, unifiers, etc. 
For the sake of completeness we give these definitions here.
We also mention a standard unification algorithm as a
decision procedure for equality constraints.
This view of unification bridges the gap
between the conventional and constraint logic programming. 

Let us start with definitions of terms and substitutions.

\begin{definition}
  Given an infinite set of variables $V$ 
  and a finite set of constructor symbols $C = \{C^i_{n_i}\}_i$,
  each with associated arity $n_i$,
  the set of \emph{terms} $T$ is inductively defined as follow:
  \begin{itemize}
    \item $\forall v \in V \ldotp v \in T$ --- every variable is a term;
    \item $\forall c_k \in C\ldotp \forall t_1 \dots t_k \in T \ldotp c_k(t_1, \dots, t_k) \in T$ --- 
       every application of $k$-ary constructor symbol to $k$ terms is a term.
  \end{itemize}
\end{definition}

From now on we will assume that terms are untyped (unsorted)
and that there exists an infinitely many constructors of any arity.

\begin{definition}
  Two terms $t$ and $u$ are \emph{syntactically equal},
  denoted as $t = u$, iff either 
  \begin{itemize}
    \item $t = v$ and $s = v$ for some variable $v$;
    \item $t = c_k(t_1, \dots, t_k)$, $s = c_k(s_1, \dots, s_k)$ and
          $\forall i \in \{1 .. k\}\ldotp t_i = s_i$.
  \end{itemize}
\end{definition}

\begin{definition}
  A substitution $\sigma$ is a function 
  from variables to terms:
  $\sigma : V \fun T$,
  s.t. $\sigma(x) \neq x$ only for a finite number of variables.
  Every substitution $\sigma$ can be represented as 
  a finite list of pairs $\{x_1 \mapsto t_1, \dots, x_n \mapsto t_n \}$. 
  By $\dom{\sigma}$ we denote the set $\{x_1, \dots, x_n\}$
  and by $\codom{\sigma}$ we denote the set $\{t_1, \dots, t_n\}$.
  We denote empty substitution as $\top$.
  We also extend the set of substitutions defined above
  with the one additional element $\bot$.
\end{definition}

\begin{definition}
  A substitution can be \emph{applied to a term}.
  The result of an application of $\sigma$ ($\sigma \neq \bot$) to $t$,
  written as $\sapp{\sigma}{t}$, is a term defined 
  in the following way:
  \begin{itemize}
    \item $\sapp{\sigma}{x} \defeq \sigma(x)$;
    \item $\sapp{\sigma}{c_k(t_1, \dots, t_k)} \defeq c_k(\sapp{\sigma}{t_1}, \dots, \sapp{\sigma}{t_k})$.
  \end{itemize}
  The result of the application of $\bot$ to any term is undefined.
\end{definition}

\begin{lemma}
  \label{lemma:app}
  Given two substitutions $\sigma$ and $\theta$,
  if $\forall v \in V\ldotp \sigma(v) = \theta(v)$ 
  then $\forall t\ldotp \sapp{\sigma}{t} = \sapp{\theta}{u}$.
\end{lemma}

\begin{proof}
  Can be proved by the induction on $t$.
\end{proof}

We are now ready to define the  
the satisfiability of \emph{equality constraint}.

\begin{definition}
  \label{def:eq-ctr}
  Equality constraint $t \equiv u$ is 
  \begin{itemize}
    \item \emph{satisfiable} if $\exists \sigma\ldotp \sapp{\sigma}{t} = \sapp{\sigma}{u}$;
          such $\sigma$ is called a \emph{unifier} of $t$ and $u$;
    \item \emph{unsatisfiable} otherwise.
  \end{itemize}
\end{definition}

% \begin{remark}
%   \label{remark:ground-sat}
%   Strictly speaking, the definitions of satisfiability/unsatisfiability 
%   should involve ground terms. 
%   That is, proper definition of satisfiability of $t \equiv s$ should 
%   be given as follow:
%   \begin{itemize}
%     \item $t \equiv s$ \emph{satisfiable} if 
%           $\exists \sigma\ldotp \isground{\sigma} \wedge \sapp{\sigma}{t} = \sapp{\sigma}{s}$.
%   \end{itemize}
%   However, it is trivial to show, 
%   that in untyped case 
%   (assuming there is at least one 0-ary constructor symbol), 
%   if $t$ and $u$ are syntactically equal, 
%   then there exists a ground substitution $\sigma$,
%   such that $\sapp{\sigma}{t} = \sapp{\sigma}{s}$
%   Conversely, if $t$ and $u$ are not syntactically equal,
%   then there exists a ground substitution $\sigma$,
%   such that $\sapp{\sigma}{t} \neq \sapp{\sigma}{s}$
%   (with an additional assumption that there are at least two distinct constructor symbols).
%   Note that in the typed case it is not true.
%   For example, consider some variable $x$.
%   Clearly, $x$ is syntactically equal to itself, $x = x$.
%   Suppose that $x$ has type $\tau$.
%   If $\tau$ is uninhabited then there is no ground terms of type $\tau$,
%   and thus there is no ground substitution for $x$.
% \end{remark}

Next, we show that the standard unification algorithm
can be seen as a decision procedure 
for checking satisfiability of equality constrains.
Before that we need to introduce several definitions.

\begin{definition}
  A term $t$ is \emph{subsumed} by the term $u$, 
  denoted as $t \subs u$,
  iff $\exists \sigma\ldotp t = \sapp{\sigma}{u}$.
  If $t \subs u$ we will also say that 
  $t$ \emph{is a more specific} term than $u$,
  or $u$ \emph{is a more general} term than $t$.
\end{definition}

\begin{definition}
  A substitution $\sigma$ is \emph{subsumed} by a substitution $\tau$, 
  denoted as $\sigma \subs \tau$,
  iff $\forall t\ldotp \sapp{\sigma}{t} \subs \sapp{\tau}{t}$.
  If $\sigma \subs \tau$ we also say that 
  $\sigma$ \emph{is a more specific substitution} than $\tau$,
  or $\sigma$ \emph{is a more general substitution} than $\tau$.
\end{definition}

\begin{definition}
  \label{def:mgu}
  Given terms $t$ and $u$ their unifier $\sigma$ is called the
  \emph{most general unifier}, 
  iff for every other unifier $\tau$
  $\sigma$ is more general that $\tau$, $\tau \subs \sigma$.
\end{definition}

\begin{theorem}
  \label{lemma:unify}
  Given terms $t$ and $u$ they are either not unifiable 
  (meaning that $\forall \sigma\ldotp \sapp{\sigma}{t} \neq \sapp{\sigma}{u}$),
  or there exists their most general unifier.
\end{theorem}

\begin{proof}
  Proof of this statement can be found in~\cite{robinson1965machine}.
  Proof of the termination and correctness of the unification algorithm,
  used by the most \textsc{MiniKanren} implementations,
  can be found in~\cite{kumar2010nominal}.
\end{proof}

From now on we will denote the most general unifier
of two terms $t$ and $u$ as $\mgu(t, u)$.
In case of $t$ is not unifiable with $u$,
we assume that $\mgu(t, u) = \bot$.

\begin{remark}
  Note we can associate with an 
  equality constraint $t \equiv u$ a
  logical first-order formula $t = u$.
  % Then, given the remark~\ref{remark:ground-sat}, 
  % the satisfiability of the constraint $t \equiv u$
  % implies the satisfiability of the formula.
  Additionally, we can associate with each substitution a logical first-order formula
  by the following rules:
  \begin{itemize}
    \item empty substitution $\top$ is associated with the truth constant $\top$;
    \item $\bot$ is associated with the falsity constant $\bot$;
    \item $\{x_1 \mapsto t_1, \dots, x_n \mapsto t_n\}$ is associated with 
          the formula $x_1 = t_1 \wedge \dots \wedge x_n = t_n$.
  \end{itemize}
  Now we can have yet another view on unification.
  We can say that giving the problem of deciding satisfiability
  of the formula $t = u$, a unification algorithm reduces it 
  to checking satisfiability of a simpler formula,
  which corresponds to a substitution. 
  Such a formula is either trivially unsatisfiable 
  (in the case of $\bot$) or trivially satisfiable.
\end{remark}

% \begin{example}
%   \label{ex:equality-sat}
%   \todo{example of unification/equality constraint}
% \end{example}

Later on we will need the notion of idempotent substitution and idempotent unifier.

\begin{definition}
  \label{def:idempotent-subs}
  Substitution $\sigma$ is \emph{idempotent} iff
  $\forall t\ldotp \sapp{\sigma}{\sapp{\sigma}{t}} = \sapp{\sigma}{t}$
\end{definition}

\begin{lemma}
  \label{lemma:idempotent-unif}
  If two terms are unifiable, there exists their idempotent unifier.
\end{lemma}

\begin{proof}
  For the proof of this statement 
  (for the case of unification algorithm, used in \textsc{MiniKanren}),
  we refer an interested reader to~\cite{kumar2010nominal}.
\end{proof}

We are ready to move on to disequality constraints.
We start with the regular (not quantified) disequalities.

\begin{definition}
  \label{def:diseq-sat}
  A disequality constraint $t \not\equiv u$ is 
  \begin{itemize}
    \item \emph{satisfiable} if $\exists \sigma\ldotp \sapp{\sigma}{t} \neq \sapp{\sigma}{u}$;
    \item \emph{unsatisfiable} otherwise.
  \end{itemize}
\end{definition}

Next lemma gives us a simple decision procedure for checking 
satisfiability of disequalities.

\begin{lemma}
  \label{lemma:diseq-sat}
  Disequality constraint $t \not\equiv u$ is 
  \begin{itemize}
    \item \emph{satisfiable} if $\mgu(t, u) \neq \top$;
    \item \emph{unsatisfiable} otherwise.
  \end{itemize}
\end{lemma}

\begin{proof}
  Let $\theta = \mgu(t, u)$.
  Let us first show that if $\theta = \top$ 
  then disequality is unsatisfiable.
  By the definition of $\top$ we have 
  $\sapp{\theta}{t} = t$ and $\sapp{\theta}{u} = u$,
  by the definition of unifier 
  $\sapp{\theta}{t} = \sapp{\theta}{u}$, 
  and thus $t = u$.
  From that, it is easy to show that 
  $\forall \sigma\ldotp \sapp{\sigma}{t} = \sapp{\sigma}{s}$,
  which means that the disequality is unsatisfiable according to 
  the definition~\ref{def:diseq-sat}.
  If $\theta \neq \top$ then 
  there exists a substitution $\sigma$,
  s.t. $\theta \subs \sigma$ (e.g. $\sigma = \top$).
  Since $\theta$ is most general unifier, 
  and $\sigma$ is more general that $\theta$,
  then $\sigma$ is not a unifier,
  and thus $\sapp{\sigma}{t} \neq \sapp{\sigma}{u}$.
\end{proof}

Given lemma~\ref{lemma:diseq-sat}, the satisfiability of
constraint $t \not\equiv u$ can be checked easily.
One need to compute $\mgu(t, u)$
and if it is not an empty substitution,
then constraint is satisfiable.

\begin{remark}
  Given disequality constraint $t \not\equiv u$,
  a substitution $\mgu(t, u)$,
  can be matched to the logical formula in the following way:
  \begin{itemize}
    \item the empty substitution $\top$ is associated with falsity constant $\bot$;
    \item $\bot$ is associated with the truth constant $\top$;
    \item $\{x_1 \mapsto t_1, \dots, x_n \mapsto t_n\}$ is associated with 
          the formula $x_1 \neq t_1 \vee \dots \vee x_n \neq t_n$.
  \end{itemize}
\end{remark}

% \begin{example}
%   \label{ex:disequality-sat}
%   \todo{example of disequality constraint}
% \end{example}

The following definition introduces universally quantified disequalities.

\begin{definition}
  Universally quantified disequality constraint 
  $\forall \overline{x}\ldotp t \not\equiv u$ is
  \begin{itemize}
    \item \emph{satisfiable} iff
          ${\exists \sigma\ldotp 
            \forall \tau, \dom{\tau} \subseteq \overline{x}\ldotp 
            \sapp{\sigma}{\sapp{\tau}{t}} \neq \sapp{\sigma}{\sapp{\tau}{u}}
          }$
    \item \emph{unsatisfiable} iff
          ${\forall \sigma\ldotp 
            \exists \tau, \dom{\tau} \subseteq \overline{x}\ldotp 
            \sapp{\sigma}{\sapp{\tau}{t}} = \sapp{\sigma}{\sapp{\tau}{u}}
          }$
  \end{itemize}
\end{definition}

For the decision procedure of this type of disequalities,
we need one auxiliary lemma.

\begin{lemma}
  Given terms $t$ and $u$ consider $\theta = \mgu(t, u)$.
  Assume without the loss of generality that 
  $\overline{x} \not\subseteq \codom{\theta}$
  (if $v \mapsto x \in \theta$ for some $x \in \overline{x}$
   consider $\hat{\theta}$ such that it is equal to $\theta$
   except that instead of mapping $v$ to $x$
   it maps $x$ to $v$).
  Universally quantified disequality constraint 
  $\forall \overline{x}\ldotp t \not\equiv u$ is
  \begin{itemize}
    \item \emph{satisfiable} if
          $\theta = \bot$ or $\dom{\theta} \not\subseteq \overline{x}$ 
    \item \emph{unsatisfiable} if
          $\dom{\theta} \subseteq \overline{x}$ 
  \end{itemize}
\end{lemma}

\begin{proof}
  \label{lemma:univ-diseq-sat}

  First, let us show that 
  $\theta = \bot$ or $\dom{\theta} \not\subseteq \overline{x}$ 
  implies satisfiability.
  We need to show that 
  $\exists \sigma\ldotp 
            \forall \tau, \dom{\tau} \subseteq \overline{x}\ldotp 
            \sapp{\sigma}{\sapp{\tau}{t}} \neq \sapp{\sigma}{\sapp{\tau}{u}}$.
  Take $\sigma = \top$. 
  Thus $\sapp{\sigma}{\sapp{\tau}{t}} = \sapp{\tau}{t}$ and 
  $\sapp{\sigma}{\sapp{\tau}{u}} = \sapp{\tau}{u}$.
  It is left to show that 
  $\forall \tau, \dom{\tau} \subseteq \overline{x}\ldotp 
  \sapp{\tau}{t} \neq \sapp{\tau}{u}$.
  If $\theta = \bot$ then $t$ and $u$ are not unifiable,
  which implies the above statement.
  Otherwise, consider some $\tau$ such that 
  $\dom{\tau} \subseteq \overline{x}$.
  If $\sapp{\tau}{t} = \sapp{\tau}{u}$ then $\tau$ is a unifier 
  of $t$ and $u$. Thus $\tau \subs \theta$.
  If we will show that $\dom{\theta} \subseteq \dom{\tau} \subseteq \overline{x}$
  we will get a contradiction with the our assumptions and
  therefore $\sapp{\tau}{t} \neq \sapp{\tau}{u}$.
  Indeed, consider $v \in \dom{\theta}$. 
  If $v \not\in \dom{\tau}$ consider two cases:
  \begin{itemize}
    \item $v \mapsto w \in \theta$ for some $w \in V$.
      From the assumptions follows that $w \not\in \overline{x}$
      and thus $w \not\in \dom{\tau}$.
      Consider the term $f(v, w)$ for some binary constructor $f$.
      It is easy to see that \\
      ${\sapp{\tau}{f(v, w)} = f(v, w) \not\subs f(w, w) = \sapp{\theta}{f(v, w)}}$,
      which contradicts $\tau \subs \theta$. 
      Thus it should be that ${v \in \dom{\tau}}$.
    \item $v \mapsto s \in \theta$ for some term $s \not\in V$.
      Then, trivially 
      $\sapp{\tau}{v} = v \not\subs s = \sapp{\theta}{v}$,
      which contradicts $\tau \subs \theta$. 
      Thus it should be that $v \in \dom{\tau}$.
  \end{itemize}

  Finally, let us show that if $\dom{\theta} \subseteq \overline{x}$ 
  then ${\forall \sigma\ldotp 
            \exists \tau, \dom{\tau} \subseteq \overline{x}\ldotp 
            \sapp{\sigma}{\sapp{\tau}{t}} = \sapp{\sigma}{\sapp{\tau}{u}}
        }$.
  Indeed, consider some $\sigma$.
  Take $\tau = \theta$. 
  Then $\sapp{\tau}{t} = \sapp{\tau}{u}$ and thus
  $\sapp{\sigma}{\sapp{\tau}{t}} = \sapp{\sigma}{\sapp{\tau}{u}}$.
\end{proof}

By the above lemma, if the substitution $\mgu(t, u)$
maps only universally quantified variables,
then the disequality is unsatisfiable
and satisfiable otherwise.

Finally, it is left to show how to check satisfiability 
of the quantified equalities of the form 
$\forall \overline{x}\ldotp \exists \overline{y}\ldotp t \equiv u$.
As we will see soon, if the constraint of this 
form is satisfiable, then the logical formula,
corresponding to the constraint, 
$\forall \overline{x}\ldotp \exists \overline{y}\ldotp t = u$
is equivalent to the formula 
$\exists \overline{y^*}\ldotp \bigwedge_i v_i = t_i$
such that $v_i \in V$ and $v_i \not\in \overline{x}$ 
and $\vars{t_i} \cap \overline{x} = \emptyset$ for all $i$
\cite{liu1999constructive}.

\begin{definition}
  Quantified equality constraint of the form
  $\forall \overline{x} \exists \overline{y}\ldotp t \equiv u$ is
  \begin{itemize}
    \item \emph{satisfiable} iff
          ${\exists \sigma\ldotp 
            \forall \tau, \dom{\tau} \subseteq \overline{x}\ldotp 
            \exists \phi, \dom{\phi} \subseteq \overline{y}\ldotp 
            \sapp{\sigma}{\sapp{\tau}{\sapp{\phi}{t}}} = \sapp{\sigma}{\sapp{\tau}{\sapp{\phi}{u}}}
          }$
    \item \emph{unsatisfiable} iff
          ${\forall \sigma\ldotp 
            \exists \tau, \dom{\tau} \subseteq \overline{x}\ldotp 
            \forall \phi, \dom{\phi} \subseteq \overline{y}\ldotp 
            \sapp{\sigma}{\sapp{\tau}{\sapp{\phi}{t}}} \neq \sapp{\sigma}{\sapp{\tau}{\sapp{\phi}{u}}}
          }$
  \end{itemize}
\end{definition}

\begin{lemma}
  If terms $t$ and $u$ are not unifiable then the constraint is unsatisfiable.
  Otherwise let $\theta$ be an idempotent unifier of $t$ and $u$
  (Lemma~\ref{lemma:idempotent-unif} states that if terms are unifiable there exists their idempotent unifier).
  Let $\theta_{y} \defeq \{y \mapsto t ~|~ y \mapsto t \in \theta \wedge y \in \overline{y}\}$.
  Let $\hat{\theta} = \{ v \mapsto t ~|~ v \mapsto t \in \theta \wedge v \not\in \overline{y} \}$.
  Then the quantified equality constraint of the form
  $\forall \overline{x} \exists \overline{y}\ldotp t \equiv u$ is
  \begin{itemize}
    \item \emph{satisfiable} if
          ${ \dom{\hat{\theta}} \cap \overline{x} = \emptyset }$ and 
          ${ \forall p \in \codom{\hat{\theta}} \ldotp \vars{p} \cap \overline{x} = \emptyset }$
    \item \emph{unsatisfiable} if
          ${ \dom{\hat{\theta}} \cap \overline{x} \neq \emptyset }$ or   
          ${ \exists p \in \codom{\hat{\theta}} \ldotp \vars{p} \cap \overline{x} \neq \emptyset }$
  \end{itemize}
\end{lemma}

\begin{proof}
  First, it is obvious that if $t$ and $u$ are not unifiable then the constraint is unsatisfiable.
  Next, let us prove the statement involving satisfiability.
  We need to show that if the given condition is met, then 
  ${\exists \sigma\ldotp 
    \forall \tau, \dom{\tau} \subseteq \overline{x}\ldotp 
    \exists \phi, \dom{\phi} \subseteq \overline{y}\ldotp 
    \sapp{\sigma}{\sapp{\tau}{\sapp{\phi}{t}}} = \sapp{\sigma}{\sapp{\tau}{\sapp{\phi}{u}}}
  }$.
  Take $\sigma = \hat{\theta}$. 
  Given an arbitrary $\tau$ such that $\dom{\tau} \subseteq \overline{x}$ take $\phi$ to be equal to $\theta_y$.
  To complete the proof we will need an auxiliary statement.
  \begin{itemize}
    \item $\forall s\ldotp \sapp{\hat{\theta}}{\sapp{\tau}{s}} = \sapp{\hat{\tau}}{\sapp{\hat{\theta}}{s}}$
      where $\hat{\tau} \defeq \{x \mapsto \sapp{\hat{\theta}}{t} ~|~ x \mapsto t \in \tau \}$.
    \begin{subproof}
      By the Lemma~\ref{lemma:app} it is sufficient to show that
      $\forall v \in V\ldotp \sapp{\hat{\theta}}{\sapp{\tau}{v}} = \sapp{\hat{\tau}}{\sapp{\hat{\theta}}{v}}$.
      Consider the cases:
      \begin{itemize}
        \item $v \in \overline{x}$. Then $\sapp{\hat{\theta}}{\sapp{\tau}{v}} = \sapp{\hat{\theta}}{\tau(v)}$.
          Since $\dom{\hat{\theta}} \cap \overline{x} = \emptyset$, $\hat{\theta}(v) = v$.
          Then $\sapp{\hat{\tau}}{\sapp{\hat{\theta}}{v}} = \sapp{\hat{\tau}}{v}$ 
          and by the construction $\sapp{\hat{\tau}}{v} = \sapp{\hat{\theta}}{\tau(v)}$.
        \item $v \not\in \overline{x}$. Then $\sapp{\hat{\theta}}{\sapp{\tau}{v}} = \hat{\theta}(v)$.
          Since $\forall p \in \codom{\hat{\theta}}\ldotp \vars{p} \cap \overline{x} = \emptyset$, 
          $\sapp{\hat{\tau}}{\sapp{\hat{\theta}}{v}} = \sapp{\hat{\theta}}{v}$ 
          and trivially $\sapp{\hat{\theta}}{v} = \hat{\theta}(v)$.
      \end{itemize}
    \end{subproof}
  \end{itemize}

  By this statement 
  $\sapp{\hat{\theta}}{\sapp{\tau}{\sapp{\theta_y}{t}}} = \sapp{\hat{\tau}}{\sapp{\hat{\theta}}{\sapp{\theta_y}{t}}}$
  and 
  $\sapp{\hat{\theta}}{\sapp{\tau}{\sapp{\theta_y}{u}}} = \sapp{\hat{\tau}}{\sapp{\hat{\theta}}{\sapp{\theta_y}{u}}}$.
  Because $\theta$ is idempotent $\sapp{\theta}{t} = \sapp{\hat{\theta}}{\sapp{\theta_y}{t}}$ 
  and $\sapp{\theta}{u} = \sapp{\hat{\theta}}{\sapp{\theta_y}{u}}$. 
  Finally, since $\theta$ is a unifier $\sapp{\theta}{t} = \sapp{\theta}{u}$.

  It is left to prove the statement involving unsatisfiability.
  In fact, we will prove more general statement.
  \begin{itemize}
    \item Let $\tilde{t}$ and $\tilde{u}$ be two arbitrary unifiable terms, 
          let $\tilde{\theta}$ be their unifier. Then \\
          $\forall \overline{x} \subseteq \dom{\tilde{\theta}} \cup \bigcup_{p \in \codom{\tilde{\theta}}}Vars(p), 
           \overline{x} \neq \emptyset \ldotp 
           \forall \sigma \ldotp \exists \tau, \dom{\tau} \subseteq \overline{x}\ldotp 
             \sapp{\sigma}{\sapp{\tau}{\tilde{t}}} \neq \sapp{\sigma}{\sapp{\tau}{\tilde{u}}}
          $
    \begin{subproof}
      By the induction on $t$:
      \begin{itemize}
        \item $\tilde{t} = v$ for some $v \in V$. Consider the cases for $u$:
        \begin{itemize}
          \item $\tilde{u} = w$ for some $w \in V$.
            Then $\tilde{\theta} = \{v \mapsto w\}$ and either $v \in \overline{x}$ or $w \in \overline{x}$. \\
            Let the former be true (the other case is similar).
            Given some arbitrary $\sigma$ \\
            take ${\tau \defeq \{ v \mapsto z ~|~ z \in V \setminus (\dom{\sigma} \cup \bigcup_{p \in \codom{\sigma}} Vars(p))\}}$.
            Then $\sapp{\sigma}{\sapp{\tau}{v}} = z$ and
            $\sapp{\sigma}{\sapp{\tau}{w}} \neq z$.
          \item $\tilde{u} = g(\tilde{u_1}, \dots, \tilde{u_m})$ for some constructor $g$.
            Then $\tilde{\theta} = \{v \mapsto u\}$.
            If $v \in \overline{x}$ then pick some constructor $f \neq g$
            (because we assume there exists an infinite number of constructor symbols, we can always do it).
            Take ${\tau \defeq \{ v \mapsto f(z_1, \dots, z_n) ~|~ z_i \in V\}}$.
            Then $\sapp{\sigma}{\sapp{\tau}{v}} = f(\tilde{t'_1}, \dots, \tilde{t'_n})$
            and $\sapp{\sigma}{\sapp{\tau}{\tilde{u}}} = g(\tilde{u'_1}, \dots, \tilde{u'_m})$,
            and thus these terms are not equal.
            If $v \not\in \overline{x}$ then take some $x \in \overline{x}$. 
            For some $i$ it should be that $x \in \vars{\tilde{u_i}}$.
            Given $\sigma$ consider $\sigma(v)$, pick some $s$ such that 
            $\sigma(v) \neq \sapp{\{x \mapsto s\}}{g(\tilde{u_1}, \dots, \tilde{u_m})}$
            (it can be done by the induction on $\sigma(v)$).
            Then $\tau \defeq \{x \mapsto s\}$.
        \end{itemize}
        \item $\tilde{t} = f(\tilde{t_1}, \dots, \tilde{t_n})$. Consider the cases for $u$:
        \begin{itemize}
          \item $\tilde{u} = w$ for some $w \in V$. Then the proof proceeds in the same way as in the previous case.
          \item $\tilde{u} = f(\tilde{u_1}, \dots, \tilde{u_n})$ 
            ($\tilde{u}$ cannot be equal to some constructor $g \neq f$ by our assumption of unifiability of terms).
            Then by our assumption $\tilde{t_1} \equiv \tilde{u_1} \wedge \dots \wedge \tilde{t_n} \equiv \tilde{u_n}$.
            For some $i$ it should be the case that 
            $\overline{x} \subseteq \dom{\tilde{\theta_i}} \cup \bigcup_{p \in \codom{\tilde{\theta_i}}}Vars(p)$
            where $\theta_i \defeq \mgu(\tilde{t_i}, \tilde{u_i})$.
            By the induction for an arbitrary $\sigma$ there exists $\tau$ such that 
            $\sapp{\sigma}{\sapp{\tau}{\tilde{t_i}}} \neq \sapp{\sigma}{\sapp{\tau}{\tilde{u_i}}}$
            and thus $\sapp{\sigma}{\sapp{\tau}{\tilde{t}}} \neq \sapp{\sigma}{\sapp{\tau}{\tilde{u}}}$.
        \end{itemize}
      \end{itemize}
    \end{subproof}
  \end{itemize}
\end{proof}

  Given this lemma, we compute $\mgu(t, u)$ in order to check the constraint 
  $\forall \overline{x} \exists \overline{y}\ldotp t \equiv u$.
  By $\mgu(t, u)$ we can construct an idempotent substitution $\theta$ 
  that is also a unifier of $t$ and $u$.
  We take $\hat{\theta}$ --- a part of $\theta$
  that does not bind existentially quantified variables $\overline{y}$.
  Then we check if $\hat{\theta}$ binds variables from $\overline{x}$, 
  or some term from codomain of $\hat{\theta}$ contains variables from $\overline{x}$.
  If it does then the constraint is unsatisfiable, 
  because in this case we can always pick an assigment for $\overline{x}$
  that will make the terms not unifiable.
  Otherwise it is satisfiable.

\subsection{Extending the Search}

\label{sec:search}

In this section we describe how \textsc{OCanren} search interacts with negation.
Also we finally present the code of negation operator itself.

In \textsc{OCanren} (as in any typical \textsc{MiniKanren} implementation)
the search is implemented on top of 
backtracking lazy stream monad~\cite{kiselyov2005backtracking}.
We present the signature of the stream module in Listing~\ref{lst:stream} 
without the actual implementation to be short.
An interested reader may refer to~\cite{kiselyov2005backtracking, hemann2013mukanren}.
During the search the current state is maintained.
The state contains accumulated constraints plus
some supplementary information stored in the environment
(for example, the identifier of last allocated variable).
A goal is simply a function
which takes a state and returns a lazy stream of states.
The Listing~\ref{lst:goals} summarizes 
the implementation of logical primitives, 
such as individual constraints, conjunction, 
disjunction, and fresh variable introduction,
based on this representation of goals.

\begin{minipage}[t]{0.47\textwidth}
\begin{lstlisting}[
  caption={Signature of the Stream module},
  label={lst:stream}
]
module Stream : sig
  type 'a t

  val empty : 'a t
  val unit : 'a -> 'a t

  val map : 
    ('a -> 'b) -> 'a t -> 'b t

  val mplus : 'a t -> 'a t -> 'a t
  
  val bind : 
    'a t -> ('a -> 'b t) -> 'b t

  val msplit : 
    'a t -> ('a, 'a t) option
end
\end{lstlisting}
\end{minipage}\hfill
\begin{minipage}[t]{0.47\textwidth}
\begin{lstlisting}[
  caption={Implementation of the key primitives},
  label={lst:goals},
  escapeinside={(*}{*)},
]
type goal = State.t -> State.t Stream.t

let (===) : 'a -> 'a -> goal = 
  fun t u st ->
    match unify s t u with
    | Some st' -> Stream.unit st'
    | None     -> Stream.empty

let (=/=) : 'a -> 'a -> goal = 
  fun t u st ->
    match add_diseq s t u with
    | Some st' -> Stream.unit st'
    | None     -> Stream.empty

let (/\) : goal -> goal -> goal = 
  fun g$_1$ g$_2$ st ->
    Stream.bind (g$_1$ st) g$_2$

let (\/) : goal -> goal -> goal = 
  fun g$_1$ g$_2$ st ->
    Stream.mplus (g$_1$ st) (g$_2$ st)

let fresh : ('a -> goal) -> goal = 
  fun f st ->
    let x = fresh_var st in f x
\end{lstlisting}
\end{minipage}

\begin{minipage}[t]{0.47\textwidth}
\begin{lstlisting}[
  caption={Implementation of the auxiliary functions},
  label={lst:negation-aux-1},
  % numbers=left,
  % escapeinside={(*}{*)},
]
module State = struct
  type t = Env.t * Subst.t * CStore.t

  ...

  let diff (e, s, cs) (e', s', cs') = 
    let e  = Env.diff e e' in
    let s  = Subst.diff s s' in
    let cs = CStore.diff cs cs' in
    (e, s, cs)

  let merge (e, s, cs) (e', s', cs') = 
    let e = Env.merge e e' in
    match Subst.merge s s' with
    | None    -> None
    | Some s  ->
      let cs = CStore.merge cs cs' in
      match CStore.recheck s cs with
      | None    -> None
      | Some cs -> (e, s, cs) 
end
\end{lstlisting}
\end{minipage}\hfill
\begin{minipage}[t]{0.47\textwidth}
\begin{lstlisting}[
  caption={Implementation of the auxiliary functions},
  label={lst:negation-aux-2},
  % numbers=left,
  % escapeinside={(*}{*)},
]
module VarMap = Map.Make(Var)

module Subst = struct
  (* Substitution is a mapping from variables to terms *)
  type t = Term.t VarMap.t
  ...

  let diff s s' = 
    VarMap.fold (fun v t a -> 
      if not (VarMap.mem v s) then 
        VarMap.add v t a
      else a
    ) s' VarMap.empty

  let merge s s' = 
    VarMap.fold (fun v t -> function
      | None   -> None
      | Some a -> unify a v t
    ) s' (Some s)
end

module CStore = struct
  (* Constraint store is a list of substitutions *)
  type t = Subst.t list
  ... 

  (* cs' must be obtained from cs by 
   * the addition of new constraints  
   *)
  let diff cs cs' = 
    if cs' = cs then []
    else
      match cs' with 
      | _ :: cs' -> diff cs cs'
      (* cs' = [] implies cs = []  *)
      | _ -> assert false

  let merge cs cs' = 
    List.append cs' cs
end
\end{lstlisting}
\end{minipage}


\begin{minipage}[htb]{\textwidth}
\begin{lstlisting}[
  caption={Implementation of the negation operator},
  label={lst:negation},
  numbers=left,
  escapeinside={(*}{*)},
]
let (~) g st =                              (*\label{line:neg-def}*)
  let sts' = g st in                        (*\label{line:neg-pos}*)
  let cexs = Stream.map (diff st) sts' in   (*\label{line:neg-diff}*)
  let sub ss cex =                          (*\label{line:neg-sub}*)
    let ss' = negate cex in                 (*\label{line:neg-neg}*)
    Stream.bind ss  (fun s ->               (*\label{line:neg-binda}*)
    Stream.bind ss' (fun s' ->              (*\label{line:neg-bindb}*)
      Stream.unit (merge s s')              (*\label{line:neg-merge}*)
    ))
  in 
  Stream.fold sub (Stream.unit st) cexs     (*\label{line:neg-fold}*)
\end{lstlisting}
\end{minipage}

Now we can define the negation operator \lstinline{~} (see Listing~\ref{lst:negation}).
Let us describe it in details.

Negation operator is a function which takes
a goal and returns a negated goal.
Because a goal is itself a function taking a state,
\lstinline{(~)} takes two arguments:
the goal \lstinline{g} and the state \lstinline{st}
(line~\ref{line:neg-def}).

The first step of constructive negation
is to run the positive version of the goal,
as code in line~\ref{line:neg-pos} does.
We run it in the current state \lstinline{st}
and thus the call \lstinline{g st} returns
a stream of refined states \lstinline{sts'}.
Each state from this stream will contain 
the constraints from the original state \lstinline{st}
as its subpart.
However, we need to negate only constraints
originated from \lstinline{g} solely.
Thus, on the line~\ref{line:neg-diff} 
we map every state from the stream \lstinline{sts'}
to its \emph{difference} with respect to 
the original substitution \lstinline{st}.
In order to compute difference of two states \lstinline{st} and \lstinline{st'}
(Listing~\ref{lst:negation-aux-1}),
given that \lstinline{st} is more general that \lstinline{st'}
we need to compute difference of their substitutions and 
disequality constraints stores.
The difference of substitution \lstinline{s'} with respect to \lstinline{s}
(Listing~\ref{lst:negation-aux-2})
is just a substitution containing all mappings from \lstinline{s'}
which are not simultaneously in \lstinline{s}.
The difference of constraint store \lstinline{c'} with respect to \lstinline{c}
is a constraint store containing all disequalities
that are in \lstinline{c'} but not in \lstinline{c}
(Listing~\ref{lst:negation-aux-2}).
As long as persistent data structures are used 
to implement substitutions and constraint stores,
the \lstinline{diff} can be computed\footnote{
In the Listing~\ref{lst:negation-aux-2} 
we present a simple representation 
of the constraint store as a list of substitutions. 
In the actual implementation, we use 
more sophisticated representation, 
that also provides the \lstinline{diff} function.
The simpler version presented here 
gives some intuition 
on how to implement \lstinline{diff} for the constraint stores.
}.

Line~\ref{line:neg-sub} defines auxiliary function \lstinline{sub}
which takes two arguments: 
a stream of states \lstinline{ss}
and some state \lstinline{cex}, and 
returns another stream.
The purpose of this function it to 
``subtract'' \lstinline{cex} from 
every element of \lstinline{ss}.
It is done as follows.
First, the state \lstinline{cex} is negated 
as described in section~\ref{sec:negation} (line~\ref{line:neg-neg}).
As we have seen, as a result of the negation of a single
state the stream of states (the disjunction of formulas) can be obtained.
Thus the result of the call \lstinline{negate cex} is 
the stream of states \lstinline{ss'}.
% Then, via two calls to monadic bind the cross product
% of two streams, \lstinline{ss} and \lstinline{ss'}, is obtained
% (lines~\ref{line:neg-binda}-\ref{line:neg-bindb}).
For every combination of some state \lstinline{s} from 
the given stream \lstinline{ss} (line~\ref{line:neg-binda}) 
and some state \lstinline{s'} from the stream \lstinline{ss'} 
representing the result of negation (line~\ref{line:neg-bindb}) 
we compute their conjunction (line~\ref{line:neg-merge}).
The conjunction of two states is computed by
the function \lstinline{merge} 
(Listings~\ref{lst:negation-aux-1},~\ref{lst:negation-aux-2}).

Finally, on the line~\ref{line:neg-fold}
\lstinline{fold} is called on the stream \lstinline{cexs},
which is a stream of answers for the positive version
of the goal \lstinline{g},
with function \lstinline{sub} defined above 
and the initial accumulator \lstinline{Stream.unit st}.
The function \lstinline{Stream.fold} is implemented
as a regular left fold over a possibly infinite list.
Intuitively with folding over stream \lstinline{cexs} 
we ``subtract'' from the original state \lstinline{st}
every answer obtained from the goal \lstinline{g}.
 
\subsection{Stratification}

\label{sec:strat}

A negation, when combined with recursion, 
might become a source of confusion for a programmer.
Consider the program in Listing~\ref{lst:game}.
It encodes a two-players game.
The positions in the game are given as 
single-character strings 
\lstinline{'a'}, \lstinline{'b'}, \lstinline{'c'} and \lstinline{'d'}.
A binary relation \lstinline{move} encodes 
the game field as the set of possible moves.
An unary relation \lstinline{winning} 
determines the set of winning positions,
meaning that if the first player
starts from some winning position, 
by making ``good'' moves the player has an 
opportunity to win the game.
According to the definition of \lstinline{winning},
the position is winning if there exists
a move from this position to some non-winning position.
Clearly, every position with no moves from it is losing.

Given the suchlike definition of \lstinline{winning},
there is no doubt, that the position \lstinline{'d'} 
is losing position,
and thus the goal \lstinline{winning 'd'} should fail,
which, in turn, means that \lstinline{winning 'c'} should succeed.
However, whether the goal \lstinline{winning 'a'} (or \lstinline{winning 'b'})
should fail or succeed is not clear.

\begin{minipage}{\textwidth}
\begin{lstlisting}[
  caption={Encoding of two-players game},
  label={lst:game}
]
let move x y = 
  (x, y) === ('a', 'b') \/
  (x, y) === ('b', 'a') \/
  (x, y) === ('b', 'c') \/
  (x, y) === ('c', 'd') 

let winning x = 
  fresh (y) (
    (move x y) /\ ~(winning y)
  )
\end{lstlisting}
\end{minipage}

The problem with the semantics of program in Listing~\ref{lst:game},
originates from the interaction of negation and recursion.
Definition of the relation \lstinline{winning} refers
to itself under negation.
Logic programs that have this property 
are called \emph{non-stratified}~\cite{przymusinski1989constructive}.
Vice versa, programs that do not have loops over negation, are called \emph{stratified}.

% TODO: do we need that ???
% Let us define stratified programs more formally.
% As stratification is a syntactic property of programs,
% we first need to define the syntax of goals and relations.

Our current implementation handles only stratified programs.
We leave the task of supporting non-stratified programs
as a direction for future work.

\section{Evaluation}

\label{sec:evaluation}

In this section, we present an evaluation of 
implemented constructive negation on a series of examples.

\subsection{If-then-else}

Using relational if-then-else operator, 
presented in section~\ref{sec:ifte},
we have implemented several 
higher-order relations over lists, namely 
\lstinline{find} (Listing~\ref{lst:eval-find}), 
\lstinline{remove}\footnote{Note, this implementation 
differs from the one in Section~\ref{sec:intro}, but 
it is easy to see that these two are semantically equivalent.} (Listing~\ref{lst:eval-remove}) 
and \lstinline{filter} (Listing~\ref{lst:eval-filter}).
These relations are almost identical (syntactically) to their
functional implementations.
We have tested that these relations can be run
in various directions and produce the expected results.
For example, the goal \lstinline{filter p q q}
with the predicate \lstinline{p} equal to

\begin{lstlisting}
  fun l -> fresh (x) (l === [x])
\end{lstlisting}

stating that the given list should be a singleton list,
starts to generate all singleton lists.
Vice versa, the goal \lstinline{filter p q []} 
with that same \lstinline{p} generates 
all lists, constrained to be not a singleton list.

Listings~\ref{lst:eval-p}-\ref{lst:eval-filter-queries} give 
more concrete examples of queries to these relations.
In the listing the syntax \lstinline{run n q g}
means running a goal \lstinline{g} with 
the free variable \lstinline{q}
taking the first \lstinline{n} answers (``\lstinline{*}'' denotes all answers).
After the sign $\leadsto$ the result of the query is given.
The result \lstinline{fail} means that the query has failed.
The result \lstinline[mathescape]|succ {{a$_1$}; ... {a$_n$}} |
means that the query has succeeded delivering $n$ answers.
Each answer represents a set of constraint on free variables.
Constraints are of two forms: equality constraints, e.g. \lstinline{q = (1, _.$_0$)}, 
or disequality constraints, e.g. \lstinline{q $\neq$ (1, _.$_0$)}.
The terms of the form \lstinline{_.$_i$} in the answer
denote some universally quantified variables.

\begin{minipage}[thb]{.3\textwidth}
\begin{lstlisting}[
  caption={A definition of \code{find} relation},
  label={lst:eval-find}
]
let find p e xs =
  fresh (x xs' ys') (
    xs === x::xs' /\
    ifte (p x)
      (e === x)
      (find p e xs')
  )
\end{lstlisting}
\end{minipage}\hfill
\begin{minipage}[thb]{.3\textwidth}
\begin{lstlisting}[
  caption={A definition of \code{remove} relation},
  label={lst:eval-remove}
]
let remove p xs ys =
  (xs === [] /\ ys === [])
  \/
  fresh (x xs' ys') (
    xs === x::xs' /\
    ifte (p x)
      (ys === xs')
      (ys === x::ys' /\ 
       remove p xs' ys')
  )
\end{lstlisting}
\end{minipage}\hfill
\begin{minipage}[thb]{.3\textwidth}
\begin{lstlisting}[
  caption={A definition of \code{filter} relation},
  label={lst:eval-filter}
]
let filter p xs ys =
  (xs === [] /\ ys === [])
  \/
  fresh (x xs' ys') (
    xs === x::xs' /\
    (ifte (p x)
      (ys === x :: ys')
      (ys === ys')) /\
    filter p xs' ys'
  )
\end{lstlisting}
\end{minipage}

% \vspace{3cm}

\begin{minipage}[thb]{0.4\textwidth}
\begin{lstlisting}[
  caption={Definition of the predicate \lstinline{p}},
  label={lst:eval-p}
]
let p l = fresh (x) (l === [x])
\end{lstlisting}
\begin{lstlisting}[
  caption={Example of queries to \lstinline{find}},
  label={lst:eval-find-queries}
]
run 3 q (fresh (e) find p e q) 
$\leadsto$ succ {
     { q = [_.$_0$] :: _.$_1$ }
     { q = _.$_0$ :: [_.$_1$] :: _.$_2$; 
         _.$_0$ $\neq$ [_.$_3$] }
     { q = _.$_0$ :: _.$_1$ :: [_.$_2$] :: _.$_3$; 
         _.$_0$ $\neq$ [_.$_4$]; _.$_1$ $\neq$ [_.$_5$] }
   }
\end{lstlisting}
\end{minipage}\hfill
\begin{minipage}[thb]{0.4\textwidth}
\begin{lstlisting}[
  caption={Example of queries to \lstinline{remove}},
  label={lst:eval-remove-queries}
]
run * q (fresh (e) remove p q [[ ]]) 
$\leadsto$ succ {
     { q = [[_.$_0$]; [ ]] }
     { q = [[ ]] }
     { q = [[ ]; [_.$_0$]] }
   }

run 3 q (fresh (e) remove p q q) 
$\leadsto$ succ {
     { q = [] }
     { q = [_.$_0$], _.$_0$ $\neq$ [_.$_1$] }
     { q = [_.$_0$; _.$_1$]; 
         _.$_0$ $\neq$ [_.$_2$]; _.$_1$ $\neq$ [_.$_3$] }
   }
\end{lstlisting}
\end{minipage}

\begin{minipage}[thb]{0.4\textwidth}
\begin{lstlisting}[
  caption={Example of queries to \lstinline{filter}},
  label={lst:eval-filter-queries}
]
run 3 q (filter p q q) 
$\leadsto$ succ {
     { q = [ ] }
     { q = [_.$_0$] }
     { q = [_.$_0$; _.$_1$] }
   }

run 3 q (filter p q [1]) 
$\leadsto$ succ {
     { q = [[1]] }
     { q = [_.$_0$; [1]]; _.$_0$ $\neq$ [_.$_1$] }
     { q = [[1]; _.$_0$]; _.$_0$ $\neq$ [_.$_1$] }
   }

run 3 q (filter p q [ ]) 
$\leadsto$ succ {
     { q = [] }
     { q = [_.$_0$]; _.$_0$ $\neq$ [_.$_1$] }
     { q = [_.$_0$; _.$_1$]; 
            _.$_0$ $\neq$ [_.$_2$]; _.$_1$ $\neq$ [_.$_3$] }
   }
\end{lstlisting}
\end{minipage}

\subsection{Universal quantification}

In the Section~\ref{sec:impl-univ} we presented 
the \lstinline{forall} goal constructor 
which is implemented through the double negation.
We have observed, that although \lstinline{forall g}
does not terminate when the goal \lstinline{g x} 
has an infinite number of answers 
(assuming \lstinline{x} is a fresh variable),
it does terminate in the case when \lstinline{g x} has 
a finite number of answers.
The behavior of \lstinline{forall} in this case is sound
even in the presence of disequality constraints or nested quantifiers. 

The Table~\ref{tab:univ} gives some concrete examples.
The left column contains the tested goals\footnote{
We typeset the goals in terms of first-order logic syntax 
instead of \textsc{OCanren} syntax for brevity and clarity.} 
and the right column gives the obtained results.
For the results we use the same notation 
as in the previous section.

\begin{table}[th]
  \centering
  \def\arraystretch{1.5}
  \begin{tabularx}{\textwidth}{|X|X|}
    \hline

    $\forall x\ldotp x = q$ & 
      \texttt{fail} \\
    \hline

    $\forall x\ldotp \exists y\ldotp x = y$ & 
      \texttt{succ \{[q = \_.$_0$]\}} \\
    \hline

    $\forall x\ldotp \exists y\ldotp x = y \wedge y = q$ &
      \texttt{fail} \\
    \hline

    $\forall x\ldotp q = (1, x)$ & 
      \texttt{fail} \\
    \hline

    $\forall x\ldotp \exists y\ldotp y = (1, x)$ & 
      \texttt{succ \{[q = \_.$_0$]\}} \\
    \hline

    $\forall x\ldotp \exists y\ldotp x = (1, y)$ &
      \texttt{fail} \\
    \hline

    $\forall x\ldotp x \neq q$ & \texttt{fail} \\
    \hline

    $\forall x\ldotp \exists y\ldotp x \neq y$ & 
      \texttt{succ \{[q = \_.$_0$]\}} \\
    \hline

    $\forall x\ldotp \exists y\ldotp x \neq y \wedge y = q$ & 
      \texttt{fail} \\
    \hline

    $\forall x\ldotp q \neq (1, x)$ & 
      \texttt{succ \{[q $\neq$ (1, \_.$_0$)]\}} \\
    \hline

    $(\exists x\ldotp q = (1, x)) \wedge (\forall x\ldotp q \neq (1, x))$ & 
      \texttt{fail} \\
    \hline

    $\forall x\ldotp (x, x) \neq (0, 1)$ & 
      \texttt{succ \{[q = \_.$_0$]\}} \\
    \hline

    $\forall x\ldotp (x, x) \neq (1, 1)$ & 
      \texttt{fail} \\
    \hline

    $\forall x\ldotp (x, x) \neq (q, 1)$ & 
      \texttt{succ \{[q $\neq$ 1]\}} \\
    \hline

    $\exists a~ b\ldotp q = (a, b) \wedge \forall x\ldotp (x, x) \neq (a, b)$ & 
      \texttt{succ \{[q = (\_.$_0$, \_.$_1$); \_.$_0$ $\neq$ \_.$_1$]\}} \\
    \hline

  \end{tabularx}
  \caption{\lstinline{forall} evaluation}
  \label{tab:univ}
\end{table}

\section{Limitations and Future Work}

\label{sec:limitations}

In this section we discuss the limitations 
of constructive negation in general 
and our implementation in particular. 
Also we consider possible directions for future work.

\subsection{Type Constraints}

Although the program written in \textsc{OCanren} typechecks statically 
(thus, for example, preventing the user from unifying two terms of distinct types),
at runtime the type information is erased.
In the presence of even regular disequality constraints
it can lead to the incorrect results.
As an example, consider the following program:

\begin{minipage}[h]{\textwidth}
\begin{lstlisting}[
  % caption={An example of unsoundness in the presence of types},
  label={lst:types-unsound}
]
type bool = true | false

let g = 
  fresh (x y z : bool) (
    (x =/= y)
    (y =/= z)
    (z =/= x)
  )
\end{lstlisting}
\end{minipage}

The goal \lstinline{g} states that there
exists at least three different non-equal 
terms of type \lstinline{bool},
which, as we know, is not true.
Yet the query \lstinline{run g} will succeed.

In order to prevent unsoundness in cases like this,
type information in the form of \emph{type constraints}
should be somehow attached to the variables at runtime.
The satisfiability of type constraints then should 
be rechecked each time when the new disequality is added to some variable.
An extension of \textsc{OCanren} with type constraints
is a direction for future work.

\subsection{Non-stratified Programs}

As we have already discussed in the section~\ref{sec:strat}
our current implementation handles only stratified logic programs.
One of the possible extensions is to support 
non-stratified programs, such as one given in Listing~\ref{lst:game},
with respect to well-founded and/or stable model semantics 
(see section~\ref{sec:related-works} for the details).

\subsection{Negation of Goals With an Infinite Number of Answers}

Consider the following program:

\begin{minipage}[h]{\textwidth}
\begin{lstlisting}[
  % caption={Relation defining a list of zeros},
  label={lst:zeros}
]
let zeros l =
  l === [0] 
  \/
  fresh (l') (
    (l === 0 :: l')
    (zeros l')
  )
\end{lstlisting}
\end{minipage}

The unary relation \lstinline{zeros} defines lists consisting of zeros.
Now, intuitively, the query \lstinline{run ~(zeros q)} should enumerate
all lists that are not built out of zeros only.
Yet this query will fail to deliver even a single answer.
Why? Consider its operational behavior.
First the positive version of the goal, that is \lstinline{zeros q}, should be executed.
Then all answers to this goal should be collected and complemented.
However, there is an infinite number of answers to \lstinline{zeros q}
and thus this process will never terminate. 

It is a significant drawback of constructive negation
that the negation of the goal cannot be computed
if the goal has an infinite number of answers.
This limitation cannot be avoided in general,
however in some cases it is possible to narrow 
the number of answers to some subgoal 
by the reordering of surrounding subgoals.
For example, the query \lstinline{run ~(zeros q) /\ (q === [1])}
can be executed in finite time by the reordering of conjuncts.
It seems that the best strategy is to delay 
negative subgoals as long as possible,
but we do not have a formal proof of that.

\section{Related Works}
\label{sec:relworks}

There is a predictable difficulty in implementing \miniKanren for a strongly typed language.
Designed in the metaprogramming-friendly and dynamically typed realm of Scheme/Racket, the original
\miniKanren implementation pays very little attention to what has a significant importance in (specifically)
ML or Haskell. In particular, one of the capstone constructs of \miniKanren~--- unification~--- has to work for
different data structures, which may have types different beyond parametricity.

There are a few ways to overcome this problem. The first one is simply to follow the untyped paradigm and
provide unification for some concrete type, rich enough to represent any reasonable data structures.
Some Haskell \miniKanren libraries\footnote{\url{https://github.com/JaimieMurdock/HK}, \url{https://github.com/rntz/ukanren}}
as well as the previous OCaml implementation\footnote{\url{https://github.com/lightyang/minikanren-ocaml}} take this approach.
As a result, the original implementation can be retold with all its elegance; the relational specifications, however,
become weakly typed. A similar approach was taken in early works on embedding Prolog into Haskell~\cite{PrologInHaskell}.

Another approach is to utilize \emph{ad hoc} polymorphism and provide a type-specific unification for each ``interesting'' type.
Some \miniKanren implementations, such as Molog\footnote{\url{https://github.com/acfoltzer/Molog}} and
MiniKanrenT\footnote{\url{https://github.com/jvranish/MiniKanrenT}}, both for Haskell, can be mentioned as examples.
While preserving strong typing, this approach requires a lot of ``boilerplate''
code to be written, so some automation --- for example, using 
Template Haskell~\cite{SheardTMH}~---
is desirable. In~\cite{TypedLogicalVariables} a separate type class was introduced to both perform the unification
and detect free logical variables in end-user data structures. The requirement for end user to provide a way to represent
logical variables in custom data structures looks superfluous for us since these logical variables would require proper
handling in the rest of the code outside the logical programming subsystem.

There is, actually, another potential approach, but we do not know if anybody has tried
it: implementing unification for a generic representation of types as sum-of-products and fixpoints of
functors~\cite{InstantGenerics, ALaCarte}. Thus, unification would work for any type for which a representation
is provided. We believe that implementing this representation would require less boilerplate code to be written.

As follows from this exposition, a typed embedding of \miniKanren in OCaml can be done with
a combination of datatype-generic programming~\cite{DGP} and \emph{ad hoc} polymorphism. There are 
a number of generic frameworks for OCaml (for example,~\cite{Deriving}). On the other hand, the support
for \emph{ad hoc} polymorphism in OCaml is weak; there is nothing comparable in power to Haskell
type classes, and even though sometimes the object-oriented layer of the language can be used to mimic
desirable behavior, the result, as a rule, is far from satisfactory. Existing proposals for \emph{ad hoc} polymorphism (for example,
modular implicits~\cite{Implicits}) require patching the compiler, which we want to avoid. Therefore, we 
take a different approach, implementing polymorphic unification once and for all logical types~--- a purely \emph{ad hoc} 
approach, since the features which would provide a less \emph{ad hoc} solution are not yet well integrated into the language. To deal
with user-defined types in the relational subsystem, we propose to use their logical representations (see Section~\ref{sec:injection}), 
which free an end user from the burden of maintaining logical variables, and we use generic programming to build conversions from and to logical
representations in a systematic manner.




\section{Future Work}

There are a few possible directions for future work. First, in this paper we did not address the performance issues. As we represent
the transformations in a very generic form with many levels of indirection, obviously, the transformations, implemented with
our framework, are at disadvantage in comparison with hard coded ones in terms of performance. We assume that the performance of transformations
can be essentially improved by applying some techniques like staging~\cite{Staged} or, perhaps, object-specific optimisations.

Another important direction is supporting more kinds of type declarations, in the first hand, GADTs and non-regular types. Although we have some
implementation ideas for this case, the solution we came up with so far makes the interface of the whole framework too cumbersome to use even for
simple cases.

Finally, the typeinfo structure we generate can be used to mimic the \emph{ad-hoc} polymorphism as it contains the implementation of
type-indexed functions. This, together with some proposed extensions~\cite{ModularImplicits}, can open interesting perspectives.



%% Acknowledgments
\begin{acks}                            %% acks environment is optional
                                        %% contents suppressed with 'anonymous'
  %% Commands \grantsponsor{<sponsorID>}{<name>}{<url>} and
  %% \grantnum[<url>]{<sponsorID>}{<number>} should be used to
  %% acknowledge financial support and will be used by metadata
  %% extraction tools.

  % This material is based upon work supported by the
  % \grantsponsor{GS100000001}{National Science
  %   Foundation}{http://dx.doi.org/10.13039/100000001} under Grant
  % No.~\grantnum{GS100000001}{nnnnnnn} and Grant
  % No.~\grantnum{GS100000001}{mmmmmmm}.  Any opinions, findings, and
  % conclusions or recommendations expressed in this material are those
  % of the author and do not necessarily reflect the views of the
  % National Science Foundation.


  We want to thank Dmitry Boulytchev and Ekaterina Verbitskaia
  for valuable comments on a draft version of the paper.
  This work was partially supported by 
  the grant \grantnum{GS100000001}{18-01-00380}
  from the \grantsponsor{GS100000001}{Russian Foundation for Basic Research}{https://www.rfbr.ru/rffi/eng}
  and the grant from JetBrains Research.

\end{acks}

%% Appendix
% \appendix
% \section{Appendix}

% The default list of authors is too long for headers}
% \renewcommand{\shortauthors}{G. Zhou et al.}

%\setmonofont[Mapping=tex-text]{CMU Typewriter Text}
\bibliography{main}

\end{document}
