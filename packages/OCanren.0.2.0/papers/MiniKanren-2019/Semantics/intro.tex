\section{Introduction}

The introductory book on \textsc{miniKanren}~\cite{TRS} describes the language by means of an evolving set of examples. In the
series of follow-up papers~\cite{MicroKanren,CKanren,CKanren1,AlphaKanren,2016,Guided} various extensions of the language were presented with
their semantics explained in terms of \textsc{Scheme} implementation. We argue that this style of semantic definition is
fragile and not self-evident since it requires the knowledge of semantics of concrete implementation language. In addition the justification of
important properties of relational programs (for example, refutational completeness~\cite{WillThesis}) becomes cumbersome. In the
area of programming languages research a formal definition for the semantics of language of interest is a \emph{de-facto} standard, and
in our opinion in its current state \textsc{miniKanren} deviates from this standard.

There were some previous attempts to define a formal semantics for \textsc{miniKanren}. \citet{RelConversion} present a variant of nondeterministic
operational semantics, and~\citet{DivTest} use another variant of finite-set semantics. None of them was capable of reflecting
the distinctive property of \textsc{miniKanren} search~--- \emph{interleaving}~\cite{Search}, thus deviating from the conventional understanding
of the language.

In this paper we present a formal semantics for core \textsc{miniKanren} and prove some its basic properties. First,
we define denotational semantics similar to the least Herbrand model for definite logic programs~\cite{LHM}; then
we describe operational semantics with interleaving in terms of labeled transition system. Finally, we prove the soundness and
completeness of the operational semantics w.r.t the denotational one. We support our development with a formal specification
using \textsc{Coq}~\cite{Coq} proof assistant\footnote{\url{https://github.com/dboulytchev/miniKanren-coq}}, thus outsourcing
the burden of proof checking to the automatic tool. 

The paper organized as follows. In Section~\ref{language} we give the syntax of the language, describe its semantics
informally and discuss some examples. Section~\ref{denotational} contains the description of denotational semantics for
the language, and Section~\ref{operational}~--- the operational semantics. In Section~\ref{equivalence} we overview the
certified proof for soundness and completeness of operational semantics. The final section concludes.
