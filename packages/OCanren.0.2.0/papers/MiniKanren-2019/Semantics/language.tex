\begin{figure}[t]
\centering
\[
\begin{array}{ccll}
  \mathcal{C} & = & \{C_i^{k_i}\} & \mbox{constructors with arities} \\
  \mathcal{T}_X & = & X \cup \{C_i^{k_i} (t_1, \dots, t_{k_i}) \mid t_j\in\mathcal{T}_X\} & \mbox{terms over the set of variables $X$} \\
  \mathcal{D} & = & \mathcal{T}_\emptyset & \mbox{ground terms}\\
  \mathcal{X} & = & \{ x, y, z, \dots \} & \mbox{syntactic variables} \\
  \mathcal{A} & = & \{ \alpha, \beta, \gamma, \dots \} & \mbox{semantic variables} \\
  \mathcal{R} & = & \{ R_i^{k_i}\} &\mbox{relational symbols with arities} \\[2mm]
  \mathcal{G} & = & \mathcal{T_X}\equiv\mathcal{T_X}   &  \mbox{unification} \\
              &   & \mathcal{G}\wedge\mathcal{G}     & \mbox{conjunction} \\
              &   & \mathcal{G}\vee\mathcal{G}       &\mbox{disjunction} \\
              &   & \mbox{\lstinline|fresh|}\;\mathcal{X}\;.\;\mathcal{G} & \mbox{fresh variable introduction} \\
              &   & R_i^{k_i} (t_1,\dots,t_{k_i}),\;t_j\in\mathcal{T_X} & \mbox{relational symbol invocation} \\[2mm]
  \mathcal{S} & = & \{R_i^{k_i} = \lambda\;x_1^i\dots x_{k_i}^i\,.\, g_i;\}\; g & \mbox{specification}
\end{array}
\]
\caption{The syntax of the source language}
\label{syntax}
\end{figure}

\begin{figure}[t]
\centering
\[
\begin{array}{c}
  \mathcal{FV}\,(x)=\{x\}\\
  \mathcal{FV}\,(C_i^{k_i}\,(t_1,\dots,t_k))=\bigcup\mathcal{FV}\,(t_i)\\[2mm]
  \mathcal{FV}\,(t_1\equiv t_2)=\mathcal{FV}\,(t_1)\cup\mathcal{FV}\,(t_2)\\
  \mathcal{FV}\,(g_1\wedge g_2)=\mathcal{FV}\,(g_1)\cup\mathcal{FV}\,(g_2)\\
  \mathcal{FV}\,(g_1\vee g_2)=\mathcal{FV}\,(g_1)\cup\mathcal{FV}\,(g_2)\\
  \mathcal{FV}\,(\mbox{\lstinline|fresh|}\;x\;.\;g)=\mathcal{FV}\,(g)\setminus\{x\}\\
  \mathcal{FV}\,(R_i^{k_i}\,(t_1,\dots,t_k))=\bigcup\mathcal{FV}\,(t_i)
\end{array}
\]
\caption{Free variables in terms and goals}
\label{free}
\end{figure}

\section{The Language}
\label{language}

In this section we introduce the syntax of the language we use throughout the paper, describe the informal semantics and give some examples.

The syntax of the language is shown on Figure~\ref{syntax}. First, we fix a set of constructors $\mathcal{C}$ with known arities and consider
a set of terms $\mathcal{T}_X$ with constructors as functional symbols and variables from $X$. We parameterize this set with an alphabet of
variables, since in the semantic description we will need \emph{two} kinds of variables. The first kind, \emph{syntactic} variables, is denoted
by $\mathcal{X}$. We also consider an alphabet of \emph{relational symbols} $\mathcal{R}$ which are used to name relational definitions.
The central syntactic category in the language is a \emph{goal}. In our case there are five types of goals: \emph{unification} of terms,
conjunction and disjunction of goals, fresh variable introduction and invocation of some relational definition. Thus, unification is used
as a constraint, and multiple constraints can be combined using conjunction, disjunction and recursion. For the sake of brevity we
abbreviate immediately nested ``\lstinline|fresh|'' constructs into the one, writing ``\lstinline|fresh $x$ $y$ $\dots$ . $g$|'' instead of
``\lstinline|fresh $x$ . fresh $y$ . $\dots$ $g$|''. The final syntactic category is \emph{specification} $\mathcal{S}$. It consists of a set
of relational definitions and a top-level goal. A top-level goal represents a search procedure which returns a stream of substitutions for
the free variables of the goal. The definition for set of free variables for both terms and goals is given on Figure~\ref{free}; as ``\lstinline|fresh|''
is the sole binding construct the definition is rather trivial. The language we defined is first-order, as goals can not be passed as parameters,
returned or constructed at runtime.

We now informally describe how relational search works. As we said, a goal represents a search procedure. This procedure takes a \emph{state} as input and returns a
stream of states; a state (among other information) contains a substitution which maps semantic variables into terms over semantic variables. Then five types of
scenarios are possible (dependent on the type of the goal):

\begin{itemize}
\item Unification ``\lstinline|$t_1$ === $t_2$|'' unifies terms $t_1$ and $t_2$ in the context of the substitution in the current state. If terms are unifiable,
  then their MGU is integrated into the substitution, and one-element stream is returned; otherwise the result is an empty stream.
\item Conjunction ``\lstinline|$g_1$ /\ $g_2$|'' applies $g_1$ to the current state and then applies $g_2$ to the each element of the result, concatenating
  the streams.
\item Disjunction ``\lstinline|$g_1$ \/ $g_2$|'' applies both its goals to the current state independently and then concatenates the results.
\item Fresh construct ``\lstinline|fresh $x$ . $g$|'' allocates a new semantic variable $\alpha$, substitutes all free occurrences of $x$ in $g$ with $\alpha$, and
  runs the goal.
\item Invocation ``$\lstinline|$R_i^{k_i}$ ($t_1$,...,$t_{k_i}$)|$'' finds a definition for relational symbol $R_i^{k_i}=\lambda x_1\dots x_{k_i}\,.\,g_i$, substitutes
  all free occurrences of formal parameter $x_j$ in $g_i$ with term $t_j$ (for all $j$) and runs the goal in the current state.
\end{itemize}

We stipulate, that the top-level goal is preceded by an implicit ``\lstinline|fresh|'' construct, which binds all its free variables, and the final substitutions for these
variables constitute the result of the goal evaluation.

Conjunction and disjunction form a monadic~\cite{Monads} interface with conjunction playing role of ``bind'' and disjunction~--- of ``mplus''. In this description
we swept a lot of important details under the carpet~--- for example, in actual implementations the components of disjunction are not evaluated in isolation, but
both disjuncts are being evaluated incrementally with the control passing from one disjunct to another (\emph{interleaving}); instead streams the implementation
can be based on ``ferns''~\cite{BottomAvoiding} to defer divergent computations, etc. 

As an example consider the following specification:

\begin{lstlisting}
  append$^o$ = fun x y xy .
    ((x === Nil) /\ (xy === y)) \/
    (fresh h t ty .
       (x  === Cons (h, t))  /\
       (xy === Cons (h, ty)) /\
       (append$^o$ y t ty)
    );
  revers$^o$ = fun x y .
    ((x === Nil) /\ (y === Nil)) \/
    (fresh h t t' .
       (x === Cons (h, t)) /\
       (append$^o$ t' (Cons (h, Nil)) y) /\
       (revers$^o$ t t') 
    );
  revers$^o$ x x
\end{lstlisting}

Here we defined two relational symbols~--- ``\lstinline|append$^o$|'' and ``\lstinline|revers$^o$|'',~--- and specified a top-level goal ``\lstinline|revers$^o$ x x|''.
The symbol ``\lstinline|append$^o$|'' defines a relational concatenation of lists~--- it takes three arguments and performs a case analysis on the first one. If the
first one is an empty list (``\lstinline|Nil|''), then the second and the third arguments are unified. Otherwise the first argument is deconstructed into a head ``\lstinline|h|''
and a tail ``\lstinline|t|'', and the tail is concatenated with the second argument using a recursive call to ``\lstinline|append$^o$|'' and additional variable ``\lstinline|ty|'', which
represents the concatenation of ``\lstinline|t|'' and ``\lstinline|y|''. Finally, we unify ``\lstinline|Cons (h, ty)|'' with ``\lstinline|xy|'' to form a final constraint. Similarly,
``\lstinline|revers$^o$|'' defines relational list reversing. The top-level goal represents a search procedure for all lists ``\lstinline|x|'', which are stable under reversing, i.e.
represent palindromes. Running it results in an infinite stream of substitutions:

\begin{lstlisting}
   $\alpha\;\mapsto\;$ Nil
   $\alpha\;\mapsto\;$ Cons ($\beta_0$, Nil)
   $\alpha\;\mapsto\;$ Cons ($\beta_0$, Cons ($\beta_0$, Nil))
   $\alpha\;\mapsto\;$ Cons ($\beta_0$, Cons ($\beta_1$, Cons ($\beta_0$, Nil)))
   $\dots$
\end{lstlisting}

where ``$\alpha$''~--- a \emph{semantic} variable, corresponding to ``\lstinline|x|'', ``$\beta_i$''~--- free semantics variables.

The notions above can be formalized in \textsc{Coq} in a natural way using inductive data types. We have made a few non-essential simplifications and modifications for the sake of convenience.

Specifically, we restrict the arities of constructors to be either zero or two:

\begin{lstlisting}[language=Coq]
   Inductive term : Set :=
   | Var : var -> term
   | Cst : con_name -> term
   | Con : con_name -> term -> term -> term.
\end{lstlisting}

Here ``\lstinline[language=Coq]{var}'' and ``\lstinline[language=Coq]{con_name}''~--- types representing variables and constructor names, whose definitions we omitted for the sake of brevity.
Similarly, we restrict relations to always have exactly one argument:

\begin{lstlisting}[language=Coq]
   Definition rel : Set := term -> goal.
\end{lstlisting}

These restrictions do not make the language less expressive in any way since we can represent a sequence of terms as a list using constructors \lstinline|Nil$^0$| and \lstinline|Cons$^2$|.

We also introduce one additional type of goals~--- \emph{failure}~--- for deliberately unsuccessful computation (empty stream). As a result, the definition of goals looks as follows:

\begin{lstlisting}[language=Coq]
   Inductive goal : Set :=
   | Fail   : goal
   | Unify  : term -> term -> goal
   | Disj   : goal -> goal -> goal
   | Conj   : goal -> goal -> goal
   | Fresh  : (var -> goal) -> goal
   | Invoke : rel_name -> term -> goal.
\end{lstlisting}

Note that in our formalization we use the higher-order abstract syntax for variable binding~\cite{HOAS}. We preferred it to the first-order syntax because it gives us the ability
to use substitution and inductive principle provided by \textsc{Coq}. However, we still need to carefully ensure some expected properties on the structure of syntax trees.
For example, we should require that the definitions of relations do not contain unbound variables:

\begin{lstlisting}[language=Coq]
   Definition closed_goal_in_context (c : list var) (g : goal) : Prop :=
     forall n, is_fv_of_goal n g -> In n c.

   Definition closed_rel (r : rel) : Prop :=
     forall (arg : term), closed_goal_in_context (fv_term arg) (r arg).
 
   Definition def : Set := {r : rel | closed_rel r}.
\end{lstlisting}

In the snippet above ``\lstinline[language=Coq]{def}'' corresponds to a set of relational symbol definitions in a specification.

We set an arbitrary environment (a map from relational symbol to the definition of relation) to use further throughout the formalization. Failure goals allow us to define it as
a total function:

\begin{lstlisting}[language=Coq]
   Definition env : Set := rel_name -> def.

   Variable Prog : env.
\end{lstlisting}

