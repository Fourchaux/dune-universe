\documentclass[acmlarge]{acmart}
\usepackage[
    type={CC},           % your choice
    modifier={by-sa},    % your choice
    version={4.0},       % your choice
]{doclicense}            % your choice, see \doclicenseThis below

\settopmatter{printacmref=false}
\fancyfoot{}

\makeatletter
\def\@formatdoi#1{}
\def\@permissionCodeOne{miniKanren.org/workshop}
\def\@copyrightpermission{\doclicenseThis} % your choice of text
\def\@copyrightowner{Copyright held by the author(s).} % your choice
\makeatother

\copyrightyear{2019}
\setcopyright{rightsretained}

\acmMonth{8}
\acmArticle{5}

%% Bibliography style
\bibliographystyle{ACM-Reference-Format}
%% Citation style
%% Note: author/year citations are required for papers published as an
%% issue of PACMPL.
\citestyle{acmauthoryear}   %% For author/year citations


%%%%%%%%%%%%%%%%%%%%%%%%%%%%%%%%%%%%%%%%%%%%%%%%%%%%%%%%%%%%%%%%%%%%%%
%% Note: Authors migrating a paper from PACMPL format to traditional
%% SIGPLAN proceedings format must update the '\documentclass' and
%% topmatter commands above; see 'acmart-sigplanproc-template.tex'.
%%%%%%%%%%%%%%%%%%%%%%%%%%%%%%%%%%%%%%%%%%%%%%%%%%%%%%%%%%%%%%%%%%%%%%


%% Some recommended packages.
\usepackage{booktabs}   %% For formal tables:
                        %% http://ctan.org/pkg/booktabs
\usepackage{subcaption} %% For complex figures with subfigures/subcaptions
                        %% http://ctan.org/pkg/subcaption


\usepackage{amsmath,amssymb}
\usepackage[russian,english]{babel}
\usepackage{amssymb}
\usepackage{mathtools}
\usepackage{listings}
\usepackage{comment}
\usepackage{indentfirst}
\usepackage{hyperref}
\usepackage{amsthm}
\usepackage{stmaryrd}
\usepackage{eufrak}
\usepackage{lstcoq}

\newtheorem{theorem}{Theorem}
\newtheorem{lemma}{Lemma}
\newtheorem{corollary}{Corollary}
\newtheorem{hyp}{Hypethesis}
\newtheorem{definition}{Definition}

\lstdefinelanguage{minikanren}{
keywords={fresh},
sensitive=true,
commentstyle=\small\itshape\ttfamily,
keywordstyle=\textbf,
identifierstyle=\ttfamily,
basewidth={0.5em,0.5em},
columns=fixed,
fontadjust=true,
literate={fun}{{$\lambda\;\;$}}1 {->}{{$\to$}}3 {===}{{$\,\equiv\,$}}1 {=/=}{{$\not\equiv$}}1 {|>}{{$\triangleright$}}3 {/\\}{{$\wedge$}}2 {\\/}{{$\vee$}}2,
morecomment=[s]{(*}{*)}
}

\lstset{
mathescape=true,
language=minikanren
}

\usepackage{letltxmacro}
\newcommand*{\SavedLstInline}{}
\LetLtxMacro\SavedLstInline\lstinline
\DeclareRobustCommand*{\lstinline}{%
  \ifmmode
    \let\SavedBGroup\bgroup
    \def\bgroup{%
      \let\bgroup\SavedBGroup
      \hbox\bgroup
    }%
  \fi
  \SavedLstInline
}

\def\transarrow{\xrightarrow}
\newcommand{\setarrow}[1]{\def\transarrow{#1}}

\def\padding{\phantom{X}}
\newcommand{\setpadding}[1]{\def\padding{#1}}

\def\subarrow{}
\newcommand{\setsubarrow}[1]{\def\subarrow{#1}}

\newcommand{\trule}[2]{\frac{#1}{#2}}
\newcommand{\crule}[3]{\frac{#1}{#2},\;{#3}}
\newcommand{\withenv}[2]{{#1}\vdash{#2}}
\newcommand{\trans}[3]{{#1}\transarrow{\padding{\textstyle #2}\padding}\subarrow{#3}}
\newcommand{\ctrans}[4]{{#1}\transarrow{\padding#2\padding}\subarrow{#3},\;{#4}}
\newcommand{\llang}[1]{\mbox{\lstinline[mathescape]|#1|}}
\newcommand{\pair}[2]{\inbr{{#1}\mid{#2}}}
\newcommand{\inbr}[1]{\left<{#1}\right>}
\newcommand{\highlight}[1]{\color{red}{#1}}
%\newcommand{\ruleno}[1]{\eqno[\scriptsize\textsc{#1}]}
\newcommand{\ruleno}[1]{\mbox{[\textsc{#1}]}}
\newcommand{\rulename}[1]{\textsc{#1}}
\newcommand{\inmath}[1]{\mbox{$#1$}}
\newcommand{\lfp}[1]{fix_{#1}}
\newcommand{\gfp}[1]{Fix_{#1}}
\newcommand{\vsep}{\vspace{-2mm}}
\newcommand{\supp}[1]{\scriptsize{#1}}
\newcommand{\sembr}[1]{\llbracket{#1}\rrbracket}
\newcommand{\cd}[1]{\texttt{#1}}
\newcommand{\free}[1]{\boxed{#1}}
\newcommand{\binds}{\;\mapsto\;}
\newcommand{\dbi}[1]{\mbox{\bf{#1}}}
\newcommand{\sv}[1]{\mbox{\textbf{#1}}}
\newcommand{\bnd}[2]{{#1}\mkern-9mu\binds\mkern-9mu{#2}}
\newcommand{\meta}[1]{{\mathcal{#1}}}
\newcommand{\dom}[1]{\mathtt{dom}\;{#1}}
\newcommand{\primi}[2]{\mathbf{#1}\;{#2}}
\renewcommand{\dom}[1]{\mathcal{D}om\,({#1})}
\newcommand{\ran}[1]{\mathcal{VR}an\,({#1})}
\newcommand{\fv}[1]{\mathcal{FV}\,({#1})}
\newcommand{\tr}[1]{\mathcal{T}r_{#1}}

\newcommand{\searchRule}[6] {
  #1, #2 \vdash (#3, #4) \xRightarrow{#5} #6}
\newcommand{\extSearchRule}[8] {
  #1, #2, #3, #4 \vdash (#5, #6) \xRightarrow{#7}_{e} #8}
\newcommand{\q}{\hspace{0.5em}}
\newcommand{\bigcdot}{\boldsymbol{\cdot}}
\newcommand{\bigslant}[2]{{\raisebox{.2em}{$#1$}\left/\raisebox{-.2em}{$#2$}\right.}}

\let\emptyset\varnothing
\let\eps\varepsilon

\sloppy

\begin{document}

%% Title information
\title{Certified Semantics for \textsc{miniKanren}} %% [Short Title] is optional;
                                           %% when present, will be used in
                                           %% header instead of Full Title.
\titlenote{This work was partially suppored by the grant 18-01-00380 from The Russian Foundation for Basic Research} %% \titlenote is optional;
                                        %% can be repeated if necessary;
                                        %% contents suppressed with 'anonymous'
%\subtitle{Subtitle}                     %% \subtitle is optional
%\subtitlenote{with subtitle note}       %% \subtitlenote is optional;
                                        %% can be repeated if necessary;
                                        %% contents suppressed with 'anonymous'


%% Author information
%% Contents and number of authors suppressed with 'anonymous'.
%% Each author should be introduced by \author, followed by
%% \authornote (optional), \orcid (optional), \affiliation, and
%% \email.
%% An author may have multiple affiliations and/or emails; repeat the
%% appropriate command.
%% Many elements are not rendered, but should be provided for metadata
%% extraction tools.

\author{Dmitry Rozplokhas}
\affiliation{%
  \institution{Higher School of Economics}}
\affiliation{%
  \institution{JetBrains Research}
  \country{Russia}}
\email{darozplokhas@edu.hse.ru}

\author{Andrey Vyatkin}
\affiliation{%
  \institution{Saint Petersburg State University}
  \country{Russia}}
\email{dewshick@gmail.com}

\author{Dmitry Boulytchev}
\affiliation{%
  \institution{Saint Petersburg State University}}
\affiliation{%
  \institution{JetBrains Research}
  \country{Russia}}
\email{dboulytchev@math.spbu.ru}



%% Abstract
%% Note: \begin{abstract}...\end{abstract} environment must come
%% before \maketitle command
\begin{abstract}
  We present two formal semantics for the core \textsc{miniKanren}. First, we give denotational
  variant which corresponds to the minimal Herbrand model for definite logic programs. Second,
  we present operational semantics which models interleaving, and prove its soundness and
  completeness w.r.t. denotational semantics. Our development is supported by formal \textsc{Coq}
  specification, thus making it certified.
\end{abstract}


%% 2012 ACM Computing Classification System (CSS) concepts
%% Generate at 'http://dl.acm.org/ccs/ccs.cfm'.
\begin{CCSXML}
<ccs2012>
<concept>
<concept_id>10003752.10003790.10003795</concept_id>
<concept_desc>Theory of computation~Constraint and logic programming</concept_desc>
<concept_significance>500</concept_significance>
</concept>
<concept>
<concept_id>10003752.10010124.10010131.10010133</concept_id>
<concept_desc>Theory of computation~Denotational semantics</concept_desc>
<concept_significance>500</concept_significance>
</concept>
<concept>
<concept_id>10003752.10010124.10010131.10010134</concept_id>
<concept_desc>Theory of computation~Operational semantics</concept_desc>
<concept_significance>500</concept_significance>
</concept>
</ccs2012>
\end{CCSXML}
\ccsdesc[500]{Theory of computation~Constraint and logic programming}
\ccsdesc[500]{Theory of computation~Denotational semantics}
\ccsdesc[500]{Theory of computation~Operational semantics}
%% End of generated code


%% Keywords
%% comma separated list
\keywords{Relational programming, denotational semantics, operational semantics, certified programming}  %% \keywords are mandatory in final camera-ready submission


%% \maketitle
%% Note: \maketitle command must come after title commands, author
%% commands, abstract environment, Computing Classification System
%% environment and commands, and keywords command.
\maketitle
\thispagestyle{empty}

\section{Introduction}

The introductory book on \textsc{miniKanren}~\cite{TRS} describes the language by means of an evolving set of examples. In the
series of follow-up papers~\cite{MicroKanren,CKanren,CKanren1,AlphaKanren,2016,Guided} various extensions of the language were presented with
their semantics explained in terms of \textsc{Scheme} implementation. We argue that this style of semantic definition is
fragile and not self-evident since it requires the knowledge of semantics of concrete implementation language. In addition the justification of
important properties of relational programs (for example, refutational completeness~\cite{WillThesis}) becomes cumbersome. In the
area of programming languages research a formal definition for the semantics of language of interest is a \emph{de-facto} standard, and
in our opinion in its current state \textsc{miniKanren} deviates from this standard.

There were some previous attempts to define a formal semantics for \textsc{miniKanren}. \citet{RelConversion} present a variant of nondeterministic
operational semantics, and~\citet{DivTest} use another variant of finite-set semantics. None of them was capable of reflecting
the distinctive property of \textsc{miniKanren} search~--- \emph{interleaving}~\cite{Search}, thus deviating from the conventional understanding
of the language.

In this paper we present a formal semantics for core \textsc{miniKanren} and prove some its basic properties. First,
we define denotational semantics similar to the least Herbrand model for definite logic programs~\cite{LHM}; then
we describe operational semantics with interleaving in terms of labeled transition system. Finally, we prove the soundness and
completeness of the operational semantics w.r.t the denotational one. We support our development with a formal specification
using \textsc{Coq}~\cite{Coq} proof assistant\footnote{\url{https://github.com/dboulytchev/miniKanren-coq}}, thus outsourcing
the burden of proof checking to the automatic tool. 

The paper organized as follows. In Section~\ref{language} we give the syntax of the language, describe its semantics
informally and discuss some examples. Section~\ref{denotational} contains the description of denotational semantics for
the language, and Section~\ref{operational}~--- the operational semantics. In Section~\ref{equivalence} we overview the
certified proof for soundness and completeness of operational semantics. The final section concludes.

\begin{figure*}[t]
\[
\begin{array}{cccll}
  &\mathcal{C} & = & \{C_i^{k_i}\} & \mbox{constructors with arities} \\
  &\mathcal{T}_X & = & X \cup \{C_i^{k_i} (t_1, \dots, t_{k_i}) \mid t_j\in\mathcal{T}_X\} & \mbox{terms over the set of variables $X$} \\
  &\mathcal{D} & = & \mathcal{T}_\emptyset & \mbox{ground terms}\\
  &\mathcal{X} & = & \{ x, y, z, \dots \} & \mbox{syntactic variables} \\
  &\mathcal{A} & = & \{ \alpha, \beta, \gamma, \dots \} & \mbox{semantic variables} \\
  &\mathcal{R} & = & \{ R_i^{k_i}\} &\mbox{relational symbols with arities} \\
  &\mathcal{G} & = & \mathcal{T_X}\equiv\mathcal{T_X}   &  \mbox{unification} \\
  &            &   & \mathcal{G}\wedge\mathcal{G}     & \mbox{conjunction} \\
  &            &   & \mathcal{G}\vee\mathcal{G}       &\mbox{disjunction} \\
  &            &   & \mbox{\lstinline|fresh|}\;\mathcal{X}\;.\;\mathcal{G} & \mbox{fresh variable introduction} \\
  &            &   & R_i^{k_i} (t_1,\dots,t_{k_i}),\;t_j\in\mathcal{T_X} & \mbox{relational symbol invocation} \\
  &\mathcal{S} & = & \{R_i^{k_i} = \lambda\;x_1^i\dots x_{k_i}^i\,.\, g_i;\}\; g & \mbox{specification}
\end{array}
\]
\caption{The syntax of the source language}
\label{syntax}
\end{figure*}

\begin{comment}
\begin{figure}[t]
%\centering
\[
\begin{array}{rcl}
  \mathcal{FV}\,(x)&=&\{x\}\\
  \mathcal{FV}\,(C_i^{k_i}\,(t_1,\dots,t_{k_i}))&=&\bigcup\mathcal{FV}\,(t_i)\\
  \mathcal{FV}\,(t_1\equiv t_2)&=&\mathcal{FV}\,(t_1)\cup\mathcal{FV}\,(t_2)\\
  \mathcal{FV}\,(g_1\wedge g_2)&=&\mathcal{FV}\,(g_1)\cup\mathcal{FV}\,(g_2)\\
  \mathcal{FV}\,(g_1\vee g_2)&=&\mathcal{FV}\,(g_1)\cup\mathcal{FV}\,(g_2)\\
  \mathcal{FV}\,(\mbox{\lstinline|fresh|}\;x\;.\;g)&=&\mathcal{FV}\,(g)\setminus\{x\}\\
  \mathcal{FV}\,(R_i^{k_i}\,(t_1,\dots,t_{k_i}))&=&\bigcup\mathcal{FV}\,(t_i)
\end{array}
\]
\caption{Free variables in terms and goals}
\label{free}
\end{figure}
\end{comment}

\section{The Language}
\label{language}
 
In this section, we introduce the syntax of the language we use throughout the paper, describe the informal semantics, and give some examples.

The syntax of the language is shown in Fig.~\ref{syntax}. First, we fix a set of constructors $\mathcal{C}$ with known arities and consider
a set of terms $\mathcal{T}_X$ with constructors as functional symbols and variables from $X$. We parameterize this set with an alphabet of
variables since in the semantic description we will need \emph{two} kinds of variables. The first kind, \emph{syntactic} variables, is denoted
by $\mathcal{X}$. The second kind, \emph{semantic} or \emph{logic} variables, is denoted by $\mathcal{A}$.
We also consider an alphabet of \emph{relational symbols} $\mathcal{R}$ which are used to name relational definitions.
The central syntactic category in the language is \emph{goal}. In our case, there are five types of goals: \emph{unification} of terms,
conjunction and disjunction of goals, fresh variable introduction, and invocation of some relational definition. Thus, unification is used
as a constraint, and multiple constraints can be combined using conjunction, disjunction, and recursion.
The final syntactic category is a \emph{specification} $\mathcal{S}$. It consists of a set
of relational definitions and a top-level goal. A top-level goal represents a search procedure which returns a stream of substitutions for
the free variables of the goal. The definition for a set of free variables for both terms and goals is conventional;
%given in Figure~\ref{free};
as ``\lstinline|fresh|''
is the sole binding construct the definition is rather trivial. The language we defined is first-order, as goals can not be passed as parameters,
returned or constructed at run time.

We now informally describe how relational search works. As we said, a goal represents a search procedure. This procedure takes a \emph{state} as input and returns a
stream of states; a state (among other information) contains a substitution that maps semantic variables into the terms over semantic variables. Then five types of
scenarios are possible (depending on the type of the goal):

\begin{itemize}
\item Unification ``\lstinline|$t_1$ === $t_2$|'' unifies terms $t_1$ and $t_2$ in the context of the substitution in the current state. If terms are unifiable,
  then their MGU is integrated into the substitution, and a one-element stream is returned; otherwise the result is an empty stream.
\item Conjunction ``\lstinline|$g_1$ /\ $g_2$|'' applies $g_1$ to the current state and then applies $g_2$ to each element of the result, concatenating
  the streams.
\item Disjunction ``\lstinline|$g_1$ \/ $g_2$|'' applies both its goals to the current state independently and then concatenates the results.
\item Fresh construct ``\lstinline|fresh $x$ . $g$|'' allocates a new semantic variable $\alpha$, substitutes all free occurrences of $x$ in $g$ with $\alpha$, and
  runs the goal.
\item Invocation ``$\lstinline|$R_i^{k_i}$ ($t_1$,...,$t_{k_i}$)|$'' finds a definition for the relational symbol \mbox{$R_i^{k_i}=\lambda x_1\dots x_{k_i}\,.\,g_i$}, substitutes
  all free occurrences of a formal parameter $x_j$ in $g_i$ with term $t_j$ (for all $j$) and runs the goal in the current state.
\end{itemize}

We stipulate that the top-level goal is preceded by an implicit ``\lstinline|fresh|'' construct, which binds all its free variables, and that the final substitutions
for these variables constitute the result of the goal evaluation.

Conjunction and disjunction form a monadic~\cite{Monads} interface with conjunction playing role of ``\lstinline|bind|'' and disjunction the role of ``\lstinline|mplus|''.
In this description, we swept a lot of important details under the carpet~--- for example, in actual implementations the components of disjunction are not evaluated in
isolation, but both disjuncts are evaluated incrementally with the control passing from one disjunct to another (\emph{interleaving})~\cite{Search};
the evaluation of some goals can be additionally deferred (via so-called ``\emph{inverse-$\eta$-delay}'')~\cite{MicroKanren}; instead of streams
the implementation can be based on ``ferns''~\cite{BottomAvoiding} to defer divergent computations, etc. In the following sections, we present
a complete formal description of relational semantics which resolves these uncertainties in a conventional way.

As an example consider the following specification. For the sake of brevity we
abbreviate immediately nested ``\lstinline|fresh|'' constructs into the one, writing ``\lstinline|fresh $x$ $y$ $\dots$ . $g$|'' instead of
``\lstinline|fresh $x$ . fresh $y$ . $\dots$ $g$|''.

\begin{tabular}{p{5.5cm}p{5.5cm}}
\begin{lstlisting}
append$^o$ = fun x y xy .
 ((x === Nil) /\ (xy === y)) \/
 (fresh h t ty .
   (x  === Cons (h, t))  /\
   (xy === Cons (h, ty)) /\
   (append$^o$ t y ty));

revers$^o$ x x
\end{lstlisting} &
\begin{lstlisting}
revers$^o$ = fun x xr .
 ((x === Nil) /\ (xr === Nil)) \/
 (fresh h t tr .
   (x === Cons (h, t)) /\
   (append$^o$ tr (Cons (h, Nil)) xr) /\
   (revers$^o$ t tr));
\end{lstlisting}
\end{tabular}

Here we defined\footnote{We respect here a conventional tradition for \textsc{miniKanren} programming to superscript all relational names with ``$^o$''.}
two relational symbols~--- ``\lstinline|append$^o$|'' and ``\lstinline|revers$^o$|'',~--- and specified a top-level goal ``\lstinline|revers$^o$ x x|''.
The symbol ``\lstinline|append$^o$|'' defines a relation of concatenation of lists~--- it takes three arguments and performs a case analysis on the first one. If the
first argument is an empty list (``\lstinline|Nil|''), then the second and the third arguments are unified. Otherwise, the first argument is deconstructed into a head ``\lstinline|h|''
and a tail ``\lstinline|t|'', and the tail is concatenated with the second argument using a recursive call to ``\lstinline|append$^o$|'' and additional variable ``\lstinline|ty|'', which
represents the concatenation of ``\lstinline|t|'' and ``\lstinline|y|''. Finally, we unify ``\lstinline|Cons (h, ty)|'' with ``\lstinline|xy|'' to form a final constraint. Similarly,
``\lstinline|revers$^o$|'' defines relational list reversing. The top-level goal represents a search procedure for all lists ``\lstinline|x|'', which are stable under reversing, i.e.
palindromes. Running it results in an infinite stream of substitutions:

\begin{lstlisting}
   $\alpha\;\mapsto\;$ Nil
   $\alpha\;\mapsto\;$ Cons ($\beta_0$, Nil)
   $\alpha\;\mapsto\;$ Cons ($\beta_0$, Cons ($\beta_0$, Nil))
   $\alpha\;\mapsto\;$ Cons ($\beta_0$, Cons ($\beta_1$, Cons ($\beta_0$, Nil)))
   $\dots$
\end{lstlisting}

where ``$\alpha$'' is a \emph{semantic} variable, corresponding to ``\lstinline|x|'', ``$\beta_i$'' are free semantic variables. Therefore, each substitution represents a set of all palindromes of a certain length.


\begin{figure}[t]
\[
\begin{array}{rcll}
  x\,[t/x] &=& t &\\
  y\,[t/x] &=& y & y\ne x\\
  C_i^{k_i}\,(t_1,\dots,t_{k_i})\,[t/x]&=&C_i^{k_i}\,(t_1\,[t/x],\dots,t_{k_i}\,[t/x])&\\[2mm]
  (t_1 \equiv t_2)\,[t/x]&=&t_1\,[t/x] \equiv t_2\,[t/x]&\\
  (g_1 \wedge g_2)\,[t/x]&=&g_1\,[t/x] \wedge g_2\,[t/x]&\\
  (g_1 \vee g_2)\,[t/x]&=&g_1\,[t/x] \vee g_2\,[t/x]&\\
  (\mbox{\lstinline|fresh|}\;x\,.\,g)\,[t/x]&=&\mbox{\lstinline|fresh|}\;x\,.\,g&\\
  (\mbox{\lstinline|fresh|}\;y\,.\,g)\,[t/x]&=&\mbox{\lstinline|fresh|}\;y\,.\,(g\,[t/x])&y\ne x\\
  (R_i^{k_i}\,(t_1,\dots,t_{k_i})\,[t/x]&=&R_i^{k_i}\,(t_1\,[t/x],\dots,t_{k_i}\,[t/x])&
\end{array}
\]
  \caption{Substitutions for terms and goals}
  \label{substitution}
\end{figure}

\section{Denotational Semantics}
\label{denotational}

In this section we present a denotational semantics for the language we defined above. We use a simple set-theoretic
approach which can be considered as an analogy to the least Herbrand model for definite logic programs~\cite{LHM}.
Strictly speaking, instead of developing it from scratch we could have just described the conversion of specifications
into definite logic form and took their least Herbrand model. However, in that case we would still need to define
the least Herbrand model semantics for definite logic programs in a certified way. In addition, while for
this concrete language the conversion to definite logic form is trivial, it may become less trivial for
its extensions (with, for examples, nominal constructs~\cite{AlphaKanren}) which we plan to do in future.

To motivate further development, we first consider the following example. Let us have the following goal:

\begin{lstlisting}
   x === Cons (y, z)
\end{lstlisting}

There are three free variables, and solving the goal delivers us the following single answer:

\begin{lstlisting}
   $\alpha\mapsto\;$ Cons ($\beta$, $\gamma$)
\end{lstlisting}

where semantic variables $\alpha$, $\beta$ and $\gamma$ correspond to the syntactic ones ``\lstinline|x|'', ``\lstinline|y|'', ``\lstinline|z|''. The
goal does not put any constraints on ``\lstinline|y|'' and ``\lstinline|z|'', so there are no bindings for ``$\beta$'' and ``$\gamma$'' in the answer.
This answer can be seen as the following ternary relation over the set of all ground terms:

\[
\{(\mbox{\lstinline|Cons ($\beta$, $\,\gamma$)|}, \beta, \gamma) \mid \beta\in\mathcal{D},\,\gamma\in\mathcal{D}\}\subset\mathcal{D}^3
\]

The order of ``dimensions'' is important, since each dimension corresponds to a certain free variable. Our main idea is to represent this relation as a set of total
functions 

\[
\mathfrak{f}:\mathcal{A}\mapsto\mathcal{D}
\]

from semantic variables to ground terms. We call these functions \emph{representing functions}. Thus, we may reformulate the same relation as

\[
\{(\mathfrak{f}\,(\alpha),\mathfrak{f}\,(\beta),\mathfrak{f}\,(\gamma))\mid\mathfrak{f}\in\sembr{\mbox{\lstinline|$\alpha$ === Cons ($\beta$, $\,\gamma$)|}}\}
\]

where we use conventional semantic brackets ``$\sembr{\bullet}$'' to denote the semantics. For the top-level goal we need to substitute its free syntactic
variables with distinct semantic ones, calculate the semantics, and build the explicit representation for the relation as shown above. The relation, obviously,
does not depend on concrete choice of semantic variables, but depends on the order in which the values of representing functions are tupled. This order can be
conventionalized, which gives us a completely deterministic semantics.

Now we implement this idea. First, for a representing function

\[
\mathfrak{f} : \mathcal{A}\to\mathcal{D}
\]

we introduce its homomorphic extension 

\[
  \overline{\mathfrak{f}}:\mathcal{T_A}\to\mathcal{D}
\]

which maps terms to terms:

\[
\begin{array}{rcl}

  \overline{\mathfrak f}\,(\alpha) & = & \mathfrak f\,(\alpha)\\
  \overline{\mathfrak f}\,(C_i^{k_i}\,(t_1,\dots.t_{k_i})) & = & C_i^{k_i}\,(\overline{\mathfrak f}\,(t_1),\dots \overline{\mathfrak f}\,(t_{k_i}))
\end{array}
\]

Let us have two terms $t_1, t_2\in\mathcal{T_A}$. If there is a unifier for $t_1$ and $t_2$ then, clearly, there is a substitution $\theta$ which
turns both $t_1$ and $t_2$ into the same \emph{ground} term (we do not require $\theta$ to be the most general). Thus, $\theta$ maps
(some) ground variables into ground terms, and its application to $t_{1(2)}$ is exactly $\overline{\theta}(t_{1(2)})$. This reasoning can be
performed in the opposite direction: a unification $t_1\equiv t_2$ defines the set of all representing functions $\mathfrak{f}$ for which
$\overline{\mathfrak{f}}(t_1)=\overline{\mathfrak{f}}(t_2)$. 

Then, the semantic function for goals is parameterized over environments which prescribe semantic functions to relational symbols:

\[
  \Gamma : \mathcal{R} \to (\mathcal{T_A}^*\to 2^{\mathcal{A}\to\mathcal{D}})
\]

An environment associates with relational symbol a function which takes a string of terms (the arguments of the relation) and returns a set of
representing functions. The signature for semantic brackets for goals is as follows:

\[
\sembr{\bullet}_{\Gamma} : \mathcal{G}\to 2^{\mathcal{A}\to\mathcal{D}}
\]

It maps a goal into the set of representing functions w.r.t. an environment $\Gamma$.

We formulate the following important \emph{completeness condition} for the semantics of a goal $g$:

\[
\forall\alpha\not\in FV\,(g)\; \forall d \in \mathcal{D}\; \forall\mathfrak{f} \in \sembr{g}\; \exists \mathfrak{f'} \in \sembr{g} \;:\; \mathfrak{f'}\,(\alpha)\; = d \wedge \forall \beta \neq \alpha:\; \mathfrak{f'}\,(\beta)\; = \mathfrak{f}\,(\beta)\; 
\]

In other words, representing functions for a goal $g$ restrict only the values of free variables of $g$ and do not introduce any ``hidden'' correlations.
This condition guarantees that our semantics is complete in the sense that it does not introduce artificial restrictions for the relation it defines. It
can be proven that the semantics of goals always satisfy this condition.

We remind conventional notions of pointwise modification of a function

\[
f\,[x\gets v]\,(z)=\left\{
\begin{array}{rcl}
  f\,(z) &,& z \ne x \\
  v      &,& z = x
\end{array}
\right.
\]

and substitution of a free variable with a term in terms and goals (see Figure~\ref{substitution}).

For a representing function $\mathfrak{f}:\mathcal{A}\to\mathcal{D}$ and a semantic variable $\alpha$ we define
the following \emph{generalization} operation:

\[
\mathfrak{f}\uparrow\alpha = \{ \mathfrak{f}\,[\alpha\gets d] \mid d\in\mathcal D\}
\]

Informally, this operation generalizes a representing function into a set of representing functions in such a way that the
values of these functions for a given variable cover the whole $\mathcal{D}$. We extend the generalization operation for sets of
representing functions $\mathfrak{F}\subseteq\mathcal{A}\to\mathcal{D}$:

\[
  \mathfrak{F}\uparrow\alpha = \bigcup_{\mathfrak{f}\in\mathfrak{F}}(\mathfrak{f}\uparrow\alpha)
\]

Now we are ready to specify the semantics for goals (see Figure~\ref{denotational_semantics_of_goals}). We've already given the motivation for
the semantics of unification: the condition $\overline{\mathfrak{f}}(t_1)=\overline{\mathfrak{f}}(t_2)$ gives us the set of all (otherwise
unrestricted) representing functions which ``equate'' terms $t_1$ and $t_2$. Set union and intersection provide a conventional interpretation
for disjunction and conjunction of goals, and the semantics of relational invocation reduces to the application of corresponding
function from the environment. The only interesting case is ``\lstinline|fresh $x$ . $g$|''. First, we take an arbitrary semantic variable $\alpha$,
not free in $g$, and substitute $x$ with $\alpha$. Then we calculate the semantics of $g\,[\alpha/x]$. The interesting part is the next step:
as $x$ can not be free in ``\lstinline|fresh $x$ . $g$|'', we need to generalize the result over $\alpha$ since in our model the semantics of a
goal specifies a relation over its free variables. We introduce some nondeterminism, by choosing arbitrary $\alpha$, but it can be proven by structural induction, that with different choices of free variable, semantics of a goal won't change. Consider the following example:

\begin{lstlisting}
   fresh y . ($\alpha$ ===  y) /\ (y === Zero)
\end{lstlisting}

As there is no invocations involved, we can safely omit the environment. Then:

\[
\begin{array}{lcr}
  \sembr{\mbox{\lstinline|fresh y . ($\alpha$ === y) $\,\wedge\,$ (y === Zero)|}}&=&\mbox{(by \textsc{Fresh$_D$})}\\[1mm]
  (\sembr{\mbox{\lstinline|($\alpha$ === $\beta$) $\,\wedge\,$ ($\beta$ === Zero)|}})\uparrow\beta&=&\mbox{(by \textsc{Conj$_D$})}\\[1mm]
  (\sembr{\mbox{\lstinline|$\alpha$ === $\beta$|}} \,\cap\, \sembr{\mbox{\lstinline|$\beta$ === Zero)|}})\uparrow\beta&=&\mbox{(by \textsc{Unify$_D$})}\\[1mm]
  (\{\mathfrak{f}\mid \overline{\mathfrak{f}}\,(\alpha)=\overline{\mathfrak{f}}\,(\beta)\} \,\cap\, \{\mathfrak{f}\mid \overline{\mathfrak{f}}\,(\beta)=\overline{\mathfrak{f}}\,(\mbox{\lstinline|Zero|})\})\uparrow\beta&=&\mbox{(by the definition of ``$\overline{\mathfrak{f}}$'')}\\[1mm]
  (\{\mathfrak{f}\mid \mathfrak{f}\,(\alpha)=\mathfrak{f}\,(\beta)\} \,\cap\, \{\mathfrak{f}\mid \mathfrak{f}\,(\beta)=\mbox{\lstinline|Zero|}\})\uparrow\beta&=&\mbox{(by the definition of ``$\cap$'')}\\[1mm]
  (\{\mathfrak{f}\mid \mathfrak{f}\,(\alpha)=\mathfrak{f}\,(\beta)=\mbox{\lstinline|Zero|}\})\uparrow\beta&=&\mbox{(by the definition of ``$\uparrow$'')}\\[1mm]
  \{\mathfrak{f}\mid \mathfrak{f}\,(\alpha)=\mbox{\lstinline|Zero|}, \mathfrak{f}\,(\beta)=d, d\in\mathcal{D}\}&=&\mbox{(by the totality of representing functions)}\\[1mm]
  \{\mathfrak{f}\mid \mathfrak{f}\,(\alpha)=\mbox{\lstinline|Zero|}\}&&
\end{array}
\]

In the end we've got the set of representing functions, each of which restricts only the value of free variable $\alpha$. 

\begin{figure}[t]
  \[
  \begin{array}{cclr}
    \sembr{t_1\equiv t_2}_\Gamma&=&\{\mathfrak f : \mathcal{A}\to\mathcal{D}\mid \overline{\mathfrak{f}}\,(t_1)=\overline{\mathfrak{f}}\,(t_2)\}& \ruleno{Unify$_D$}\\
    \sembr{g_1\wedge g_2}_\Gamma&=&\sembr{g_1}_\Gamma\cap\sembr{g_1}_\Gamma&\ruleno{Conj$_D$}\\
    \sembr{g_1\vee g_2}_\Gamma&=&\sembr{g_1}_\Gamma\cup\sembr{g_1}_\Gamma&\ruleno{Disj$_D$}\\
    \sembr{\mbox{\lstinline|fresh|}\,x\,.\,g}_\Gamma&=&(\sembr{g\,[\alpha/x]}_\Gamma)\uparrow\alpha,\;\alpha\not\in FV(g)& \ruleno{Fresh$_D$}\\
    \sembr{R\,(t_1,\dots,t_k)}_\Gamma&=&(\Gamma\,R)\,t_1\dots t_k & \ruleno{Invoke$_D$}
  \end{array}
  \]
  \caption{Denotational semantics of goals}
  \label{denotational_semantics_of_goals}
\end{figure}

The final component is the semantics of specifications. Given a specification

\[
\{R_i=\lambda\,x_1^i\dots x_{k_i}^i\,.\,g_i;\}_{i=1}^n\;g
\]

we have to construct a correct environment $\Gamma_0$ and then take the semantics of the top-level goal:

\[
\sembr{\{R_i=\lambda\,x_1^i\dots x_{k_i}^i\,.\,g_i;\}_{i=1}^n\;g}=\sembr{g}_{\Gamma_0}
\]

As the set of definitions can be mutually recursive we apply the fixed point approach. We consider the following
function

\[
\mathcal{F} : (\mathcal{T_A}^*\to 2^{\mathcal{A}\to\mathcal{D}})^n\to (\mathcal{T_A}^*\to 2^{\mathcal{A}\to\mathcal{D}})^n
\]

which represents a semantic for the set of definitions abstracted over themselves. The definition of this function is
rather standard:

\begin{gather*}
    \begin{array}{rcl}
      \mathcal{F}\,(p_1,\dots,p_n)& = &(t^1_1\dots t^1_{k_1}\mapsto\sembr{g^1\,[t^1_1/x^1_1,\dots,t^1_{k_1}/x^1_{k_1}]}_\Gamma,\\
                                  &  &\phantom{(}\dots\\
                                  &  &\phantom{(}t^n_1\dots t^n_{k_n}\mapsto\sembr{g^n\,[t^n_1/x^n_1,\dots,t^n_{k_n}/x^n_{k_n}]}_\Gamma)
    \end{array}\\
    \mbox{where}\;\Gamma\, R_i=p_i
\end{gather*}

Here $p_i$ is a semantic function for $i$-th definition; we build an environment $\Gamma$ which associates each relational symbol
$R_i$ with $p_i$ and construct a $n$-dimensional vector-function, where $i$-th component corresponds to a function which
calculates the semantics of $i$-th relational definition application to terms w.r.t. the environment $\Gamma$. Finally,
we take the least fixed point of $\mathcal{F}$ and define the top-level environment as follows:

\[
\Gamma_0\,R_i=(fix\;\mathcal{F})\,[i]
\]

where ``$[i]$'' denotes the $i$-th component of a vector-function.

The least fixed point exists by Knaster-Tarski~\cite{TarskiKnaster} theorem~--- the set $(\mathcal{T_A}^*\to 2^{\mathcal{A}\to\mathcal{D}})^n$
forms a complete lattice, and $\mathcal{F}$ is monotonic. 

To formalize denotational semantics in \textsc{Coq} we can define representing functions simply as \textsc{Coq} functions:

\begin{lstlisting}[language=Coq]
   Definition repr_fun : Set := var -> ground_term.
\end{lstlisting}

We define the semantics via inductive proposition ``\lstinline|in_denotational_sem_goal|'' such that

\[
\forall g,\mathfrak{f}\;:\;\mbox{\lstinline|in_denotational_sem_goal|}\;g\;\mathfrak{f}\Longleftrightarrow\mathfrak{f}\in\sembr{g}_\Gamma
\]

The definition is as follows:

\begin{lstlisting}[language=Coq]
   Inductive in_denotational_sem_goal : goal -> repr_fun -> Prop :=
   | dsgUnify  : forall f t1 t2, apply_repr_fun f t1 = apply_repr_fun f t2 ->
                            in_denotational_sem_goal (Unify t1 t2) f

   | dsgDisjL  : forall f g1 g2, in_denotational_sem_goal g1 f ->
                            in_denotational_sem_goal (Disj g1 g2) f

   | dsgDisjR  : forall f g1 g2, in_denotational_sem_goal g2 f ->
                            in_denotational_sem_goal (Disj g1 g2) f

   | dsgConj   : forall f g1 g2, in_denotational_sem_goal g1 f ->
                            in_denotational_sem_goal g2 f ->
                            in_denotational_sem_goal (Conj g1 g2) f

   | dsgFresh  : forall f fn a fg, (~ is_fv_of_goal a (Fresh fg)) ->
                              in_denotational_sem_goal (fg a) fn ->
                              (forall x, x <> a -> fn x = f x) ->
                              in_denotational_sem_goal (Fresh fg) f

   | dsgInvoke : forall r t f, in_denotational_sem_goal (proj1_sig (Prog r) t) f ->
                          in_denotational_sem_goal (Invoke r t) f.
\end{lstlisting}

Here we refer to a fixpoint ``\lstinline[language=Coq]|apply_repr_fun|'' which calculates the extension ``$\overline{\bullet}$'' for a representing
function, and inductive proposition ``\lstinline[language=Coq]|is_fv_of_goal|'' which encodes the set of free variables for a goal.

Recall that the environment ``\lstinline[language=Coq]|Prog|'' maps every relational symbol to the definition of relation,
which is a pair of a function from terms to goals and a proof that it has no unbound variables.
So in the last case ``\lstinline[language=Coq]|(proj1_sig (Prog r) t)|'' simply takes the body of the corresponding relation;
thus ``\lstinline[language=Coq]|Prog|'' in \textsc{Coq} specification plays role of a global environment $\Gamma$.

It is interesting that in \textsc{Coq} implementation we do not need to refer to Tarski-Knaster theorem explicitly since
the least fixpoint semantic is implicitly provided by inductive definitions.

\section{Operational Semantics}
\label{operational}

In this section we describe operational semantics of \textsc{miniKanren}, which corresponds to the known
implementations with interleaving search. The semantics will be given in the form of labeled transition system (LTS). From now on we
assume the set of semantic variables to be linearly ordered ($\mathcal{A}=\{\alpha_1,\alpha_2,\dots\}$).

We introduce the notion of substitution

\[
  \sigma : \mathcal{A}\to\mathcal{T_A}
\]

as a (partial) mapping from semantic variables to terms over the set of semantic variables. We denote $\Sigma$ the
set of all substitutions, $\dom{\sigma}$~--- the domain for a substitution $\sigma$,
$\ran{\sigma}=\bigcup_{\alpha\in\mathcal{D}om\,(\sigma)}\fv{\sigma\,(\alpha)}$~--- its range (the set of all free variables in the image).

The states in the transition system have the following shape

\[
S = \mathcal{G}\times\Sigma\times\mathbb{N}\mid S\oplus S \mid S \otimes \mathcal{G}
\]

As we will see later, an evaluation of a goal is separated into elementary steps, and these steps are performed interchangeably for different subgoals. 
Thus, a state has a tree-like structure with intermediate nodes corresponding to partially-evaluated conjunctions (``$\otimes$'') or
disjunctions (``$\oplus$''). A leaf in the form $\inbr{g, \sigma, n}$ determines a goal in a context, where $g$~--- a goal, $\sigma$~--- a substitution accumulated so far,
and $n$~--- a natural number, which corresponds to a number of semantic variables used to this point. For a conjunction node its right child is always a goal since
it cannot be evaluated unless some result is provided by the left conjunct.

We also need extended states

\[
\overline{S} = \diamond \mid S
\]

where $\diamond$ symbolizes the end of evaluation, and the following well-formedness condition:

\begin{definition}
  Well-formedness condition for extended states:
  
  \begin{itemize}
  \item $\diamond$ is well-formed;
  \item $\inbr{g, \sigma, n}$ is well-formed iff $\fv{g}\cup\dom{\sigma}\cup\ran{\sigma}\subset\{\alpha_1,\dots,\alpha_n\}$;
  \item $s_1\oplus s_2$ is well-formed iff $s_1$ and $s_2$ well-formed;
  \item $s\otimes g$ is well-formed iff $s$ is well-formed and for all leaf triplets $\inbr{\_,\_,n}$ in $s$ $\fv{g}\subseteq\{\alpha_1,\dots,\alpha_n\}$.
  \end{itemize}
  
\end{definition}

Informally the well-formedness restricts the set of states to those in which all goals use only allocated variables.

Finally, we define the set of labels:

\[
L = \circ \mid \Sigma\times \mathbb{N}
\]

The label ``$\circ$'' is used to mark those steps which do not provide an answer; otherwise a transition is labeled by a pair of a substitution and a number of allocated
variables. The substitution is one of the answers, and the number is threaded through the derivation to keep track of allocated variables; we ignore it in further explanations.

\begin{figure}
  \[
  \begin{array}{cr}
    \inbr{t_1 \equiv t_2, \sigma, n} \xrightarrow{\circ} \Diamond , \, \, \nexists\; mgu\,(t_1, t_2, \sigma) &\ruleno{UnifyFail} \\[2mm]
    \inbr{t_1 \equiv t_2, \sigma, n} \xrightarrow{(mgu\,(t_1, t_2, \sigma),\, n)} \Diamond & \ruleno{UnifySuccess} \\[2mm]
    \inbr{g_1 \lor g_2, \sigma, n} \xrightarrow{\circ} \inbr{g_1, \sigma, n} \oplus \inbr{g_2, \sigma, n} & \ruleno{Disj} \\[2mm]
    \inbr{g_1 \land g_2, \sigma, n} \xrightarrow{\circ} \inbr{ g_1, \sigma, n} \otimes g_2 & \ruleno{Conj} \\[2mm]
    \inbr{\mbox{\lstinline|fresh|}\, x\, .\, g, \sigma, n} \xrightarrow{\circ} \inbr{g\,[\bigslant{\alpha_{n + 1}}{x}], \sigma, n + 1} & \ruleno{Fresh}\\[2mm]
    \dfrac{R_i^{k_i}=\lambda\,x_1\dots x_{k_i}\,.\,g}{\inbr{R_i^{k_i}\,(t_1,\dots,t_{k_i}),\sigma,n} \xrightarrow{\circ} \inbr{g\,[\bigslant{t_1}{x_1}\dots\bigslant{t_{k_i}}{x_{k_i}}], \sigma, n}} & \ruleno{Invoke}\\[5mm]
    \dfrac{s_1 \xrightarrow{\circ} \Diamond}{(s_1 \oplus s_2) \xrightarrow{\circ} s_2} & \ruleno{DisjStop}\\[5mm]
    \dfrac{s_1 \xrightarrow{r} \Diamond}{(s_1 \oplus s_2) \xrightarrow{r} s_2} & \ruleno{DisjStopAns}\\[5mm]
    \dfrac{s \xrightarrow{\circ} \Diamond}{(s \otimes g) \xrightarrow{\circ} \Diamond} &\ruleno{ConjStop}\\[5mm]
    \dfrac{s \xrightarrow{(\sigma, n)} \Diamond}{(s \otimes g) \xrightarrow{\circ} \inbr{g, \sigma, n}}  & \ruleno{ConjStopAns}\\[5mm]
    \dfrac{s_1 \xrightarrow{\circ} s'_1}{(s_1 \oplus s_2) \xrightarrow{\circ} (s_2 \oplus s'_1)} &\ruleno{DisjStep}\\[5mm]
    \dfrac{s_1 \xrightarrow{r} s'_1}{(s_1 \oplus s_2) \xrightarrow{r} (s_2 \oplus s'_1)} &\ruleno{DisjStepAns}\\[5mm]
    \dfrac{s \xrightarrow{\circ} s'}{(s \otimes g) \xrightarrow{\circ} (s' \otimes g)} &\ruleno{ConjStep}\\[5mm]
    \dfrac{s \xrightarrow{(\sigma, n)} s'}{(s \otimes g) \xrightarrow{\circ} (\inbr{g, \sigma, n} \oplus (s' \otimes g))} & \ruleno{ConjStepAns} 
  \end{array}
  \]
  \caption{Operational semantics of interleaving search}
  \label{lts}
\end{figure}

The transition rules are shown on Figure~\ref{lts}. The first two rules specify the semantics of unification. If two terms are not unifiable under the current substitution
$\sigma$ then the evaluation stops with no answer; otherwise it stops with the answer equal to the most general unifier.

The next two rules describe the steps performed when disjunction (conjunction) is encountered on the top level of the current goal. For disjunction it schedules both goals (using ``$\oplus$'') for
evaluating in the same context as the parent state, for conjunction~--- schedules the left goal and postpones the right one (using ``$\otimes$'').

The rule for ``\lstinline|fresh|'' substitutes bound syntactic variable with a newly allocated semantic one and proceeds with the goal; no answer provided at this step.

The rule for relation invocation finds a corresponding definition, substitutes its formal parameters with the actual ones, and proceeds with the body.

The rest of the rules specify the steps performed during the evaluation of two remaining types of the states~--- conjunction and disjunction. In all cases the left state
is evaluated first. If its evaluation stops with a result then the right state (or goal) is scheduled for evaluation, and the label is propagated. If there is no result then
the conjunction evaluation stops with no result (\textsc{ConjStop}) as well while the disjunction evaluation proceeds with the right state (\textsc{DisjStop}).

The last four rules describe \emph{interleaving}, which occurs when the evaluation of the left state suspends with some residual state (with or without an answer). In the case of disjunction
the answer (if any) is propagated, and the constituents of the disjunction are swapped (\textsc{DisjStep}, \textsc{DisjStepAns}). In case of conjunction, if the evaluation step in
the left conjunct did not provide any answer, the evaluation is continued in the same order since there is still no information to proceed with the evaluation of the right
conjunct (\textsc{ConjStep}); if there is some answer, then the disjunction of the right conjunct in the context of the answer and the remaining conjunction is
scheduled for evaluation (\textsc{ConjStepAns}).

The introduced transition system is completely deterministic. There was, however, some freedom in choosing the order of evaluation for conjunction and
disjunction states. For example, instead of evaluating the left substate first we could choose to evaluate the right one, etc. In each concrete case we would
end up with a different (but still deterministic) system which would prescribe different semantics to a concrete goal. This choice reflects the inherent
non-deterministic nature of search in relational (and, more generally, logical) programming. However, as long as deterministic search procedures
are sound and complete, we can consider them ``equivalent''\footnote{There still can be differences in observable behavior of concrete goals under different
sound and complete search strategies: a goal can be refutationally complete~\cite{WillThesis} under one strategy and non-complete under another.}.

A derivation sequence for a certain state determines a \emph{trace}~--- a finite or infinite sequence of answers. We may define a set of finite or infinite
sequences $X^\omega$ over an alphabet $X$ as a set of functions from natural numbers into a lifted set $X_\bot=X\cup\{\bot\}$:

\[
X^\omega=\{\omega : \mathbb{N}\to X_\bot\ \mid \forall n\in\mathbb{N},\, \omega\,(n)=\bot\Rightarrow \omega\,(n+1)=\bot\}
\]

Informally speaking, we represent a sequence as a function which maps positions (treated as natural numbers) into the elements of the sequence. We use ``$\bot$''
to specify that there is no element at given position, and we stipulate, that there are no ``holes'' in this representation: if there is no element at given
position then there are no elements at greater positions as well. 

For this representation we may define the empty sequence $\epsilon$ and operations of prepending a sequence $\omega$ with an element $a$ and taking a suffix of
a sequence $\omega$ from a position $n$ as follows:

\begin{gather*}
  \epsilon = i \mapsto \bot\\[2mm]
  a\omega = i \mapsto \left\{
  \begin{array}{rcl}
    a &,& i = 0\\
    \omega\,(i-1)&,&\mbox{otherwise}
  \end{array}
  \right.\\[2mm]
  \omega\,[n:]=i\mapsto\omega\,(n+i)
\end{gather*}

For a given state $s$ a trace $\tr{s}\in L^\omega$ is a sequence of labels, defined as follows simultaneously with the sequence of states $\{s_i\}$:

\[
\begin{array}{ccccl}
  \multicolumn{5}{c}{s_o=s}\\
  \tr{s}\,(n)=a &,& s_{n+1}=s'&\mbox{ if }& s_n\ne\diamond,\, s_n\xrightarrow{a} x'\\
  \tr{s}\,(n)=\bot&,&s_{n+1}=\diamond&\mbox{ if }& s_n=\diamond
\end{array}
\]

The trace corresponds to the stream of answers in the reference \textsc{miniKanren} implementations.

To formalize the operational part in \textsc{Coq} we first need to define all preliminary notions from unification theory~\cite{Unification} which our semantics uses.

In particular, we need to implement the notion of the most general unifier (MGU). As is it well-known~\cite{UnificationMcBride} all standard recursive algorithms for calculating
MGU are not decreasing on argument terms, so we can't define it as a simple recursive function in \textsc{Coq} due to the termination check. There is no such obstacle when we define
MGU as a proposition:

\begin{lstlisting}[language=Coq]
  Inductive MGU : term -> term -> option subst -> Set := ...
\end{lstlisting}

However, we still need to use a well-founded induction to prove the existence of the most general unifier and its defining properties:

\begin{lstlisting}[language=Coq]
  Lemma MGU_ex : forall t1 t2, {r & MGU t1 t2 r}.
  
  Definition unifier (s : subst) (t1 t2 : term) : Prop := apply_subst s t1 = apply_subst s t2.

  Lemma MGU_unifies:
    forall t1 t2 s, MGU t1 t2 (Some s) -> unifier s t1 t2.
  
  Definition more_general (m s : subst) : Prop :=
    exists (s' : subst), forall (t : term), apply_subst s t = apply_subst s' (apply_subst m t).

  Lemma MGU_most_general :
    forall (t1 t2 : term) (m : subst),
      MGU t1 t2 (Some m) ->
      forall (s : subst), unifier s t1 t2 -> more_general m s.

  Lemma MGU_non_unifiable :
    forall (t1 t2 : term),
      MGU t1 t2 None -> forall s,  ~ (unifier s t1 t2).
\end{lstlisting}

For this well-founded induction we use the number of free variables in argument terms as a well-founded order on pairs of terms:

\begin{lstlisting}[language=Coq]
  Definition terms := term * term.

  Definition fvOrder (t : terms) := length (union (fv_term (fst t)) (fv_term (snd t))).

  Definition fvOrderRel (t p : terms) := fvOrder t < fvOrder p.

  Lemma fvOrder_wf : well_founded fvOrderRel.
\end{lstlisting}

After this preliminary work, the described transition relation can be encoded naturally as an inductively defined proposition (here ``\lstinline|state'|''
stands for an extended state):

\begin{lstlisting}[language=Coq]
  Inductive eval_step : state -> label -> state' -> Set := ...
\end{lstlisting}

We state the fact that our system is deterministic through existence and uniqueness of a transition for every state:

\begin{lstlisting}[language=Coq]
  Lemma eval_step_ex : forall (st : state), {l : label & {st' : state' & eval_step st l st'}}.

  Lemma eval_step_unique :
    forall (st : state) (l1 l2 : label) (st'1 st'2 : state'),
      eval_step st l1 st'1 -> eval_step st l2 st'2 -> l1 = l2 /\ st'1 = st'2.
\end{lstlisting}

To work with (possibly) infinite sequences we use the standard approach in \textsc{Coq}~--- coinductively defined streams:

\begin{lstlisting}[language=Coq]
  Context {A : Set}.

  CoInductive stream : Set :=
  | Nil : stream
  | Cons : A -> stream -> stream.
\end{lstlisting}

Although the definition of the datatype is coinductive some of its properties we are working with make sense only when defined inductively:

\begin{lstlisting}[language=Coq]
  Inductive in_stream : A -> stream -> Prop :=
  | inHead : forall x t, in_stream x (Cons x t)
  | inTail : forall x h t, in_stream x t -> in_stream x (Cons h t).

  Inductive finite : stream -> Prop :=
  | fNil : finite Nil
  | fCons : forall h t, finite t -> finite (Cons h t).
\end{lstlisting}

Then we define a trace coinductively as a stream of labels in transition steps and prove that there exists a unique trace from any extended state:

\begin{lstlisting}[language=Coq]
  Definition trace : Set := $@$stream label.

  CoInductive op_sem : state' -> trace -> Set :=
  | osStop : op_sem Stop Nil
  | osState : forall st l st' t, eval_step st l st' ->
                            op_sem st' t ->
                            op_sem (State st) (Cons l t).

  Lemma op_sem_ex (st' : state') : {t : trace & op_sem st' t}.

  Lemma op_sem_unique :
    forall st' t1 t2, op_sem st' t1 -> op_sem st' t2 -> equal_streams t1 t2.
\end{lstlisting}

Note, for the equality of streams we need to define a new coinductive proposition instead of using the standard syntactic equality in order for coinductive proofs to work~\cite{CPDT}.

One thing we can prove using operational semantics is the \emph{interleaving} properties of disjunction. Specifically, we can prove that a trace for a disjunction is
a one-by-one interleaving of streams for its disjuncts:

\begin{lstlisting}[language=Coq]
  CoInductive interleave : stream -> stream -> stream -> Prop :=
  | interNil : forall s s', equal_streams s s' -> interleave Nil s s'
  | interCons : forall h t s rs, interleave s t rs -> interleave (Cons h t) s (Cons h rs).

  Lemma sum_op_sem : forall st1 st2 t1 t2 t, op_sem (State st1) t1 ->
                                        op_sem (State st2) t2 ->
                                        op_sem (State (Sum st1 st2)) t ->
                                        interleave t1 t2 t.
\end{lstlisting}

This allows us to prove the expected properties of interleaving in a more general setting of arbitrary streams:

\begin{itemize}
\item  the elements of the interleaved stream are exactly those of two interleaved streams;
\item  the interleaved stream is finite iff both interleaving streams are finite.
\end{itemize}

The corresponding \textsc{Coq} lemmas are as follows:

\begin{lstlisting}[language=Coq]
  Lemma interleave_in : forall s1 s2 s, interleave s1 s2 s ->
                   forall x, in_stream x s <-> in_stream x s1 \/ in_stream x s2.

  Lemma interleave_finite : forall s1 s2 s, interleave s1 s2 s ->
                   (finite s <-> finite s1 /\ finite s2).
\end{lstlisting}

\section{Semantics Equivalence}
\label{equivalence}

Now when we defined two different kinds of semantics for \textsc{miniKanren} we can relate them and show that the results given by these two semantics are the same for any specification.
This will actually say something important about the search in the language: since operational semantics describes precisely the behavior of the search and denotational semantics
ignores the search and describes what we \emph{should} get from mathematical point of view, by proving their equivalence we establish \emph{completeness} of the search which
means that the search will get all answers satisfying the described specification and only those.

But first, we need to relate the answers produced by these two semantics as they have different forms: a trace of substitutions (along with numbers of allocated variables)
for operational and a set of representing functions for denotational. We can notice that the notion of representing function is close to substitution, with only two differences:

\begin{itemize}
\item representing function is total;
\item terms in the domain of representing function are ground.
\end{itemize}

Therefore we can easily extend (perhaps ambiguously) any substitution to a representing function by composing it with an arbitrary representing function and that will
preserve all variable dependencies in the substitution. So we can define a set of representing functions corresponding to substitution as follows:

\[
[\sigma] = \{\overline{\mathfrak f} \circ \sigma \mid \mathfrak{f}:\mathcal{A}\mapsto\mathcal{D}\}
\]

And \emph{denotational analog} of an operational semantics (a set of representing functions corresponding to answers in the trace) for given extended state $s$ is
then defined as a union of sets for all substitution in the trace:

\[
\sembr{s}_{op} = \cup_{(\sigma, n) \in \tr{s}} [\sigma]
\]

This allows us to state theorems relating two semantics.

\begin{theorem}[Operational semantics soundness]
For any specification $\{\dots\}\; g$, for which the indices of all free variables in $g$ are limited by some number $n$

\[
\sembr{\inbr{g, \epsilon, n}}_{op} \subset \sembr{\{\dots\}\; g}.
\]
\end{theorem}

It can be proven by nested induction, but first, we need to generalize the statement so that the inductive hypothesis would be strong enough for the inductive step.
To do so, we define denotational semantics not only for goals but for arbitrarily extended states. Note that this definition does not need to have any intuitive
interpretation, it is introduced only for proof to go smoothly. The definition of the denotational semantics for extended states is on Figure~\ref{denotational_semantics_of_states}.
The generalized version of the theorem uses it:

\begin{figure}[t]
  \[
  \begin{array}{ccl}
    \sembr{\Diamond}_\Gamma&=&\emptyset\\
    \sembr{\inbr{g, \sigma, n}}_\Gamma&=&\sembr{g}_\Gamma\cap[\sigma]\\
    \sembr{s_1 \oplus s_2}_\Gamma&=&\sembr{s_1}_\Gamma\cup\sembr{s_2}_\Gamma\\
    \sembr{s \otimes g}_\Gamma&=&\sembr{s}_\Gamma\cap\sembr{g}_\Gamma\\
  \end{array}
  \]
  \caption{Denotational semantics of states}
  \label{denotational_semantics_of_states}
\end{figure}

\begin{lemma}[Generalized soundness]
For any top-level environment $\Gamma_0$ acquired from some specification, for any well-formed (w.r.t. that specification) extended state $s$

\[
\sembr{s}_{op} \subset \sembr{s}_{\Gamma_0}.
\]
\end{lemma}

It can be proven by induction on the number of steps in which a given answer (more accurately, the substitution that contains it) occurs in the trace.
The induction step is proven by structural induction on the extended state $s$.

It would be tempting to formulate the completeness of operational semantics as the inverse inclusion, but it does not hold in such generality. The reason for
this is that denotational semantics encodes only dependencies between the free variables of a goal, which is reflected by the completeness condition, while
operational semantics may also contain dependencies between semantic variables allocated in ``\lstinline|fresh|''. Therefore we formulate the completeness
with representing functions restricted on the semantic variables allocated in the beginning (which includes all free variables of a goal). This does not
compromise our promise to prove the completeness of the search as \textsc{miniKanren} provides the result as substitutions only for queried variables,
which are allocated in the beginning.

\begin{theorem}[Operational semantics completeness]
For any specification $\{\dots\}\; g$, for which the indices of all free variables in $g$ are limited by some number $n$

\[
\{\mathfrak{f}|_{\{\alpha_1,\dots,\alpha_n\}} \mid \mathfrak{f} \in \sembr{\{\dots\}\; g}\} \subset \{\mathfrak{f}|_{\{\alpha_1,\dots,\alpha_n\}} \mid \mathfrak{f} \in \sembr{\inbr{g, \epsilon, n}}_{op}\}.
\]
\end{theorem}


Similarly to the soundness, this can be proven by nested induction, but the generalization is required. This time it is enough to generalize it from goals
to states of the shape $\inbr{g, \sigma, n}$. We also need to introduce one more auxiliary semantics --- bounded denotational semantics:

\[
\sembr{\bullet}^l : \mathcal{G} \to 2^{\mathcal{A}\to\mathcal{D}}
\]

Instead of always unfolding the definition of a relation for invocation goal, it does so only given number of times. So for a given set of relational
definitions $\{R_i^{k_i} = \lambda\;x_1^i\dots x_{k_i}^i\,.\, g_i;\}$ the definition of bounded denotational semantics is exactly the same as in usual denotational semantics,
except that for the invocation case:

\[
\sembr{R_i^{k_i}\,(t_1,\dots,t_{k_i})}^{l+1} = \sembr{g_i[t_1/x_1^i, \dots, t_{k_i}/x_{k_i}^i]}^{l}
\]

It is convenient to define bounded semantics for level zero as an empty set:

\[
\sembr{g}^{0} = \emptyset
\]

Bounded denotational semantics is an approximation of a usual denotational semantics and it is clear that any answer in usual denotational semantics will also be in
bounded denotational semantics for some level:

\begin{lemma}
$\sembr{g}_{\Gamma_0} \subset \cup_l \sembr{g}^l$
\end{lemma}

Formally it can be proven using the definition of the least fixed point from Tarski-Knaster theorem: the set on the right-hand side is a closed set.

Now the generalized version of the completeness theorem is as follows:

\begin{lemma}[Generalized completeness]
For any set of relational definitions, for any level $l$, for any well-formed (w.r.t. that set of definitions) state $\inbr{g, \sigma, n}$,

\[
\{\mathfrak{f}|_{\{\alpha_1,\dots,\alpha_n\}} \mid \mathfrak{f} \in \sembr{g}^l \cap [\sigma]\} \subset \{\mathfrak{f}|_{\{\alpha_1,\dots,\alpha_n\}} \mid \mathfrak{f} \in \sembr{\inbr{g, \sigma, n}}_{op}\}.
\]
\end{lemma}

It is proven by induction on the level $l$. The induction step is proven by structural induction on the goal $g$.

The proofs of both theorems are certified in \textsc{Coq}, although the proofs for a number of (obvious) technical facts about representing functions and computation of the most
general unifier as well as some properties of denotational semantics, proven informally in Section~\ref{denotational}, are
admitted for now. For completeness we can not just use the induction on proposition \lstinline|in_denotational_sem_goal|, as it would be natural to expect,
because the inductive principle it provides is not flexible enough. So we need to define bounded denotational semantics in our formalization too and perform
induction on the level explicitly:

\begin{lstlisting}[language=Coq]
   Inductive in_denotational_sem_lev_goal : nat -> goal -> repr_fun -> Prop :=
   ...
   | dslgInvoke : forall l r t f,
        in_denotational_sem_lev_goal l (proj1_sig (Prog r) t) f ->
        in_denotational_sem_lev_goal (S l) (Invoke r t) f.
\end{lstlisting}

The lemma relating bounded and unbounded denotational semantics is translated into \textsc{Coq}:

\begin{lstlisting}[language=Coq]
   Lemma in_denotational_sem_some_lev: forall (g : goal) (f : repr_fun),
        in_denotational_sem_goal g f ->
        exists l, in_denotational_sem_lev_goal l g f.
\end{lstlisting}

The statements of the theorems are as follows:

\begin{lstlisting}[language=Coq]
   Theorem search_correctness: forall (g : goal) (k : nat) (f : repr_fun) (t : trace),
      closed_goal_in_context (first_nats k) g) ->
      op_sem (State (Leaf g empty_subst k)) t) ->
      in_denotational_analog t f ->
      in_denotational_sem_goal g f.
      
   Theorem search_completeness: forall (g : goal) (k : nat) (f : repr_fun) (t : trace),
      closed_goal_in_context (first_nats k) g) ->
      op_sem (State (Leaf g empty_subst k)) t) ->
      in_denotational_sem_goal g f ->
      exists (f' : repr_fun), (in_denotational_analog t f') /\
                         forall (x : var), In x (first_nats k) -> f x = f' x.
\end{lstlisting}

One important immediate corollary of these theorems is the correctness of certain program transformations. Since the results obtained by the search on a
specification are exactly the results from the mathematical model of this specification, after the transformations of relations that do not change their
mathematical meaning the search will obtain the same results. Note that this way we guarantee only the stability of results as the set of ground terms,
the other aspects of program behavior, such as termination, may be affected. This allows us to safely (to a certain extent) apply such natural
transformations as:

\begin{itemize}
\item changing the order of constituents in conjunction or disjunction;
\item swapping conjunction and disjunction using distributivity;
\item moving fresh variable introduction.
\end{itemize}

and even transform relational definitions to some kinds of normal form (like all fresh variables introduction on the top level with the
conjunctive normal form inside), which may be convenient, for example, for metacomputation.

\section{Future Work}

There are a few possible directions for future work. First, in this paper we did not address the performance issues. As we represent
the transformations in a very generic form with many levels of indirection, obviously, the transformations, implemented with
our framework, are at disadvantage in comparison with hard coded ones in terms of performance. We assume that the performance of transformations
can be essentially improved by applying some techniques like staging~\cite{Staged} or, perhaps, object-specific optimisations.

Another important direction is supporting more kinds of type declarations, in the first hand, GADTs and non-regular types. Although we have some
implementation ideas for this case, the solution we came up with so far makes the interface of the whole framework too cumbersome to use even for
simple cases.

Finally, the typeinfo structure we generate can be used to mimic the \emph{ad-hoc} polymorphism as it contains the implementation of
type-indexed functions. This, together with some proposed extensions~\cite{ModularImplicits}, can open interesting perspectives.



%Text of paper \ldots


%% Acknowledgments
%\begin{acks}                            %% acks environment is optional
                                        %% contents suppressed with 'anonymous'
  %% Commands \grantsponsor{<sponsorID>}{<name>}{<url>} and
  %% \grantnum[<url>]{<sponsorID>}{<number>} should be used to
  %% acknowledge financial support and will be used by metadata
  %% extraction tools.
 % This material is based upon work supported by the
  %\grantsponsor{GS100000001}{National Science
   % Foundation}{http://dx.doi.org/10.13039/100000001} under Grant
  %No.~\grantnum{GS100000001}{nnnnnnn} and Grant
  %No.~\grantnum{GS100000001}{mmmmmmm}.  Any opinions, findings, and
  %conclusions or recommendations expressed in this material are those
  %of the author and do not necessarily reflect the views of the
  %National Science Foundation.
%\end{acks}


%% Bibliography
\bibliography{main}


%% Appendix
%\appendix
%\section{Appendix}

%Text of appendix \ldots

\end{document}
