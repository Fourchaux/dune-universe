\documentclass[acmlarge]{acmart}
\usepackage[
    type={CC},           % your choice
    modifier={by-sa},    % your choice
    version={4.0},       % your choice
]{doclicense}            % your choice, see \doclicenseThis below


\usepackage{alltt}
%\usepackage{pslatex}
%\usepackage{epigraph}
%\usepackage{verbatim}
\usepackage{latexsym}
\usepackage{array}
%\usepackage{comment}
%\usepackage{makeidx}
%\usepackage{indentfirst}
%\usepackage{verbatim}
%\usepackage{color}
%\usepackage{url}
%\usepackage{xspace}
%\usepackage{hyperref}
%\usepackage{stmaryrd}
\usepackage{amsmath, amsthm}
%\usepackage{graphicx}
%\usepackage{euscript}
\usepackage{mathtools}
%\usepackage{mathrsfs}
%\usepackage{multirow,bigdelim}
%\usepackage{subcaption}
%\usepackage{placeins}
\usepackage{csvsimple}
\usepackage{array}

%\geometry{
%     top=18pt, bottom=14pt, inner=21pt, outer=21pt,
%     paperwidth=5.5in, paperheight=8.5in,
%     }
     
\settopmatter{printacmref=false}
\fancyfoot{}
 
\makeatletter
\def\@formatdoi#1{}
\def\@permissionCodeOne{miniKanren.org/workshop}
\def\@copyrightpermission{\doclicenseThis} 
\def\@copyrightowner{Copyright held by the author(s).}
\makeatother

\copyrightyear{2019}
\setcopyright{rightsretained}

\acmMonth{8}
\acmArticle{3} % your article number, same as in HotCRP



%% Bibliography style
\bibliographystyle{ACM-Reference-Format}
%% Citation style
%% Note: author/year citations are required for papers published as an
%% issue of PACMPL.
\citestyle{acmauthoryear}   %% For author/year citations


%%%%%%%%%%%%%%%%%%%%%%%%%%%%%%%%%%%%%%%%%%%%%%%%%%%%%%%%%%%%%%%%%%%%%%
%% Note: Authors migrating a paper from PACMPL format to traditional
%% SIGPLAN proceedings format must update the '\documentclass' and
%% topmatter commands above; see 'acmart-sigplanproc-template.tex'.
%%%%%%%%%%%%%%%%%%%%%%%%%%%%%%%%%%%%%%%%%%%%%%%%%%%%%%%%%%%%%%%%%%%%%%


%% Some recommended packages.
\usepackage{booktabs}   %% For formal tables:
                        %% http://ctan.org/pkg/booktabs
\usepackage{subcaption} %% For complex figures with subfigures/subcaptions
                        %% http://ctan.org/pkg/subcaption
\usepackage{multirow}

\usepackage{placeins}

\usepackage{listings}
\lstdefinelanguage{ocanren}{
keywords={run, conde, fresh, let, in, match, with, when, class, type,
object, method, of, rec, repeat, until, while, not, do, done, as, val, inherit,
new, module, sig, deriving, datatype, struct, if, then, else, open, private, virtual, include, success, failure,
true, false},
sensitive=true,
commentstyle=\small\itshape\ttfamily,
keywordstyle=\ttfamily\textbf,
identifierstyle=\ttfamily,
basewidth={0.5em,0.5em},
columns=fixed,
mathescape=true,
fontadjust=true,
literate={fun}{{$\lambda$}}1 {->}{{$\to$}}3 {===}{{$\equiv$}}1 {=/=}{{$\not\equiv$}}1 {|>}{{$\triangleright$}}3 {\\/}{{$\vee$}}2 {/\\}{{$\wedge$}}2 {^}{{$\uparrow$}}1,
morecomment=[s]{(*}{*)}
}

\lstset{
%mathescape=true,
%basicstyle=\small,
%identifierstyle=\ttfamily,
%keywordstyle=\bfseries,
%commentstyle=\scriptsize\rmfamily,
%basewidth={0.5em,0.5em},
%fontadjust=true,
language=ocanren
}

\newcommand{\lstquot}[1]{``\lstinline{#1}''}
\newcommand{\sembr}[1]{\llbracket{#1}\rrbracket}
\newcommand\false{$f\!alse$}
\newcommand\myif{i\!f}


\def\transarrow{\xrightarrow}
\newcommand{\setarrow}[1]{\def\transarrow{#1}}

\def\padding{\phantom{X}}
\newcommand{\setpadding}[1]{\def\padding{#1}} 

\def\subarrow{}
\newcommand{\setsubarrow}[1]{\def\subarrow{#1}}

\newcommand{\trule}[2]{\dfrac{#1}{#2}}
\newcommand{\crule}[3]{\dfrac{#1}{#2},\;{#3}}
\newcommand{\withenv}[2]{{#1}\vdash{#2}}
\newcommand{\trans}[3]{{#1}\transarrow{\padding{\textstyle #2}\padding}\subarrow{#3}}
\newcommand{\ctrans}[4]{{#1}\transarrow{\padding#2\padding}\subarrow{#3},\;{#4}}
\newcommand{\llang}[1]{\mbox{\lstinline[mathescape]|#1|}}
\newcommand{\pair}[2]{\inbr{{#1}\mid{#2}}}
\newcommand{\inbr}[1]{\left<{#1}\right>}
\newcommand{\highlight}[1]{\color{red}{#1}}
%\newcommand{\ruleno}[1]{\eqno[\scriptsize\textsc{#1}]}
\newcommand{\ruleno}[1]{\mbox{[\textsc{#1}]}}
\newcommand{\rulename}[1]{\textsc{#1}}
\newcommand{\inmath}[1]{\mbox{$#1$}}
\newcommand{\lfp}[1]{fix_{#1}}
\newcommand{\gfp}[1]{Fix_{#1}}
\newcommand{\vsep}{\vspace{-2mm}}
\newcommand{\supp}[1]{\scriptsize{#1}}
\renewcommand{\sembr}[1]{\llbracket{#1}\rrbracket}
\newcommand{\cd}[1]{\texttt{#1}}
\newcommand{\free}[1]{\boxed{#1}}
\newcommand{\binds}{\;\mapsto\;}
\newcommand{\dbi}[1]{\mbox{\bf{#1}}}
\newcommand{\sv}[1]{\mbox{\textbf{#1}}}
\newcommand{\bnd}[2]{{#1}\mkern-9mu\binds\mkern-9mu{#2}}
\newcommand{\meta}[1]{{\mathcal{#1}}}
\newcommand{\dom}[1]{\mathtt{dom}\;{#1}}
%\newcommand{\primi}[2]{\mathbf{#1}\;{#2}}
\renewcommand{\dom}[1]{\mathcal{D}om\,({#1})}
\newcommand{\ran}[1]{\mathcal{VR}an\,({#1})}
\newcommand{\fv}[1]{\mathcal{FV}\,({#1})}
\newcommand{\tr}[1]{\mathcal{T}r_{#1}}
\newcommand{\diseq}{\not\equiv}
\newcommand{\reprfunset}{\mathcal{R}}
\newcommand{\reprfun}{\mathfrak{f}}
\newcommand{\cstore}{\Omega}
\newcommand{\cstoreinit}{\cstore_\epsilon^{init}}
\newcommand{\csadd}[3]{add(#1, #2 \diseq #3)}  %{#1 + [#2 \diseq #3]}
\newcommand{\csupdate}[2]{update(#1, #2)}  %{#1 \cdot #2}
\newcommand{\primi}[1]{\mathbf{#1}}
\newcommand{\sem}[1]{\llbracket #1 \rrbracket}
\newcommand{\ir}{\ensuremath{\mathcal{S}}}
\usepackage{tikz}
\newcommand*\circled[1]{\tikz[baseline=(char.base)]{
    \node[shape=circle,draw,inner sep=1pt] (char) {#1};}}

\let\emptyset\varnothing
\let\eps\varepsilon

\sloppy 

\newtheorem{Observation}{Observation}

\begin{document}

\title[Relational Synthesis of Pattern Matching]{Relational Synthesis for Pattern Matching}    

\titlenote{This work was partially supported by the grant 18-01-00380 from The Russian Foundation for Basic Research} %% \titlenote is optional;


\author{Dmitry Kosarev}
\email{Dmitrii.Kosarev@pm.me}

\author{Dmitry Boulytchev}
\email{dboulytchev@math.spbu.ru}    

\affiliation{
  \institution{Saint Petersburg State University}
  \country{Russia}                   
}

\affiliation{
  \institution{JetBrains Research}   
  \country{Russia}                   
}


%% Abstract
%% Note: \begin{abstract}...\end{abstract} environment must come
%% before \maketitle command
\begin{abstract}
  We apply relational programming techniques to the problem of synthesizing efficient implementation for a pattern matching construct. Although in principle
  pattern matching can be implemented in a trivial way, the result suffers from inefficiency in terms of both performance and code size. Thus, in implementing functional languages alternative, more elaborate  approaches are widely used. However, as there are multiple kinds and flavors of pattern
  matching constructs, these approaches have to be specifically developed and justified for each concrete inhabitant of the pattern matching ``zoo.'' We formulate the
  pattern matching synthesis problem in relational terms and develop optimizations which improve the efficiency of the synthesis and guarantee the
  optimality of the result. Our approach is based on relational representations of both the high-level semantics of pattern matching and the semantics of
  an intermediate-level implementation language. This choice make our approach, in principle, more scalable as we only need to modify the high-level semantics in order
  to synthesize the implementation of a new feature. Our evaluation on a set of small samples, partially taken from existing literature shows, that our framework is
  capable of synthesizing optimal implementations quickly. Our in-depth stress evaluation on a number of artificial benchmarks, however,
  has shown the need for future improvements.
\end{abstract}


%% 2012 ACM Computing Classification System (CSS) concepts
%% Generate at 'http://dl.acm.org/ccs/ccs.cfm'.
\begin{CCSXML}
<ccs2012>
<concept>
<concept_id>10011007.10011006.10011008.10011009.10011015</concept_id>
<concept_desc>Software and its engineering~Constraint and logic languages</concept_desc>
<concept_significance>500</concept_significance>
</concept>
<concept>
<concept_id>10011007.10011006.10011041.10011047</concept_id>
<concept_desc>Software and its engineering~Source code generation</concept_desc>
<concept_significance>500</concept_significance>
</concept>
</ccs2012>
\end{CCSXML}

\ccsdesc[500]{Software and its engineering~Constraint and logic languages}
\ccsdesc[500]{Software and its engineering~Source code generation}
%% End of generated code


%% Keywords
%% comma separated list
\keywords{relational programming, relational interpreters, pattern matching}  %% \keywords are mandatory in final camera-ready submission


%% \maketitle
%% Note: \maketitle command must come after title commands, author
%% commands, abstract environment, Computing Classification System
%% environment and commands, and keywords command.
\maketitle

\thispagestyle{empty}

\section{Introduction}
\label{sec:intro}

Algebraic data types (ADT) are an important tool in functional programming which deliver a way to represent flexible and easy to manipulate data structures.
To inspect the contents of an ADT's values a generic construct~--- \emph{pattern matching}~--- is used. The importance of pattern matching efficient
implementation stimulated the development of various advanced techniques which provide good results in practice. The objective of our work is to use these
results as a baseline for a case study of relational synthesis\footnote{We have to note that this term is overloaded and can be used to refer to completely
different approaches than we utilize.}~--- an approach for program synthesis based on application of relational programming~\cite{TRS,WillThesis}, and,
in particular, relational interpreters~\cite{unified} and relational conversion~\cite{conversion}. Relational programming can be considered as a specific form
of constraint logic programming centered around \textsc{miniKanren}\footnote{\url{http://minikanren.org}}, a combinator-based DSL, implemented for a number of host languages.
Unlike \textsc{Prolog}, which employs a deterministic depth-first search, \textsc{miniKanren} advocates a 
%completely 
more
declarative approach, in which a user is not
allowed to rely on a concrete search discipline, which means, that the specifications, written in \textsc{miniKanren}, are understood much more symmetrically.
The distinctive feature of \textsc{miniKanren} is complete \emph{interleaving search}~\cite{search}. The basic constraint is unification with occurs check, although
advanced implementations support other primitive constructs, such as disequality or finite-domain constraints~\cite{CKanren}. Syntactically, \textsc{miniKanren} is mutually
convertible to \textsc{Prolog}, but, unlike latter, makes use of explicit logical connectives (conjunction and disjunction), existential quantification and unification.
 
A distinctive application of relational programming is implementing \emph{relational interpreters}~\cite{Untagged}. Unlike conventional interpreters, which for a program and
input value produce output, relational interpreters can operate in various directions: for example, they are capable of computing an input value for a given
program and a given output, or even synthesize a program for a given pairs of input-output values. The latter case forms a basis for program synthesis~\cite{eigen,unified}.

Our approach is based on relational representation of the source language pattern matching semantics on the one hand, and
the semantics of the intermediate-level implementation language on the other. We formulate the condition necessary for a correct and complete implementation of pattern matching and use it to
construct a top-level goal which represents a search procedure for all correct and complete implementations. We also present a number of techniques which make it possible to come up with an
\emph{optimal} solution as well as optimizations to improve the performance of the search. Similarly to many other prior works we use the size of the synthesized code, which can be measured
statically, to distinguish better programs. Our implementation\footnote{\url{https://github.com/Kakadu/pat-match/tree/aplas2020}} makes use of \textsc{OCanren}\footnote{\url{https://github.com/JetBrains-Research/OCanren}}~---
 a typed implementation of \textsc{miniKanren} for \textsc{OCaml}~\cite{OCanren}, and \textsc{noCanren}\footnote{\url{https://github.com/Lozov-Petr/noCanren}}~--- 
a converter from the subset of plain \textsc{OCaml} into \textsc{OCanren}~\cite{conversion}. An initial  evaluation, performed for a set of benchmarks taken from other papers, showed our synthesizer performing well.
However, being aware of some pitfalls of our approach, we came up with a set of counterexamples on which it did not provide any results in observable time, so we do not consider the problem
completely solved. We also started to work on mechanized 
formalization\footnote{\url{https://github.com/dboulytchev/Coq-matching-workout}},
written in \textsc{Coq}~\cite{Coq}, to make the justification of our approach more solid and easier to verify, but this formalization is not yet complete. 

 

\begin{comment}
We apply relational programming techniques to the problem of synthesizing efficient implementation for a pattern matching construct.
Although in principle pattern matching can be implemented in a trivial way, the result suffers from inefficiency in terms of both
performance and code size. Thus, in implementing functional languages alternative, more elaborate  approaches are widely used.
However, as there are multiple kinds and flavors of pattern matching constructs, these approaches have to be specifically developed
and justified for each concrete inhabitant of the pattern matching ``zoo''. We formulate the pattern matching synthesis problem in
declarative terms and apply relational programming, a specific form of constraint logic programming, to develop a 
develop optimizations which improve the efficiency of the synthesis and guarantee the
optimality of the result. 
\end{comment}

\section{The Pattern Matching Synthesis Problem}

We describe here a simplified view on pattern matching which does not incorporate some practically important aspects of the construct such as
name bindings in patterns, guards or even semantic actions in branches. In a purified form, however, it  represents the essence of pattern
matching as an ``inspect-and-branch'' procedure. Other features can be easily added later once a solution for the essential part of the problem
is found.

First, we introduce a finite set of \emph{constructors} $\mathcal C$, equipped with arities, a set of values $\mathcal{V}$
and a set of patterns $\mathcal{P}$:
 
\[
 \begin{array}{rcll}
    \mathcal{C} & = & \{ C_1^{k_1}, \dots, C_n^{k_n} \}\\
    \mathcal{V} & = & \mathcal{C}\,\mathcal{V}^*\\  
    \mathcal{P} & = & \_ \mid \mathcal{C}\,\mathcal{P}^*
 \end{array}
\]

We define a matching of a value $v$ (\emph{scrutinee}) against an ordered non-empty sequence of patterns $p_1,\dots,p_k$ by means of the following
relation

\[
\setarrow{\xrightarrow}
\trans{\inbr{v;\,p_1,\dots,p_k}}{}{i},\,1\le i\le k+1
\]

\noindent which gives us the index of the leftmost matched pattern or $k+1$ if no such pattern exists. We use an auxiliary relation $\inbr{;}\subseteq\mathcal{V}\times\mathcal{P}$
to specify the notion of a value matched by an individual pattern (see Fig.~\ref{fig:match1pat}). The rule \ruleno{Wildcard} says that
a wildcard pattern ``\_'' matches any value, and \ruleno{Constructor} specifies that a constructor pattern matches exactly those values which
have the same constructor at the top level and all subvalues matched by corresponding subpatterns. The definition of ``$\xrightarrow{}{\!\!}$'' is
shown on Fig.~\ref{fig:matchpatts}. An auxiliary relation
 ``$\xrightarrow{}{}_{\!\!*}$'' 
is introduced to specify the left-to-right matching strategy, and we
use current index as an environment. An important rule, $\ruleno{MatchOtherwise}$ specifies that if we exhausted all the patterns with no matching we stop with
the current index (which in this case is equal to the number of patterns plus one).

\begin{figure}[t]
   \renewcommand*{\arraystretch}{2}
   \[
   \begin{array}{cr}
     \inbr{v;\,\_} & \ruleno{Wildcard} \\
     \trule{\forall i\;\inbr{v_i;\,p_i}}{\inbr{C^k\,v_1\dots v_k;\,C^k\,p_1\dots p_k}},\,k\ge 0 & \ruleno{Constructor}
   \end{array}
   \]
   \caption{Matching against a single pattern}
   \label{fig:match1pat}
\end{figure}

\begin{figure}[t]
   \renewcommand*{\arraystretch}{3}
   \setarrow{\xrightarrow}
   \setsubarrow{_*}
   \[
   \begin{array}{cr}
     \trule{\inbr{v;\,p_1}}
           {\withenv{i}{\trans{\inbr{v;\,p_1,\dots,p_k}}{}{i}}} & \ruleno{MatchHead}\\
     \trule{\neg\inbr{v;\,p_1}\qquad\withenv{i+1}{\trans{\inbr{v;\,p_2,\dots,p_k}}{}{j}}}
           {\withenv{i}{\trans{\inbr{v;\,p_1,\dots,p_k}}{}{j}}} & \ruleno{MatchTail}\\
     \withenv{i}{\trans{\inbr{v;\,\varepsilon}}{}{i}} & \ruleno{MatchOtherwise}\\
     \trule{\withenv{1}{\trans{\inbr{v;\,p_1,\dots,p_k}}{}{i}}}
           {\setsubarrow{}\trans{\inbr{v;\,p_1,\dots,p_k}}{}{i}} & \ruleno{Match}
   \end{array}
   \]
   \caption{Matching against an ordered sequence of patterns}
   \label{fig:matchpatts}
\end{figure}

The relation ``$\xrightarrow{}{}\!\!$'' gives us a \emph{declarative} semantics of pattern matching. Since we are interested in
synthesizing implementations, we need a \emph{programmatical} view on the same problem. Thus, we introduce a language $\mathcal S$
(the ``switch'' language) of test-and-branch constructs:

\[
\begin{array}{rccl}
  \mathcal M & = &       & \bullet \\
             &   & \mid  & \mathcal M\,[\mathbb{N}] \\
  \ir        & = &       & \primi{return}\,\mathbb{N} \\
             &   & \mid  & \primi{switch}\;\mathcal{M}\;\primi{with}\; [\mathcal{C}\; \primi{\rightarrow}\; \ir]^*\;\primi{otherwise}\;\ir
\end{array}
\]
 
Here $\mathcal{M}$ stands for a \emph{matching expression}, which is either a reference to a scrutinee ``$\bullet$'' or
a (multiply) indexed subexpression of a scrutinee. Programs in the switch language can discriminate on the
structure of matching expressions, testing their top-level constructors and eventually returning natural numbers as results.
The switch language is similar to the intermediate representations for pattern matching code used in 
previous works on pattern matching implementation~\cite{maranget2001,maranget2008}, and switch programs are analogous to
\emph{decision trees}.

The semantics of the switch language is given by mean of relations ``$\xrightarrow{}{}_{\!\!\!\mathcal M}$'' and ``$\xrightarrow{}{}_{\!\!\mathcal S}$''
(see Fig.~\ref{fig:matchexpr} and \ref{fig:test-and-branch}). The first one describes the semantics of matching expression, while
the second describes the semantics of the switch language itself. In both cases the scrutinee $v$ is used as an environment ($v\vdash$).


\begin{figure}[t]
  \renewcommand*{\arraystretch}{2}
  \setarrow{\xrightarrow}
  \setsubarrow{_{\mathcal M}}
  \[
  \begin{array}{cr}
    \withenv{v}{\trans{\bullet}{}{v}} & \ruleno{Scrutinee} \\
    \trule{\withenv{v}{\trans{m}{}{C^k v_1\dots v_k}}}{\withenv{v}{\trans{m[i]}{}{v_i}}} & \ruleno{SubMatch} 
  \end{array}
  \]
  \caption{Semantics of matching expression}
  \label{fig:matchexpr}
\end{figure}

\begin{figure}[t]
  \setarrow{\xrightarrow}
  \setsubarrow{_{\mathcal S}}
  \[
  \begin{array}{cr}
    \withenv{v}{\trans{\primi{return}\;i}{}{i}} & \ruleno{Return}\\[10mm]
    
    \trule{\renewcommand*{\arraystretch}{1}
           \begin{array}{c}        
              {\setsubarrow{_{\mathcal M}}\withenv{v}{\trans{m}{}{C^k\ v_1 \dots v_k}}} \\
              \withenv{v}{\trans{s}{}{i}}
           \end{array}
          }    
          {\withenv{v}{\trans{\primi{switch}\;m\;\primi{with}\;[C^k\to s]s^*\;\primi{otherwise}\;s^\prime}{}{i}}} & \ruleno{SwitchMatched}\\[10mm]
          
    \trule{\renewcommand*{\arraystretch}{1}
           \begin{array}{c}        
             {\setsubarrow{_{\mathcal M}}\withenv{v}{\trans{m}{}{D^n\  v_1 \dots v_n}}}\\
             C^k\ne D^n\\
             \withenv{v}{\trans{\primi{switch}\;m\;\primi{with}\;s^*\;\primi{otherwise}\;s^\prime}{}{i}}
           \end{array}
          }
          {\withenv{v}{\trans{\primi{switch}\;m\;\primi{with}\;[C^k\to s]s^*\;\primi{otherwise}\;s^\prime}{}{i}}} & \ruleno{SwitchNotMatched}\\[10mm]
          
    \trule{\withenv{v}{\trans{s}{}{i}}}{\withenv{v}{\trans{\primi{switch}\;m\;\primi{with}\;\varepsilon\;\primi{otherwise}\;s}{}{i}}} & \ruleno{SwitchOtherwise}
  \end{array}
  \]
  \caption{Semantics of switch programs}
  \label{fig:test-and-branch}
\end{figure}

The following observations can be easily proven by structural induction.

\begin{Observation}
  For arbitrary pattern the set of matching values is non-empty:

  \[
  \forall p\in\mathcal P : \{v\in\mathcal V\mid \inbr{v;\,p}\}\ne\emptyset
  \]
\end{Observation}

\begin{Observation}
  Relations ``$\xrightarrow{}{}\!\!\!$'' and ``$\xrightarrow{}{}_{\!\!\mathcal S}$'' are functional and deterministic respectively:

  \[
  \setarrow{\xrightarrow}
  \begin{array}{rcl}
    \forall p_1,\dots,p_k\in\mathcal P,\,\forall v\in \mathcal V,\,\forall \pi\in\mathcal S & : & |\{i\in\mathbb N\mid \trans{\inbr{v;\,p_1,\dots,p_k}}{}{i}\}|=1 \\
                                                                 &  & {\setsubarrow{_{\mathcal S}}|\{i\in\mathbb N\mid \withenv{v}{\trans{\pi}{}{i}}\}|\le 1}
  \end{array}
  \]
\end{Observation}

With these definitions, we can formulate the \emph{pattern matching synthesis problem} as follows: for a given ordered sequence of patterns $p_1,\dots,p_k$ find
a switch program $\pi$, such that

\[
\setarrow{\xrightarrow}
\forall v\in \mathcal V,\; \forall 1\le i\le n+1 : \trans{\inbr{v;\,p_1,\dots,p_n}}{}{i}\Longleftrightarrow{\setsubarrow{_{\mathcal S}}\withenv{v}{\trans{\pi}{}{i}}}\eqno{(\star)}
\]

In other words, program $\pi$ delivers a correct and complete implementation for pattern matching.

\section{Pattern Matching Synthesis, Relationally}
\label{sec:relationally}

In this section we describe a relational formulation for the pattern matching synthesis problem. Practically,
this amounts to constructing a goal with a free variable corresponding to the switch program to synthesize
for (arbitrary) list of patterns. In order to come up with a tractable goal certain steps have to be performed.
We first describe the general idea, and then consider these steps in detail.

Our idea of using relational programming for pattern matching synthesis is based on the following observations:

\begin{itemize}
\item For the switch language we can implement a relational interpreter\footnote{Conventionally for \textsc{miniKanren},
  the names of relations are superscripted by ``$^o$''.} $eval^o_\ir$ with the following property: for
  arbitrary $v\in\mathcal V$, $\pi\in\ir$ and $i\in\mathbb N$
 
  \[
  \setarrow{\xrightarrow}
  \setsubarrow{_\ir}
   eval^o_\ir\, v\, \pi\, i \Longleftrightarrow \withenv{v}{\trans{\pi}{}{i}}
  \]

  In other words, $eval^o_\ir$ interprets a program $\pi$ for a scrutinee $v$ and returns exactly the same branch (if any)
  which is prescribed by the semantics of the switch language. 
  
\item On the other hand, we can directly encode the declarative semantics of pattern matching as a relational
  program $match^o$ such that for arbitrary $v\in\mathcal V$, $p_i\in\mathcal P$ and $i\in\mathbb N$

  \[
  \setarrow{\xrightarrow}
  match^o\,v\,p_1,\dots,p_k\,i \Longleftrightarrow \trans{\inbr{v;\,p_1,\dots,p_k}}{}{i}
  \]

  Again, $match^o$ succeeds with $1\le i\le k$ iff $p_i$ is the leftmost pattern, matching $v$; otherwise it
  succeeds with $i=k+1$.
\end{itemize}

We address the construction of relational interpreters for both semantics in Section~\ref{sec:relints}.

Being relational, both $eval^o_\ir$ and $match^o$ do not just succeed or fail for ground arguments, but also can be \emph{queried} for
arguments with free logical variables, thus performing a search for all substitutions for these variables which make the
relation hold. This observation leads us to the idea of utilizing the definition of the pattern matching
synthesis problem, replacing ``$\xrightarrow{}{}\!\!$'' with $match^o$, ``$\xrightarrow{}{}_{\!\!\!\mathcal S}$`` with $eval^o$,
and $\pi$ with a free logical variable $\circled{?}$, which gives us the goal

\[
\forall v\in \mathcal V,\; \forall 1\le i\le n+1 : match^o\,v\,p_1,\dots,p_n\,i\Longleftrightarrow eval^o\,v\,\circled{?}\,i
\]

\noindent This goal, however, is problematic from relational point of view for a number of reasons.

First, \textsc{miniKanren} provides rather a limited support for universal quantification. Apart from being inefficient from
a performance standpoint, existing implementations either do not coexist with disequality constraints~\cite{eigen}
or do not support quantified goals with an infinite number of answers~\cite{moiseenko}. As we will see below, both restrictions are
violated in our case. Second, there is no direct support for the equivalence of goals (``$\Leftrightarrow$''). Thus,
reducing the original synthesis problem to a viable relational goal involves some ``massaging''.

We eliminate the universal quantification over the infinite set of scrutinees, replacing it by a \emph{finite}
conjunction over a \emph{complete set of samples}. For a sequence of patterns $p_1,\dots,p_k$ a
complete set of samples is a finite set of values $\mathcal{E}(p_1,\dots,p_k)\subseteq\mathcal{V}$ with the following
property:

\[
\setarrow{\xrightarrow}
\begin{array}{rcl}
  \forall\pi\in\mathcal S\! &\!: & [\forall v\in\mathcal{E}(p_1,\dots,p_k),\,\forall i\in\mathbb{N} : \trans{\inbr{v;\,p_1,\dots,p_k}}{}{i}\!\Longleftrightarrow \!{\setsubarrow{_{\mathcal S}}\withenv{v}{\trans{\pi\!}{}{i}}}]\!\Rightarrow\\
                          &   & [\forall v\in\mathcal V,\,\forall i\in\mathbb{N} : \trans{\inbr{v;\,p_1,\dots,p_k}}{}{i} \Longleftrightarrow  {\setsubarrow{_{\mathcal S}}\withenv{v}{\trans{\pi}{}{i}}}]
\end{array}
\]

In other words, if a program implements a correct and complete pattern matching for all values in a complete set of samples, then this
program implements a correct and complete pattern matching for all values. The idea of using a complete set of samples originates from the following observation: each pattern
describes a (potentially infinite) set of values, and pattern matching splits the set of all values into equivalence classes, each corresponding to a certain matching pattern.
Moreover, the values of different classes can be distinguished only by looking down to a \emph{finite} depth (as different patterns can be distinguished in this way).
The generation of a complete sample set will be addressed below (see Section~\ref{sec:samples}). Example-based program synthesis is not a completely new technique in relational
programming~\cite{unified}; in our case, however, we can ensure the correctness of the synthesis result, while in previous reports it had to be established externally.

\setarrow{\xrightarrow}

To eliminate the universal quantification over the set of answers we rely on the functionality of declarative pattern matching semantics. Indeed, given a fixed sequence $p_1,\dots,p_k$
of patterns, for every value $v$ there is exactly one answer value $i$, such that $\trans{\inbr{v;\,p_1,\dots,p_k}}{}{i}$. We can reformulate this property as

\[
\exists i:\, \trans{\inbr{v;\,p_1,\dots,p_k}}{}{i} \Longrightarrow  
\Big(\forall j : \trans{\inbr{v;\,p_1,\dots,p_k}}{}{j} \Longrightarrow  j=i\Big)
\]

Thus, we can replace universal quantification over the sets of answers by existential one, for which we have an efficient relational counterpart~--- the ``\lstinline|fresh|''
construct.

Following the same argument, we may replace the equivalence with conjunction: indeed, if

\[
\setarrow{\xrightarrow}
\trans{\inbr{v;\,p_1,\dots,p_k}}{}{i}
\]

for some $i$, then (by functionality), for any other $j\ne i$

\[
\setarrow{\xrightarrow}
\neg\;(\trans{\inbr{v;\,p_1,\dots,p_k}}{}{j})
\]

A correct pattern matching implementation $\pi$ should satisfy the condition

\[
\setarrow{\xrightarrow}
\setsubarrow{_{\mathcal S}}
\withenv{v}{\trans{\pi}{}{i}}
\]

But, by the determinism of the switch language semantics, it immediately follows, that for arbitrary $j\ne i$

\[
\setarrow{\xrightarrow}
\setsubarrow{_{\mathcal S}}
\neg\;(\withenv{v}{\trans{\pi}{}{j}})
\]

\begin{comment}
Alternatively\footnote{\color{red} Reviewer N1 said that passage about bool argument is unclear and may be omitted (or described with more details)}, we could switch to a more explicit relational representation of both semantics, adding an extra boolean argument to
both $eval^o_{\mathcal S}$ and $match^o$ and using the same fresh variable $b$ in the query of interest:

\[
match^o\,v\,p_1,\dots,p_k\,i\,b \wedge eval^o_{\mathcal S}\,v\,\pi\,i\,b
\]
\end{comment}
Thus, the goal we eventually came up with is

\[
\bigwedge_{v\in\mathcal{E}\,(p_1,\dots,p_k)}\mbox{\lstinline|fresh ($i$)|}\; \{match^o\ v\,\ p_1,\dots,p_k\ i \ \wedge \ eval^o_{\mathcal S}\,v\,\circled{?}\ i \}
\eqno{(\star\star)}
\]

From a relational point of view this is a pretty conventional goal which can be solved by virtually any decent \textsc{miniKanren} implementation in
which the relations $eval^o_{\mathcal S}$ and $match^o$ can be encoded.

Finally, we can make the following important observation. Obviously, any pattern matching synthesis problem has at least one trivial solution.
This, due to the completeness of relational interleaving search~\cite{search,certifiedSemantics}, means that the goal above \emph{can not diverge} with
no results. Actually it is rather easy to see that any pattern matching synthesis problem has \emph{infinitely many} solutions: indeed, having just
one it is always possible to ``pump'' it with superfluous ``$\primi{otherwise}$'' clauses; thus, the goal above is \emph{refutationally
complete}~\cite{WillThesis,DivergenceTest}. These observations justify the totality of our synthesis approach. In Section~\ref{sec:optimization} we show
how we can make it provide optimal solution.

\subsection{Constructing Relational Interpreters}
\label{sec:relints}

In this section we address the implementation of relations $eval^o_{\mathcal S}$ and $match^o$. In principle, it amounts to accurate encoding of
relations 
``$\xRightarrow{}{}\!\!$'' and ``$\xRightarrow{}{}_{\!\!\mathcal S}$'' 
in \textsc{miniKanren} (in our case, \textsc{OCanren}). We, however,
make use of a relational conversion~\cite{conversion} tool, called \textsc{noCanren}, which automatically converts a subset of \textsc{OCaml} into
\textsc{OCanren}. Thus, both interpreters are in fact implemented in \textsc{OCaml} and repeat corresponding inference rules almost
literally in a familiar functional style. For example, functional implementation of a declarative semantics looks like follows:

\begin{lstlisting}
   let rec $\inbr{v;\,p}$ =
     match ($v$, $p$) with
     | (_, Wildcard) -> true
     | ($C^k\;v^*$, $C^k\;p^*$) -> list_all $\inbr{;}$ (list_combine $v^*$ $p^*$)
     | _             -> false

  let $match^o$ $v$ $p^*$ =
    let rec inner $i$ $p^*$ =
      match $p^*$ with
      | []      -> $i$
      | $p$ :: $p^*$ -> if $\inbr{v;\,p}$ then $i$ else inner S($i$) $p^*$
    in inner O $p^*$
\end{lstlisting}

We mixed here the concrete syntax of \textsc{OCaml} and mathematical notation, used in the definition of the relation in question, to underline their similarity;
the actual implementation only a few lines of code longer. Note, we use here natural numbers in Peano form and custom list processing functions in order
to apply relational conversion later.

Using relational conversion saves a lot of efforts as \textsc{OCanren} specifications tend to be much more verbose; in addition
relational conversion implements some ``best practices'' in relational programming (for example, moves unifications forward in
conjunctions and puts recursive calls last). Finally, it has to be taken into account that relational conversion of pattern matching introduces
disequality constraints.

\subsection{Dealing with a Complete Set of Samples}
\label{sec:samples}

As we mentioned above, a complete set of samples plays an important role in our approach: it allows us to eliminate universal quantification over the
set of all values. As we replace the universal quantifier with a finite conjunction with one conjunct per sample value reducing the size of
the set is an important task. At the present time, however, we build an excessively large (worst case exponential of depth) number of samples. We discuss
the issues with this choice in Section~\ref{sec:eval} and consider developing a more advanced approach as the main direction for
improvement.

Our construction of a complete set of samples is based upon the following simple observations. We simultaneously define the \emph{depth} measure
for patterns and sequences of patterns as follows:

\[
\begin{array}{rcl}
   d\,(p_1,\dots,p_k)     & = & max\, \{ d\,(p_i)\}\\
   d\,(\_)                 & = & 0 \\
   d\,(C^k\,p_1,\dots p_k) & = & 1 + d\,(p_1,\dots,p_k)
\end{array}
\]

\noindent As a sequence of patterns is the single input in our synthesis approach we will call its depth \emph{synthesis depth}.

Similarly, we define the depth of matching expressions

\[
\begin{array}{rcl}
  d_{\mathcal M}\,(\bullet) & = & 1 \\
  d_{\mathcal M}\,(m\,[i]) & = & 1 + d_{\mathcal M}\,(m)\\
\end{array}
\]

and switch programs:

\[
\begin{array}{rcl}
  d_{\mathcal S}\,(\primi{return}\;i)&=&0\\
  d_{\mathcal S}\,(\primi{switch}\;m\;\primi{of}\;C_1\to s_1,\dots,C_k\to s_k\;\primi{otherwise}\;s)&=&\\
  \multicolumn{3}{c}{\qquad\qquad\qquad\qquad\qquad\qquad\qquad\qquad\qquad\qquad max\,\{d_{\mathcal M}\,(m),\,d_{\mathcal S}\,(s_i),\,d_{\mathcal S}\,(s)\}}
\end{array}
\]

Informally, the depth of a switch program tells us how deep the program can look into a value. 

From the definition of $\inbr{;}$ it immediately follows that a pattern $p$ can only discriminate values up to its depth $d\,(p)$: changing a value at the depth greater
or equal than $d\,(p)$ cannot affect the fact of matching/non matching. This means that we need only consider switch programs of depth no greater than the synthesis depth.
But for these programs the set of all values with height no greater than the synthesis depth forms a complete set of samples. Indeed, if the height of a value less or
equal to the synthesis depth, then this value is a member of complete set of samples and by definition the behavior of the synthesized program on this value is
correct. Otherwise there exists some value $s$ from the complete set of samples, such that given value can be obtained as an ``extension'' of $s$ at the
depth greater than the synthesis depth. Since neither declarative semantics nor switch programs can discriminate values at this depth, the behavior for a given value
will coincide with the correct-by-definition behavior for  $s$.

The implementation of complete set generation, again, is done using relational conversion. The enumeration of all terms up to a certain depth
can be acquired from a function which calculates the depth of a term: indeed, converting it into a relation and then running with \emph{fixed} depth
and \emph{free} term arguments delivers what we need. Thus, we add an extra conjunct which performs the enumeration of all values to the
relational goal $(\star\star)$, arriving at

\[
depth^o\,v\,n\wedge\mbox{\lstinline|fresh ($i$)|}\; \{match^o\,v\,p_1,\dots,p_k\,i \wedge eval^o_{\mathcal S}\,v\,\circled{?}\,i\}
\eqno{(\star\star\star)}
\]

Here $n$ is a precomputed synthesis depth in Peano form.

\begin{comment}
\begin{figure}[ht]
\begin{subfigure}[t]{0.2\linewidth}
  \[
  \{A^1,\,B^0,\,C^1,\,D^0\}
  \]
\vskip6mm
\caption{Constructors}
\label{fig:constructors}
\end{subfigure}
\hspace{0.5cm}
\begin{subfigure}[t]{0.26\linewidth}
  \[
  \begin{array}{c}
    C^1\,(A^1\,(B^0))\\
    C^1\,(\_)\\
    \_
  \end{array}
\]
\caption{Patterns}
\label{fig:patterns}
\end{subfigure}
\hspace{0.5cm}
\begin{subfigure}[t]{0.33\linewidth}
  \[
  \begin{array}{lcl}
     B             & \mapsto & 2 \\
     D             & \mapsto & 2 \\
     A\, (B)       & \mapsto & 2 \\
     A\, (D)       & \mapsto & 2 \\
     C\, (B)       & \mapsto & 1 \\
     C\, (D)       & \mapsto & 1 \\
     A\, (A\, (B)) & \mapsto & 2 \\
     A\, (A\, (D)) & \mapsto & 2 \\
     A\, (C\, (B)) & \mapsto & 2 \\
     A\, (C\, (D)) & \mapsto & 2 \\
     C\, (A\, (B)) & \mapsto & 0 \\
     C\, (A\, (D)) & \mapsto & 1 \\
     C\, (C\, (B)) & \mapsto & 1 \\
     C\, (C\, (D)) & \mapsto & 1 
  \end{array}
  \]
\caption{Generated samples}
\label{fig:samples}
\end{subfigure}
\caption{Complete set of samples example} 
\label{fig:complete-set-example}
\end{figure}
\end{comment}

\section{Implementation and Optimizations}
\label{sec:optimization}

In this section we address two aspects of our solution: a number of optimizations which make the search more efficient, and
the way it ends up with the optimal solution.

The relational goal in its final form, presented in the previous section, does not demonstrate good performance. Thus, we apply a number
of techniques, some of which require extending the implementation of the search. Namely, we apply the following optimizations:

\begin{itemize}
\item We make use of type information to restrict the subset of constructors which may appear in a certain branch of
  program being synthesized.
\item After a complete set of samples is generated, we use it to put auxiliary constraints on matching expressions. For example,
  if we can detect that a matching expression points to a subexpression of scrutinee which can start with a single constructor (like
  tuples), we can prohibit it from being considered during the synthesis.
\item We implement \emph{structural constraints} which allow us to restrict the shape of terms during the search, and
  utilize them to implement pruning.
\end{itemize}

In our formalization we do not make any use of types since as a rule type information does not affect matching. In addition,
utilizing the properties of a concrete type system would make our approach too coupled with this particular type system, hampering
its reusability for other languages. Nevertheless we may use a certain abstraction of type system which would deliver only
that part of information which is essential for our approach to function. Currently, we calculate the type of any matching expression in
the program being synthesized and from this type extract the subset of constructors which can appear when branching on this expression
is performed. The number of these constructors restricts the number of branches which a corresponding $\primi{switch}$ expression can have.
In our implementation we assume the constructor set ordered, and we consider only ordered branches, which restricts branching even more.


Our approach to finding an optimal solution in fact implements branch-and-bound strategy. The birds-eye view of our plan is as follows:

\begin{itemize}
\item We construct a trivial solution, which gives us the first estimate.
\item During the search we prune all partial solutions whose size exceeds the current estimate. We can do this due to the top-down nature of partial solution construction.
\item When we come up with a better solution we remember it and update current estimate.
\end{itemize}
% estimate ~ rough, estimation ~ approximate

\noindent This strategy inevitably delivers us the optimal solution since there are only finitely many switch programs, shorter than trivial solution.

In order to implement this strategy we extended \textsc{OCanren} with a new primitive called \emph{structural constraint}, which may
fail on some terms depending on some criterion specified by an end-user. Structural constraints can be seen as a generalization of
some known constraints\footnote{The constraint  \lstinline|symbol$^o$| is similar to \lstinline{symbol?} function in Scheme. The constraint \lstinline|absent$^o$| ensures that specific term is not a subterm of another term.}
 like \lstinline|absent$^o$| or \lstinline|symbol$^o$|
in existing \textsc{miniKanren} implementations~\cite{Untagged}, 
so they can be widely used in solving other problems as well. Note, we could implement other constraints we considered (on the
depth of switch programs, on the type of scrutinee) as structural.
However, our experience has shown that this leads to
a less efficient implementation. Since these constraints are inherent to the problem, we kept them hardcoded.

\subsection{Reducing the Complete Set of Samples}
\label{sec:reduced-samples}

Although in general our approach requires an exponential number of samples to be generated, in some cases a complete set of examples can be reduced.
For example, for the following pattern matching problem

\[
\begin{array}{l}
\mbox{\lstinline|(_, _ :: _ :: _)|} \\
\mbox{\lstinline|(_, _ :: _)|}
\end{array}
\]

the synthesized program should not investigate the left subtree of the scrutinee since its contents can not alter the behaviour of pattern matching.

The set of admissible matching values $s^\cup$ also can be restricted using the same arguments which we described in Section~\ref{sec:samples}.
This set essentially describes the paths to the ``interesting'' subexpressions of the scrutinee, and it can be computed statically before
the synthesis procedure:

\[
\begin{array}{rcl}
   s\,(m, C\ p_1 \dots p_k)     & = & \{m\}\cup \bigcup\limits_{i=1}^{k} s(m[i], p_i)\\
   s\,(m,\_)                 & = & \varnothing \\
   s^\cup\,(p_1,\dots, p_k) & = & \bigcup\limits_{i=1}^{k} s(\bullet, p_i)
\end{array}
\]

For the example above, the set  $s^\cup$ is

\[
\{\bullet, \bullet[1], \bullet[1][1]\}
\]

The complete set of samples then can be the following 3-element set:

\[
\begin{array}{l}
  \mbox{\lstinline|($\underline{[]}$, [])|}\\
  \mbox{\lstinline|($\underline{[]}$, $\;\underline{42}\;$ :: [])|}\\
  \mbox{\lstinline|($\underline{[]}$, $\;\underline{42}\;$ :: $\;\underline{42}\;$ :: $\;\underline{[]}$)|}
\end{array}
\]

\noindent where underlined expressions are chosen arbitrarily. A straightforward algorithm from the section~\ref{sec:samples} would generate the larger set of $2^3$ examples.

The set $s^\cup$ can be used for sample enumeration in the following manner. During the enumeration we hold current matching expression which will be used to
access current subtree of the sample. If that expression does not belong to $s^\cup$, we can choose an arbitrary inhabitant; if not we enumerate all
possible top-level constructors for this subexpression and recurse. The correctness of this algorithm relies on the fact that if an expression does
not belong to $s^\cup$, then all its extensions also do not belong to $s^\cup$.



\section{Evaluation}

\label{sec:evaluation}

In this section, we present an evaluation of 
implemented constructive negation on a series of examples.

\subsection{If-then-else}

Using relational if-then-else operator, 
presented in section~\ref{sec:ifte},
we have implemented several 
higher-order relations over lists, namely 
\lstinline{find} (Listing~\ref{lst:eval-find}), 
\lstinline{remove}\footnote{Note, this implementation 
differs from the one in Section~\ref{sec:intro}, but 
it is easy to see that these two are semantically equivalent.} (Listing~\ref{lst:eval-remove}) 
and \lstinline{filter} (Listing~\ref{lst:eval-filter}).
These relations are almost identical (syntactically) to their
functional implementations.
We have tested that these relations can be run
in various directions and produce the expected results.
For example, the goal \lstinline{filter p q q}
with the predicate \lstinline{p} equal to

\begin{lstlisting}
  fun l -> fresh (x) (l === [x])
\end{lstlisting}

stating that the given list should be a singleton list,
starts to generate all singleton lists.
Vice versa, the goal \lstinline{filter p q []} 
with that same \lstinline{p} generates 
all lists, constrained to be not a singleton list.

Listings~\ref{lst:eval-p}-\ref{lst:eval-filter-queries} give 
more concrete examples of queries to these relations.
In the listing the syntax \lstinline{run n q g}
means running a goal \lstinline{g} with 
the free variable \lstinline{q}
taking the first \lstinline{n} answers (``\lstinline{*}'' denotes all answers).
After the sign $\leadsto$ the result of the query is given.
The result \lstinline{fail} means that the query has failed.
The result \lstinline[mathescape]|succ {{a$_1$}; ... {a$_n$}} |
means that the query has succeeded delivering $n$ answers.
Each answer represents a set of constraint on free variables.
Constraints are of two forms: equality constraints, e.g. \lstinline{q = (1, _.$_0$)}, 
or disequality constraints, e.g. \lstinline{q $\neq$ (1, _.$_0$)}.
The terms of the form \lstinline{_.$_i$} in the answer
denote some universally quantified variables.

\begin{minipage}[thb]{.3\textwidth}
\begin{lstlisting}[
  caption={A definition of \code{find} relation},
  label={lst:eval-find}
]
let find p e xs =
  fresh (x xs' ys') (
    xs === x::xs' /\
    ifte (p x)
      (e === x)
      (find p e xs')
  )
\end{lstlisting}
\end{minipage}\hfill
\begin{minipage}[thb]{.3\textwidth}
\begin{lstlisting}[
  caption={A definition of \code{remove} relation},
  label={lst:eval-remove}
]
let remove p xs ys =
  (xs === [] /\ ys === [])
  \/
  fresh (x xs' ys') (
    xs === x::xs' /\
    ifte (p x)
      (ys === xs')
      (ys === x::ys' /\ 
       remove p xs' ys')
  )
\end{lstlisting}
\end{minipage}\hfill
\begin{minipage}[thb]{.3\textwidth}
\begin{lstlisting}[
  caption={A definition of \code{filter} relation},
  label={lst:eval-filter}
]
let filter p xs ys =
  (xs === [] /\ ys === [])
  \/
  fresh (x xs' ys') (
    xs === x::xs' /\
    (ifte (p x)
      (ys === x :: ys')
      (ys === ys')) /\
    filter p xs' ys'
  )
\end{lstlisting}
\end{minipage}

% \vspace{3cm}

\begin{minipage}[thb]{0.4\textwidth}
\begin{lstlisting}[
  caption={Definition of the predicate \lstinline{p}},
  label={lst:eval-p}
]
let p l = fresh (x) (l === [x])
\end{lstlisting}
\begin{lstlisting}[
  caption={Example of queries to \lstinline{find}},
  label={lst:eval-find-queries}
]
run 3 q (fresh (e) find p e q) 
$\leadsto$ succ {
     { q = [_.$_0$] :: _.$_1$ }
     { q = _.$_0$ :: [_.$_1$] :: _.$_2$; 
         _.$_0$ $\neq$ [_.$_3$] }
     { q = _.$_0$ :: _.$_1$ :: [_.$_2$] :: _.$_3$; 
         _.$_0$ $\neq$ [_.$_4$]; _.$_1$ $\neq$ [_.$_5$] }
   }
\end{lstlisting}
\end{minipage}\hfill
\begin{minipage}[thb]{0.4\textwidth}
\begin{lstlisting}[
  caption={Example of queries to \lstinline{remove}},
  label={lst:eval-remove-queries}
]
run * q (fresh (e) remove p q [[ ]]) 
$\leadsto$ succ {
     { q = [[_.$_0$]; [ ]] }
     { q = [[ ]] }
     { q = [[ ]; [_.$_0$]] }
   }

run 3 q (fresh (e) remove p q q) 
$\leadsto$ succ {
     { q = [] }
     { q = [_.$_0$], _.$_0$ $\neq$ [_.$_1$] }
     { q = [_.$_0$; _.$_1$]; 
         _.$_0$ $\neq$ [_.$_2$]; _.$_1$ $\neq$ [_.$_3$] }
   }
\end{lstlisting}
\end{minipage}

\begin{minipage}[thb]{0.4\textwidth}
\begin{lstlisting}[
  caption={Example of queries to \lstinline{filter}},
  label={lst:eval-filter-queries}
]
run 3 q (filter p q q) 
$\leadsto$ succ {
     { q = [ ] }
     { q = [_.$_0$] }
     { q = [_.$_0$; _.$_1$] }
   }

run 3 q (filter p q [1]) 
$\leadsto$ succ {
     { q = [[1]] }
     { q = [_.$_0$; [1]]; _.$_0$ $\neq$ [_.$_1$] }
     { q = [[1]; _.$_0$]; _.$_0$ $\neq$ [_.$_1$] }
   }

run 3 q (filter p q [ ]) 
$\leadsto$ succ {
     { q = [] }
     { q = [_.$_0$]; _.$_0$ $\neq$ [_.$_1$] }
     { q = [_.$_0$; _.$_1$]; 
            _.$_0$ $\neq$ [_.$_2$]; _.$_1$ $\neq$ [_.$_3$] }
   }
\end{lstlisting}
\end{minipage}

\subsection{Universal quantification}

In the Section~\ref{sec:impl-univ} we presented 
the \lstinline{forall} goal constructor 
which is implemented through the double negation.
We have observed, that although \lstinline{forall g}
does not terminate when the goal \lstinline{g x} 
has an infinite number of answers 
(assuming \lstinline{x} is a fresh variable),
it does terminate in the case when \lstinline{g x} has 
a finite number of answers.
The behavior of \lstinline{forall} in this case is sound
even in the presence of disequality constraints or nested quantifiers. 

The Table~\ref{tab:univ} gives some concrete examples.
The left column contains the tested goals\footnote{
We typeset the goals in terms of first-order logic syntax 
instead of \textsc{OCanren} syntax for brevity and clarity.} 
and the right column gives the obtained results.
For the results we use the same notation 
as in the previous section.

\begin{table}[th]
  \centering
  \def\arraystretch{1.5}
  \begin{tabularx}{\textwidth}{|X|X|}
    \hline

    $\forall x\ldotp x = q$ & 
      \texttt{fail} \\
    \hline

    $\forall x\ldotp \exists y\ldotp x = y$ & 
      \texttt{succ \{[q = \_.$_0$]\}} \\
    \hline

    $\forall x\ldotp \exists y\ldotp x = y \wedge y = q$ &
      \texttt{fail} \\
    \hline

    $\forall x\ldotp q = (1, x)$ & 
      \texttt{fail} \\
    \hline

    $\forall x\ldotp \exists y\ldotp y = (1, x)$ & 
      \texttt{succ \{[q = \_.$_0$]\}} \\
    \hline

    $\forall x\ldotp \exists y\ldotp x = (1, y)$ &
      \texttt{fail} \\
    \hline

    $\forall x\ldotp x \neq q$ & \texttt{fail} \\
    \hline

    $\forall x\ldotp \exists y\ldotp x \neq y$ & 
      \texttt{succ \{[q = \_.$_0$]\}} \\
    \hline

    $\forall x\ldotp \exists y\ldotp x \neq y \wedge y = q$ & 
      \texttt{fail} \\
    \hline

    $\forall x\ldotp q \neq (1, x)$ & 
      \texttt{succ \{[q $\neq$ (1, \_.$_0$)]\}} \\
    \hline

    $(\exists x\ldotp q = (1, x)) \wedge (\forall x\ldotp q \neq (1, x))$ & 
      \texttt{fail} \\
    \hline

    $\forall x\ldotp (x, x) \neq (0, 1)$ & 
      \texttt{succ \{[q = \_.$_0$]\}} \\
    \hline

    $\forall x\ldotp (x, x) \neq (1, 1)$ & 
      \texttt{fail} \\
    \hline

    $\forall x\ldotp (x, x) \neq (q, 1)$ & 
      \texttt{succ \{[q $\neq$ 1]\}} \\
    \hline

    $\exists a~ b\ldotp q = (a, b) \wedge \forall x\ldotp (x, x) \neq (a, b)$ & 
      \texttt{succ \{[q = (\_.$_0$, \_.$_1$); \_.$_0$ $\neq$ \_.$_1$]\}} \\
    \hline

  \end{tabularx}
  \caption{\lstinline{forall} evaluation}
  \label{tab:univ}
\end{table}

\section{Future Work}

There are a few possible directions for future work. First, in this paper we did not address the performance issues. As we represent
the transformations in a very generic form with many levels of indirection, obviously, the transformations, implemented with
our framework, are at disadvantage in comparison with hard coded ones in terms of performance. We assume that the performance of transformations
can be essentially improved by applying some techniques like staging~\cite{Staged} or, perhaps, object-specific optimisations.

Another important direction is supporting more kinds of type declarations, in the first hand, GADTs and non-regular types. Although we have some
implementation ideas for this case, the solution we came up with so far makes the interface of the whole framework too cumbersome to use even for
simple cases.

Finally, the typeinfo structure we generate can be used to mimic the \emph{ad-hoc} polymorphism as it contains the implementation of
type-indexed functions. This, together with some proposed extensions~\cite{ModularImplicits}, can open interesting perspectives.



%\section{Appendix}
%\input{lst} 

\begin{comment}
%% Acknowledgments
\begin{acks}                            %% acks environment is optional
                                        %% contents suppressed with 'anonymous'
  %% Commands \grantsponsor{<sponsorID>}{<name>}{<url>} and
  %% \grantnum[<url>]{<sponsorID>}{<number>} should be used to
  %% acknowledge financial support and will be used by metadata
  %% extraction tools.
  This material is based upon work supported by the
  \grantsponsor{GS100000001}{Russian Foundation for Basic Research}{https://www.rfbr.ru/rffi/eng} under Grant
  No.~\grantnum{GS100000001}{18-01-00380} and by the grant from JetBrains Research. 
  %Any opinions, findings, and
  %conclusions or recommendations expressed in this material are those
  %of the author and do not necessarily reflect the views of the
  %National Science Foundation.
\end{acks}
\end{comment}

\bibliography{references}

\end{document}
