\documentclass[french,11pt]{article}
\usepackage[utf8x]{inputenc}
%\usepackage[T1]{fontenc}
\usepackage[francais]{babel}
\usepackage[resetfonts]{cmap}
\usepackage{fancyvrb}
\begin{VerbatimOut}{ot1tt.cmap}
%!PS-Adobe-3.0 Resource-CMap
%%DocumentNeededResources: ProcSet (CIDInit)
%%IncludeResource: ProcSet (CIDInit)
%%BeginResource: CMap (TeX-OT1-0)
%%Title: (TeX-OT1-0 TeX OT1 0)
%%Version: 1.000
%%EndComments
/CIDInit /ProcSet findresource begin
12 dict begin
begincmap
/CIDSystemInfo
<< /Registry (TeX)
/Ordering (OT1)
/Supplement 0
>> def
/CMapName /TeX-OT1TT-0 def
/CMapType 2 def
1 begincodespacerange
<00> <7F>
endcodespacerange
8 beginbfrange
<00> <01> <0000>
<09> <0A> <0000>
<23> <26> <0000>
<28> <3B> <0000>
<3F> <5B> <0000>
<5D> <5E> <0000>
<61> <7A> <0000>
<7B> <7C> <0000>
endbfrange
40 beginbfchar
<02> <0000>
<03> <0000>
<04> <0000>
<05> <0000>
<06> <0000>
<07> <0000>
<08> <0000>
<0B> <0000>
<0C> <0000>
<0D> <0000>
<0E> <0000>
<0F> <0000>
<10> <0000>
<11> <0000>
<12> <0000>
<13> <0000>
<14> <0000>
<15> <0000>
<16> <0000>
<17> <0000>
<18> <0000>
<19> <0000>
<1A> <0000>
<1B> <0000>
<1C> <0000>
<1D> <0000>
<1E> <0000>
<1F> <0000>
<21> <0000>
<22> <0000>
<27> <0000>
<3C> <0000>
<3D> <0000>
<3E> <0000>
<5C> <0000>
<5F> <0000>
<60> <0000>
<7D> <0000>
<7E> <0000>
<7F> <0000>
endbfchar
endcmap
CMapName currentdict /CMap defineresource pop
end
end
%%EndResource
%%EOF
\end{VerbatimOut}
\usepackage{fullpage,fancyhdr}
\usepackage{listings}
\usepackage{amssymb}
\usepackage{url}
\usepackage{xcolor}
\newtheorem{question}{Question}
\usepackage{color}
\usepackage{pifont}
\definecolor{sh_comment}{rgb}{0.12, 0.38, 0.18 }
\definecolor{sh_keyword}{rgb}{0.37, 0.08, 0.25}
\definecolor{sh_string}{rgb}{0.06, 0.10, 0.98}
\lstset {
 language=caml,
 rulesepcolor=\color{white},
 showspaces=false,showtabs=false,tabsize=2,
 showstringspaces=false,d
 numberstyle=\none,
 basicstyle= \normalsize\ttfamily\color{olive},
 stringstyle=\color{sh_string},
 keywordstyle = \color{sh_keyword}\bfseries,
 commentstyle=\color{sh_comment}\itshape,
 captionpos=b,
 inputencoding=utf8,
 extendedchars=true,
}
\usepackage[left=1cm,right=1cm,top=1cm,bottom=3.5cm]{geometry}
\newenvironment{qu}{\begin{question}\normalfont}{\end{question}
\noindent\hfil\rule{0.5\textwidth}{.4pt}\hfil
}
\lstnewenvironment{caml}{\lstset{language=caml}}{}
\newcommand{\co}[1]{\lstinline[language=caml]{#1}}

\def\subjectheader{
Vous devez rendre votre copie sous la forme d'un fichier OCaml sur la plateforme
\url{http://syntaxerror.lacl.fr}. Attention elle ne fait que vérifier
que le code compile et les signatures des fonctions.
Si le fichier ne compile pas, il ne sera pas corrigé !

Ce travail est individuel et doit être réalisé uniquement avec les
documents du cours: vos notes, les slides et les TPs précédents. Si
vos réponses sont trop similaires à celles d'un autre étudiant ou à du code
trouvé sur internet, il ne sera pas corrigé.
Les fonctions de la librairie standard OCaml (comme
\lstinline|List.fold_left|) sont interdites sauf
celles explicitement mentionnées par le sujet.

Il y a trois genres de questions:
\begin{itemize}
\item Les questions à réponse unique A, B, C ou D le type attendu est le type\\ \co{type qcm = A | B | C | D} il y a toujours exactement une réponse valide ;
\item Les questions à réponse multiple, le type attendu est le type \co{type qlist = int list}, vous devez renvoyer une liste contenant le numéro des réponses valides ;
\item Les questions ouvertes, une expression OCaml est attendue sont type est précisé dans l'énoncé de la question.
\end{itemize}

Vous ne devez pas définir les types \co{qcm, qlist} dans votre rendu.
}

\pagestyle{fancy}
\lhead{Programmation Fonctionnelle}
