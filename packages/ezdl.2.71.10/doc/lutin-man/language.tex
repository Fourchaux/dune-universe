



\section{The \lutin\ language}


%%%%%%%%%%%%%%%%%%%%%%%%%%%%%%%%%%%%%%%%%%%%%%%%%%%%%%%%%%%%%%%%%%%%%%%%%
\subsection{Data types}

There exists 3 pre-defined data types: \key{bool}, \key{int}, and
\key{real}. 

Structured data-types (arrays, enums, structures) and user-defined
abstract types are not yet implemented (coming soon hopefully).

%%%%%%%%%%%%%%%%%%%%%%%%%%%%%%%%%%%%%%%%%%%%%%%%%%%%%%%%%%%%%%%%%%%%%%%%%
\subsection{Nodes}

Nodes are  entry points for \lutin\  programs.  Nodes are  made of an
interface  declaration and  a body.  \lutin\ nodes  can be  reused in
other \lutin\  nodes (as \lustre  nodes); they can also  be top-level
programs. 

\subsubsection{Support variables}

The \key{node}\footnote{used to be called \key{system} in the earlier
  versions  of  \lutin,  to  highlight the  difference  with  \lustre
  nodes.}  interface declares the  set of input and output variables;
they  are  called  the  {\em  support variables}.   During  the  node
execution,  actual   input  values   are  provided  by   the  program
environment.  Output  variable ranges can  specified (or not)  in the
declaration. By default, numeric value ranges from -10000 to 10000.

%% \begin{example}
%% \begin{program}
%% \key{node} foo(x:bool) \key{returns} (t:real; i:\key{int}) = \syn{NodeBodyStatement} 
%% \end{program}
%% \end{example}



\begin{example}
\begin{program}
\key{node} foo(x:bool) \key{returns} (t:real [0.0;100.0]; i:\key{int}) =\\
 \syn{NodeBodyStatement} 
\end{program}
\end{example}




The node body is made of statements that we describe below.


\subsubsection{Local variables}

Node   body   statements   can   be   made  of   a   local   variable
declarations.  Such  variables  are  declared  with  the  \key{exist}
keyword.

\begin{example}
\begin{program}
 \key{exist} y \key{:} real \key{in} \syn{st}\\ 
 \key{exist} z \key{:} int [-100000; 100000] \key{in} \syn{st}
\end{program}
\end{example}

In their scope, local variables  are similar to outputs; we call them
the {\em controllable variables}.


\subsubsection{Memory variables}

Any expression  may refer to the  previous value of  a variable using
the \key{pre}  keyword.  The value of \prg{\key{pre}  x} is inherited
from the past and cannot be modified.  Memories are therefore similar
to inputs; we sometimes call them the {\em uncontrollable variables}.


A memory variable  \prg{\key{pre} x} doesn't need to  be declared, as
long as the variable \prg{x} is declared.


% This is the {\em non-backtracking principle}.

\begin{example}
\begin{program}
\key{if} x \key{then} (t > \key{pre} t) \key{else} (t <= \key{pre} t )
\end{program}
describes any valuation  of the support where \prg{t}  is higher than
its value  at the  previous instant when  \prg{x} is true,  and lower
otherwise.
\end{example}

Note that \key{pre} can only operates over variables. For example,
\key{pre} \prg{(t+10.0)} is forbidden.


%%%%%%%%%%%%%%%%%%%%%%%%%%%%%%%%%%%%%%%%%%%%%%%%%%%%%%%%%%%%%%%%%%%%%%%%%
\subsection{Trace Statements}

%% \todo{La terminologie ``Basic  statements'' est vraiment pas terrible **)
%%   je  trouve (R1)  : Trace  statement  ? Regular  statement ?  Kleene **)
%%   statement ? Pattern statement ?} **)

A node body can be made  of statements that describes how support and
local variables  evolve at each instant.  Such  statements are called
{\em  trace  expressions}  (or  a  {\em  trace  statements}).   Trace
expressions are either atomic, or composed of other trace expressions
using operators inspired by regular expressions, and described below.


\subsubsection{Atomic Trace Statements (Constraints)}


An atomic  trace is simply a  relation (or a  constraint) between the
program variables.  A constraint is a trace of length 1.


\begin{example}
\begin{program}
\prg{x} \key{and} (0.0 < t) \key{and} (t <= 10.0)
\end{program}

 The  constraint  above  describes   any  valuation  of  the  support
 variables where \prg{x} is true and \prg{t} is between \prg{0.0} and
 \prg{10.0}.
\end{example}

Note  that, during  the  execution, if  \prg{x}  is an  input of  the
current node, and if \prg{x} is  false at the current instant, then
the constraint is unsatisfiable.



%%%% l'utilsateur (le  débutant en particulier)  se fout de  la facon
%%%% dont sont compilés les alternatives !!!


%% In this  case, the constraint dynamically raises  a {\em deadlock}
%% exception.

%% The  deadlocks Handling  is central  in \lutin:  most of  the language
%% constructs are defined in terms of deadlock catching, as explained in
%% the sequel.


Atomic statements can be combined to describe longer traces using the
trace operators described below.

An atomic statement is said to be  {\em startable} if it is made of a
satisfiable constraint.  When an atomic statement is not satisfiable,
we say that it {\em deadlocks}.




\subsubsection{Sequence}

If \syn{st1} and \syn{st2} are 2 trace expressions,
 \syn{st1} \key{fby} \syn{st2}
is a trace expression that 
\begin{itemize}
\item  behaves  as \syn{st1},  and  when  it  terminates, behaves  as
  \syn{st2};
\item deadlocks as soon as \syn{t1} or \syn{t2} deadlocks.
\end{itemize}

The sequence \syn{st1} \key{fby}  \syn{st2} is {\em startable} if and
only if \syn{st1} is startable.


\subsubsection{Choice}

If  \syn{st1}, \dots  \syn{stn}  are n  trace  expressions, \OB  \BAR
\syn{st1} \BAR ... \BAR \syn{stn}  \CB\ (the first \BAR\ is optional)
behaves  as  follows: randomly  choose  one  of  the {\em  startable}
statements  from \syn{st1}, \dots,  \syn{stn}.  If  none of  them are
startable, the  whole statement  deadlocks.  \OB \BAR  \syn{st1} \BAR
...   \BAR \syn{stn}  \CB\ is  startable if  and only  if one  of the
\syn{sti} is startable.




\paragraph{Weighted choice.}
 
In a choice, the random selection of a particular startable statement
is uniform.  For instance,  if $k$ of  $n$ statements  are startable,
each of them is chosen with a probability of $1/k$.

This  is the  reason  why the  choice  is not  a binary,  associative
statement:\\  \OB \syn{st1}  \BAR  \OB \syn{st2}  \BAR \syn{st3}  \CB
\CB\\ is  not {\em stockastically} equivalent to\\  \OB \OB \syn{st1}
\BAR \syn{st2} \CB \BAR \syn{st3} \CB

In order  to influence  the probabilities, the  user may  assign {\em
  weights} to the branches of a choice:\\

\OB 
\BAR \syn{w1}\COL\ \syn{st1} $\cdots$
\BAR \syn{wn}\COL\ \syn{stn}
\CB \\


Weights  (\syn{wi})  may be  any  {\em  integer  expression} made  of
constants  and uncontrollable  variables.  In  other terms,  only the
environment  and the past  may influence  the probabilities.   If not
specified, the  weight is equal to  1, and the first  bar is optional
(e.g., '\OB \syn{st1} \BAR \syn{st2}  \CB' is equivalent to '\OB \BAR
1:  \syn{st1} \BAR  1: \syn{st2}  \CB').  Weights  do not  define the
probability to be chosen among the choices, but the probability to be
chosen  among  the {\em  possible}  choices,  (i.e., among  startable
statements).

\paragraph{Priority Choice.}

 \OB \SBAR \syn{st1} \SBAR ... \SBAR \syn{stn} \CB\
behaves as \syn{st1} if \syn{st1} is startable, otherwise
behaves as \syn{st2} if \syn{st2} is startable, etc.
If none of them are startable, the whole statement deadlocks.
The first \SBAR\ is optionnal.


\subsubsection{Loops}

%\paragraph{Infinite loops.}

   \key{loop}   \syn{st}  terminates   {\em  normally}   if  \syn{st}
   deadlocks; otherwise, it  behaves as \syn{st} \key{fby} \key{loop}
   \syn{st}. This can be read as ``repeat the behavior of \syn{st} as
   long as possible''.



\paragraph{Nested loops.}

The execution of nested loops may results on infinite,
instantaneous loops.

\begin{example}
If \syn{c} is a non satisfiable constraint, the statement
\begin{program}
\key{loop} \key{loop} \syn{c}
\end{program}
keeps the control but do nothing.
\end{example}

We consider programs that  generates such instantaneous loops as {\em
  incorrect}  (this  is  quite   similar  to  infinite  recursion  in
classical languages).

Statically checking  if a program  is free of instantaneous  loops is
undecidable. One solution is  to adopt a statical criterium rejecting
all incorrect programs, but also some correct ones.

Typically, a program is certainly  free of instantaneous loop if each
control branch whitin a loop contains a statement that ``takes time''
(i.e., a constraint).

\begin{example}
The (potentially) incorrect program:
\begin{program}
\key{loop} \key{loop} \syn{c}
\end{program}
can be safely replaced by: 
\begin{program}
\key{loop} \OB \syn{c} \key{fby} \key{loop} \syn{c} \CB
\end{program}
\end{example}

The opposite solution is to accept a priori any programs and generate
a  runtime  error  if   an  instantaneous  loops  arises  during  the
execution.  This is the solution adopted in the operational semantics
(Section~\ref{semantics}).





%%%%%%%%%%%%%%%%%%%%%%%%%%%%%%%%%%%%%%%%%%%%%%%%%%%%%%%%%%%%%%%%%%%%%%%%%
\subsection{Exceptions}

\subsubsection{Defining and Raising Exceptions}

Global    exceptions    can   be    declared    outside   the    main
node:

   \key{exception} \syn{ident}

   or locally within a trace statement:

 \key{exception} \syn{ident} \key{in} \syn{st}



An existing exception \syn{ident} can be raised with the statement:
\key{raise} \syn{ident}\\

\subsubsection{Catching exceptions}

An  exception  can  be   caught  with  the  statement:\\  \key{catch}
\syn{ident} \key{in} \syn{st1}  \key{do} \syn{st2}\\ If the exception
is raised in \syn{st1}, the control immediatelly passes to \syn{st2}.
If  the  ``\key{do}''  part  is  omitted,  the  statement  terminates
normally.



\subsubsection{The predefined \key{Deadlock} exception}

When a  trace expression  deadlocks, the \key{Deadlock}  exception is
raised.  In fact, this exception  is internal and cannot be redefined
nor raised by the user.   The only possible use of the \key{Deadlock}
in programs is one try to catch it: 

\begin{example}
\begin{program}
  \key{catch} \key{Deadlock} \key{in} \syn{st1} \key{do} \syn{st2}
\end{program}
\end{example}

Cf Section~\ref{try}.





\subsubsection{Non determinism and deadlocks}
The general rule is that, if a statement can start, then it must 
start; this is the {\em reactivity principle}.
%For the time being, this principle concerns only the \BAR\ construct
%(both \SBAR\ and \key{loop} are deterministic).


%%%%%%%%%%%%%%%%%%%%%%%%%%%%%%%%%%%%%%%%%%%%%%%%%%%%%%%%%%%%%%%%%%%%%%%%%
\subsection{Parallel composition}

\subsubsection{The \ES\ operator}

In order to put in parallel several statements, one can write:\\
\OB \ES \syn{st1} \ES ... \ES \syn{stn} \CB\\
where the first \ES\ can be omitted.

This  statement executes  in  parallel all  the statements  \syn{st1}
... \syn{stn}.  All along the parallel execution each branch produces
its own  constraint; the conjunction of these  local constraints gives
the global constraint.

If one branch terminates  normally, the other branches continue.  The
whole statement terminates when the last branches terminates.

If (at  least) one  branch raises an  exception, the  whole statement
raises the exception.


\subsubsection{Parallelism versus stockastic directives}


It  is impossible  to define  a  parallel composition  which is  fair
according to the stockastic directive.


\begin{example}
Consider the statement:\\ 
\OB~\OB \BAR 1000: \syn{X} \BAR \syn{Y} \CB
~\ES~
\OB \BAR 1000: \syn{A}  \BAR \syn{B} \CB
~\CB\\
where \syn{X}, \syn{A}, \syn{X$\wedge$B}, \syn{A$\wedge$Y} are all
satisfiable, but not \syn{X$\wedge$A}:
\begin{minitemize}
\item the priority can be given to \syn{X$\wedge$B}, which 
does not respect the stockastic directive of the second branch,
\item or to \syn{A$\wedge$Y}, which
does not respect the stockastic directive of the first branch.
\end{minitemize}
\end{example}
In order to solve the problem, the stockastic directives are not
treated in parallel, but in {\em sequence}, from left to right:
\begin{minitemize}
\item the first branch makes its choice, according its local
stockastic directives, 
\item the second one  branch makes its choice, according to what has
been chosen by the first one etc. 
\end{minitemize}
In the example, the priority is then given to
\syn{X$\wedge$B}. 

Finally, the treatment is:
\begin{itemize}
\item parallel w.r.t. constraints (it's a conjunction),
\item but sequential w.r.t.  weights directives (left to right).
\end{itemize}

Note that the concrete syntax (\ES) reflects the fact the operation
is not commutative.


\subsubsection{Parallelism and exceptions}
There is no notion  of ``muti-raising'', even when several statements
are  executed  in parallel.   In  a  parallel composition,  exception
raising  are, like  stockastic directives,  treated in  sequence from
left to right.


%%%%%%%%%%%%%%%%%%%%%%%%%%%%%%%%%%%%%%%%%%%%%%%%%%%%%%%%%%%%%%%%%%%%%%%%%
\subsection{Calling nodes from nodes: a cheap parallel composition mechanism}

The \key{run}/\key{:=}/\key{in} construct is another way of executing
code in parallel.  It is actually the only way of calling Lutin nodes
from other nodes.  Nodes are executed, and the result of this
execution is re-injected in the calling nodes (as if an external call
is made).

Let i1, ...., im be m uncontrollable variables, and 
o1, ..., on be n controllable ones, then

\key{run} (o1, ..., on) \key{:=} a\_node(i1, ...., im) \key{in} \syn{st}

will execute the node a\_node, and compute values for o1, ..., on that
will be substituted in \syn{st}. In \syn{st} o1, ..., om are therefore
uncontrollable.



Follows a working example that makes use of \key{run} statements:


\begin{example}
\begin{program}
\key{node} N() \key{returns} (\key{y}:int) = y = 0 \key{fby} \key{loop} y = pre y+1 \\
\key{node} incr(x:int) returns (y:\key{int}) =  \key{loop} y = x+1  \\
\key{node} use\_run() \key{returns}(x:\key{int}) = \\
~~~  \key{exist} a,b:\key{int in} \\
~~~  \key{run} a := N() \key{in} \\
~~~  \key{run} b := incr(a) \key{in} \\
~~~  \key{run} x := incr(b) \key{in} \\
~~~    \key{loop} true
\end{program}
\end{example}

 If the chosen values during a run make a constraint unsatisfiable, no
 backtracking occurs.  For instance, the following expression is very
 likely to deadlock:

\begin{example}
\begin{program}
  \key{run} x := within(0,100) \key{in} x = 42
\end{program}
\end{example}

 Indeed, if 42 is not chosen during the call to the \key{within} node
 (which definition is obvious), the constraint x=42 will fail.


This form of parallelism is cheaper, because contraints are solved
locally.  But of course it is less powerful: constraints are not
merged, but solved in sequence (no backtracking).  More detail on the
\key{run} semantics is provided in Section~\ref{run-function}.







%%%%%%%%%%%%%%%%%%%%%%%%%%%%%%%%%%%%%%%%%%%%%%%%%%%%%%%%%%%%%%%%%%%%%%%%%
\subsection{Some sugared shortcuts}

In this  section, we present a set  of operators that do  not add any
expressing power  to the language, but  that ought to  make the \lutin
programmer's life more harmonious.

\subsubsection{Propagating a constraint into a scope}

Very often, one wants to  define some constraints that should hold in
all (or  most) of the  program statements. One  way to do this  is to
to put it in parallel with the statement, and to raise an exception as 
soon as one branch finishes.

\begin{example}
\begin{program}
\key{trap} stop \key{in} \{\\ 
   \ES \ \key{loop} \{  \syn{exp} \} \key{fby} \key{raise} stop\\ 
   \ES \  \syn{st} \key{fby} \key{raise} stop\\
 \} 
\end{program}
\end{example}

Because  this  is  very  useful,  we defined  a  dedicated  construct
(\key{assert}) that has exactly the same semantics:

\begin{example}
\begin{program}
  \key{assert} \syn{exp} \key{in} \syn{st}
\end{program}
\end{example}
In other  words, the constraint  \syn{exp} (a Boolean  expression) is
distributed  (propagated) in  all  the constraints  of the  statement
\syn{st}.



\subsubsection{Random loops}
 
Random loops are defined by constraining the number of iterations. 
There are actually two pre-defined kinds of random loops:\\
\begin{itemize}
\item  Interval:   \LOOPI{\syn{min}}{\syn{max}}  \\  the   number  of
  iteration  should  be   comprized  between  the  integer  constants
  \syn{min} and \syn{max} (which  must satisfy $0 \leq \syn{min} \leq
  \syn{max}$).
\item  Average: \LOOPA{\syn{av}}{\syn{sd}}\\  the  average number  of
  iteration should  be \syn{av}, with a  standard deviation \syn{sd}.
  The behavior is defined if and only if $4*\syn{sd} < \syn{av}$.
\end{itemize}

\paragraph{Note.}
Random  loops are  following  the {\em  reactivity principle},  which
means that the  actual number of loops may  significantly differ from
the  ``expected'' one  since  looping may  sometimes  be required  or
impossible,  according  to  the  satisfiability of  constraints.  The
precise semantics is given in~\S\ref{semantics}.



\subsubsection{Define and catch an exception}

The following statement:\\
\key{trap} \syn{x} \key{in} \syn{st1} \key{do} \syn{st2}\\
is a shortcut for:
\key{exception} \syn{x} \key{in}
\key{catch} \syn{x} \key{in} \syn{st1} \key{do} \syn{st2}

\subsubsection{Catching deadlocks}
\label{try}

  \key{catch} \key{Deadlock} \key{in} \syn{st1} \key{do} \syn{st2}


can be written:\\

%A dynamic deadlock can be caught using the statement:\\

\key{try} \syn{st1} \key{do} \syn{st2}\\

If  a deadlock  is  raised  during the  execution  of \syn{st1},  the
control  passes  immediately to  \syn{st2}.  If \syn{st1}  terminates
normally, the  whole statement terminates  and the control  passes to
the sequel.


%%%%%%%%%%%%%%%%%%%%%%%%%%%%%%%%%%%%%%%%%%%%%%%%%%%%%%%%%%%%%%%%%%%%%%%%%
\subsection{Combinators}
\label{combinators}


 {\em  Combinator} were  introduced  in the  language  to allow  code
 reuse. It's a kind of  well-typed macro. One can define a combinator
 with the \key{let} statement:\\

{\tt  \key{let}  \syn{id}  (\syn{Params})  :~\syn{Type}  =  \syn{St1}
  \key{in} \syn{St2}}\\

\begin{itemize}

\item Such a  definition may appear at top-level,  outside a node, in
  which case the "{\tt \key{in} \syn{St2}}" is absent.

\item Classical scoping rules apply for \syn{St1}: free variables are
  first binded  to the  \syn{Params} declaration, otherwise  they are
  binded to the scope in which the whole statement appears.

\item  The   "{\tt  (\syn{Params})}"   part  is  optional;   with  no
  parameters, the declaration simply  means that \syn{id} is an alias
  for the expression \syn{St1} within \syn{St2}.

\item The "{\tt :~\syn{Type}}" is  optional; when absent, the type is
  deduced from the expression \syn{St1}.

\item  The  type  is   either  a  data-type  (\key{bool},  \key{int},
  \key{real}) or  the type  \key{trace}, meaning that  the expression
  "\syn{St1}"  (and   thus  the  identifier   "\syn{id}")  denotes  a
  behaviour.   \key{trace} is  an  abstract type.   It  does not  say
  anything about the support variables of the denoted behaviour.
\end{itemize}

 Here  is  an  example  of  (global)  Boolean  combinator  over  data
 expressions: \

\begin{example}
\begin{program}
\key{let} within\=(x,~min,~max:~\key{real}):~\key{bool}~=  (min <= x) \key{and} (x <= max)
\end{program}
\end{example}
~\\

Here  is an example  of trace  combinator.  It  takes two  traces and
returns a trace that:
\begin{itemize}
\item runs the two trace arguments in parallel,
\item terminates when the second one terminates.
\end{itemize}
\begin{example}
\begin{program}
\key{let} as\_long\_as(X, Y : \key{trace}) : \key{trace} = \\
    \> \key{trap} \= Stop \key{in}\\
    \>      \> X  \EA~\OB Y \key{fby} \key{raise} Stop\CB\\
    \> \CB
\end{program}
\end{example}


\subsubsection{Reference declarations}
\label{ref-declaration}

If one wants to acces to the previous value of a variable, one has to
declare in  the combinator profile that  it is a  reference using the
\key{ref} keyword.


\begin{example}
\begin{program}
\key{let} foo(pt: real \key{ref}, t: real) : \key{bool} = \\
       \> \key{if} \key{pre} pt < pt \key{then} pt < t \key{else} t < pt 
\end{program}
\end{example}

Another  example  of the  use  of  reference  variables is  given  in
Section~\ref{up-and-down}.

\subsubsection{Pre-defined combinators: nor, xor, and \#}
\label{Pre-defined-combinators}

Some useful combinators are predefined to state that, among a list of
Boolean expressions:

\begin{itemize}
\item none is true (\key{not})
\item exactly one is true (\key{xor})
\item exactly zero or one is true (\key{\#})
\end{itemize}


\begin{example}
  \begin{program}
    \key{node} N() \key{returns} (\key{x,y,t}:bool) = \\
        \{ \\
          | \key{nor}(x,y,z) or \key{xor}(x,y,z) \\
          | \#(x,y,t) -- actually equivalent to the previous line\\
        \}
\end{program}
\end{example}



%%%%%%%%%%%%%%%%%%%%%%%%%%%%%%%%%%%%%%%%%%%%%%%%%%%%%%%%%%%%%%%%%%%%%%%%%
\subsection{Calling external code}


In order to use external code from \lutin, we provide a mechanism based
on dynamic (shared) libraries (a.k.a.  {\tt .so} or {\tt .dll} files). 
%
Such dynamic libraries should be built and used according to certain
conventions that we describe in this section.

Moreover, the  type of imported  functions should be declared  in the
\lutin\  file, and  of course,  the declared  types should  match their
definitions in the library. For example,  in order to be able to call
the {\tt sin} and the {\tt cos} extern functions in a lutin file, one
have to declare them like that:

\begin{example}
A \lutin\ program calling 2 extern functions  {\tt sin()} and {\tt cos()}.
\begin{alltt}
\begin{small}
\input{polar.lut}
\end{small}
\end{alltt}
\end{example}

Another example,  as well as  the extern library  compilation process
and    the    \lutin\    interpreter    options,   is    provided    in
Section~\ref{call-extern-c-code}


{\bf BEWARE: } if the types you declare in the Lutin file does not
match their definitions, it might run silently returning wrong values!


