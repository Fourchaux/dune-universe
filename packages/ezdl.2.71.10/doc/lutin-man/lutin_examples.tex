\section{Examples}

\subsection{Up and down}
\label{up-and-down}


The {\tt examples/lutin/up\_and\_down}  directory of the \lutin\ distribution
contains a complete running (via the Makefile) example.

\begin{example}
The {\tt ud.lut} file.
\begin{alltt}
\begin{small}
\input{ud.lut}
\end{small}
\end{alltt}
\end{example}


This program  first defines 3  combinators: {\tt between},  {\tt up},
and  {\tt down}.   {\tt between}  is  used to  constraint a  variable
between a min and a max. Is  is used by the {\tt up} combinator, that
constraint a  controllable variable to be between  its previous value
and its previous value plus  a constant (delta).  The parameter of up
needs to  be declared as  reference, so that  it possible to  use its
previous value (cf~\ref{ref-declaration}).

Then comes the definition of the main node. At the first instant, the
output x is chosen between  the minimum and the maximum. Then, either
it goes  up or it goes  down. If it goes  up (resp down),  it does so
until the maximum (resp minimum)  value is exceeded, and then it goes
down (resp up), and so on forever.



\begin{figure}
\includegraphics[width=15cm]{ud.jpg}
\caption{
This  image has  been obtained  with the  command {\tt  lutin  -l 100
  ud.lut -main main > ud.rif ; gnuplot-rif -jpg ud.rif}
}\label{gnuplot-ud}
\end{figure}

\subsection{The crazy rabbit}

The  {\tt   examples/lutin/crazy\_rabbit}  directory  of   the  \lutin
distribution contains a bigger program.


\begin{example}
The {\tt rabbit.lut} file.
\begin{alltt}
\begin{small}
\input{rabbit.lut}
\end{small}
\end{alltt}
\end{example}


\subsection{Calling external code}
\label{call-extern-c-code}

The  {\tt  examples/lutin/external\_code}  directory of  the  \lutin
distribution contains a complete running (via the Makefile) example
of calling extern code from \lutin.

This  directory  contains a  C  file {\tt  foo.c}  that  defines a  C
function {\tt  rand\_up\_to}. 


\begin{example}
The {\tt foo.c} file.
\begin{alltt}
\begin{small}
\input{foo.c}
\end{small}
\end{alltt}
\end{example}

This C function,  as well as two other function that  are part of the
standard  C  math library  is  are used  in  the  \lutin\ program  {\tt
  call\_external\_c\_code.lut}.

\begin{example}
The {\tt call\_external\_c\_code.lut} file. 
\begin{alltt}
\begin{small}
\input{call_external_c_code.lut}
\end{small}
\end{alltt}
\end{example}



One needs  to generate a  shared lib from  this C file  (foo.so under
unix, or foo.dll  under windows), and to pass  this shared library to
the \lutin\  interpreter via the  {\tt -L foo.so} option.   Since the
\lutin\ file  also uses  the {\tt sin}  and the {\tt  sqrt} functions
that are part of the standard math library, one also need to pass the
{\tt -L libm.so} option.  For instance

\begin{alltt}
  lutin call_external_c_code.lut -m Fun_Call -L libm.so -L obj/foo.so
\end{alltt}

All  this compilation process  is illustrated  in the  {\tt Makefile}
contained in the directory.

