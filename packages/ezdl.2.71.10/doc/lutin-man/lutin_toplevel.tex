\section{Executing  \lutin programs}



%%%%%%%%%%%%%%%%%%%%%%%%%%%%%%%%%%%%%%%%%%%%%%%%%%%%%%%%%%%%%%%%%%%%%%%%%
\subsection{The toplevel interpreter}

Here is the output of {\tt lutin --help}:
\begin{alltt}
\input{lutin}
\end{alltt}



%%%%%%%%%%%%%%%%%%%%%%%%%%%%%%%%%%%%%%%%%%%%%%%%%%%%%%%%%%%%%%%%%%%%%%%%%
\subsection{The C and the \ocaml API}
\label{api}

It is  possible to call  the \lutin interpreter  from C or  from \Ocaml
programs.

\paragraph{Calling the \lutin interpreter from C.}

In order to do that from C, one can use the functions provided in the
\verb+luc4c_stubs.h+   header   file   (that   should   be   in   the
distribution).    A    complete    example    can   be    found    in
\verb+examples/lutin/C/+.  It  contains, a C file, a  \lutin file that
is  called  in  the C  file,  and  a  Makefile that  illustrates  the
different compilers  and options  that should be  used to  generate a
stand-alone executable.

\paragraph{Calling the \lutin interpreter from \Ocaml.}



In order call \lutin from \ocaml, one can use the functions provided in
the  \verb+luc4ocaml.mli+   interface  file  (or   cf  the  ocamlcdoc
\href{http://www-verimag.imag.fr/DIST-TOOLS/SYNCHRONE/lurette/doc/luc4ocaml/Luc4ocaml.html}{generated  html  files}).   A  complete  example can  be  found  in
\verb+examples/lutin/ocaml/+.




%%%%%%%%%%%%%%%%%%%%%%%%%%%%%%%%%%%%%%%%%%%%%%%%%%%%%%%%%%%%%%%%%%%%%%%%%
\subsection{Tools that can be used in conjunction with \lutin}



Some tools developed in the Verimag  lab might be useful in you write
\lutin\  programs. In  this section,  we  list the  tools and  describe
briefly how they can be used in conjunction with \lutin.



\subsubsection{\lustre}

Using the {\tt lutin --2c-4lustre <string>} option and the C API
described in Section\ref{api}, one can call the
\lutin interpreter from  a lustre node.
A complete example can be found in \verb+examples/lutin/lustre/+.

\subsubsection{\luciole}


\luciole is GUI that provides buttons and slide bars to
ease the execution of \lustre programs.


Using the  {\tt lutin --2c-4luciole}  option, one can use  the \lutin
interpreter in conjunction with Luciole.  This can be very handy when
writing \lutin programs.
A complete example can be found in \verb+examples/lutin/luciole/+.

\todo{Faire une copie d'ecran illustrant une simu luciole/lutin.}


\subsubsection{\lurette}

\lurette is  a tool that  automates the testing of  reactive programs,
for example written \lustre. The \lutin program interpreter is embedded
into \lurette;  it is  mainly used to  program the environment  of the
System Under Test  (a.k.a. SUT).  Hence, \lurette is  able to test the
program into a  simulated environment.  The SUT inputs  are the \lutin
outputs, and vice versa.

Therefore, \lutin is used to  close the reactive programs by providing
inputs. From a lutin-centric point  of view, a \lutin program could use
\lurette and \lustre to close the \lutin program.

A complete example can be found in {\tt examples/lutin/xlurette}.

\subsubsection{\rdbg}
Lutin programs can be debugged with \rdbg{} (\url{http://rdbg.forge.imag.fr/}).

\subsubsection{\checkrif}

A  tool that performs  post-mortem oracle  checking using  the
 Lustre expanded code (.ec) interpreter \ecexe.

Here is the output of {\tt check-rif --help}:
\begin{alltt}
\input{checkrif}
\end{alltt}

\subsubsection{\simtochro}

\simtochro is  a program written par Yann  R\'emond that displays
data files that follows the  RIF convention.  For example, to display
à RIF file, one can launch the command : {\tt sim2chrogtk -ecran -in
  data.rif }

\subsubsection{\gnuplotrif}

{\gnuplotrif} is another tool that displays RIF files.  Sometimes
it performs a better job than \simtochro, sometimes not.


Here is the output of {\tt gnuplot-rif --help}:
\begin{alltt}
\input{gnuplotrif}
\end{alltt}


An    example    is    provided   in    Figure~\ref{gnuplot-ud}    of
Section~\ref{up-and-down}.



%%%%%%%%%%%%%%%%%%%%%%%%%%%%%%%%%%%%%%%%%%%%%%%%%%%%%%%%%%%%%%%%%%%%%%%%%

\section{Known bugs and issues}

\subsection{Numeric solver issues}
\label{lucky-numeric-solver}

Since we target the test of real-time software, we put the emphasis
on the efficiency of the solver.


In order to  solve numeric linear constraints, we  use the library of
convex  polyhedron  {\sc   Polka}~\cite{polka}  which  is  reasonably
efficient, at  least for small dimension of  manipulated polyhedra --
the  algorithms complexity  is exponential  in the  dimension  of the
polyhedron. Polyhedron of dimension  bigger that $15$ generally leads
to unreasonable response time.

Note however that independent  variables -- namely, variables that do
not  appear  in the  same  constraint  --  are handled  in  different
polyhedra. This means  that the limitation of 15  dimensions does not
lead to a limitation of  15 variables.  Fortunately, having more than
15 variables that are truly interdependent in the same cycle ought to
be quite rare.


\subsubsection{Solving integer constraints in dimension $n \geq 2$}
When the dimension is greater than  2, for the sake of efficiency, we
do not use classical methods such as linear logic for solving integer
constraints:  we solve those  constraints in  the domain  of rational
numbers  and then we  truncate.  The  problem is  of course  that the
result may not be a solution of the constraints.

In  such  a  case,  we  chose  to  pretend  that  the  constraint  is
unsatisfiable  (after   a  few   more  tries  according   to  various
heuristics), which can be wrong, but which is safe in some sense. The
right solution  there would  be to call  an integer solver,  which is
very expensive, and yet to be done.


\subsubsection{Fairness versus efficiency}

A \lutin program can be  interpreted in two different modes; one that
emphasises the  fairness of the  draw; the other one  that emphasises
the  efficiency.  Indeed,  suppose  we want  to  solve the  following
constraint:

$$ ((b \wedge \alpha_1) \vee (\overline{b} \wedge \alpha_2)) \wedge \alpha_3 \wedge (\alpha_4 \vee \alpha_5) $$

where $b$ is a Boolean, and where $\alpha_i$ are atomic numeric
constraints of the form: $\sum_i a_i x_i < cst$.  The first step is to
find solution from the Boolean point of view. This leads to the four solutions:

$$ b \alpha_1 \overline{\alpha_2} \alpha_3 \overline{\alpha_4} \alpha_5, \ \ \ \
 b \alpha_1 \overline{\alpha_2} \alpha_3 \alpha_4 \overline{\alpha_5}, \ \ \ \
 \overline{b} \overline{\alpha_1} \alpha_2 \alpha_3 \overline{\alpha_4} \alpha_5, \ \ \ \ 
 \overline{b} \overline{\alpha_1} \alpha_2 \alpha_3 \alpha_4 \overline{\alpha_5}$$

\noindent
Now, suppose that:
$$\alpha_1 = 100 > x, \ \
\alpha_2 = 200 > x, \ \
\alpha_3 = x > 0, \ \
\alpha_4 = x > x, \ \
\alpha_5 = x > 1$$

\noindent
 where $x$ is an integer variable that has to be generated by \lutin. We
 use the convex polyhedron  library to solve the numeric constraints,
 which lead respectively to the following sets of solutions:

$$S1 = b \wedge x \in [2;100]; \ \ 
S2 = b \wedge x = 0; \ \ 
S3 = b \wedge \overline{x} \in [2;200]; \ \ 
S4 = b \wedge \overline{x} = 0 $$


In order  to perform a fair draw  among the set of  all solutions, we
need to compute the number of  solutions in each of the set $Si$. But
this  computation  is  very  very  expensive for  polyhedron  of  big
dimension.  Moreover, as  we use Binary Decision Diagrams~\cite{cudd}
to solve the Boolean part,  associating a volume to each numeric part
results in a lost of sharing in BBDs.

Therefore, we have  adopted a pragmatic approach:
\begin{itemize}
\item implement an efficient mode that is fair with respect to the Boolean part only; 
\item implement a fair mode that performs an approximation of the polyhedron volume.
\end{itemize}

The polyhedron volume is approximated by the smallest hypercube
containing the polyhedron.  Note that this leads to no approximation
for polyhedron of dimension 1 (intervals), and reasonable
approximation in dimension 2. But the error made increases
exponentionally in the dimension.
%
Therefore, for polyhedron of big dimension, it is better to use the
efficient mode, and to rely only the probability defined by
transition weights.



Note that when there are only Boolean variables as output or local
variables, the two modes are completely equivalent.

\subsubsection{Fair mode and precision and the computations}

In the fair  mode, we compute an approximation  of polyhedron volume.
But  how to  mix set  of solutions  that involves  both  integers and
floats (which are necessarily computed by distinct polyhedra)?

The solution we have adopted is the following: relate both domain via
the precision of the computations, which is a parameter of the \lutin
programs interpreter. For example, with a precision of 2 digits after
the  dot, we  consider  that the  set  $x \in  [0;3]$ contains  $300$
solutions.



%%%%%%%%%%%%%%%%%%%%%%%%%%%%%%%%%%%%%%%%%%%%%%%%%%%%%%%%%%%%%%%%%%%%%%%%%
\subsection{Last breath}

 
 Before stopping (Vanish exception), the \lutin interpreter generates
 one dummy vector of values that should be ignored.

