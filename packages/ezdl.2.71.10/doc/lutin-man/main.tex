
\documentclass[11pt,a4paper,twoside,titlepage]{article}

\usepackage{color}
\usepackage[pdftex, backref, colorlinks, linkcolor=blue]{hyperref}
\usepackage{longtable}
\usepackage{tabularx}
\usepackage[pdftex]{graphicx}
\usepackage{fancyhdr}
\usepackage{bold-extra}
\usepackage{alltt}

\definecolor{snow}{rgb}{1.000,0.980,0.980}
\definecolor{GhostWhite}{rgb}{0.973,0.973,1.000}
\definecolor{WhiteSmoke}{rgb}{0.961,0.961,0.961}
\definecolor{gainsboro}{rgb}{0.863,0.863,0.863}
\definecolor{FloralWhite}{rgb}{1.000,0.980,0.941}
\definecolor{OldLace}{rgb}{0.992,0.961,0.902}
\definecolor{linen}{rgb}{0.980,0.941,0.902}
\definecolor{AntiqueWhite}{rgb}{0.980,0.922,0.843}
\definecolor{PapayaWhip}{rgb}{1.000,0.937,0.835}
\definecolor{BlanchedAlmond}{rgb}{1.000,0.922,0.804}
\definecolor{bisque}{rgb}{1.000,0.894,0.769}
\definecolor{PeachPuff}{rgb}{1.000,0.855,0.725}
\definecolor{NavajoWhite}{rgb}{1.000,0.871,0.678}
\definecolor{moccasin}{rgb}{1.000,0.894,0.710}
\definecolor{cornsilk}{rgb}{1.000,0.973,0.863}
\definecolor{ivory}{rgb}{1.000,1.000,0.941}
\definecolor{LemonChiffon}{rgb}{1.000,0.980,0.804}
\definecolor{seashell}{rgb}{1.000,0.961,0.933}
\definecolor{honeydew}{rgb}{0.941,1.000,0.941}
\definecolor{MintCream}{rgb}{0.961,1.000,0.980}
\definecolor{azure}{rgb}{0.941,1.000,1.000}
\definecolor{AliceBlue}{rgb}{0.941,0.973,1.000}
\definecolor{lavender}{rgb}{0.902,0.902,0.980}
\definecolor{LavenderBlush}{rgb}{1.000,0.941,0.961}
\definecolor{MistyRose}{rgb}{1.000,0.894,0.882}
\definecolor{white}{rgb}{1.000,1.000,1.000}
\definecolor{black}{rgb}{0.000,0.000,0.000}
\definecolor{DarkSlateGray}{rgb}{0.184,0.310,0.310}
\definecolor{DarkSlateGrey}{rgb}{0.184,0.310,0.310}
\definecolor{DimGray}{rgb}{0.412,0.412,0.412}
\definecolor{DimGrey}{rgb}{0.412,0.412,0.412}
\definecolor{SlateGray}{rgb}{0.439,0.502,0.565}
\definecolor{SlateGrey}{rgb}{0.439,0.502,0.565}
\definecolor{LightSlateGray}{rgb}{0.467,0.533,0.600}
\definecolor{LightSlateGrey}{rgb}{0.467,0.533,0.600}
\definecolor{gray}{rgb}{0.745,0.745,0.745}
\definecolor{grey}{rgb}{0.745,0.745,0.745}
\definecolor{LightGrey}{rgb}{0.827,0.827,0.827}
\definecolor{LightGray}{rgb}{0.827,0.827,0.827}
\definecolor{MidnightBlue}{rgb}{0.098,0.098,0.439}
\definecolor{navy}{rgb}{0.000,0.000,0.502}
\definecolor{NavyBlue}{rgb}{0.000,0.000,0.502}
\definecolor{CornflowerBlue}{rgb}{0.392,0.584,0.929}
\definecolor{DarkSlateBlue}{rgb}{0.282,0.239,0.545}
\definecolor{SlateBlue}{rgb}{0.416,0.353,0.804}
\definecolor{MediumSlateBlue}{rgb}{0.482,0.408,0.933}
\definecolor{LightSlateBlue}{rgb}{0.518,0.439,1.000}
\definecolor{MediumBlue}{rgb}{0.000,0.000,0.804}
\definecolor{RoyalBlue}{rgb}{0.255,0.412,0.882}
\definecolor{blue}{rgb}{0.000,0.000,1.000}
\definecolor{DodgerBlue}{rgb}{0.118,0.565,1.000}
\definecolor{DeepSkyBlue}{rgb}{0.000,0.749,1.000}
\definecolor{SkyBlue}{rgb}{0.529,0.808,0.922}
\definecolor{LightSkyBlue}{rgb}{0.529,0.808,0.980}
\definecolor{SteelBlue}{rgb}{0.275,0.510,0.706}
\definecolor{LightSteelBlue}{rgb}{0.690,0.769,0.871}
\definecolor{LightBlue}{rgb}{0.678,0.847,0.902}
\definecolor{PowderBlue}{rgb}{0.690,0.878,0.902}
\definecolor{PaleTurquoise}{rgb}{0.686,0.933,0.933}
\definecolor{DarkTurquoise}{rgb}{0.000,0.808,0.820}
\definecolor{MediumTurquoise}{rgb}{0.282,0.820,0.800}
\definecolor{turquoise}{rgb}{0.251,0.878,0.816}
\definecolor{cyan}{rgb}{0.000,1.000,1.000}
\definecolor{LightCyan}{rgb}{0.878,1.000,1.000}
\definecolor{CadetBlue}{rgb}{0.373,0.620,0.627}
\definecolor{MediumAquamarine}{rgb}{0.400,0.804,0.667}
\definecolor{aquamarine}{rgb}{0.498,1.000,0.831}
\definecolor{DarkGreen}{rgb}{0.000,0.392,0.000}
\definecolor{DarkOliveGreen}{rgb}{0.333,0.420,0.184}
\definecolor{DarkSeaGreen}{rgb}{0.561,0.737,0.561}
\definecolor{SeaGreen}{rgb}{0.180,0.545,0.341}
\definecolor{MediumSeaGreen}{rgb}{0.235,0.702,0.443}
\definecolor{LightSeaGreen}{rgb}{0.125,0.698,0.667}
\definecolor{PaleGreen}{rgb}{0.596,0.984,0.596}
\definecolor{SpringGreen}{rgb}{0.000,1.000,0.498}
\definecolor{LawnGreen}{rgb}{0.486,0.988,0.000}
\definecolor{green}{rgb}{0.000,1.000,0.000}
\definecolor{chartreuse}{rgb}{0.498,1.000,0.000}
\definecolor{MediumSpringGreen}{rgb}{0.000,0.980,0.604}
\definecolor{GreenYellow}{rgb}{0.678,1.000,0.184}
\definecolor{LimeGreen}{rgb}{0.196,0.804,0.196}
\definecolor{YellowGreen}{rgb}{0.604,0.804,0.196}
\definecolor{ForestGreen}{rgb}{0.133,0.545,0.133}
\definecolor{OliveDrab}{rgb}{0.420,0.557,0.137}
\definecolor{DarkKhaki}{rgb}{0.741,0.718,0.420}
\definecolor{khaki}{rgb}{0.941,0.902,0.549}
\definecolor{PaleGoldenrod}{rgb}{0.933,0.910,0.667}
\definecolor{LightGoldenrodYellow}{rgb}{0.980,0.980,0.824}
\definecolor{LightYellow}{rgb}{1.000,1.000,0.878}
\definecolor{yellow}{rgb}{1.000,1.000,0.000}
\definecolor{gold}{rgb}{1.000,0.843,0.000}
\definecolor{LightGoldenrod}{rgb}{0.933,0.867,0.510}
\definecolor{goldenrod}{rgb}{0.855,0.647,0.125}
\definecolor{DarkGoldenrod}{rgb}{0.722,0.525,0.043}
\definecolor{RosyBrown}{rgb}{0.737,0.561,0.561}
\definecolor{IndianRed}{rgb}{0.804,0.361,0.361}
\definecolor{SaddleBrown}{rgb}{0.545,0.271,0.075}
\definecolor{sienna}{rgb}{0.627,0.322,0.176}
\definecolor{peru}{rgb}{0.804,0.522,0.247}
\definecolor{burlywood}{rgb}{0.871,0.722,0.529}
\definecolor{beige}{rgb}{0.961,0.961,0.863}
\definecolor{wheat}{rgb}{0.961,0.871,0.702}
\definecolor{SandyBrown}{rgb}{0.957,0.643,0.376}
\definecolor{tan}{rgb}{0.824,0.706,0.549}
\definecolor{chocolate}{rgb}{0.824,0.412,0.118}
\definecolor{firebrick}{rgb}{0.698,0.133,0.133}
\definecolor{brown}{rgb}{0.647,0.165,0.165}
\definecolor{DarkSalmon}{rgb}{0.914,0.588,0.478}
\definecolor{salmon}{rgb}{0.980,0.502,0.447}
\definecolor{LightSalmon}{rgb}{1.000,0.627,0.478}
\definecolor{orange}{rgb}{1.000,0.647,0.000}
\definecolor{DarkOrange}{rgb}{1.000,0.549,0.000}
\definecolor{coral}{rgb}{1.000,0.498,0.314}
\definecolor{LightCoral}{rgb}{0.941,0.502,0.502}
\definecolor{tomato}{rgb}{1.000,0.388,0.278}
\definecolor{OrangeRed}{rgb}{1.000,0.271,0.000}
\definecolor{red}{rgb}{1.000,0.000,0.000}
\definecolor{HotPink}{rgb}{1.000,0.412,0.706}
\definecolor{DeepPink}{rgb}{1.000,0.078,0.576}
\definecolor{pink}{rgb}{1.000,0.753,0.796}
\definecolor{LightPink}{rgb}{1.000,0.714,0.757}
\definecolor{PaleVioletRed}{rgb}{0.859,0.439,0.576}
\definecolor{maroon}{rgb}{0.690,0.188,0.376}
\definecolor{MediumVioletRed}{rgb}{0.780,0.082,0.522}
\definecolor{VioletRed}{rgb}{0.816,0.125,0.565}
\definecolor{magenta}{rgb}{1.000,0.000,1.000}
\definecolor{violet}{rgb}{0.933,0.510,0.933}
\definecolor{plum}{rgb}{0.867,0.627,0.867}
\definecolor{orchid}{rgb}{0.855,0.439,0.839}
\definecolor{MediumOrchid}{rgb}{0.729,0.333,0.827}
\definecolor{DarkOrchid}{rgb}{0.600,0.196,0.800}
\definecolor{DarkViolet}{rgb}{0.580,0.000,0.827}
\definecolor{DarkViolet2}{rgb}{0.330,0.000,0.420}
\definecolor{BlueViolet}{rgb}{0.541,0.169,0.886}
\definecolor{purple}{rgb}{0.627,0.125,0.941}
\definecolor{MediumPurple}{rgb}{0.576,0.439,0.859}
\definecolor{thistle}{rgb}{0.847,0.749,0.847}
\definecolor{snow1}{rgb}{1.000,0.980,0.980}
\definecolor{snow2}{rgb}{0.933,0.914,0.914}
\definecolor{snow3}{rgb}{0.804,0.788,0.788}
\definecolor{snow4}{rgb}{0.545,0.537,0.537}
\definecolor{seashell1}{rgb}{1.000,0.961,0.933}
\definecolor{seashell2}{rgb}{0.933,0.898,0.871}
\definecolor{seashell3}{rgb}{0.804,0.773,0.749}
\definecolor{seashell4}{rgb}{0.545,0.525,0.510}
\definecolor{AntiqueWhite1}{rgb}{1.000,0.937,0.859}
\definecolor{AntiqueWhite2}{rgb}{0.933,0.875,0.800}
\definecolor{AntiqueWhite3}{rgb}{0.804,0.753,0.690}
\definecolor{AntiqueWhite4}{rgb}{0.545,0.514,0.471}
\definecolor{bisque1}{rgb}{1.000,0.894,0.769}
\definecolor{bisque2}{rgb}{0.933,0.835,0.718}
\definecolor{bisque3}{rgb}{0.804,0.718,0.620}
\definecolor{bisque4}{rgb}{0.545,0.490,0.420}
\definecolor{PeachPuff1}{rgb}{1.000,0.855,0.725}
\definecolor{PeachPuff2}{rgb}{0.933,0.796,0.678}
\definecolor{PeachPuff3}{rgb}{0.804,0.686,0.584}
\definecolor{PeachPuff4}{rgb}{0.545,0.467,0.396}
\definecolor{NavajoWhite1}{rgb}{1.000,0.871,0.678}
\definecolor{NavajoWhite2}{rgb}{0.933,0.812,0.631}
\definecolor{NavajoWhite3}{rgb}{0.804,0.702,0.545}
\definecolor{NavajoWhite4}{rgb}{0.545,0.475,0.369}
\definecolor{LemonChiffon1}{rgb}{1.000,0.980,0.804}
\definecolor{LemonChiffon2}{rgb}{0.933,0.914,0.749}
\definecolor{LemonChiffon3}{rgb}{0.804,0.788,0.647}
\definecolor{LemonChiffon4}{rgb}{0.545,0.537,0.439}
\definecolor{cornsilk1}{rgb}{1.000,0.973,0.863}
\definecolor{cornsilk2}{rgb}{0.933,0.910,0.804}
\definecolor{cornsilk3}{rgb}{0.804,0.784,0.694}
\definecolor{cornsilk4}{rgb}{0.545,0.533,0.471}
\definecolor{ivory1}{rgb}{1.000,1.000,0.941}
\definecolor{ivory2}{rgb}{0.933,0.933,0.878}
\definecolor{ivory3}{rgb}{0.804,0.804,0.757}
\definecolor{ivory4}{rgb}{0.545,0.545,0.514}
\definecolor{honeydew1}{rgb}{0.941,1.000,0.941}
\definecolor{honeydew2}{rgb}{0.878,0.933,0.878}
\definecolor{honeydew3}{rgb}{0.757,0.804,0.757}
\definecolor{honeydew4}{rgb}{0.514,0.545,0.514}
\definecolor{LavenderBlush1}{rgb}{1.000,0.941,0.961}
\definecolor{LavenderBlush2}{rgb}{0.933,0.878,0.898}
\definecolor{LavenderBlush3}{rgb}{0.804,0.757,0.773}
\definecolor{LavenderBlush4}{rgb}{0.545,0.514,0.525}
\definecolor{MistyRose1}{rgb}{1.000,0.894,0.882}
\definecolor{MistyRose2}{rgb}{0.933,0.835,0.824}
\definecolor{MistyRose3}{rgb}{0.804,0.718,0.710}
\definecolor{MistyRose4}{rgb}{0.545,0.490,0.482}
\definecolor{azure1}{rgb}{0.941,1.000,1.000}
\definecolor{azure2}{rgb}{0.878,0.933,0.933}
\definecolor{azure3}{rgb}{0.757,0.804,0.804}
\definecolor{azure4}{rgb}{0.514,0.545,0.545}
\definecolor{SlateBlue1}{rgb}{0.514,0.435,1.000}
\definecolor{SlateBlue2}{rgb}{0.478,0.404,0.933}
\definecolor{SlateBlue3}{rgb}{0.412,0.349,0.804}
\definecolor{SlateBlue4}{rgb}{0.278,0.235,0.545}
\definecolor{RoyalBlue1}{rgb}{0.282,0.463,1.000}
\definecolor{RoyalBlue2}{rgb}{0.263,0.431,0.933}
\definecolor{RoyalBlue3}{rgb}{0.227,0.373,0.804}
\definecolor{RoyalBlue4}{rgb}{0.153,0.251,0.545}
\definecolor{blue1}{rgb}{0.000,0.000,1.000}
\definecolor{blue2}{rgb}{0.000,0.000,0.933}
\definecolor{blue3}{rgb}{0.000,0.000,0.804}
\definecolor{blue4}{rgb}{0.000,0.000,0.545}
\definecolor{DodgerBlue1}{rgb}{0.118,0.565,1.000}
\definecolor{DodgerBlue2}{rgb}{0.110,0.525,0.933}
\definecolor{DodgerBlue3}{rgb}{0.094,0.455,0.804}
\definecolor{DodgerBlue4}{rgb}{0.063,0.306,0.545}
\definecolor{SteelBlue1}{rgb}{0.388,0.722,1.000}
\definecolor{SteelBlue2}{rgb}{0.361,0.675,0.933}
\definecolor{SteelBlue3}{rgb}{0.310,0.580,0.804}
\definecolor{SteelBlue4}{rgb}{0.212,0.392,0.545}
\definecolor{DeepSkyBlue1}{rgb}{0.000,0.749,1.000}
\definecolor{DeepSkyBlue2}{rgb}{0.000,0.698,0.933}
\definecolor{DeepSkyBlue3}{rgb}{0.000,0.604,0.804}
\definecolor{DeepSkyBlue4}{rgb}{0.000,0.408,0.545}
\definecolor{SkyBlue1}{rgb}{0.529,0.808,1.000}
\definecolor{SkyBlue2}{rgb}{0.494,0.753,0.933}
\definecolor{SkyBlue3}{rgb}{0.424,0.651,0.804}
\definecolor{SkyBlue4}{rgb}{0.290,0.439,0.545}
\definecolor{LightSkyBlue1}{rgb}{0.690,0.886,1.000}
\definecolor{LightSkyBlue2}{rgb}{0.643,0.827,0.933}
\definecolor{LightSkyBlue3}{rgb}{0.553,0.714,0.804}
\definecolor{LightSkyBlue4}{rgb}{0.376,0.482,0.545}
\definecolor{SlateGray1}{rgb}{0.776,0.886,1.000}
\definecolor{SlateGray2}{rgb}{0.725,0.827,0.933}
\definecolor{SlateGray3}{rgb}{0.624,0.714,0.804}
\definecolor{SlateGray4}{rgb}{0.424,0.482,0.545}
\definecolor{LightSteelBlue1}{rgb}{0.792,0.882,1.000}
\definecolor{LightSteelBlue2}{rgb}{0.737,0.824,0.933}
\definecolor{LightSteelBlue3}{rgb}{0.635,0.710,0.804}
\definecolor{LightSteelBlue4}{rgb}{0.431,0.482,0.545}
\definecolor{LightBlue1}{rgb}{0.749,0.937,1.000}
\definecolor{LightBlue2}{rgb}{0.698,0.875,0.933}
\definecolor{LightBlue3}{rgb}{0.604,0.753,0.804}
\definecolor{LightBlue4}{rgb}{0.408,0.514,0.545}
\definecolor{LightCyan1}{rgb}{0.878,1.000,1.000}
\definecolor{LightCyan2}{rgb}{0.820,0.933,0.933}
\definecolor{LightCyan3}{rgb}{0.706,0.804,0.804}
\definecolor{LightCyan4}{rgb}{0.478,0.545,0.545}
\definecolor{PaleTurquoise1}{rgb}{0.733,1.000,1.000}
\definecolor{PaleTurquoise2}{rgb}{0.682,0.933,0.933}
\definecolor{PaleTurquoise3}{rgb}{0.588,0.804,0.804}
\definecolor{PaleTurquoise4}{rgb}{0.400,0.545,0.545}
\definecolor{CadetBlue1}{rgb}{0.596,0.961,1.000}
\definecolor{CadetBlue2}{rgb}{0.557,0.898,0.933}
\definecolor{CadetBlue3}{rgb}{0.478,0.773,0.804}
\definecolor{CadetBlue4}{rgb}{0.325,0.525,0.545}
\definecolor{turquoise1}{rgb}{0.000,0.961,1.000}
\definecolor{turquoise2}{rgb}{0.000,0.898,0.933}
\definecolor{turquoise3}{rgb}{0.000,0.773,0.804}
\definecolor{turquoise4}{rgb}{0.000,0.525,0.545}
\definecolor{cyan1}{rgb}{0.000,1.000,1.000}
\definecolor{cyan2}{rgb}{0.000,0.933,0.933}
\definecolor{cyan3}{rgb}{0.000,0.804,0.804}
\definecolor{cyan4}{rgb}{0.000,0.545,0.545}
\definecolor{DarkSlateGray1}{rgb}{0.592,1.000,1.000}
\definecolor{DarkSlateGray2}{rgb}{0.553,0.933,0.933}
\definecolor{DarkSlateGray3}{rgb}{0.475,0.804,0.804}
\definecolor{DarkSlateGray4}{rgb}{0.322,0.545,0.545}
\definecolor{aquamarine1}{rgb}{0.498,1.000,0.831}
\definecolor{aquamarine2}{rgb}{0.463,0.933,0.776}
\definecolor{aquamarine3}{rgb}{0.400,0.804,0.667}
\definecolor{aquamarine4}{rgb}{0.271,0.545,0.455}
\definecolor{DarkSeaGreen1}{rgb}{0.757,1.000,0.757}
\definecolor{DarkSeaGreen2}{rgb}{0.706,0.933,0.706}
\definecolor{DarkSeaGreen3}{rgb}{0.608,0.804,0.608}
\definecolor{DarkSeaGreen4}{rgb}{0.412,0.545,0.412}
\definecolor{SeaGreen1}{rgb}{0.329,1.000,0.624}
\definecolor{SeaGreen2}{rgb}{0.306,0.933,0.580}
\definecolor{SeaGreen3}{rgb}{0.263,0.804,0.502}
\definecolor{SeaGreen4}{rgb}{0.180,0.545,0.341}
\definecolor{PaleGreen1}{rgb}{0.604,1.000,0.604}
\definecolor{PaleGreen2}{rgb}{0.565,0.933,0.565}
\definecolor{PaleGreen3}{rgb}{0.486,0.804,0.486}
\definecolor{PaleGreen4}{rgb}{0.329,0.545,0.329}
\definecolor{SpringGreen1}{rgb}{0.000,1.000,0.498}
\definecolor{SpringGreen2}{rgb}{0.000,0.933,0.463}
\definecolor{SpringGreen3}{rgb}{0.000,0.804,0.400}
\definecolor{SpringGreen4}{rgb}{0.000,0.545,0.271}
\definecolor{green1}{rgb}{0.000,1.000,0.000}
\definecolor{green2}{rgb}{0.000,0.933,0.000}
\definecolor{green3}{rgb}{0.000,0.804,0.000}
\definecolor{green4}{rgb}{0.000,0.545,0.000}
\definecolor{chartreuse1}{rgb}{0.498,1.000,0.000}
\definecolor{chartreuse2}{rgb}{0.463,0.933,0.000}
\definecolor{chartreuse3}{rgb}{0.400,0.804,0.000}
\definecolor{chartreuse4}{rgb}{0.271,0.545,0.000}
\definecolor{OliveDrab1}{rgb}{0.753,1.000,0.243}
\definecolor{OliveDrab2}{rgb}{0.702,0.933,0.227}
\definecolor{OliveDrab3}{rgb}{0.604,0.804,0.196}
\definecolor{OliveDrab4}{rgb}{0.412,0.545,0.133}
\definecolor{DarkOliveGreen1}{rgb}{0.792,1.000,0.439}
\definecolor{DarkOliveGreen2}{rgb}{0.737,0.933,0.408}
\definecolor{DarkOliveGreen3}{rgb}{0.635,0.804,0.353}
\definecolor{DarkOliveGreen4}{rgb}{0.431,0.545,0.239}
\definecolor{khaki1}{rgb}{1.000,0.965,0.561}
\definecolor{khaki2}{rgb}{0.933,0.902,0.522}
\definecolor{khaki3}{rgb}{0.804,0.776,0.451}
\definecolor{khaki4}{rgb}{0.545,0.525,0.306}
\definecolor{LightGoldenrod1}{rgb}{1.000,0.925,0.545}
\definecolor{LightGoldenrod2}{rgb}{0.933,0.863,0.510}
\definecolor{LightGoldenrod3}{rgb}{0.804,0.745,0.439}
\definecolor{LightGoldenrod4}{rgb}{0.545,0.506,0.298}
\definecolor{LightYellow1}{rgb}{1.000,1.000,0.878}
\definecolor{LightYellow2}{rgb}{0.933,0.933,0.820}
\definecolor{LightYellow3}{rgb}{0.804,0.804,0.706}
\definecolor{LightYellow4}{rgb}{0.545,0.545,0.478}
\definecolor{yellow1}{rgb}{1.000,1.000,0.000}
\definecolor{yellow2}{rgb}{0.933,0.933,0.000}
\definecolor{yellow3}{rgb}{0.804,0.804,0.000}
\definecolor{yellow4}{rgb}{0.545,0.545,0.000}
\definecolor{gold1}{rgb}{1.000,0.843,0.000}
\definecolor{gold2}{rgb}{0.933,0.788,0.000}
\definecolor{gold3}{rgb}{0.804,0.678,0.000}
\definecolor{gold4}{rgb}{0.545,0.459,0.000}
\definecolor{goldenrod1}{rgb}{1.000,0.757,0.145}
\definecolor{goldenrod2}{rgb}{0.933,0.706,0.133}
\definecolor{goldenrod3}{rgb}{0.804,0.608,0.114}
\definecolor{goldenrod4}{rgb}{0.545,0.412,0.078}
\definecolor{DarkGoldenrod1}{rgb}{1.000,0.725,0.059}
\definecolor{DarkGoldenrod2}{rgb}{0.933,0.678,0.055}
\definecolor{DarkGoldenrod3}{rgb}{0.804,0.584,0.047}
\definecolor{DarkGoldenrod4}{rgb}{0.545,0.396,0.031}
\definecolor{RosyBrown1}{rgb}{1.000,0.757,0.757}
\definecolor{RosyBrown2}{rgb}{0.933,0.706,0.706}
\definecolor{RosyBrown3}{rgb}{0.804,0.608,0.608}
\definecolor{RosyBrown4}{rgb}{0.545,0.412,0.412}
\definecolor{IndianRed1}{rgb}{1.000,0.416,0.416}
\definecolor{IndianRed2}{rgb}{0.933,0.388,0.388}
\definecolor{IndianRed3}{rgb}{0.804,0.333,0.333}
\definecolor{IndianRed4}{rgb}{0.545,0.227,0.227}
\definecolor{sienna1}{rgb}{1.000,0.510,0.278}
\definecolor{sienna2}{rgb}{0.933,0.475,0.259}
\definecolor{sienna3}{rgb}{0.804,0.408,0.224}
\definecolor{sienna4}{rgb}{0.545,0.278,0.149}
\definecolor{burlywood1}{rgb}{1.000,0.827,0.608}
\definecolor{burlywood2}{rgb}{0.933,0.773,0.569}
\definecolor{burlywood3}{rgb}{0.804,0.667,0.490}
\definecolor{burlywood4}{rgb}{0.545,0.451,0.333}
\definecolor{wheat1}{rgb}{1.000,0.906,0.729}
\definecolor{wheat2}{rgb}{0.933,0.847,0.682}
\definecolor{wheat3}{rgb}{0.804,0.729,0.588}
\definecolor{wheat4}{rgb}{0.545,0.494,0.400}
\definecolor{tan1}{rgb}{1.000,0.647,0.310}
\definecolor{tan2}{rgb}{0.933,0.604,0.286}
\definecolor{tan3}{rgb}{0.804,0.522,0.247}
\definecolor{tan4}{rgb}{0.545,0.353,0.169}
\definecolor{chocolate1}{rgb}{1.000,0.498,0.141}
\definecolor{chocolate2}{rgb}{0.933,0.463,0.129}
\definecolor{chocolate3}{rgb}{0.804,0.400,0.114}
\definecolor{chocolate4}{rgb}{0.545,0.271,0.075}
\definecolor{firebrick1}{rgb}{1.000,0.188,0.188}
\definecolor{firebrick2}{rgb}{0.933,0.173,0.173}
\definecolor{firebrick3}{rgb}{0.804,0.149,0.149}
\definecolor{firebrick4}{rgb}{0.545,0.102,0.102}
\definecolor{brown1}{rgb}{1.000,0.251,0.251}
\definecolor{brown2}{rgb}{0.933,0.231,0.231}
\definecolor{brown3}{rgb}{0.804,0.200,0.200}
\definecolor{brown4}{rgb}{0.545,0.137,0.137}
\definecolor{salmon1}{rgb}{1.000,0.549,0.412}
\definecolor{salmon2}{rgb}{0.933,0.510,0.384}
\definecolor{salmon3}{rgb}{0.804,0.439,0.329}
\definecolor{salmon4}{rgb}{0.545,0.298,0.224}
\definecolor{LightSalmon1}{rgb}{1.000,0.627,0.478}
\definecolor{LightSalmon2}{rgb}{0.933,0.584,0.447}
\definecolor{LightSalmon3}{rgb}{0.804,0.506,0.384}
\definecolor{LightSalmon4}{rgb}{0.545,0.341,0.259}
\definecolor{orange1}{rgb}{1.000,0.647,0.000}
\definecolor{orange2}{rgb}{0.933,0.604,0.000}
\definecolor{orange3}{rgb}{0.804,0.522,0.000}
\definecolor{orange4}{rgb}{0.545,0.353,0.000}
\definecolor{DarkOrange1}{rgb}{1.000,0.498,0.000}
\definecolor{DarkOrange2}{rgb}{0.933,0.463,0.000}
\definecolor{DarkOrange3}{rgb}{0.804,0.400,0.000}
\definecolor{DarkOrange4}{rgb}{0.545,0.271,0.000}
\definecolor{coral1}{rgb}{1.000,0.447,0.337}
\definecolor{coral2}{rgb}{0.933,0.416,0.314}
\definecolor{coral3}{rgb}{0.804,0.357,0.271}
\definecolor{coral4}{rgb}{0.545,0.243,0.184}
\definecolor{tomato1}{rgb}{1.000,0.388,0.278}
\definecolor{tomato2}{rgb}{0.933,0.361,0.259}
\definecolor{tomato3}{rgb}{0.804,0.310,0.224}
\definecolor{tomato4}{rgb}{0.545,0.212,0.149}
\definecolor{OrangeRed1}{rgb}{1.000,0.271,0.000}
\definecolor{OrangeRed2}{rgb}{0.933,0.251,0.000}
\definecolor{OrangeRed3}{rgb}{0.804,0.216,0.000}
\definecolor{OrangeRed4}{rgb}{0.545,0.145,0.000}
\definecolor{red1}{rgb}{1.000,0.000,0.000}
\definecolor{red2}{rgb}{0.933,0.000,0.000}
\definecolor{red3}{rgb}{0.804,0.000,0.000}
\definecolor{red4}{rgb}{0.545,0.000,0.000}
\definecolor{DeepPink1}{rgb}{1.000,0.078,0.576}
\definecolor{DeepPink2}{rgb}{0.933,0.071,0.537}
\definecolor{DeepPink3}{rgb}{0.804,0.063,0.463}
\definecolor{DeepPink4}{rgb}{0.545,0.039,0.314}
\definecolor{HotPink1}{rgb}{1.000,0.431,0.706}
\definecolor{HotPink2}{rgb}{0.933,0.416,0.655}
\definecolor{HotPink3}{rgb}{0.804,0.376,0.565}
\definecolor{HotPink4}{rgb}{0.545,0.227,0.384}
\definecolor{pink1}{rgb}{1.000,0.710,0.773}
\definecolor{pink2}{rgb}{0.933,0.663,0.722}
\definecolor{pink3}{rgb}{0.804,0.569,0.620}
\definecolor{pink4}{rgb}{0.545,0.388,0.424}
\definecolor{LightPink1}{rgb}{1.000,0.682,0.725}
\definecolor{LightPink2}{rgb}{0.933,0.635,0.678}
\definecolor{LightPink3}{rgb}{0.804,0.549,0.584}
\definecolor{LightPink4}{rgb}{0.545,0.373,0.396}
\definecolor{PaleVioletRed1}{rgb}{1.000,0.510,0.671}
\definecolor{PaleVioletRed2}{rgb}{0.933,0.475,0.624}
\definecolor{PaleVioletRed3}{rgb}{0.804,0.408,0.537}
\definecolor{PaleVioletRed4}{rgb}{0.545,0.278,0.365}
\definecolor{maroon1}{rgb}{1.000,0.204,0.702}
\definecolor{maroon2}{rgb}{0.933,0.188,0.655}
\definecolor{maroon3}{rgb}{0.804,0.161,0.565}
\definecolor{maroon4}{rgb}{0.545,0.110,0.384}
\definecolor{VioletRed1}{rgb}{1.000,0.243,0.588}
\definecolor{VioletRed2}{rgb}{0.933,0.227,0.549}
\definecolor{VioletRed3}{rgb}{0.804,0.196,0.471}
\definecolor{VioletRed4}{rgb}{0.545,0.133,0.322}
\definecolor{magenta1}{rgb}{1.000,0.000,1.000}
\definecolor{magenta2}{rgb}{0.933,0.000,0.933}
\definecolor{magenta3}{rgb}{0.804,0.000,0.804}
\definecolor{magenta4}{rgb}{0.545,0.000,0.545}
\definecolor{orchid1}{rgb}{1.000,0.514,0.980}
\definecolor{orchid2}{rgb}{0.933,0.478,0.914}
\definecolor{orchid3}{rgb}{0.804,0.412,0.788}
\definecolor{orchid4}{rgb}{0.545,0.278,0.537}
\definecolor{plum1}{rgb}{1.000,0.733,1.000}
\definecolor{plum2}{rgb}{0.933,0.682,0.933}
\definecolor{plum3}{rgb}{0.804,0.588,0.804}
\definecolor{plum4}{rgb}{0.545,0.400,0.545}
\definecolor{MediumOrchid1}{rgb}{0.878,0.400,1.000}
\definecolor{MediumOrchid2}{rgb}{0.820,0.373,0.933}
\definecolor{MediumOrchid3}{rgb}{0.706,0.322,0.804}
\definecolor{MediumOrchid4}{rgb}{0.478,0.216,0.545}
\definecolor{DarkOrchid1}{rgb}{0.749,0.243,1.000}
\definecolor{DarkOrchid2}{rgb}{0.698,0.227,0.933}
\definecolor{DarkOrchid3}{rgb}{0.604,0.196,0.804}
\definecolor{DarkOrchid4}{rgb}{0.408,0.133,0.545}
\definecolor{purple1}{rgb}{0.608,0.188,1.000}
\definecolor{purple2}{rgb}{0.569,0.173,0.933}
\definecolor{purple3}{rgb}{0.490,0.149,0.804}
\definecolor{purple4}{rgb}{0.333,0.102,0.545}
\definecolor{MediumPurple1}{rgb}{0.671,0.510,1.000}
\definecolor{MediumPurple2}{rgb}{0.624,0.475,0.933}
\definecolor{MediumPurple3}{rgb}{0.537,0.408,0.804}
\definecolor{MediumPurple4}{rgb}{0.365,0.278,0.545}
\definecolor{thistle1}{rgb}{1.000,0.882,1.000}
\definecolor{thistle2}{rgb}{0.933,0.824,0.933}
\definecolor{thistle3}{rgb}{0.804,0.710,0.804}
\definecolor{thistle4}{rgb}{0.545,0.482,0.545}
\definecolor{gray0}{rgb}{0.000,0.000,0.000}
\definecolor{grey0}{rgb}{0.000,0.000,0.000}
\definecolor{gray1}{rgb}{0.012,0.012,0.012}
\definecolor{grey1}{rgb}{0.012,0.012,0.012}
\definecolor{gray2}{rgb}{0.020,0.020,0.020}
\definecolor{grey2}{rgb}{0.020,0.020,0.020}
\definecolor{gray3}{rgb}{0.031,0.031,0.031}
\definecolor{grey3}{rgb}{0.031,0.031,0.031}
\definecolor{gray4}{rgb}{0.039,0.039,0.039}
\definecolor{grey4}{rgb}{0.039,0.039,0.039}
\definecolor{gray5}{rgb}{0.051,0.051,0.051}
\definecolor{grey5}{rgb}{0.051,0.051,0.051}
\definecolor{gray6}{rgb}{0.059,0.059,0.059}
\definecolor{grey6}{rgb}{0.059,0.059,0.059}
\definecolor{gray7}{rgb}{0.071,0.071,0.071}
\definecolor{grey7}{rgb}{0.071,0.071,0.071}
\definecolor{gray8}{rgb}{0.078,0.078,0.078}
\definecolor{grey8}{rgb}{0.078,0.078,0.078}
\definecolor{gray9}{rgb}{0.090,0.090,0.090}
\definecolor{grey9}{rgb}{0.090,0.090,0.090}
\definecolor{gray10}{rgb}{0.102,0.102,0.102}
\definecolor{grey10}{rgb}{0.102,0.102,0.102}
\definecolor{gray11}{rgb}{0.110,0.110,0.110}
\definecolor{grey11}{rgb}{0.110,0.110,0.110}
\definecolor{gray12}{rgb}{0.122,0.122,0.122}
\definecolor{grey12}{rgb}{0.122,0.122,0.122}
\definecolor{gray13}{rgb}{0.129,0.129,0.129}
\definecolor{grey13}{rgb}{0.129,0.129,0.129}
\definecolor{gray14}{rgb}{0.141,0.141,0.141}
\definecolor{grey14}{rgb}{0.141,0.141,0.141}
\definecolor{gray15}{rgb}{0.149,0.149,0.149}
\definecolor{grey15}{rgb}{0.149,0.149,0.149}
\definecolor{gray16}{rgb}{0.161,0.161,0.161}
\definecolor{grey16}{rgb}{0.161,0.161,0.161}
\definecolor{gray17}{rgb}{0.169,0.169,0.169}
\definecolor{grey17}{rgb}{0.169,0.169,0.169}
\definecolor{gray18}{rgb}{0.180,0.180,0.180}
\definecolor{grey18}{rgb}{0.180,0.180,0.180}
\definecolor{gray19}{rgb}{0.188,0.188,0.188}
\definecolor{grey19}{rgb}{0.188,0.188,0.188}
\definecolor{gray20}{rgb}{0.200,0.200,0.200}
\definecolor{grey20}{rgb}{0.200,0.200,0.200}
\definecolor{gray21}{rgb}{0.212,0.212,0.212}
\definecolor{grey21}{rgb}{0.212,0.212,0.212}
\definecolor{gray22}{rgb}{0.220,0.220,0.220}
\definecolor{grey22}{rgb}{0.220,0.220,0.220}
\definecolor{gray23}{rgb}{0.231,0.231,0.231}
\definecolor{grey23}{rgb}{0.231,0.231,0.231}
\definecolor{gray24}{rgb}{0.239,0.239,0.239}
\definecolor{grey24}{rgb}{0.239,0.239,0.239}
\definecolor{gray25}{rgb}{0.251,0.251,0.251}
\definecolor{grey25}{rgb}{0.251,0.251,0.251}
\definecolor{gray26}{rgb}{0.259,0.259,0.259}
\definecolor{grey26}{rgb}{0.259,0.259,0.259}
\definecolor{gray27}{rgb}{0.271,0.271,0.271}
\definecolor{grey27}{rgb}{0.271,0.271,0.271}
\definecolor{gray28}{rgb}{0.278,0.278,0.278}
\definecolor{grey28}{rgb}{0.278,0.278,0.278}
\definecolor{gray29}{rgb}{0.290,0.290,0.290}
\definecolor{grey29}{rgb}{0.290,0.290,0.290}
\definecolor{gray30}{rgb}{0.302,0.302,0.302}
\definecolor{grey30}{rgb}{0.302,0.302,0.302}
\definecolor{gray31}{rgb}{0.310,0.310,0.310}
\definecolor{grey31}{rgb}{0.310,0.310,0.310}
\definecolor{gray32}{rgb}{0.322,0.322,0.322}
\definecolor{grey32}{rgb}{0.322,0.322,0.322}
\definecolor{gray33}{rgb}{0.329,0.329,0.329}
\definecolor{grey33}{rgb}{0.329,0.329,0.329}
\definecolor{gray34}{rgb}{0.341,0.341,0.341}
\definecolor{grey34}{rgb}{0.341,0.341,0.341}
\definecolor{gray35}{rgb}{0.349,0.349,0.349}
\definecolor{grey35}{rgb}{0.349,0.349,0.349}
\definecolor{gray36}{rgb}{0.361,0.361,0.361}
\definecolor{grey36}{rgb}{0.361,0.361,0.361}
\definecolor{gray37}{rgb}{0.369,0.369,0.369}
\definecolor{grey37}{rgb}{0.369,0.369,0.369}
\definecolor{gray38}{rgb}{0.380,0.380,0.380}
\definecolor{grey38}{rgb}{0.380,0.380,0.380}
\definecolor{gray39}{rgb}{0.388,0.388,0.388}
\definecolor{grey39}{rgb}{0.388,0.388,0.388}
\definecolor{gray40}{rgb}{0.400,0.400,0.400}
\definecolor{grey40}{rgb}{0.400,0.400,0.400}
\definecolor{gray41}{rgb}{0.412,0.412,0.412}
\definecolor{grey41}{rgb}{0.412,0.412,0.412}
\definecolor{gray42}{rgb}{0.420,0.420,0.420}
\definecolor{grey42}{rgb}{0.420,0.420,0.420}
\definecolor{gray43}{rgb}{0.431,0.431,0.431}
\definecolor{grey43}{rgb}{0.431,0.431,0.431}
\definecolor{gray44}{rgb}{0.439,0.439,0.439}
\definecolor{grey44}{rgb}{0.439,0.439,0.439}
\definecolor{gray45}{rgb}{0.451,0.451,0.451}
\definecolor{grey45}{rgb}{0.451,0.451,0.451}
\definecolor{gray46}{rgb}{0.459,0.459,0.459}
\definecolor{grey46}{rgb}{0.459,0.459,0.459}
\definecolor{gray47}{rgb}{0.471,0.471,0.471}
\definecolor{grey47}{rgb}{0.471,0.471,0.471}
\definecolor{gray48}{rgb}{0.478,0.478,0.478}
\definecolor{grey48}{rgb}{0.478,0.478,0.478}
\definecolor{gray49}{rgb}{0.490,0.490,0.490}
\definecolor{grey49}{rgb}{0.490,0.490,0.490}
\definecolor{gray50}{rgb}{0.498,0.498,0.498}
\definecolor{grey50}{rgb}{0.498,0.498,0.498}
\definecolor{gray51}{rgb}{0.510,0.510,0.510}
\definecolor{grey51}{rgb}{0.510,0.510,0.510}
\definecolor{gray52}{rgb}{0.522,0.522,0.522}
\definecolor{grey52}{rgb}{0.522,0.522,0.522}
\definecolor{gray53}{rgb}{0.529,0.529,0.529}
\definecolor{grey53}{rgb}{0.529,0.529,0.529}
\definecolor{gray54}{rgb}{0.541,0.541,0.541}
\definecolor{grey54}{rgb}{0.541,0.541,0.541}
\definecolor{gray55}{rgb}{0.549,0.549,0.549}
\definecolor{grey55}{rgb}{0.549,0.549,0.549}
\definecolor{gray56}{rgb}{0.561,0.561,0.561}
\definecolor{grey56}{rgb}{0.561,0.561,0.561}
\definecolor{gray57}{rgb}{0.569,0.569,0.569}
\definecolor{grey57}{rgb}{0.569,0.569,0.569}
\definecolor{gray58}{rgb}{0.580,0.580,0.580}
\definecolor{grey58}{rgb}{0.580,0.580,0.580}
\definecolor{gray59}{rgb}{0.588,0.588,0.588}
\definecolor{grey59}{rgb}{0.588,0.588,0.588}
\definecolor{gray60}{rgb}{0.600,0.600,0.600}
\definecolor{grey60}{rgb}{0.600,0.600,0.600}
\definecolor{gray61}{rgb}{0.612,0.612,0.612}
\definecolor{grey61}{rgb}{0.612,0.612,0.612}
\definecolor{gray62}{rgb}{0.620,0.620,0.620}
\definecolor{grey62}{rgb}{0.620,0.620,0.620}
\definecolor{gray63}{rgb}{0.631,0.631,0.631}
\definecolor{grey63}{rgb}{0.631,0.631,0.631}
\definecolor{gray64}{rgb}{0.639,0.639,0.639}
\definecolor{grey64}{rgb}{0.639,0.639,0.639}
\definecolor{gray65}{rgb}{0.651,0.651,0.651}
\definecolor{grey65}{rgb}{0.651,0.651,0.651}
\definecolor{gray66}{rgb}{0.659,0.659,0.659}
\definecolor{grey66}{rgb}{0.659,0.659,0.659}
\definecolor{gray67}{rgb}{0.671,0.671,0.671}
\definecolor{grey67}{rgb}{0.671,0.671,0.671}
\definecolor{gray68}{rgb}{0.678,0.678,0.678}
\definecolor{grey68}{rgb}{0.678,0.678,0.678}
\definecolor{gray69}{rgb}{0.690,0.690,0.690}
\definecolor{grey69}{rgb}{0.690,0.690,0.690}
\definecolor{gray70}{rgb}{0.702,0.702,0.702}
\definecolor{grey70}{rgb}{0.702,0.702,0.702}
\definecolor{gray71}{rgb}{0.710,0.710,0.710}
\definecolor{grey71}{rgb}{0.710,0.710,0.710}
\definecolor{gray72}{rgb}{0.722,0.722,0.722}
\definecolor{grey72}{rgb}{0.722,0.722,0.722}
\definecolor{gray73}{rgb}{0.729,0.729,0.729}
\definecolor{grey73}{rgb}{0.729,0.729,0.729}
\definecolor{gray74}{rgb}{0.741,0.741,0.741}
\definecolor{grey74}{rgb}{0.741,0.741,0.741}
\definecolor{gray75}{rgb}{0.749,0.749,0.749}
\definecolor{grey75}{rgb}{0.749,0.749,0.749}
\definecolor{gray76}{rgb}{0.761,0.761,0.761}
\definecolor{grey76}{rgb}{0.761,0.761,0.761}
\definecolor{gray77}{rgb}{0.769,0.769,0.769}
\definecolor{grey77}{rgb}{0.769,0.769,0.769}
\definecolor{gray78}{rgb}{0.780,0.780,0.780}
\definecolor{grey78}{rgb}{0.780,0.780,0.780}
\definecolor{gray79}{rgb}{0.788,0.788,0.788}
\definecolor{grey79}{rgb}{0.788,0.788,0.788}
\definecolor{gray80}{rgb}{0.800,0.800,0.800}
\definecolor{grey80}{rgb}{0.800,0.800,0.800}
\definecolor{gray81}{rgb}{0.812,0.812,0.812}
\definecolor{grey81}{rgb}{0.812,0.812,0.812}
\definecolor{gray82}{rgb}{0.820,0.820,0.820}
\definecolor{grey82}{rgb}{0.820,0.820,0.820}
\definecolor{gray83}{rgb}{0.831,0.831,0.831}
\definecolor{grey83}{rgb}{0.831,0.831,0.831}
\definecolor{gray84}{rgb}{0.839,0.839,0.839}
\definecolor{grey84}{rgb}{0.839,0.839,0.839}
\definecolor{gray85}{rgb}{0.851,0.851,0.851}
\definecolor{grey85}{rgb}{0.851,0.851,0.851}
\definecolor{gray86}{rgb}{0.859,0.859,0.859}
\definecolor{grey86}{rgb}{0.859,0.859,0.859}
\definecolor{gray87}{rgb}{0.871,0.871,0.871}
\definecolor{grey87}{rgb}{0.871,0.871,0.871}
\definecolor{gray88}{rgb}{0.878,0.878,0.878}
\definecolor{grey88}{rgb}{0.878,0.878,0.878}
\definecolor{gray89}{rgb}{0.890,0.890,0.890}
\definecolor{grey89}{rgb}{0.890,0.890,0.890}
\definecolor{gray90}{rgb}{0.898,0.898,0.898}
\definecolor{grey90}{rgb}{0.898,0.898,0.898}
\definecolor{gray91}{rgb}{0.910,0.910,0.910}
\definecolor{grey91}{rgb}{0.910,0.910,0.910}
\definecolor{gray92}{rgb}{0.922,0.922,0.922}
\definecolor{grey92}{rgb}{0.922,0.922,0.922}
\definecolor{gray93}{rgb}{0.929,0.929,0.929}
\definecolor{grey93}{rgb}{0.929,0.929,0.929}
\definecolor{gray94}{rgb}{0.941,0.941,0.941}
\definecolor{grey94}{rgb}{0.941,0.941,0.941}
\definecolor{gray95}{rgb}{0.949,0.949,0.949}
\definecolor{grey95}{rgb}{0.949,0.949,0.949}
\definecolor{gray96}{rgb}{0.961,0.961,0.961}
\definecolor{grey96}{rgb}{0.961,0.961,0.961}
\definecolor{gray97}{rgb}{0.969,0.969,0.969}
\definecolor{grey97}{rgb}{0.969,0.969,0.969}
\definecolor{gray98}{rgb}{0.980,0.980,0.980}
\definecolor{grey98}{rgb}{0.980,0.980,0.980}
\definecolor{gray99}{rgb}{0.988,0.988,0.988}
\definecolor{grey99}{rgb}{0.988,0.988,0.988}
\definecolor{gray100}{rgb}{1.000,1.000,1.000}
\definecolor{grey100}{rgb}{1.000,1.000,1.000}
\definecolor{DarkGrey}{rgb}{0.663,0.663,0.663}
\definecolor{DarkGray}{rgb}{0.663,0.663,0.663}
\definecolor{DarkBlue}{rgb}{0.000,0.000,0.545}
\definecolor{DarkCyan}{rgb}{0.000,0.545,0.545}
\definecolor{DarkMagenta}{rgb}{0.545,0.000,0.545}
\definecolor{DarkRed}{rgb}{0.545,0.000,0.000}
\definecolor{LightGreen}{rgb}{0.565,0.933,0.565}



\setlength{\textwidth}{161mm} %21cm - (2 * 1cm)
\setlength{\textheight}{224mm}
\setlength{\oddsidemargin}{0pt}
\setlength{\evensidemargin}{0pt}
%\setlength{\topmargin}{0pt}
%\setlength{\partopsep}{0pt}
\setlength{\parskip}{0pt}
\setlength{\parindent}{0pt}


%%% mly2bnf customization

%\newcommand{\bnftoken}[1]{{\color{blue2}\bfseries\ttfamily #1}}
\newcommand{\bnftoken}[1]{\mbox{\color{blue2}\bfseries\ttfamily #1}}
\newcommand{\bnfleftident}[1]{{\color{purple4}\itshape #1}}
\newcommand{\bnfrightident}[1]{~\hspace{-1ex}{\color{purple4}\itshape #1}}
\newcommand{\bnfdef}{\mbox{\color{red}$::=$}}
\newcommand{\bnfor}{\mbox{\color{red3}$|$}}
\newcommand{\bnfopt}[1]{\mbox{\color{red3}$\lbrack$} #1 \mbox{\color{red3}$\rbrack$}}
\newcommand{\bnflist}[1]{\mbox{\color{red3}$\lbrace$} #1 \mbox{\color{red3}$\rbrace$}}
\newcommand{\bnfgroup}[1]{\mbox{\color{red3}$($} #1 \mbox{\color{red3}$)$}}
%short cuts ...
\newcommand{\key}[1]{\mbox{\bnftoken{#1}}}
\newcommand{\prg}[1]{\mbox{\bfseries\ttfamily #1}}
\newcommand{\syn}[1]{\mbox{\bnfrightident{#1}}}


%\newcommand {\key}[1]{\mbox{\texttt{\textbf{#1}}}}
%\newcommand {\key}[1]{\mbox{\tt #1}}
%\newcommand {\syn}[1]{\mbox{\it #1}}
\newcommand {\opt}[1]{\mbox{$[$ #1  $]$}}
\newcommand {\oom}[1]{\mbox{$\{$ #1 $\}^+$}}
\newcommand {\zom}[1]{\mbox{$\{$ #1 $\}^*$}}
\newcommand {\comm}[1]{\framebox{#1}}

\newcommand{\COL}{{\tt \symbol{"3A}}}  %{
\newcommand{\OB}{{\tt \symbol{"7B}}}  %{
\newcommand{\CB}{{\tt \symbol{"7D}}}  %}
\newcommand{\BAR}{{\tt \symbol{"7C}}} %|
\newcommand{\SBAR}{{\tt \symbol{"7C}\symbol{"3E}}}  %|>
\newcommand{\OS}{{\tt \symbol{"5B}}}  %[
\newcommand{\CS}{{\tt \symbol{"5D}}}  %]
\newcommand{\TI}{{\tt \symbol{"7E}}}  %~
\newcommand{\EE}{{\tt \symbol{"26}\symbol{"26}}}  %&&
\newcommand{\ES}{{\tt \symbol{"26}\symbol{"3E}}}  %&>
\newcommand{\OP}{{\tt \symbol{"28}}}  %(
\newcommand{\CP}{{\tt \symbol{"29}}}  %)
\newcommand{\LOOPI}[2]{{\tt loop\OS\mbox{#1},\mbox{#2}\CS}}
\newcommand{\LOOPA}[2]{{\tt loop\TI\mbox{#1}:\mbox{#2}}}
\newcommand{\EA}{\key{\symbol{"26}\symbol{"3E}}}  %&>


\newenvironment{program}{\tt\vspace{0.0cm}\par \begin{minipage}{\fboxrule}\begin{tabbing} XX \= XX  \= XX  \= XX \= XX \= XX \= XX \= XX \= XX \= XX \= XX \= XX \= XX \= \+\kill}{\end{tabbing} \end{minipage}\vspace{0.0cm}\rm\noindent\par}
\newenvironment{smallprogram}{\vspace{0.3cm}\par \begin{minipage}{\fboxrule}\small\tt\begin{tabbing} XX \= XX \= XX \= XX \= XX \= XX \= XX \= XX \= XX \= XX \= XX \= \+\kill}{\end{tabbing} \end{minipage}\vspace{0.3cm}\rm\noindent\par}
\newenvironment{tinyprogram}{\tt\footnotesize\hspace{-5mm}\par \begin{minipage}{\fboxrule}\begin{tabbing} X \= X \= X \= X \= X \= X \= X \= X \= X \= X \= X \= \+\kill}{\end{tabbing} \end{minipage}\vspace{0.3cm}\rm\noindent\par}

\newenvironment{example}
{\vspace{1ex}\noindent
\begin{tabular}{|p{15.8cm}|}
{\bfseries $\triangleright$ Example:}
}
{\end{tabular}\vspace{1ex}}

\newenvironment{definition}
{\vspace{1ex}\noindent
\begin{tabular}{||p{15.8cm}||}
{\bfseries $\triangleright$ Definition:}
}
{\end{tabular}\vspace{1ex}}

%\newenvironment{minitemize}{\begin{itemize}
%\renewcommand{\labelitemi}{\mbox{--}}
%\renewcommand{\labelitemii}{\mbox{$\bullet$}}
%\setlength{\itemsep}{0mm}
%\setlength{\topsep}{0mm}
%}{\end{itemize}}

\newcommand{\bug}{{\bf bug}}
\newcommand{\todo}[1]{{\bf todo : } \emph{#1}}


\newenvironment{minitemize}{\begin{itemize}
\setlength{\itemsep}{0mm}
\setlength{\topsep}{0mm}
}{\end{itemize}}

\usepackage{xspace}
\newcommand{\lutin}{{\sc Lutin}\xspace}
\newcommand{\lustre}{{\sc Lustre}\xspace}
\newcommand{\lurette}{{\sc Lurette}\xspace}
\newcommand{\rdbg}{{\sc RDBG}\xspace}
\newcommand{\simtochro}{{\sc Sim2chro}\xspace}
\newcommand{\simtochrogtk}{{\sc Sim2chrogtk}\xspace}
\newcommand{\luciole}{{\sc Luciole}\xspace}
\newcommand{\gnuplotrif}{{\sc Gnuplot-rif}\xspace}
\newcommand{\checkrif}{{\sc check-rif}\xspace}
\newcommand{\ecexe}{{\sc ecexe}\xspace}

\newcommand{\Ocaml}{{\sc Ocaml}\xspace}
\newcommand{\ocaml}{{\sc Ocaml}\xspace}
\newcommand{\kwd}[1]{{\bf\color{red4}#1}}
\newcommand{\kwdd}[1]{{\color{DarkSlateGrey}#1}}
\newcommand{\kwddd}[1]{{\bf\color{green4}#1}}
\newcommand{\commentaire}[1]{{\color{blue3}#1}}

\gdef\menhirversion{20201122}


\ifx\pdftexversion\undefined
  \usepackage[dvips]{graphicx}
\else
  \usepackage[pdftex]{graphicx}
\fi



\fancyhead[LO, RE]{\color{grey} \hyperlink{toc}{Table of contents}}
\fancyhead[RO, LE]{\color{grey} \thepage}
% \fancyhead[RO, LE]{\color{grey} \thepage /\pageref{LastPage}}

\fancyfoot[LO, RE]{\color{grey} }
\fancyfoot[RO, LE]{\color{grey} }

\pagestyle{fancy}

\title{\lutin Reference manual\\ Version \versionname-\version{}}
\author{Pascal Raymond\\Erwan Jahier}
\date{last commit : \versiondate{} (sha:\sha)}

\begin{document}



%\maketitlepage

\maketitle

\tableofcontents
\section*{Abstract.}
A reactive  system indefinitely responds  to its environment.  We are
particularly  interested here in  control and  embedded applications,
where  the  environment is  often  the  physical  world.  During  the
development  of such  systems, non-determinism  is often  useful, for
describing a partially designed system and/or its environment.

\lutin  is a  language designed  to describe  and simulate  such non
deterministic reactive systems.


Executing  a  \lutin  program  consists  in  randomly  generating  a
particular  behaviour consistent  with its  definition.  In  order to
guide  the  generation, the  language  provides  some constructs  for
controlling the random choices.





\newpage


%\input{lutyacc.tex}


\section{An overview of the language}
%\section{The language principles}
\label{lutin-section}


Synchronous  programs~\cite{signal,esterel,lustre}  deterministically
produce  outputs   from  input  values.   To  be   able  to  compile,
synchronous programs need to be fully deterministic.
 However,  sometimes, we  want to  be able  to  describe synchronous
systems in a non deterministic manner.

\begin{itemize}
\item  If  one wants  to  describe  (and  simulate) an  intrinsically
  non-deterministic system.   A typical example  is when one  want to
  describe  the environment  of a  reactive program;  it can  be very
  useful for testing and simulation purposes.
  
\item  Another potential  use of  the animation  of non-deterministic
  code  is when  one  wants to  simulate  partially written  reactive
  programs (some  components are missing).  The idea is then  to take
  advantage  of  program  signatures,  pre/post conditions,  or  code
  chunks to simulate those programs the more realisticly as possible,
  taking  into account  the  available constraints,  and drawing  the
  non-deterministic parts.  This can be  very useful to  simulate and
  test applications at every stage of the development process.
\end{itemize}

We call an \emph{non-deterministic  program} such pieces of code that
produce their  outputs non-deterministically.  \lutin\  is a language
to  describe   such  non-deterministic  programs.    \lutin\  program
describes a  set of data-flow  constraints over Booleans  and numeric
values, that are combined with an explicit control-structure based on
regular expressions.
%
\lutin\  can be seen as a language to program stochastic processes
(Markov chains).





\subsection{Symbolic state/transition systems}
The   basic   qualitative   model   consists  in   a   very   general
state/transition system, characterised by:

\begin{itemize}
\item  a  memory:   a  finite  set  of  variables   with  no  special
  restrictions on  their domains (to simplify, we  will consider here
  just boolean, integer and floating values);
\item  an interface: variables  are declared  as inputs,  outputs, or
  locals;
\item a finite control  structure based on regular expressions, whose
  atoms represent reactions of the machine.
\end{itemize}

A  global state of  the system  is then  a pair  made of  the current
control point  (the {\em control-state}), and a  current valuation of
its  memory (the  {\em  data-state}).  

\subsection{Synchronous relations}
We adopt  the synchronous approach  for the reactions: all  values in
the memory are changing  simultaneously when a reaction is performed.
The  previous   value  of  the  memory  corresponds   to  the  source
data-state,  and  the current  value  to  the  next data-state.   The
program statements denote what are the possible values of the current
memory  depending on  the  current data-state.   This information  is
quite general:  it is a {\em  relation} between the  past and current
values of the variables.   In particular, no syntactic distinction is
made between uncontrollable (inputs and past values) and controllable
(locals and  outputs) variables.  Performing a  reaction will consist
in  finding solutions  to such  a  formula.  This  problem induces  a
restriction: we suppose that, once  reduced according to the past and
input   values,  the   constraints  are   solvable  by   some  actual
procedure\footnote{concretely, we have  developed a constraint solver
  for mixed boolean/linear constraints.}.


\subsection{Weights}
Since we have to deal with uncontrollable variables, defining a sound
notion  of distribution  must  be done  carefully:  depending on  its
variables,  a  formula  may   be  infeasible,  and  thus  its  actual
probability is zero.  In other terms, if we want to use probabilistic
distributions, we would  have to define a reaction as  a map from the
tuple $\langle$source state, past  values, input values$\rangle$ to a
distribution  over  the   pairs  $\langle$controllable  values,  next
state$\rangle$.  Expressing  and exploiting this kind  of model would
be too  complex. We prefer  a pragmatic approach  where probabilities
are introduced in a more symbolic way.

The main  idea is to  keep the distinction between  the probabilistic
information  and the constraint  information.  Since  constraints are
influencing probabilities  (zero or non-zero),  this information does
not express  the probability to be  drawn, but the  probability to be
{\em tried}.   Therefore, we do  not use distributions (i.e.,  set of
positive values the  sum of which is 1) but  {\em weights}.  A weight
is  a  positive  integer:   if  two  possible  reactions  (i.e.,  the
corresponding   constraints  are   both  satisfiable)   are  labelled
respectively with the  weights $w$ and $w'$, then  the probability to
perform the  former is  $w/w'$ times the  probability to  perform the
latter.



\subsection{Static weights versus dynamic weights}
The simplest solution is to  define weights as constants, but in this
case, the expressive power can be too weak. With such static weights,
the    uncontrollable   variables    qualitatively    influence   the
probabilities  (zero or not,  depending on  the constraints)  but not
quantitatively: the idea  is then to define {\em  dynamic weights} as
numerical  functions  of  the  inputs and  the  past-values.   Taking
numerical past-values into account can be particularly useful. A good
example is  when simulating an  {\em alive process} where  the system
has a  known average  life expectancy before  breaking down;  at each
reaction, the probability to  work properly depends {\em numerically}
on an internal counter of the process age.



\subsection{Global concurrency}
Concurrency  (i.e.,  parallel  execution)  is  a  central  issue  for
reactive  systems. The  problem  of merging  sequential and  parallel
constructs  has   been  largely  studied:   classical  solutions  are
hierarchical       automata        ``\`a       la       StateCharts''
\cite{syncchart96,maraninchi92},  or  statement-based languages  like
Esterel~\cite{esterel}.   Our  opinion  is  that  deeply  merging
sequence  and parallelism is  a problem  of high-level  language, and
that  it  is sufficient  to  have  a  notion of  global  parallelism:
intuitively, local  parallelism can always  be made global  by adding
extra  idle states.   As a  consequence, concurrency  is a  top level
notion  in  our model:  a  complete system  is  a  set of  concurrent
program,  each one  producing its  own constraints  on  the resulting
global behaviour.


\subsection{More reading}

Some case studies that use Lutin can be found in \cite{tacas,sies}.
A description of the constraint solving algorithms is done here:
\cite{jahier-cstva06}.
A Lutin tutorial is also available in \href{http://www-verimag.imag.fr/DIST-TOOLS/SYNCHRONE/lurette/doc/lutin-tuto/lutin-tuto-html.html}{html} and \href{http://www-verimag.imag.fr/DIST-TOOLS/SYNCHRONE/lurette/doc/lutin-tuto/lutin-tuto-pdf.pdf}{pdf}.  


\newpage

\begin{figure*}[t]
\[
\begin{array}{cccll}
  &\mathcal{C} & = & \{C_i^{k_i}\} & \mbox{constructors with arities} \\
  &\mathcal{T}_X & = & X \cup \{C_i^{k_i} (t_1, \dots, t_{k_i}) \mid t_j\in\mathcal{T}_X\} & \mbox{terms over the set of variables $X$} \\
  &\mathcal{D} & = & \mathcal{T}_\emptyset & \mbox{ground terms}\\
  &\mathcal{X} & = & \{ x, y, z, \dots \} & \mbox{syntactic variables} \\
  &\mathcal{A} & = & \{ \alpha, \beta, \gamma, \dots \} & \mbox{semantic variables} \\
  &\mathcal{R} & = & \{ R_i^{k_i}\} &\mbox{relational symbols with arities} \\
  &\mathcal{G} & = & \mathcal{T_X}\equiv\mathcal{T_X}   &  \mbox{unification} \\
  &            &   & \mathcal{G}\wedge\mathcal{G}     & \mbox{conjunction} \\
  &            &   & \mathcal{G}\vee\mathcal{G}       &\mbox{disjunction} \\
  &            &   & \mbox{\lstinline|fresh|}\;\mathcal{X}\;.\;\mathcal{G} & \mbox{fresh variable introduction} \\
  &            &   & R_i^{k_i} (t_1,\dots,t_{k_i}),\;t_j\in\mathcal{T_X} & \mbox{relational symbol invocation} \\
  &\mathcal{S} & = & \{R_i^{k_i} = \lambda\;x_1^i\dots x_{k_i}^i\,.\, g_i;\}\; g & \mbox{specification}
\end{array}
\]
\caption{The syntax of the source language}
\label{syntax}
\end{figure*}

\begin{comment}
\begin{figure}[t]
%\centering
\[
\begin{array}{rcl}
  \mathcal{FV}\,(x)&=&\{x\}\\
  \mathcal{FV}\,(C_i^{k_i}\,(t_1,\dots,t_{k_i}))&=&\bigcup\mathcal{FV}\,(t_i)\\
  \mathcal{FV}\,(t_1\equiv t_2)&=&\mathcal{FV}\,(t_1)\cup\mathcal{FV}\,(t_2)\\
  \mathcal{FV}\,(g_1\wedge g_2)&=&\mathcal{FV}\,(g_1)\cup\mathcal{FV}\,(g_2)\\
  \mathcal{FV}\,(g_1\vee g_2)&=&\mathcal{FV}\,(g_1)\cup\mathcal{FV}\,(g_2)\\
  \mathcal{FV}\,(\mbox{\lstinline|fresh|}\;x\;.\;g)&=&\mathcal{FV}\,(g)\setminus\{x\}\\
  \mathcal{FV}\,(R_i^{k_i}\,(t_1,\dots,t_{k_i}))&=&\bigcup\mathcal{FV}\,(t_i)
\end{array}
\]
\caption{Free variables in terms and goals}
\label{free}
\end{figure}
\end{comment}

\section{The Language}
\label{language}
 
In this section, we introduce the syntax of the language we use throughout the paper, describe the informal semantics, and give some examples.

The syntax of the language is shown in Fig.~\ref{syntax}. First, we fix a set of constructors $\mathcal{C}$ with known arities and consider
a set of terms $\mathcal{T}_X$ with constructors as functional symbols and variables from $X$. We parameterize this set with an alphabet of
variables since in the semantic description we will need \emph{two} kinds of variables. The first kind, \emph{syntactic} variables, is denoted
by $\mathcal{X}$. The second kind, \emph{semantic} or \emph{logic} variables, is denoted by $\mathcal{A}$.
We also consider an alphabet of \emph{relational symbols} $\mathcal{R}$ which are used to name relational definitions.
The central syntactic category in the language is \emph{goal}. In our case, there are five types of goals: \emph{unification} of terms,
conjunction and disjunction of goals, fresh variable introduction, and invocation of some relational definition. Thus, unification is used
as a constraint, and multiple constraints can be combined using conjunction, disjunction, and recursion.
The final syntactic category is a \emph{specification} $\mathcal{S}$. It consists of a set
of relational definitions and a top-level goal. A top-level goal represents a search procedure which returns a stream of substitutions for
the free variables of the goal. The definition for a set of free variables for both terms and goals is conventional;
%given in Figure~\ref{free};
as ``\lstinline|fresh|''
is the sole binding construct the definition is rather trivial. The language we defined is first-order, as goals can not be passed as parameters,
returned or constructed at run time.

We now informally describe how relational search works. As we said, a goal represents a search procedure. This procedure takes a \emph{state} as input and returns a
stream of states; a state (among other information) contains a substitution that maps semantic variables into the terms over semantic variables. Then five types of
scenarios are possible (depending on the type of the goal):

\begin{itemize}
\item Unification ``\lstinline|$t_1$ === $t_2$|'' unifies terms $t_1$ and $t_2$ in the context of the substitution in the current state. If terms are unifiable,
  then their MGU is integrated into the substitution, and a one-element stream is returned; otherwise the result is an empty stream.
\item Conjunction ``\lstinline|$g_1$ /\ $g_2$|'' applies $g_1$ to the current state and then applies $g_2$ to each element of the result, concatenating
  the streams.
\item Disjunction ``\lstinline|$g_1$ \/ $g_2$|'' applies both its goals to the current state independently and then concatenates the results.
\item Fresh construct ``\lstinline|fresh $x$ . $g$|'' allocates a new semantic variable $\alpha$, substitutes all free occurrences of $x$ in $g$ with $\alpha$, and
  runs the goal.
\item Invocation ``$\lstinline|$R_i^{k_i}$ ($t_1$,...,$t_{k_i}$)|$'' finds a definition for the relational symbol \mbox{$R_i^{k_i}=\lambda x_1\dots x_{k_i}\,.\,g_i$}, substitutes
  all free occurrences of a formal parameter $x_j$ in $g_i$ with term $t_j$ (for all $j$) and runs the goal in the current state.
\end{itemize}

We stipulate that the top-level goal is preceded by an implicit ``\lstinline|fresh|'' construct, which binds all its free variables, and that the final substitutions
for these variables constitute the result of the goal evaluation.

Conjunction and disjunction form a monadic~\cite{Monads} interface with conjunction playing role of ``\lstinline|bind|'' and disjunction the role of ``\lstinline|mplus|''.
In this description, we swept a lot of important details under the carpet~--- for example, in actual implementations the components of disjunction are not evaluated in
isolation, but both disjuncts are evaluated incrementally with the control passing from one disjunct to another (\emph{interleaving})~\cite{Search};
the evaluation of some goals can be additionally deferred (via so-called ``\emph{inverse-$\eta$-delay}'')~\cite{MicroKanren}; instead of streams
the implementation can be based on ``ferns''~\cite{BottomAvoiding} to defer divergent computations, etc. In the following sections, we present
a complete formal description of relational semantics which resolves these uncertainties in a conventional way.

As an example consider the following specification. For the sake of brevity we
abbreviate immediately nested ``\lstinline|fresh|'' constructs into the one, writing ``\lstinline|fresh $x$ $y$ $\dots$ . $g$|'' instead of
``\lstinline|fresh $x$ . fresh $y$ . $\dots$ $g$|''.

\begin{tabular}{p{5.5cm}p{5.5cm}}
\begin{lstlisting}
append$^o$ = fun x y xy .
 ((x === Nil) /\ (xy === y)) \/
 (fresh h t ty .
   (x  === Cons (h, t))  /\
   (xy === Cons (h, ty)) /\
   (append$^o$ t y ty));

revers$^o$ x x
\end{lstlisting} &
\begin{lstlisting}
revers$^o$ = fun x xr .
 ((x === Nil) /\ (xr === Nil)) \/
 (fresh h t tr .
   (x === Cons (h, t)) /\
   (append$^o$ tr (Cons (h, Nil)) xr) /\
   (revers$^o$ t tr));
\end{lstlisting}
\end{tabular}

Here we defined\footnote{We respect here a conventional tradition for \textsc{miniKanren} programming to superscript all relational names with ``$^o$''.}
two relational symbols~--- ``\lstinline|append$^o$|'' and ``\lstinline|revers$^o$|'',~--- and specified a top-level goal ``\lstinline|revers$^o$ x x|''.
The symbol ``\lstinline|append$^o$|'' defines a relation of concatenation of lists~--- it takes three arguments and performs a case analysis on the first one. If the
first argument is an empty list (``\lstinline|Nil|''), then the second and the third arguments are unified. Otherwise, the first argument is deconstructed into a head ``\lstinline|h|''
and a tail ``\lstinline|t|'', and the tail is concatenated with the second argument using a recursive call to ``\lstinline|append$^o$|'' and additional variable ``\lstinline|ty|'', which
represents the concatenation of ``\lstinline|t|'' and ``\lstinline|y|''. Finally, we unify ``\lstinline|Cons (h, ty)|'' with ``\lstinline|xy|'' to form a final constraint. Similarly,
``\lstinline|revers$^o$|'' defines relational list reversing. The top-level goal represents a search procedure for all lists ``\lstinline|x|'', which are stable under reversing, i.e.
palindromes. Running it results in an infinite stream of substitutions:

\begin{lstlisting}
   $\alpha\;\mapsto\;$ Nil
   $\alpha\;\mapsto\;$ Cons ($\beta_0$, Nil)
   $\alpha\;\mapsto\;$ Cons ($\beta_0$, Cons ($\beta_0$, Nil))
   $\alpha\;\mapsto\;$ Cons ($\beta_0$, Cons ($\beta_1$, Cons ($\beta_0$, Nil)))
   $\dots$
\end{lstlisting}

where ``$\alpha$'' is a \emph{semantic} variable, corresponding to ``\lstinline|x|'', ``$\beta_i$'' are free semantic variables. Therefore, each substitution represents a set of all palindromes of a certain length.


\newpage

\section{Syntax}

%%% Include yacc syntax
\input{lutyacc.tex}


%%%%%%%%%%%%%%%%%%%%%%%%%%%%%%%%%%%%%%%%%%%%%%%%%%%%%%%%%%%%%%%%%%%%%%%%%
\subsection{Lexical conventions}

\begin{itemize}
\item
One-line comments start with \key{--} and  stop at the the end of the
line.
\item
Multi-line comments start with \key{(*} and end at the next following
\key{*)}. Multi-line comments cannot be nested.
\item
\bnfrightident{Ident} stands for identifier, following the C standard
(\bnftoken{[\_a-zA-Z][\_a-zA-Z0-9]*}),
\item
\bnfrightident{Floating} and \bnfrightident{Integer} stands for decimal floating point and integer
notations, following C standard,
\end{itemize}

%%%%%%%%%%%%%%%%%%%%%%%%%%%%%%%%%%%%%%%%%%%%%%%%%%%%%%%%%%%%%%%%%%%%%%%%%
\subsection{Syntax notation (EBNF)}
\begin{itemize}
\item Keywords are displayed like that: \bnftoken{keyword}.
\item Grammatical symbols like that: \bnfleftident{GramaticalSymbol}.
\item Optional parts like that: \bnfopt{something}.
\item List (0 or more)  parts like that: \bnflist{something}.
\item Grouped parts like that: \bnfgroup{something}.
\end{itemize}


%%%%%%%%%%%%%%%%%%%%%%%%%%%%%%%%%%%%%%%%%%%%%%%%%%%%%%%%%%%%%%%%%%%%%%%%%
\subsection{Syntax rules}


 Those syntax rules are automatically extracted from the yacc.



\paragraph{\lutin\ files.}

A  Lutin file  ({\tt .lut})  is  a list  of declarations.   Top-level
declarations can be combinator, exception, or node declarations.\\

\decls


\paragraph{Variable and combinator Parameter Declaration.}
Both are declared  with their type. The \bnftoken{ref}  type flag may
only  appear in  combinator parameter  declaration.  A  default value
(\bnftoken{=}\bnfleftident{Exp})   may   only   appear  in   variable
declaration.   Range  annotations  are  only meaningful  for  numeric
variables.

\varparams

\types

\paragraph{Trace expressions.}

 A Trace expression is a statement of type \bnftoken{trace}.

%, by opposition to data expressions (\syn{Exp}).

\statements

Trace expressions  are surrounded by braces, and  data expressions by
parenthesis.

\paragraph{Data Expressions.}
 A  data   expression  is   a  statement  of   type  \bnftoken{bool},
 \bnftoken{int},  or  \bnftoken{real}.   They  are  almost  classical
 algebraic   expressions,   except   for   the   special   "operator"
 \bnftoken{pre} which requires a variable identifier.


\expressions


Ident references, with or without  arguments, appear in both trace or
data expressions.  Arguments can be any expressions.


\identref


%%%%%%%%%%%%%%%%%%%%%%%%%%%%%%%%%%%%%%%%%%%%%%%%%%%%%%%%%%%%%%%%%%%%%%%%%
\subsection{Priorities}

Priorities are the following, from lower precedence to higher precedence.
In the same level, the default is to group binary operators left-to-right
(note that it may result in type errors).

\begin{itemize}
\item \bnftoken{else},
\item \bnftoken{=>}, logical implication, group {\bf right-to-left},
\item \bnftoken{or},
\item \bnftoken{xor},
\item \bnftoken{and},
\item \bnftoken{=}, \bnftoken{<>},
\item \bnftoken{>}, \bnftoken{<}, \bnftoken{>=}, \bnftoken{<=},
\item \bnftoken{+}, \bnftoken{-} (binary),
\item \bnftoken{*}, \bnftoken{/}, \bnftoken{div}, \bnftoken{mod},
\item \bnftoken{not},
\item \bnftoken{-} (unary).
\end{itemize}



\newpage

\newcommand{\trans}[1]{\mbox{$\stackrel{#1}{\rightarrow}$}}
\newcommand{\vanish}{\raisebox{-0mm}{$\;\rotatebox{90}{$\scriptstyle \hookleftarrow$}$}}
\newcommand{\diewith}[1]{\raisebox{-0mm}{$\;\rotatebox{90}{$\scriptstyle \hookrightarrow$}^{#1}$}}
\newcommand{\catch}[2]{\mbox{$[#2]_{#1}$}}
\newcommand{\catchdo}[3]{\mbox{$[#1\stackrel{#2}{\hookrightarrow}#3]$}}
\newcommand{\try}[1]{\mbox{$[#1]_{\delta}$}}
\newcommand{\trydo}[2]{\catchdo{#1}{\delta}{#2}}
\newcommand{\deadlock}{\diewith{\delta}}
\newcommand{\merge}[2]{\mbox{$#1\;\&\;#2$}}
\newcommand{\WLOOP}[2]{\mbox{$#1^{(\omega_c, \omega_s)}_{#2}$}}

\newcommand{\RUN}{\mbox{\it Run}}
\newcommand{\RUNE}{\mbox{${\cal R}_e$}}

\newcommand{\ACTIONS}{\mbox{$\cal A$}}
\newcommand{\TRACES}{\mbox{$\cal T$}}
\newcommand{\CONSTRAINTS}{\mbox{$\cal C$}}
\newcommand{\TERMINATIONS}{\mbox{$\cal X$}}

\newcommand{\ITE}[3]{(#1)?\;#2\,:\,#3}
\newcommand{\LET}{\mbox{\it let}}
\newcommand{\IN}{\mbox{\it in}}
\newcommand{\WHERE}{\mbox{\it where}}

%%%%%%%%%%%%%%%%%%%%%%%%%%%%%%%%%%%%%%%%%%%%%%%%%%%%%%%%%%%%%%%%%%%%%%%%%
%%%%%%%%%%%%%%%%%%%%%%%%%%%%%%%%%%%%%%%%%%%%%%%%%%%%%%%%%%%%%%%%%%%%%%%%%
\section{Semantics}
\label{semantics}

%%%%%%%%%%%%%%%%%%%%%%%%%%%%%%%%%%%%%%%%%%%%%%%%%%%%%%%%%%%%%%%%%%%%%%%%%
\subsection{Abstract syntax}

The semantics is defined according to the following abstract syntax,
where:
\begin{minitemize}
\item we only consider binary priority choice and parallel composition,
since they are left-associative,
\item we define the empty-behaviour ($\varepsilon$)
and the empty-behaviour filter ($t\setminus\varepsilon$),
which are not available in the concrete syntax, but useful for defining
the semantics,
\item random loops are {\em normalized} by expliciting their
weight functions:
\begin{minitemize}
\item the stop function $\omega_s$ takes the number of iteration already performed
and returns the relative weight of the ``stop'' choice, 
\item the continue function $\omega_c$ takes the number of iteration already performed
and returns the relative weight of the ``continue'' choice. 
\end{minitemize}
These functions  are completly determined  by the ``profile''  of the
loop in the  concrete syntax (interval or average,  together with the
corresponding  static  arguments).  See~\S\ref{loop-profiles}  for  a
precise definition of these weight functions.
\item the actual number of (already) performed iterations is syntacticaly attached
to the loop; this is convenient to define the semantics in terms
of rewriting. In the main statement, this flag is obviously set
to $0$.
\end{minitemize}

\begin{minipage}{70mm}
\begin{tabular}{rl}
empty behaviour: & $\varepsilon$ \\
atomic constraint: & $c$ \\
raise: & $\diewith{x}$ \\
sequence: & $t \;\cdot\; t'$ \\
priority: & $t \;\succ\; t'$ \\
parallel: & $\merge{t}{t'}$
\end{tabular}
\end{minipage}\begin{minipage}{6cm}
\begin{tabular}{rl}
empty filter: & $t\setminus\varepsilon$ \\
catch: & $\catchdo{t}{x}{t'}$ \\
%choice: & $t/w \;|\; t'/w'$ \\
choice: & $|_{i=1}^n\;\;t_i/w_i$\\
random loop: & $\WLOOP{t}{i}$ \\
priority loop: & $t^*$
\end{tabular}
\end{minipage}

\TRACES\ denotes the set of trace expressions, and \CONSTRAINTS\ the
set of constraints.
%% \begin{definition} **)
%% \end{definition} **)

%%%%%%%%%%%%%%%%%%%%%%%%%%%%%%%%%%%%%%%%%%%%%%%%%%%%%%%%%%%%%%%%%%%%%%%%%
\subsection{The run function}
\label{run-function}
The semantics of an execution step is given by a function
taking an environment $e$ and a (trace) expression $t$:
$\RUN(e,t)$.
%The semantics accepts any programs, comprising 
%incorrect ones that can generates instantaneous loops.
%When an instantaneous loop is detected a
%fatal run-time errors is raised.

This function returns an {\em action} which is either:
\begin{minitemize}
\item a transition $\trans{c}{n}$, which means that $t$ produces
a constraint $c$ and rewrite itself in the (next) trace $n$,
\item a termination $\diewith{x}$, where $x$ is a termination flag
which is either $\varepsilon$ (normal termination),
$\delta$ (deadlock) or some user-defined exception.
\end{minitemize}

\ACTIONS\ denotes the set of actions, and \TERMINATIONS\ denotes the
 set of termination flags.
%% \begin{definition} **)
%% \end{definition} **)

The run function is inductively defined using a 
recursive function $\RUNE(t, g, s)$ where the parameters
$g$ and $s$ are continuation functions returning actions.
\begin{minitemize}
\item $g: \CONSTRAINTS \times \TRACES \mapsto \ACTIONS$
is the {\em goto} function, defining how
a local transition should be treated according to the calling context.
\item $s: \TERMINATIONS \mapsto \ACTIONS$
is the {\em stop} function, defining how
a local termination should be treated according to the calling context.
\end{minitemize}

At the top level, \RUNE\ is simply called with the trivial continuations:
\begin{eqnarray}
\RUN(e,t) & = & \RUNE(t,\;\;
\lambda(c,v).\trans{c}{v},\;\;
\lambda x. \diewith{x}
)
\end{eqnarray}


%%%%%%%%%%%%%%%%%%%%%%%%%%%%%%%%%%%%%%%%%%%%%%%%%%%%%%%%%%%%%%%%%%%%%%%%%
\subsection{The recursive run function}

\subsubsection{Basic traces.}

The empty behavior raises the termination flag in the current context:
\[
\RUNE(\varepsilon, g, s) = s(\varepsilon)
\]
A raise statement terminates with the corresponding flag:
\[
\RUNE(\diewith{x}, g, s) = s(x)
\]
A constraint generates a goto or raises a deadlock, depending on its 
satisfiability in the environment:
\[
\RUNE(c, g, s) = \ITE{e\models c}{g(c, \varepsilon)}{s(\delta)}
\]

\subsubsection{Sequence.}

\[
\RUNE(t \cdot t', g, s) = \RUNE(t, g', s')
\]
where:
\begin{eqnarray*}
%g' & = & \lambda c n. g(c, n \cdot t')\\
%s' & = & \lambda x. \ITE{x = \varepsilon}{\RUNE(t', g, s)}{s(x)}
g'(c,n) & = & g(c, n \cdot t')\\
s'(x) & = & \ITE{x = \varepsilon}{\RUNE(t', g, s)}{s(x)}
\end{eqnarray*}

\subsubsection{Priority choice.}

There is no continuation here: just a deterministic
choice between the two branches.
The second branch is taken if and only if the first branch deadlocks
in the current context.

\[
\RUNE(t \succ t', g, s) =
	\ITE{r \neq \diewith{\delta}}{r}{\RUNE(t', g, s)}
	\;\;\;\WHERE\;\;\;
   r = \RUNE(t, g, s))
\]

\subsubsection{Empty filter.}
This internal construct is introduced to ease the definition
of the loops. Intuitively, it forbids the core $t$ to terminate 
immediately.
\[
\RUNE(t\setminus\varepsilon, g, s) = \RUNE(t, g, s')
\]
where:
\begin{eqnarray*}
s'(x) & = & \ITE{x = \varepsilon}{\diewith{\deadlock}}{s(x)}
\end{eqnarray*}


\subsubsection{Priority loop.}

The semantics is defined according to the equivalence:
\begin{eqnarray*}
t^* & \Leftrightarrow & (t\setminus\varepsilon)\cdot t^* \succ \varepsilon
\end{eqnarray*}

\subsubsection{Catch.}
Note that $z$ is a catchable exception (either $\delta$ or a user-defined
exception).
\[
\RUNE(\catchdo{t}{z}{t'}, g, s) = \RUNE(t, g', s')
\]
where:
\begin{eqnarray*}
g'(c,n) & = & g(c, \catchdo{n}{z}{t'})\\
s'(x) & = & \ITE{x = z}{\RUNE(t', g, s)}{s(x)}
\end{eqnarray*}

\subsubsection{Parallel composition.}
\[
\RUNE(\merge{t}{t'}, g, s) = \RUNE(t, g', s')
\]
where:
\begin{eqnarray*}
s'(x) & = & \ITE{x = \varepsilon}{\RUNE(t', g, s)}{s(x)}\\
g'(c,n) & = & \RUNE(t', g'', s'') \;\;\;\mbox{with:}\\
s''(x) & = & \ITE{x = \varepsilon}{g(c,n)}{s(x)}\\
g''(c',n') & = & g(c \wedge c', \merge{n}{n'})
\end{eqnarray*}

\subsubsection{Weighted choice.}

\newcommand{\SORTE}{\mbox{$\mbox{\it Sort}_e$}}

The evaluation of the weights, and the (random) total ordering
of the branches according those actual weights are both 
performed by the environment:\\
$\SORTE(t_1/w_1, \cdots, t_n/w_n)$ returns:
\begin{itemize}
\item a priority expression $t_{\sigma(1)} \succ \cdots \succ t_{\sigma(k)}$ 
reflecting the priorities that have been (randomly) assigned 
to the branches; note that $k$ may be less than $n$, since some branches may have
an actual weight of $0$.  
\item the deadlock expression $\diewith{\delta}$ if all weights
are evaluated to $0$.
\end{itemize}
See~\S\ref{random-sort} for the precise definition of \SORTE.

\[
\RUNE(|_{i=1}^n\;\;t_i/w_i, g, s) = 
\RUNE(\SORTE(t_1/w_1, \cdots, t_n/w_n), g, s)
\]

\subsubsection{Random loop.}

The semantics is defined according to the equivalence:
\begin{eqnarray*}
\WLOOP{t}{i}
& \Leftrightarrow &
(t\setminus\varepsilon)\cdot \WLOOP{t}{i+1} / \omega_c(i)
\;\;\;|\;\;\;
\varepsilon / \omega_s(i)
\end{eqnarray*}

%%%%%%%%%%%%%%%%%%%%%%%%%%%%%%%%%%%%%%%%%%%%%%%%%%%%%%%%%%%%%%%%%%%%%%%%%
\subsection{The execution environment}

\subsubsection{Random sort of weighted choices}
\label{random-sort}

\subsection{Predefined loop profiles}
\label{loop-profiles}


\newpage
\section{Executing  \lutin programs}



%%%%%%%%%%%%%%%%%%%%%%%%%%%%%%%%%%%%%%%%%%%%%%%%%%%%%%%%%%%%%%%%%%%%%%%%%
\subsection{The toplevel interpreter}

Here is the output of {\tt lutin --help}:
\begin{alltt}
\input{lutin}
\end{alltt}



%%%%%%%%%%%%%%%%%%%%%%%%%%%%%%%%%%%%%%%%%%%%%%%%%%%%%%%%%%%%%%%%%%%%%%%%%
\subsection{The C and the \ocaml API}
\label{api}

It is  possible to call  the \lutin interpreter  from C or  from \Ocaml
programs.

\paragraph{Calling the \lutin interpreter from C.}

In order to do that from C, one can use the functions provided in the
\verb+luc4c_stubs.h+   header   file   (that   should   be   in   the
distribution).    A    complete    example    can   be    found    in
\verb+examples/lutin/C/+.  It  contains, a C file, a  \lutin file that
is  called  in  the C  file,  and  a  Makefile that  illustrates  the
different compilers  and options  that should be  used to  generate a
stand-alone executable.

\paragraph{Calling the \lutin interpreter from \Ocaml.}



In order call \lutin from \ocaml, one can use the functions provided in
the  \verb+luc4ocaml.mli+   interface  file  (or   cf  the  ocamlcdoc
\href{http://www-verimag.imag.fr/DIST-TOOLS/SYNCHRONE/lurette/doc/luc4ocaml/Luc4ocaml.html}{generated  html  files}).   A  complete  example can  be  found  in
\verb+examples/lutin/ocaml/+.




%%%%%%%%%%%%%%%%%%%%%%%%%%%%%%%%%%%%%%%%%%%%%%%%%%%%%%%%%%%%%%%%%%%%%%%%%
\subsection{Tools that can be used in conjunction with \lutin}



Some tools developed in the Verimag  lab might be useful in you write
\lutin\  programs. In  this section,  we  list the  tools and  describe
briefly how they can be used in conjunction with \lutin.



\subsubsection{\lustre}

Using the {\tt lutin --2c-4lustre <string>} option and the C API
described in Section\ref{api}, one can call the
\lutin interpreter from  a lustre node.
A complete example can be found in \verb+examples/lutin/lustre/+.

\subsubsection{\luciole}


\luciole is GUI that provides buttons and slide bars to
ease the execution of \lustre programs.


Using the  {\tt lutin --2c-4luciole}  option, one can use  the \lutin
interpreter in conjunction with Luciole.  This can be very handy when
writing \lutin programs.
A complete example can be found in \verb+examples/lutin/luciole/+.

\todo{Faire une copie d'ecran illustrant une simu luciole/lutin.}


\subsubsection{\lurette}

\lurette is  a tool that  automates the testing of  reactive programs,
for example written \lustre. The \lutin program interpreter is embedded
into \lurette;  it is  mainly used to  program the environment  of the
System Under Test  (a.k.a. SUT).  Hence, \lurette is  able to test the
program into a  simulated environment.  The SUT inputs  are the \lutin
outputs, and vice versa.

Therefore, \lutin is used to  close the reactive programs by providing
inputs. From a lutin-centric point  of view, a \lutin program could use
\lurette and \lustre to close the \lutin program.

A complete example can be found in {\tt examples/lutin/xlurette}.

\subsubsection{\rdbg}
Lutin programs can be debugged with \rdbg{} (\url{http://rdbg.forge.imag.fr/}).

\subsubsection{\checkrif}

A  tool that performs  post-mortem oracle  checking using  the
 Lustre expanded code (.ec) interpreter \ecexe.

Here is the output of {\tt check-rif --help}:
\begin{alltt}
\input{checkrif}
\end{alltt}

\subsubsection{\simtochro}

\simtochro is  a program written par Yann  R\'emond that displays
data files that follows the  RIF convention.  For example, to display
à RIF file, one can launch the command : {\tt sim2chrogtk -ecran -in
  data.rif }

\subsubsection{\gnuplotrif}

{\gnuplotrif} is another tool that displays RIF files.  Sometimes
it performs a better job than \simtochro, sometimes not.


Here is the output of {\tt gnuplot-rif --help}:
\begin{alltt}
\input{gnuplotrif}
\end{alltt}


An    example    is    provided   in    Figure~\ref{gnuplot-ud}    of
Section~\ref{up-and-down}.



%%%%%%%%%%%%%%%%%%%%%%%%%%%%%%%%%%%%%%%%%%%%%%%%%%%%%%%%%%%%%%%%%%%%%%%%%

\section{Known bugs and issues}

\subsection{Numeric solver issues}
\label{lucky-numeric-solver}

Since we target the test of real-time software, we put the emphasis
on the efficiency of the solver.


In order to  solve numeric linear constraints, we  use the library of
convex  polyhedron  {\sc   Polka}~\cite{polka}  which  is  reasonably
efficient, at  least for small dimension of  manipulated polyhedra --
the  algorithms complexity  is exponential  in the  dimension  of the
polyhedron. Polyhedron of dimension  bigger that $15$ generally leads
to unreasonable response time.

Note however that independent  variables -- namely, variables that do
not  appear  in the  same  constraint  --  are handled  in  different
polyhedra. This means  that the limitation of 15  dimensions does not
lead to a limitation of  15 variables.  Fortunately, having more than
15 variables that are truly interdependent in the same cycle ought to
be quite rare.


\subsubsection{Solving integer constraints in dimension $n \geq 2$}
When the dimension is greater than  2, for the sake of efficiency, we
do not use classical methods such as linear logic for solving integer
constraints:  we solve those  constraints in  the domain  of rational
numbers  and then we  truncate.  The  problem is  of course  that the
result may not be a solution of the constraints.

In  such  a  case,  we  chose  to  pretend  that  the  constraint  is
unsatisfiable  (after   a  few   more  tries  according   to  various
heuristics), which can be wrong, but which is safe in some sense. The
right solution  there would  be to call  an integer solver,  which is
very expensive, and yet to be done.


\subsubsection{Fairness versus efficiency}

A \lutin program can be  interpreted in two different modes; one that
emphasises the  fairness of the  draw; the other one  that emphasises
the  efficiency.  Indeed,  suppose  we want  to  solve the  following
constraint:

$$ ((b \wedge \alpha_1) \vee (\overline{b} \wedge \alpha_2)) \wedge \alpha_3 \wedge (\alpha_4 \vee \alpha_5) $$

where $b$ is a Boolean, and where $\alpha_i$ are atomic numeric
constraints of the form: $\sum_i a_i x_i < cst$.  The first step is to
find solution from the Boolean point of view. This leads to the four solutions:

$$ b \alpha_1 \overline{\alpha_2} \alpha_3 \overline{\alpha_4} \alpha_5, \ \ \ \
 b \alpha_1 \overline{\alpha_2} \alpha_3 \alpha_4 \overline{\alpha_5}, \ \ \ \
 \overline{b} \overline{\alpha_1} \alpha_2 \alpha_3 \overline{\alpha_4} \alpha_5, \ \ \ \ 
 \overline{b} \overline{\alpha_1} \alpha_2 \alpha_3 \alpha_4 \overline{\alpha_5}$$

\noindent
Now, suppose that:
$$\alpha_1 = 100 > x, \ \
\alpha_2 = 200 > x, \ \
\alpha_3 = x > 0, \ \
\alpha_4 = x > x, \ \
\alpha_5 = x > 1$$

\noindent
 where $x$ is an integer variable that has to be generated by \lutin. We
 use the convex polyhedron  library to solve the numeric constraints,
 which lead respectively to the following sets of solutions:

$$S1 = b \wedge x \in [2;100]; \ \ 
S2 = b \wedge x = 0; \ \ 
S3 = b \wedge \overline{x} \in [2;200]; \ \ 
S4 = b \wedge \overline{x} = 0 $$


In order  to perform a fair draw  among the set of  all solutions, we
need to compute the number of  solutions in each of the set $Si$. But
this  computation  is  very  very  expensive for  polyhedron  of  big
dimension.  Moreover, as  we use Binary Decision Diagrams~\cite{cudd}
to solve the Boolean part,  associating a volume to each numeric part
results in a lost of sharing in BBDs.

Therefore, we have  adopted a pragmatic approach:
\begin{itemize}
\item implement an efficient mode that is fair with respect to the Boolean part only; 
\item implement a fair mode that performs an approximation of the polyhedron volume.
\end{itemize}

The polyhedron volume is approximated by the smallest hypercube
containing the polyhedron.  Note that this leads to no approximation
for polyhedron of dimension 1 (intervals), and reasonable
approximation in dimension 2. But the error made increases
exponentionally in the dimension.
%
Therefore, for polyhedron of big dimension, it is better to use the
efficient mode, and to rely only the probability defined by
transition weights.



Note that when there are only Boolean variables as output or local
variables, the two modes are completely equivalent.

\subsubsection{Fair mode and precision and the computations}

In the fair  mode, we compute an approximation  of polyhedron volume.
But  how to  mix set  of solutions  that involves  both  integers and
floats (which are necessarily computed by distinct polyhedra)?

The solution we have adopted is the following: relate both domain via
the precision of the computations, which is a parameter of the \lutin
programs interpreter. For example, with a precision of 2 digits after
the  dot, we  consider  that the  set  $x \in  [0;3]$ contains  $300$
solutions.



%%%%%%%%%%%%%%%%%%%%%%%%%%%%%%%%%%%%%%%%%%%%%%%%%%%%%%%%%%%%%%%%%%%%%%%%%
\subsection{Last breath}

 
 Before stopping (Vanish exception), the \lutin interpreter generates
 one dummy vector of values that should be ignored.



\newpage
\section{Examples}

\subsection{Up and down}
\label{up-and-down}


The {\tt examples/lutin/up\_and\_down}  directory of the \lutin\ distribution
contains a complete running (via the Makefile) example.

\begin{example}
The {\tt ud.lut} file.
\begin{alltt}
\begin{small}
\input{ud.lut}
\end{small}
\end{alltt}
\end{example}


This program  first defines 3  combinators: {\tt between},  {\tt up},
and  {\tt down}.   {\tt between}  is  used to  constraint a  variable
between a min and a max. Is  is used by the {\tt up} combinator, that
constraint a  controllable variable to be between  its previous value
and its previous value plus  a constant (delta).  The parameter of up
needs to  be declared as  reference, so that  it possible to  use its
previous value (cf~\ref{ref-declaration}).

Then comes the definition of the main node. At the first instant, the
output x is chosen between  the minimum and the maximum. Then, either
it goes  up or it goes  down. If it goes  up (resp down),  it does so
until the maximum (resp minimum)  value is exceeded, and then it goes
down (resp up), and so on forever.



\begin{figure}
\includegraphics[width=15cm]{ud.jpg}
\caption{
This  image has  been obtained  with the  command {\tt  lutin  -l 100
  ud.lut -main main > ud.rif ; gnuplot-rif -jpg ud.rif}
}\label{gnuplot-ud}
\end{figure}

\subsection{The crazy rabbit}

The  {\tt   examples/lutin/crazy\_rabbit}  directory  of   the  \lutin
distribution contains a bigger program.


\begin{example}
The {\tt rabbit.lut} file.
\begin{alltt}
\begin{small}
\input{rabbit.lut}
\end{small}
\end{alltt}
\end{example}


\subsection{Calling external code}
\label{call-extern-c-code}

The  {\tt  examples/lutin/external\_code}  directory of  the  \lutin
distribution contains a complete running (via the Makefile) example
of calling extern code from \lutin.

This  directory  contains a  C  file {\tt  foo.c}  that  defines a  C
function {\tt  rand\_up\_to}. 


\begin{example}
The {\tt foo.c} file.
\begin{alltt}
\begin{small}
\input{foo.c}
\end{small}
\end{alltt}
\end{example}

This C function,  as well as two other function that  are part of the
standard  C  math library  is  are used  in  the  \lutin\ program  {\tt
  call\_external\_c\_code.lut}.

\begin{example}
The {\tt call\_external\_c\_code.lut} file. 
\begin{alltt}
\begin{small}
\input{call_external_c_code.lut}
\end{small}
\end{alltt}
\end{example}



One needs  to generate a  shared lib from  this C file  (foo.so under
unix, or foo.dll  under windows), and to pass  this shared library to
the \lutin\  interpreter via the  {\tt -L foo.so} option.   Since the
\lutin\ file  also uses  the {\tt sin}  and the {\tt  sqrt} functions
that are part of the standard math library, one also need to pass the
{\tt -L libm.so} option.  For instance

\begin{alltt}
  lutin call_external_c_code.lut -m Fun_Call -L libm.so -L obj/foo.so
\end{alltt}

All  this compilation process  is illustrated  in the  {\tt Makefile}
contained in the directory.



\bibliographystyle{abbrv}

\bibliography{bib}

\end{document}
