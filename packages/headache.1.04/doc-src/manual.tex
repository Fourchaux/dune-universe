%%%%%%%%%%%%%%%%%%%%%%%%%%%%%%%%%%%%%%%%%%%%%%%%%%%%%%%%%%%%%%%%%%%%%%%%%%
%                                                                        %
%                               Headache                                 %
%                                                                        %
%          Vincent Simonet, Projet Cristal, INRIA Rocquencourt           %
%                                                                        %
%  Copyright 2002                                                        %
%  Institut National de Recherche en Informatique et en Automatique.     %
%  All rights reserved.  This file is distributed under the terms of     %
%  the GNU Library General Public License.                               %
%                                                                        %
%  Vincent.Simonet@inria.fr           http://cristal.inria.fr/~simonet/  %
%                                                                        %
%%%%%%%%%%%%%%%%%%%%%%%%%%%%%%%%%%%%%%%%%%%%%%%%%%%%%%%%%%%%%%%%%%%%%%%%%%

%BEGIN LATEX
\documentclass{article}
\setlength{\textheight}{240mm}
\setlength{\textwidth}{160mm}
\setlength{\topmargin}{-22mm}
\setlength{\headheight}{5mm}
\setlength{\headsep}{12mm}
\setlength{\footskip}{16mm}
\setlength{\oddsidemargin}{2.2mm}
\setlength{\evensidemargin}{2.2mm}
%END LATEX
%HEVEA \documentclass{article}

\usepackage{alltt}
\usepackage{moreverb}
\usepackage{url}

%BEGIN LATEX
\newcommand{\mytt}[1]{\texttt{#1}}
%END LATEX
%HEVEA \newcommand{\mytt}[1]{`\texttt{#1}`}

\newcommand{\headache}{\mytt{headache}}



\title{\headache{}}
\author{Vincent Simonet}
\date{November, 2002}

\begin{document}

\maketitle

%HEVEA This manual is also available in \ahref{manual.txt}{plain text},
%HEVEA \ahref{manual.ps.gz}{PostScript} and \ahref{manual.pdf}{PDF}.

% *************************************************************************
\section{Overview}

It is a common usage to put at the beginning of source code files a
short header giving, for instance, some copyright informations.
\headache{} is a simple and lightweight tool for managing easily these
headers.  Among its functionalities, one may mention:
\begin{itemize}
\item Headers must generally be generated as \emph{comments} in source
  code files.  \headache{} deals with different files types and generates
  for each of them headers in an appropriate format.
\item Headers automatically detects existing headers and removes them.
  Thus, you can use it to update headers in a set of files.
\end{itemize}

\headache{} is distributed under the terms of the \emph{GNU Library General
  Public License}.  See file \mytt{LICENSE} of the distribution for
more information.


% *************************************************************************
\section{Compilation and installation}

Building \headache{} requires \emph{Objective Caml} (version 3.06 or up,
available at \url{http://caml.inria.fr/}) and \emph{GNU Make}.
In addition, from version \texttt{1.03-utf8}, the build requires the Unicode library \emph{Camomile} and \emph{OCamlbuild}.

\paragraph{Instructions}

\begin{alltt}
  make && sudo make INSTALLDIR=/usr/local/bin install
\end{alltt}
Build the executable and install it into the specified directory.

\headache{} is available through \emph{OPAM} (available at
\url{http://opam.ocaml.org/}), the \emph{OCaml Package Manager}.
This is the preferred installation method.
Be sure to install opam v1.2 or higher.
Then the following sequence of commands should install the package:
\begin{alltt}
  opam init
  opam install depext
  opam depext headache
  opam install headache
\end{alltt}

% *************************************************************************
\section{Usage}

Let us illustrate the use of this tool with a small example.  Assume
you have a small project mixing C and Caml code consisting in three
files \mytt{foo.c}, \mytt{bar.ml} and \mytt{bar.mli}, and you want to
equip them with some header.  First of all, write a \emph{header
  file}, i.e.\ a plain text file including the information headers
must mention.  An example of such a file is given in
figure~\ref{figure:header}.  In the following, we assume this file is
named \mytt{myheader} and is in the same directory as source files.

Then, in order to generate headers, just run the command:
\begin{alltt}
  headache -h myheader foo.c bar.ml bar.mli
\end{alltt}
Each file is equipped with an header including the text given in the
header file \mytt{myheader}, surrounded by some extra characters
depending on its format making it a comment (e.g.\ \mytt{(*} and
\mytt{*)} in \mytt{.ml} files).  If you update informations in the
header file \mytt{myheader}, you simply need to re-run the above
command to update headers in source code files: existing ones are
automatically removed.

Similarly, running:
\begin{alltt}
  headache -r foo.c bar.ml bar.mli
\end{alltt}
removes any existing in files \mytt{foo.c}, \mytt{bar.ml} and
\mytt{bar.mli}.  Files which do not have a header are kept unchanged.

The current headers of files can be extracted:
\begin{alltt}
  headache -e foo.c bar.ml bar.mli
\end{alltt}
prints on the standard output the current headers of the files \mytt{foo.c},
\mytt{bar.ml} and \mytt{bar.mli}. All files are kept unchanged.

\begin{figure}
%BEGIN LATEX
\begin{center}
%END LATEX
\begin{boxedverbatim}

                             Headache
               Automatic generation of files headers

        Vincent Simonet, Projet Cristal, INRIA Rocquencourt

Copyright 2002
Institut National de Recherche en Informatique et en Automatique.
All rights reserved.  This file is distributed under the terms of
the GNU Library General Public License.

Vincent.Simonet@inria.fr           http://cristal.inria.fr/~simonet/
\end{boxedverbatim}
%BEGIN LATEX
\end{center}
%END LATEX
  \caption{An example of header file}
  \label{figure:header}
\end{figure}



% *************************************************************************
\section{Configuration file}

File types and format of header may be specified by a
\emph{configuration file}.  By default, the default builtin
configuration file given in figure~\ref{figure:config} is used.  You
can also use your own configuration file thanks to the \mytt{-c}
option:
\begin{alltt}
  headache -c myconfig -h myheader foo.c bar.ml bar.mli
\end{alltt}

In order to write your own configuration, you can follow the example
given in figure~\ref{figure:config}.  A configuration file consists in
a list of \emph{entries} separated by the character \mytt{|}.  Each
of them is made of two parts separated by an \mytt{->}:
\begin{itemize}
\item The first one is a \emph{regular expression}.  Regular
  expression are enclosed within double quotes and have the same
  syntax as in Gnu Emacs.  \headache{} determines file types according to
  file basenames; thus, each file is dealt with using the first line
  its name matches.
\item The second one describes the format of headers for files of this
  type.  It consists of the name of a \emph{model} (e.g.\ 
  \mytt{frame}), possibly followed by a list of arguments.  Arguments
  are named: \mytt{open:"(*"} means that the value of the argument
  \mytt{open} is \mytt{(*}.
\end{itemize}
\headache{} currently supports three \emph{models}:
\begin{itemize}
\item \mytt{frame}.  With this model, headers are generated in a
  frame.  This model requires three arguments: \mytt{open} and
  \mytt{close} (the opening and closing sequences for comments) and
  \mytt{line} (the character used to make the horizontal lines of the
  frame).  Two optional arguments may be used \mytt{margin} (a string
  printed between the left and right side of the frame and the border,
  by default two spaces) and \mytt{width} (the width of the inside of
  the frame, default is 68).
\item \mytt{lines}.  Headers are typeset between two lines.  Three
  arguments must be provided: \mytt{open} and \mytt{close} (the
  opening and closing sequences for comments), \mytt{line} (the
  character used to make the horizontal lines).  Three optional
  arguments are allowed: \mytt{begin} (a string typeset at the
  beginning of each line, by default two spaces), \mytt{last} (a
  string typeset at the beginning of the last line) and \mytt{width}
  (the width of the lines, default is 70).
\item \mytt{no}.  This model generates no header and has no argument.
\end{itemize}


It is possible to change the default builtin configuration file at
compile time.  For this, just edit the file \mytt{config\_builtin.txt}
present in the source distribution before building the software.


\begin{figure}
%BEGIN LATEX
\begin{center}
%END LATEX
\begin{boxedverbatim}

# Objective Caml source
  ".*\\.ml[il]?" -> frame open:"(*" line:"*" close:"*)"
| ".*\\.fml[i]?" -> frame open:"(*" line:"*" close:"*)"
| ".*\\.mly"     -> frame open:"/*" line:"*" close:"*/"
# C source
| ".*\\.[chy]"    -> frame open:"/*" line:"*" close:"*/"
# Latex
| ".*\\.tex"     -> frame open:"%"  line:"%" close:"%"
# Misc
| ".*Makefile.*" -> frame open:"#"  line:"#" close:"#"
| ".*README.*"   -> frame open:"*"  line:"*" close:"*"
| ".*LICENSE.*"  -> frame open:"*"  line:"*" close:"*"
\end{boxedverbatim}
%BEGIN LATEX
\end{center}
%END LATEX
  \caption{The default builtin configuration file}
  \label{figure:config}
\end{figure}

It is also possible to add entries into your own configuration file that
specify when the first line of the processed file has to be skipped.
As previously, these \emph{entries} are separated by the character \mytt{|}
and each of them is made of two parts separated by an \mytt{->}:
\begin{itemize}
\item Again, the first part is a \emph{regular expression} used by \headache{}
  to determine the file type. But here, it is according to
  its full filename (including the pathname).
\item The second part specifies when the first line must be skipped.
  It consists of the keyword \mytt{skip} followed by the named
  argument \mytt{match:} a \emph{regular expression}.
  When the first line of such a file type matches,
  \headache{} does not modify that line and considers that the header must
  start at the second line.
\end{itemize}

\begin{figure}
%BEGIN LATEX
\begin{center}
%END LATEX
\begin{boxedverbatim}
# Script file
| ".*\\.sh" -> frame open:"#"  line:"#" close:"#"
| ".*\\.sh" -> skip match:"#!.*"
\end{boxedverbatim}
%BEGIN LATEX
\end{center}
%END LATEX
  \caption{Example of a configuration file for skipping the shebang line of shell scripts}
  \label{figure:example}
\end{figure}

\end{document}
%%% Local Variables:
%%% mode: latex
%%% TeX-master: t
%%% End:
