\documentclass[a4paper,12pt]{article}

\usepackage{graphicx,tikz}
\usepackage{times,mathptmx}

% \newcommand{\meshpoint}[3]{%
%   \begin{pgfscope}
%     \pgfsetlinewidth{0pt}
%     \pgfpathcircle{\pgfpointxy{#2}{#3}}{1pt}
%     \pgfusepath{fill}
%   \end{pgfscope}}

\begin{document}

\setlength{\unitlength}{75mm}
\noindent
\begin{tikzpicture}
  \pgfsetlinewidth{1pt}
  \pgfsetxvec{\pgfpoint{0.5\linewidth}{0mm}}
  \pgfsetyvec{\pgfpoint{0mm}{0.5\linewidth}}
  \input{testmesh1.tex}
  \newcommand{\meshpoint}[3]{%
    \node[fill=white,inner sep=0pt] at (#2,#3) {%
      \textcolor{red}{\fontsize{3pt}{5pt}\selectfont #1}};}
  % Superpose the original mesh with thinner lines so that differences
  % are easily detected.
  \pgfsetlinewidth{0.3pt}
  \input{testmesh.tex}

  \begin{scope}[xshift=\unitlength]
    \input{testmesh1.tex}
  \end{scope}
\end{tikzpicture}

\begin{tikzpicture}
  \pgfsetxvec{\pgfpoint{0.45\linewidth}{0mm}}
  \pgfsetyvec{\pgfpoint{0mm}{0.45\linewidth}}
  \input{levels.tex}
\end{tikzpicture}


% Include the EPS file of Scilab
\IfFileExists{triangle.eps}{%
  \rotatebox{-90}{%
    \includegraphics[height=\linewidth]{triangle}%
  }%
}{}

\end{document}

%%% Local Variables:
%%% mode: latex
%%% TeX-master: t
%%% End:
