\documentclass[12pt]{article}

\usepackage[utf8]{inputenc}
\usepackage{graphicx}
\usepackage{floatflt}
\usepackage{blindtext}
\usepackage{enumitem}
\usepackage{amsthm}
\usepackage{subfig}
\usepackage{listings}
\usepackage{listingsutf8}
\usepackage{amsmath}
\usepackage{framed}
\usepackage{titling}
\usepackage{minibox}
\usepackage{float}
\usepackage{wrapfig}
\usepackage{longtable}
\usepackage[strict]{changepage}
\usepackage{pgfplots}
\usepackage{tikz}
\usetikzlibrary{matrix}
\pgfplotsset{width=11cm,compat=1.9}
\usepgfplotslibrary{external}
\tikzexternalize

\definecolor{vgreen}{RGB}{104,180,104}
\definecolor{vblue}{RGB}{49,49,255}
\definecolor{vorange}{RGB}{255,143,102}
\lstdefinestyle{bash} {language=bash, basicstyle=\ttfamily,
	keywordstyle=\color{vblue}, identifierstyle=\color{black},
	commentstyle=\color{vgreen}, tabsize=4,
%	moredelim=*[s][\colorIndex]{[}{]},
	literate=*{:}{:}1}

\lstdefinestyle{caml} {language=caml, basicstyle=\ttfamily, columns=[c]fixed,
 keywordstyle=\color{vblue}, identifierstyle=\color{black},
 commentstyle=\color{vgreen}, upquote=true, commentstyle=, breaklines=true,
 showstringspaces=false, stringstyle=\color{blue},
 literate={'"'}{\textquotesingle "\textquotesingle}3}

\title{Minicaml, a purely functional, didactical programming language with an
interactive REPL.}
\author{Alessandro Cheli\\Course taught by Prof. Gianluigi Ferrari and Prof. Francesca Levi}
\begin{document}
\begin{titlingpage}
\maketitle

\begin{abstract}
\textbf{minicaml} is a small, purely functional interpreted programming language
with a didactical purpose. It is based on the Prof. Gianluigi Ferrari and
Prof. Francesca Levi's minicaml, an evaluation example to show students
attending the Programming 2 course at the University of Pisa how interpreters
work. It is an interpreted language with a Caml-like syntax, featuring
interchangeable eager and lazy evaluation, a didactical REPL
that shows each AST expression and each evaluation step.
\end{abstract}
\end{titlingpage}


Test
\end{document}