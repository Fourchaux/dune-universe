\chapter{Using the RFSM compiler}
\label{cha:rfsmc}

The RFSM compiler can be used to
\begin{itemize}
\item produce graphical representations of FSM models and programs (using the \verb|.dot| format),
\item simulate programs, generating execution traces (\verb|.vcd| format),
\item generate C, SystemC or VHDL code from FSM models and programs.
\end{itemize}

This chapter describes how to invoke compiler on the command-line. On Unix systems, this is
done from a terminal running a shell interpreter. On Windows, from an MSYS or Cygwin
terminal.

\medskip
The compiler is invoked with a command like :

%rfsmc [options] file\textsubscript{1} ... file\textsubscript{n}
\begin{FVerbatim}[commandchars=\\\{\}]
rfsmc [options] \emph{source_files}
\end{FVerbatim}

\medskip
There must be at least one source file. If several are given, all happens as if a single one,
obtained by concatening all of them, in the given order, was used. 

\medskip
The complete set of options is described in App.~\ref{cha:compiler-options}.

\medskip
The set of generated files depends on the selected target. The output file \texttt{rfsm.output}
contains the list of the generated file.

\section{Generating graphical representations}
\label{sec:gener-graph-repr}

%rfsmc -dot f\textsubscript{1}.fsm ... f\textsubscript{n}.fsm
\begin{FVerbatim}[commandchars=\\\{\}]
rfsmc [-options] -dot \emph{source_files}
\end{FVerbatim}

The previous command generates a graphical representation of each FSM model 
contained in the given source file(s). If the source file(s) contain(s) FSM instances, involving global IOs
and shared objects, it also generates a graphical representation of the the corresponding system. 

The graphical representations use the \verb|.dot| format and can be viewed
with the \texttt{Graphviz} suite of tools\footnote{Available freely from
  \texttt{http://www.graphviz.org}.}.

The representation for the FSM model \verb|m| is generated in file \verb|m.dot|. When generated, the representation
for the system is written in file \verb|main.dot| by default. The name of this file can be changed
with the \verb|-main| option.

By default, the generated \verb|.dot| files are written in the current directory. This can be changed with the
\verb|-target_dir| option.

\section{Running the simulator}
\label{sec:running-simulator}

\begin{FVerbatim}[commandchars=\\\{\}]
rfsmc [-options] -sim \emph{source_files}
\end{FVerbatim}

The previous command runs simulator on the program described in the given source files, writing
an execution trace in VCD (Value Change Dump) format.

The generated \verb|.vcd| file can be viewed using a VCD visualizing application such as
\verb|gtkwave|\footnote{gtkwave.sourceforge.net}.

By default, the VCD file is named \verb|main.vcd|. This name can be changed using the \verb|-main| option.

By default, the VCD file is written in the current directory. This can be changed with the
\verb|-target_dir| option.

\section{Generating C code}
\label{sec:gener-c-code}

\begin{FVerbatim}[commandchars=\\\{\}]
rfsmc [-options] -ctask \emph{source_files}
\end{FVerbatim}

For each FSM model \verb|m| contained in the listed source file(s), the previous command generates a file
\verb|m.c| containing a C-based implementation of the corresponding behavior.

By default, the generated code is written in the current directory. This can be changed with the
\verb|-target_dir| option.

\section{Generating SystemC code}
\label{sec:gener-syst-code}

\begin{FVerbatim}[commandchars=\\\{\}]
rfsmc [-options] -systemc \emph{source_files}
\end{FVerbatim}

If the source file(s) only contain(s) FSM \emph{models}, then, for each listed FSM model \texttt{m}, 
the previous command generates a pair of files \verb|m.h| and \verb|m.cpp| containing the
  interface and implementation of the SystemC module implementing this model.

\medskip
If the source file(s) contain(s) FSM \emph{instances}, involving global IOs
and shared objects, it generates
\begin{itemize}
\item for each FSM instance \verb|m|, a pair of files \verb|m.h| and \verb|m.cpp| containing the
  interface and implementation of the SystemC module implementing this instance,
\item for each global input \verb|i|, a pair of files \verb|inp_i.h|
  and \verb|inp_i.cpp| containing the interface and implementation of the SystemC module describing
  this input (generating the associated stimuli, in particular),
\item a file \verb|main.cpp| containing the description of the \emph{testbench} for simulating the
  program.
\end{itemize}

The name of the file containing the \emph{testbench} can be changed with the \verb|main| option.

\medskip
By default, the generated code is written in the current directory. This can be changed with the
\verb|-target_dir| option.

\medskip Simulation itself is performed by compiling the generated code and running the executable,
using the standard SystemC toolchain.  In order to simplify this, the RFSM compiler also generates a
customized \emph{Makefile} so that compiling and running the code generated by the SystemC backend
can be performed by simply invoking \verb|make|. For this, the compiler simply needs to know where
to find the predefined template from which this \emph{Makefile} is built. This is achieved by using
the \verb|-lib| option when invoking the compiler. For example, provided that RFSM has been
installed in directory \verb|/usr/local/rfsm|, the following command

\begin{FVerbatim}[commandchars=\\\{\}]
rsfmc -systemc -lib /usr/local/rfsm/lib -target_dir ./systemc \emph{source_file(s)}
\end{FVerbatim}

will write in directory \verb|./systemc| the generated source files and the corresponding
\verb|Makefile|. Compiling these files and running the resulting application is then simply achieved
by typing

\begin{verbatim}
cd ./systemc
make 
\end{verbatim}

\medskip
\textbf{Note}. The generated \emph{Makefile} uses platform-specific definitions which have been
written in a file named \verb|platform| located in RSFM library directory
(\verb|/usr/local/rfsm/lib/etc/plaform| in the example above). This file is generated by
the installation process from the values given to the \verb|configure| script. Depending on your
local SystemC installation, some definitions given in the \verb|platform| file may have to be
adusted.

\section{Generating VHDL code}
\label{sec:generating-vhdl-code}

\begin{FVerbatim}[commandchars=\\\{\}]
rfsmc [-options] -vhdl \emph{source_files}
\end{FVerbatim}

If the source file(s) only contain(s) FSM \emph{models}, then, for each listed FSM model \texttt{m}, 
the previous command generates file \verb|m.vhd| containing the entity and architecture describing
this model.

\medskip
If the source file(s) contain(s) FSM \emph{instances}, involving global IOs
and shared objects, it generates
\begin{itemize}
\item for each FSM instance \verb|m|, a file \verb|m.vhd| containing an entity and architecture
  description for this instance,
\item a file \verb|main_top.vhd| containing the description of the \emph{top level} model of the
  system,
\item a file \verb|main_tb.vhd|containing the description of the \emph{testbench} for
  simulating the system.
\end{itemize}

\medskip The name of the files containing the \emph{top level} description \emph{testbench} can be
changed with the \verb|main| option.

\medskip
By default, the generated code is written in the current directory. This can be changed with the
\verb|-target_dir| option.

\medskip
The produced files can then compiled, simulated and synthetized using a standard VHDL
toolchain\footnote{We use GHDL for simulation and Altera/Quartus for synthesis.}.

\medskip
As for the SystemC backend, the RFSM compiler simplifies the compilation and simulation of the
generated code by also generating a dedicated \emph{Makefile}. For example,
and, again, provided that RFSM has been installed in directory \verb|/usr/local/rfsm|, the following
command

\begin{FVerbatim}[commandchars=\\\{\}]
rsfmc -vhdl -lib /usr/local/rfsm/lib -target_dir ./vhdl \emph{source_file(s)}
\end{FVerbatim}

will write in directory \verb|./vhdl| the generated source files and the corresponding
\verb|Makefile|. Compiling these files and running the resulting application is then simply achieved
by typing

\begin{verbatim}
cd ./vhdl
make 
\end{verbatim}

\section{Using \texttt{rfsmmake}}
\label{sec:rfsmmake}

The current distribution provides a script named \verb|rfsmmake| aiming at easing the use of the
RSFM compiler in a command line environment. With this tool, the only thing required is to write a
small \emph{project description} (\verb|.pro| file\footnote{The \texttt{.pro} file is also used by
  the GUI described in chapter~\ref{cha:gui}.}). Invoking \verb|rfsmmake| will then
automatically build a top-level \emph{Makefile} which can be used to invoke the compiler, generate
code and exploit the generated products.

Suppose, for instance, that the application is made of two source files, \verb|foo.fsm|, containing the FSM model(s), and
\verb|main.fsm|, containing the global declarations and FSM instanciations (the so-called
\emph{testbench}). Writing the following lines in file \verb|main.pro|

\begin{lstlisting}[language=make,frame=single]
SRCS=foo.fsm main.fsm
DOT_OPTS= ...
SIM_OPTS= ...
SYSTEMC_OPTS= ...
VHDL_OPTS= ...
\end{lstlisting}

\noindent
and invoking

\begin{verbatim}
rfsmmake main.pro
\end{verbatim}

\noindent
will generate a file \verb|Makefile| in the current directory. 
Then, simply typing\footnote{Please refer to the generated \emph{Makefile} for
  a complete list of targets.}
  \begin{itemize}
  \item \verb|make dot| will generate the \verb|.dot| and lauch the corresponding viewer,
  \item \verb|make sim.run| to run the simulation using the interpreter (\verb|make sim.show| to display results),
  \item \verb|make ctask.code| will invoke the C backend C and generate the corresponding code,
  \item \verb|make systemc.code| will invoke the SystemC backend  and generate the corresponding code,
  \item \verb|make systemc.run| will invoke the SystemC backend, generate the corresponding
    code, compile it and run the corresponding simulation,
  \item \verb|make vhdl.code| will invoke the VHDL backend  and generate the corresponding code,
  \item \verb|make vhdl.run| will invoke the VHDL backend, generate the corresponding
    code, compile it and run the corresponding simulation,
  \item \verb|make sim.show| (resp \verb|make systemc.show| and \verb|make vhdl.show|) will display
    the simulation traces generated by the interpreter (resp. SystemC and VHDL simulation).
  \end{itemize}


%%% Local Variables: 
%%% mode: latex
%%% TeX-master: "rfsm"
%%% End: 
