\chapter{Syntax}
\label{chap:syntax}

This section is a brief, informal, presentation of the syntax used to write RFSM programs. A
precise description, in BNF, can be found in Appendix~A.

\bigskip
An RSFM program comprises three sections, containing, respectively~:
\begin{itemize}
\item declarations of \textbf{FSM models},
\item declarations of \textbf{global objects},
\item declarations of \textbf{FSM instanciations}.
\end{itemize}

\section{FSM models}
\label{sec:fsm-models}

An \textbf{FSM model} describe the interface and behavior of a \emph{reactive finite state
  machine}. This description can be \emph{generic} in the sense that it can depend on a set of
parameters whose value will only be specified when the model will be instanciated.

The general form of an FSM model is given in listing.~\ref{lst:fsm-model-gen}.

\begin{lstlisting}[language=Rfsm,frame=single,caption=Overall syntax for FSM models,label=lst:fsm-model-gen]
fsm model <@\emph{parameter declarations}@> @\emph{name}@ (
     @\emph{io declaration}@
  {
  states: @\emph{state declaration}@;
  vars: @\emph{variable declaration}@;
  trans: @\emph{transition descriptions}@;
  itrans: @\emph{initial transition description}@;
  }
\end{lstlisting}

where
\begin{itemize}
\item \emph{name} is the name of the defined model,
\item \emph{parameter declarations} is an optional list of generic parameters,
\item \emph{io declarations} list the inputs and outputs of the FSM,
\item \emph{state declarations} list the states of the FSM,
\item \emph{variable declarations} is an optional list of internal variables,
\item \emph{transition descriptions} describe all the transitions of the FSM,
\item \emph{initial transition description} describes the initial transition of the FSM.
\end{itemize}

\medskip
\noindent
Within FSM models, \textbf{transitions} are denoted 

\begin{verbatim}
                           src -- cond | acts -> dst
\end{verbatim}

where
\begin{itemize}
\item \texttt{src} is the source state,
\item \texttt{dst} is the destination state,
\item \texttt{cond} is the condition trigerring the transition,
\item \texttt{acts} is a list of actions performed when then transition is enabled.
\end{itemize}

\medskip
A \textbf{condition} involves exactly one \emph{triggering event} and, possibly, a conjunction of boolean
conditions called \emph{guards}. The triggering event must be listed in the inputs. The guards may
involve inputs and/or internal variables.

\medskip
The \texttt{actions} associated to a transition may involve outputs and/or internal variables.
The set of actions may be empty. In this case, the transition is denoted :

\begin{verbatim}
                           src -- cond -> dst
\end{verbatim}


\medskip
The semantics is that the transition is enabled whenever
the FSM is in the source state, the triggering event occurs and all the guards evaluate to
true. The associated actions are then performed and the FSM moves to the destination state.

\medskip The listing in Fig.~\ref{fig:fsm-model-ex} gives an example of a FSM model declaration. The
declared model is that of a simplified mouse controler. It has two inputs, \verb|Top| and
\verb|Clic|, and two outputs, \verb|SimpleClic| and \verb|DoubleClic|. All inputs an and outputs are
of type \verb|event|. Input \verb|Top| is here supposed to be periodic and hence provide an time
base. Whenever an event occurs on the input \verb|Clic|, an event is emitted either on
\verb|DoubleClic| or \verb|SimpleClic| depending on whether the \verb|Clic| event is followed by
another before \verb|D| events occur on the \verb|Top| input, where \verb|D| is a parameter of the
model. The description of the behavior employs two states, named \verb|Idle| and \verb|Wait|, an
internal variable, \verb|ctr|, and four transitions rules, listed in the \verb|trans:| section.  The
first rules says that a transition from state \verb|Idle| to state \verb|Wait| happens whenever and
event occurs on input \verb|Clic| and that the internal variable \verb|ctr| is then reset to 0. The
second rule says that a transition from state \verb|Wait| to state \verb|Idle| happens whenever and
event occurs on input \verb|Clic| and that an event is then emitted on output \verb|DoubleClic|. The
third rule says that a transition from state \verb|Wait| to itself happens whenever and event occurs
on input \verb|Top| provided that, at this instant, the value of variable \verb|ctr| is less than
\verb|D-1|. The variable \verb|ctr| is then incremented. The fourth and last rule says that a
transition from state \verb|Wait| to state \verb|Idle| happens, emitting an event on output
\verb|SimpleClic|, whenever and event occurs on input \verb|Top| and that, at this instant, the
value of variable \verb|ctr| is equal to \verb|D-1|. Finally, the initial transition, given after
the keyword \verb|itrans:| designates the state \verb|Idle| as the initial state (with no associated
action here). 

A graphical representation of this FSM is given on the right in Fig.~\ref{fig:fsm-model-ex} (this
diagram has been automatically generated by the RFSM compiler, as explained in
Chap.~\ref{chap:using}). 

\begin{figure}[t]
  \begin{minipage}[b]{0.6\linewidth}
   \centering
\begin{lstlisting}[language=Rfsm,frame=single,basicstyle=\small]
fsm model ctlr<D:int> (
  in Top: event,
  in Clic: event,
  out SimpleClic: event,
  out DoubleClic: event)
  {
  states:  Idle, Wait;
  vars:  ctr: int<0..D>;
  trans:
    Idle -- Clic | ctr:=0 -> Wait,
    Wait -- Clic | DoubleClic -> Idle,
    Wait -- Top.ctr<D-1 | ctr:=ctr+1 -> Wait,
    Wait -- Top.ctr=D-1 | SimpleClic -> Idle;
  itrans: -> Idle;
  }
\end{lstlisting}
  \end{minipage}
  \begin{minipage}[b]{0.4\linewidth}
   \includegraphics[height=5cm]{figs/mousectlr-dot}
   \centering
  \end{minipage}
\hfill
  \caption{Example of an FSM model description (with its graphical representation)}
  \label{fig:fsm-model-ex}
\end{figure}



\section{Globals}
\label{sec:globals}

There are three types of global values : inputs, outputs and shared objects. Global inputs and
outputs represent the interface of the system to the external world and shared objects are used as
communication and synchronisation media between the FSMs composing this system.

For global outputs and shared objects, the declaration simply gives a name and a type. For example :

\begin{lstlisting}[language=Rfsm,frame=single,basicstyle=\small]
output SimpleClic: event
shared C0: bool
\end{lstlisting}

For global inputs, the declaration also specifies the \textbf{stimuli} which are attached to the
corresponding input for simulating the system. There are three types of stimuli : periodic and
sporadic stimuli for inputs of type \verb|event| and value changes for scalar inputs.
Periodic stimuli are specified with a period, a starting time and an ending time. Sporadic stimuli
are simply a list of dates at which the corresponding input event occurs. Value changes are given as
list of pairs \verb|t:v|, where \verb|t| is a date and \verb|v| the value assigned to the
corresponding input at this date. 

Examples:

\begin{lstlisting}[language=Rfsm,frame=single,basicstyle=\small]
input Top: event = periodic(10,10,120)
input Clic: event = sporadic(25,75,95)
input Enable : bool = value_changes (0:0, 25:1, 35:0)
\end{lstlisting}

\section{FSM instances}
\label{sec:fsm-instances}

Systems are described by \emph{instanciating} -- and, possibly, inter-connecting -- FSM
models. Instanciating a model creates a ``copy'' of the corresponding FSM for which
\begin{itemize}
\item the value of the declared generic parameter (ex: \verb|D| in the model of
  listing~\ref{fig:fsm-model-ex}), is fixed,
\item the declared inputs and outputs are actually connected to global inputs, outputs or shared
  objects.
\end{itemize}

The syntax for model instanciation is as follows

\begin{lstlisting}[language=Rfsm,frame=single,caption=Overall syntax for FSM models]
fsm @\emph{instance name}@ = @\emph{model name}@ <@\emph{parameter values}@> (@\emph{actual inputs
and outputs})
\end{lstlisting}

Of course, the type of the declared and actual IOs and parameters must match.

\medskip
For example, provided that globals \verb|Top|, \verb|Clic|, \verb|SimpleClic| and \verb|DoubleClic|
have been previously declared with the correct type,  writing

\begin{lstlisting}[language=Rfsm,frame=single,basicstyle=\small]
fsm c1 = ctlr<5>(Top,Clic,SimpleClic,DoubleClic)
\end{lstlisting}

will create an instance of the FSM model described in
Fig.~\pageref{fig:fsm-model-ex} named \verb|c1| and for which \verb|D=5|.


%%% Local Variables: 
%%% mode: latex
%%% TeX-master: rfsm
%%% End: 
