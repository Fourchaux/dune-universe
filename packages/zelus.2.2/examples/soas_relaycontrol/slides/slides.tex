% vim:nojs:spelllang=en_us tw=76 sw=2 sts=2 fo+=awn fmr={-{,}-} et ts=8
\documentclass{beamer}

\usepackage[T1]{fontenc}
\usepackage[utf8]{inputenc}
\usepackage{amsmath}

\mode<presentation>
{
  \usetheme{default} %Pittsburgh, default, boxes
  \setbeamertemplate{navigation symbols}{}
  \setbeamercovered{transparent}
  \setbeamersize
  {
    text margin left = 1.0em,
    text margin right = 1.0em
  }
  \setbeamercolor{section in toc}{fg=red}
  \setbeamercolor{section in toc shaded}{fg=structure}
  \setbeamertemplate{section in toc shaded}[default][100]
}

\usepackage{tikz}
\usetikzlibrary{shapes,arrows,calc}

\tikzset{appear/.code args={<#1>}{%
  \pgfkeysalso{fill=gray!10,draw=gray!50,text=gray!50}
  \pgfkeysalso{pin edge={latex-,color=gray!50,thick}}
  \only<#1->{\pgfkeysalso{fill=blue!10,draw=black,text=black}}
  \only<#1->{\pgfkeysalso{pin edge={color=black,latex-,thick}}}
  \only<#1>{\pgfkeysalso{fill=red!50,draw=black,text=black}}
}}

\tikzset{text appear/.code args={<#1>}{%
  \pgfkeysalso{text=gray!50}
  \only<#1->{\pgfkeysalso{text=black}}
}}

\tikzset{line appear/.code args={<#1>}{%
  \pgfkeysalso{color=gray!50}
  \only<#1->{\pgfkeysalso{color=black}}
}}

\begin{document}

\begin{frame}{Block diagram of a self-oscillating adaptive system} %{-{1

\centering

\tikzstyle{block}=[draw, fill=blue!10, rectangle,
                   minimum height=2.5em, minimum width=4em]
\tikzstyle{sum}=[draw, fill=blue!10, circle]
\tikzstyle{input}=[coordinate]
\tikzstyle{output}=[coordinate]
\tikzstyle{pinstyle}=[pin edge={latex-,black,thick},pin distance=.4cm]

\scalebox{.8}{
\begin{tikzpicture}[auto, node distance=2cm,>=latex,thick]
    \node[input, name=input] {};

    \node[block, right of=input,node distance=1.2cm] (model) {Model};
    \draw[draw,->] (input) -- (model);

    \node[sum, right of=model] (sum1) {$\Sigma$};
    \draw[draw,->] (model) -- node {$y_m$} (sum1);

    \node[block, right of=sum1, appear=<2>] (filter) {$G_f(s)$};
    \node[above of=filter,node distance=.7cm, text appear=<2>] {Filter};
    \node[coordinate, left of=filter,node distance=1.1cm] (afterfilter) {};
    \draw[->] (sum1) -- node[below] {$e$} (filter);

    \node[sum, right of=filter,
          pin={[pinstyle,appear=<4>]below:Dither},
          appear=<4>]
      (sum2) {$\Sigma$};
    \draw[->] (filter) -- (sum2);

    \node[block, right of=sum2] (relay) {};
    \draw[->] (sum2) -- (relay);
    \draw[very thick] (relay.center) |- +(-.4,-.2);
    \draw[very thick] (relay.center) |- +(.4,.2);

    \node[block, right of=relay,node distance=2.5cm] (process) {$G_p(s)$};
    \node[above of=process,node distance=.7cm] {Process};
    \draw[->] (relay) -- (process);

    \node[output, right of=process] (output) {};
    \draw[->] (process) -- node [name=y] {$y$}(output);

    \node[block, below of=sum2,node distance=2.5cm,minimum width=2.5em]
      (feedback) {$-1$};
    \draw[->] (y) |- (feedback);
    \draw[->] (feedback) -| (sum1);

    \node[block, above of=filter,appear=<3>] (ffilter) {Filter};
    \draw[->,line appear=<3>] (afterfilter) |- (ffilter);

    \node[block, above of=relay,text width=3.5em,align=center,appear=<3>]
      (gchanger) {Gain\\changer};
    \draw[->,line appear=<3>] (ffilter) -- (gchanger);
    \draw[->,line appear=<3>] (gchanger) -- (relay);


\end{tikzpicture}}

\bigskip
\textcolor{gray}{\footnotesize Figure 10.7, Åström and Wittenmark, Adaptive 
Control, 2nd edition.}

\end{frame}
%--   }-}1%%%%%%%%%%%%%%%%%%%%%%%%%%%%%%%%%%%%%%%%%%%%%%%%%%%%%%%%%%%%

\begin{frame}{Model elements} %{-{1

\begin{block}{Process transfer function}
\[
    G_p(s) = \frac{K\alpha}{s(s + 1)(s + \alpha)}
\]
\end{block}

\begin{block}{Compensation network}
\[
    G_f(s) = 1.2\frac{s + 5}{s + 15}
\]
\end{block}

\begin{block}{Gain changer}
\begin{equation*}
  d = \begin{cases}
    d_1                              & \mbox{if }\lvert e\rvert > e_l\\
    d_2 + (d_1 - d_2)e^{-(t-t_0)/T}  & \mbox{if }\lvert e\rvert < e_l
\end{cases}
\end{equation*}
\end{block}

\end{frame}
%--   }-}1%%%%%%%%%%%%%%%%%%%%%%%%%%%%%%%%%%%%%%%%%%%%%%%%%%%%%%%%%%%%

\end{document}

